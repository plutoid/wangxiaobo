\chapter{大学四年级}

一 


在大学里的第四年,以前空空荡荡的信箱忽然满了起来,我开始收到推销各种东西的邮寄广告:时装、皮衣、首饰、化妆品、成套的唱片、CD、LD、丛书、文库,等等。有些东西过去买不起,有些东西人家不卖给我们;现在这些东西我都有了,堆在双层床的顶上。到目前为止,我还没付过钱,全是赊购。它们不仅是商品,还是我已经长大的证明。有一样东西人家在努力推销,我还没有买,那就是公寓的入住权。我今年已经二十二岁了,再有一年,就要毕业,搬出学生宿舍,住进黑铁公寓。以前的事情未必值得记述,对我来说,大学的四年级是第一个值得记录的年度。 

所有上过大学的人,都必须住在有营业执照的公寓里。据说公寓里特别好,别人想住都住不进去。假如你生在我们的时代,对这些想必已经耳熟能详,但你也可能生在后世,所以我要说给你知道——假如有样东西人人都说好,那它一定不好,这是一定之理。我有一个表哥,开着一所黑铁公寓。我和他说,想到公寓里看看。他说,我正要搬家,你就不用过来了。他正要搬进我们学校对面的旧仓库,正在那里装修房子。闲着没事时我常去看看,但装修公司的人不让我进去,说是这种地方不准学生来看。我说我是业主的表弟,表哥让我来看看工程质量,他们才让我进去了。 

我表哥的公寓里地下铺着黑色的水磨石,四壁上涂着黑色的油漆。整个楼层黑得一塌糊涂,看起来倒是蛮别致的。地面和四壁都做好了之后,在装修公司的泛光灯照耀之下,这地方像个夜里开放的溜冰场。但这地方想要住人的话,就得隔成房间才对。后来他们开始打隔断——水磨石地面上早就留好了地脚,他们在地脚上竖起了若干铁柱子,在铁柱子之间架起了铁栅栏,又在铁栅栏上涂上了黑漆。一面做这些事,一面往里面搬粗笨家具。等到这些活做好了之后,这地方倒像个动物园,放着很多关动物的笼子。和兽笼不同的是,每一间里都有一个小小的卫生间,有床,有桌子,这就让你不得不相信,这些笼子是给人住的:狮子老虎既不会坐抽水马桶,也不会坐椅子。我在滑溜溜的地面上走着,冷风刺着我的耳朵。时值冬日,北风在拆去了窗框的方洞中呼啸着。工人正把这些洞砌起来,此后这里会是一所没有窗户的房子,不点灯会伸手不见五指。我想不明白,为什么就不能留着窗户。 

我表哥的房子装修好了,他搬了过来,带着他的家具、杂物,还有六个房客。家具装在大卡车上,由搬家公司的人搬上楼去,房客装在一辆黑玻璃的面包车上,一直没有露面。那辆面包车窗子像黑铁公寓的窗子一样,装着铁栅栏,有个武装警卫坐在车里,还有几个站在了周围。等到一切都安顿好了,才把面包车门打开,请房客们下车。原来这些房客都是女的。有两位有四十来岁,看上去像学校里的教授。有三位有三十来岁,看上去像学校里的讲师。还有一位只有二十多岁,像一个研究生,或者是高年级同学。大家都拖着沉重的脚镣,手里提着一个黑塑料垃圾袋,里面盛着换洗衣服,只有那个女孩没提塑料袋。她们从车上下来,顺着墙根站成了一排,等着我表哥清点人数。 

我表哥搬家那天,北京城里刮着大风,天空被尘暴弄得灰蒙蒙的,照在地面上的阳光也变得惨白。有两位房客戴着花头巾,有三位房客戴着墨镜,其他人没有戴。我表哥说:老师们,搬家是好事情,大家高兴一点——这回的房子真不赖。但她们听了无动于衷,谁也不肯高兴。我想这是很自然的,披枷戴锁站在过往行人面前,谁也高兴不起来。我听说监狱里的犯人犯了错误时,就给他们戴上脚镣作为惩罚——这还是因为他们已经在监狱里,没别的地方可送了。我们不过是多读了几本书而已,又没招谁惹谁,干吗要戴这种东西。当然,给犯人戴的脚镣是生铁铸的,房客们戴的脚镣是不锈钢做的,样子非常的小巧别致。但它仍然是脚镣,不是别的东西。我表哥见我在发愣,就解释说:这不是搬家吗,万一跑丢一个就不好了——咱们平时不戴这种东西。我表哥像别的老北京一样,喜欢说“咱们”来套近乎,但我觉得他这个“咱们”十足虚伪,因为他没戴这种东西。这些房客里有五个戴着手铐或者拇指铐——这后一种东西也非常的小巧,像两个连在一起的顶针,把两手的大拇指铐在了一起。不过这也不是什么好东西,因为假如没有钥匙,不把大拇指砍掉是取不下来的,而把拇指砍掉了就会立刻成为残废。她们双手并在前面提着袋子,像动物园里的狗熊在作揖。我表哥又说:手铐出门时才戴,不是总戴着的。那个年轻的女孩倒是没戴手铐,双手被一条麂皮绳子反绑在了身后。她挺起胸膛,好像就要从容就义的样子。我表哥解释说:这位老师讨厌手铐,所以用根绳子。他还对我说,要是你将来讨厌手铐,或者对铁器过敏的话,也可以用根绳子——他是在和我说笑话。我听说癌病房里的病人总拿死和别人开玩笑,已婚的女人和未婚的女人间总拿性来开玩笑。但我觉得这个笑话十足虚伪,因为他自己并没有用根绳子嘛。所有公寓的人肘弯都扣着一根铁环,被一根铁链串在一起,只有我表哥例外,这件事让人看着实在有气。 



有句话我们经常听说:知识分子是社会的精英——而我正要变成一个知识分子,或者说,一个精英。以前我听到这里就满意了,现在不满意。现在我觉得更重要的是:应该怎么对待这些精英。这些房客们都穿着郑重的秋季服装——呢子的上衣和裙子,这些衣服都是很贵的;脸上涂了很重的粉,嘴唇涂得鲜艳欲滴。只有一个人例外:那个年,以的女防水同有化妆。她穿着花格衬衫,袖子挽到肘上,那个扣住手臂的钢环被掩在袖子里。下襟束在腰带里,那条小牛皮的腰带好像是名牌。腿上穿着褪色的牛仔裤,脚下穿一双雪白的运动鞋。那条不锈钢的脚镣亮晶晶的,镣环扣在套着白袜子的脚腕上。背着手,姿势挺拔,四下张望着——她排在队尾。混在这样一群人里她非常抢眼,我不禁盯住了她。她的领口敞开着,露出了锁骨和一部分胸口,随着呼吸平缓地起伏着。后来她转过身去背对着我——她的小臂修长,手腕被黑色的皮条纠缠着。有时她握紧拳头,把双手往上举着,这样双臂这就构成个W形;有时又把手放下来,平静地搭在对面的手臂上。与此同时,别的房客低着头,一动都不动。直到一切都安顿好了,我表哥才说:好,进去吧。房客们从黑铁公寓的前门鱼贯而入,像一伙被逮住的女贼。那个女孩走在最后,她在我脚上踩了一脚,说:小傻冒!看什么你?既然她说我是傻冒,想必我就是傻冒了,但她也该告诉我,我到底傻在哪里。我还想和她说几句,但她已经走过去了。电动的铁门哗啦啦地关上,把别人都挡在了门外。 



二 

我住的宿舍离学校的南墙很近,学校的南墙又和我表哥开的公寓很近,有一段南墙是砌锅炉的耐火砖砌的,黄碜碜的,看起来很古怪。墙下有窄窄的一条草坪,出了南墙就能看见,总没人浇水,但草还活着。草坪里种了一丛丛的月季,夏天草坪上满是西瓜皮。草坪前面是马路,过了马跃就到了黑铁公寓门前。人们说,所有的聪明人都住在公寓里,住在公寓外面的人都不够聪明。聪明人被人像大蒜一样拴成一串,这件事却未必聪明。你知道的吧,这世界上最不幸的事就是:吃了千辛万苦,做成一件傻事情。 

黑铁公寓是一座四四方方的混凝土城堡,从外面看起来是浅灰色的,但它名副其实,因为它里面非常的黑。在高高的天花板上,亮着一盏遥远的水银灯,照着这间宽大的房子,好像一座篮球馆内部的样子,但是这里没有篮球架子。从底层的中央乘升降机到达四楼,你会发现自己在十字交叉的通道的中心。每条通道通向一个窗子,窗子的大小刚够区别白天和黑夜。在通道两边,雕花的黑漆铁栏杆后面,就是黑铁公寓的房间——房间里的一切都一览无余,你怎么也不肯同意,像这样的小房间可以要那么多的房钱。但是人家也不需要你同意,他们径直把你推进其中的一间,然后你就得为这间房子付钱了。隆冬时节,黑铁公寓里面流动着透明的暖风,从铺在地面上的橡胶地毯上方流过,黑铁公寓里面一尘不染,多亏了有效的中央空调系统。这里有第一流的房间服务——一日三餐都有人从铁门上的送饭口送进来。从这个口子送进来的还有内衣和卫生纸、袋装茶和袋装咖啡——在动物园里,人们也是这样给笼养的猛兽送东西,只是不送袋装咖啡——住在这个笼子里,你大概也用不着别的东西。这个地方过去是座旧仓库,现在是黑铁公寓。打听了这所公寓的房钱之后,我得出了这样一个结论:这黑铁公寓可真是够黑的。 

经过深思熟虑,我在表哥那里打了一份工。大学四年级功课不忙,现在放寒假,我又需要钱。至于为什么要到表哥那里打工,我也说不清楚:深思熟虑的结果往往就是说不清楚。上工的头一天,我表哥说道:咱们这里什么都好,就是少了一样东西——他让我猜猜是什么。我想了半天没有想出来,他告诉我说:这里有七个房间,但只有六个房客,所以少了一个房客,空了一个房间。402室就是穿着的。算数我是会的,但我没有注意过这件事。我倒注意到他说到空了一间房时看了我一眼,我马上就感到不舒服。他让我想想该怎么办,我又没想出来。他告诉我说:应该去买一个来。原来房客还可以买卖。这件事我不知道,想不出来也怪不得我啦。他打电话请人来替班,我们俩开车去了房客市场,后来是股票交易所,现在卖人——什么能赚钱就卖什么,用我表哥的话说,什么牛逼这里就卖什么,这话把我逼入了两难境界。如果说房客,也就是社会的精英,是不够牛逼的货物,我没法同意,这等于说我也不够牛逼。但若说他们是牛逼的货物,我也不喜欢——谁也不愿被比作 一个牛逼。 



市场里熙熙攘攘,有很多摊位,每个摊位上都拴着好几个很牛逼的货物,穿着打扮和我表哥的房客搬家时差不多,但每人手里都有一把折扇,假如有人来问,就打开来遮着脸,隔着扇子和他说话——看起来像日本的艺妓。假如人成为商品,就应该遮着脸。 

你未必去过那个房客市场,但你早晚是要去的:不是作为买主,而是作为货物。这间房子很高,没有天花板,在透光的塑料瓦中央有一个长方形的天窗。从底下看上去,天窗就像个亭子,或者说,像一道长廊。盯着它看得久了,脑海还会冒出来些木字边的中国字:“榭”、“枋”之类;这些建筑都是木头造的,但现在天然的木头很少了,这个天窗是角铁焊出来的。你正看得出神,忽然手上一阵冰凉。低头一看,眼前是一件黑皮茄克和一个秃头,他正把戴着黑皮手套的手放在你手腕上。当然,你是货物,对方是主顾。此时你如梦方醒,连忙用扇子把脸遮上。对方问道:你是干什么的?你要告诉他,是学中文的,除了从口袋里掏毕业证给他看,还要告诉他:我每月都有作品在刊物上发表。对方小声嘟囔道:这才几个钱哪。然后他后退半步,上上下下打量着你,摇摇头说:你该减减肥了。为了回答这种轻蔑,你要挺起胸膛,收紧肚皮,刷地把扇子一收,朗声说道:大家评评理,我这样子难道还算胖吗?有人给你鼓掌,都是卖主。有人嘘你,都是买主。有人一声不吭,都是货物。所有的货物都一声不吭,抬头看着天窗。 

我表哥说,有些公寓的房客多房间少,有些公寓房客少房间多,互相之间需要调剂。这是合乎道理的,但此地交易的方法实在古怪。看好了货以后,把他带到市场中心的公平秤那里,卸掉了手铐脚镣,脱掉外衣和裤子,往磅上一站:论斤约,每斤一百块。不管秃顶大胖子还是苗条小姑娘,都是这个价钱——这算是卖肉,也该分个等级。要是有什么争论,也都围绕这分量。买主指着房客说道:早上你给他揣了不少吧?这是指早饭而言。卖主则说,甭管揣了多少,你看看现在都几点了。这就是说,现在已经过了十点,早饭都消化了。我觉得这种买卖方法实在太笨,禁不住嘟囔了出来。我表哥听到了,就问我:照你看,应该怎么卖?我就提出了一个公式:用房客的收入乘一个权数,加他的预期寿命(这可以从他的健康状况估计出来)乘第二个权数,减掉他的消费。我表哥听了就说:扯淡。像你这么会算账,我都该进公寓,还开什么公寓呢……还是得论斤约!这话听得我目瞪口呆,因为它包含着精深的道理:有件事情你看着很笨,但别人都那么做,那就是因为不这么做就要倒霉 ——有这么一条,一切聪明与笨都要倒过来说。我表哥一点都不笨,甚至还可以说很精明——像这么精明的人却没有考上大学。也许这另有内情,但我不敢想下去了。 

从理论上说,我表哥是个文盲。他受过九年义务教育,但所有的功课都是零分,既不识字又不会算数。像这样的人才能开公寓,因为他不会和房客串通一气。实际上没有比这更虚伪的事了:现在哪有文盲呢。就拿我表哥来说吧,我不仅会算数,而且三位以下的加减法心算起来比我还要快。他还有阅读的嗜好,床底下的纸箱子里放了那么大一堆话本小说。在市场上他看过了一个待售房客的文凭,回过头来问我:表弟,这个词是什么意思:A-N-T-H-R-O-P-O-L-O-G-Y。气得我差点骂了出来:别斗孙子了!你要是不认识这个字,这么长一个单词,怎么能拼得一个字母都不错呢? 

我说表哥精明,还表现在他知道买大胖子不值。这种人不光是压秤,而且往往有一身的病,有时会犯心脏病,有时会中风。不管犯了哪种病,结果总是一样——用他的话来说:叫做“砸在手里了”。他专找苗条的人打听。终于找到了一个苗条小姑娘,看样子不超过四十公斤,明眸皓齿,虽瘦精神却旺盛,大概在三十年之内不会有砸在手里的问题。他很中意。一问职业,却是个画家。我表哥就嚷了起来:画家不要!都是穷光蛋,扔在街上都没有拣的!女孩很受打击,蹲在地下就哭起来了。我也蹲下去安慰她——她说自己毕业一年多了,每天都被牵出来卖,不得安生,也没法工作。要是今天再卖不出去,回去就自杀——但看她的样子不像是当真的。她一眼就看出我不是个买主,就问我是学什么的。我说是学应用数学的。她说你没这个问题——专业好,人又瘦,会很好卖。想到自己好卖,稍微有点得意,过了一会,又连打几个寒噤。 



三 

一般以为,有学问的人聪明,必须把他们关进公寓里,没有学问的人比较笨,让他们在外面跑跑没有什么——这个看法是错误的。有学问的人往往很笨,没有学问的人反而很聪明。这是因为假如学问会给人带来好处,聪明人就不会不要它,或者有了学问也不让你知道。因为这个原故,黑铁公寓里的房客就是一伙傻瓜,但她们都以为公寓里有个比她们还大的傻瓜,那就是我。 



每天早上我要从床上爬起来,送403室的房客去上班。这张床放在公寓的走廊里,紧贴403室。这位阿姨身材颀长,肤色黝黑,刚起床时头发乱糟糟地垂在脸两旁,像个印地安人。洗漱之后,她要把头发编成一根辫子。在我看来,这比任何一种发式都要麻烦。然后她又给脸化妆,这段时间也是非常的漫长。我还没有活到等女人的年龄,所以禁不住催促道:阿姨,能不能快一点?她答道:小表弟,不要急嘛。我要去上班。有两件事使我感到不快:第一,我不喜欢她强调自己要上班。在这所公寓里,只有她要上班,因为她是银行的职员。第二,我不喜欢她叫我表弟——我不是她的表弟。弄完了脸以后,她取出一叠衣服:外衣放在下面,内衣放在上面,都叠得整整齐齐,脱掉身上的梳装袍,仔仔细细穿戴起来——古代的武士上阵前披挂也没有她仔细。她穿的是一套暗色的男式西服,里面是薄薄的毛衣,所以显示出婀娜的曲线。我没看见她的大衣在哪里,看来她不准备穿大衣。今天外面在刮西北风,最高气温是零下10度。有句老话叫做“爱俏不穿棉,冻死不可怜”。我没有提醒她外面冷。既然是冻死不可怜,我可怜她干什么。 

403室的阿姨终于穿戴整齐,戴上了耳环,隔着铁栅栏让我看“可以不可以”。我答道:很可以。就打开铁门走了进去,手里拿了一个黑色的公文箱。这回轮到我问她可以不可以。她叹了一口气,把手伸了过来——这不是公文箱,而是一种手铐的式样。我怀着暗藏的快意,把她的双手铐在皮箱的把手上。 

北京的三环路两旁的人行道上有一些铁柱子,以前我不知道是干什么的。早上有些铁柱边上有人,一只手拿着一张报纸在看。此时北风正烈,会把报纸吹走。吹走了一份,他会从大衣口袋里拿出另一份。在旧报纸飞走之后,新报纸展开之前,你会看到他的一只手被铐在柱上的一个铁环里。这就是黑铁公寓的房客,在等上班的班车。我把403的房客带到过街天桥下,那里有一根铁柱子,是银行的班车站。此时我穿着一件破旧的蓝棉大衣,把头缩在领子里,从口袋里掏出一条铁链和一把大锁来,说道:伸伸手,阿姨。只要她一伸手,我就可以把铁链从她腋下穿过去,往铁柱子上一套,把她锁在这里,然后我就可以回去睡懒觉——班车司机有开锁的钥匙。但是她不伸手,反而把双臂夹紧说:你陪陪我。我偏过头来,看着她,用很不讨人喜欢的口吻说道:为什么呀?这座天桥底下是个风口,别的地方刮着五级风,这里有七级。403的房客跺着脚,把双手缩在袖口里,往四下看着,忽然把嘴凑到我耳畔说道:我怕在这里碰上性骚扰。这倒是个使我不能推托的理由。我往四下看着,看到几团废报纸神速地呼呼飞地之没看到有人经过。现在没人不等于总没人,我不好意思就这么溜掉。 



早上六点钟,黑铁公寓笼罩在一团黑暗的温暖里。虽然这里总是这么黑,但人的生物钟还在起作用,所有的房间里同有一丝声音,大家都在睡着。我睡在走廊的行军床上,被一阵刺耳的闹钟声吵醒,然后一盏雪亮的泛光灯直射我的面门。我像蝙蝠、像猫头鹰一样,讨厌这种突如其来的白光。403室的房客在白光下起身,脱下身上的睡袍,在卫生间里出出进进。我和她说过,换个红色的暗室灯就不会这么晃人。但她瞪着我看了好半天,然后说道:红灯怎么成?我要化妆。我要去上班,不化妆怎么成?我无话可说,只能眯着眼睛看她出出进进。她的样子当然无可挑剔,否则也不能在银行里做事。但我总觉得她小腹那里黑蓬蓬的一片,像生了一个大黑痣——起码那地方就难看得很。后来在马路边上,我心里一直想着那个大黑痣,对她的种种暗示就无动于衷——她在我身边不停地跳着脚,说道:冷啊,冷。我知道她的意思:她希望我把这件蓝色的破大衣解开,让她钻进来。但我不肯这么做:我不愿担上性骚扰的恶名。 

早上七点钟,灰白色的街道变成了淡蓝色,路边的楼房的墙壁出现了红色的光斑。这个红蓝两色的世界只有一个寓意,那就是冷。我从桥底下探出头去,看到天空明亮,空气透明。风在割我的脸。403室的房客转过身去躲避迎面来的风,她忽然叫道:你看。我转头看去,见到一个小个子,身穿一件破旧的军棉袄,双手揣在袖子里,从桥边走过。我没看到他的脸,只看到那一头乱发像板刷一样竖着。他走起路来一拐一拐的,看来小时缺钙给了他一双O形腿。我想他是一个四川来北京打工的民工。开头我不知道她叫 看什么,后来想起了她说自己常在等车时遇到性骚扰——这就是她说的骚扰者吧。我在心里冷笑了一下说:别扯淡了,人家会骚扰你吗? 



我表哥常常关照我说,要尊重房客。起初我觉得这种叮嘱是多此一举:我自己将来也是房客,我会不尊重自己吗?但后来发现这不是多此一举,在天桥底下403喋喋不休时,要不是想起了表哥的叮嘱,我早就出言顶撞了。她说到银行里的种种好处,不但发工资,还发东西:香水、唇膏、山美子牌的内衣(看来她穿在里面的就是山美子了,样子是有点怪,但她不说我是看不出来的),还发香烟,我表哥抽的骆驼牌香烟就是她们那里发的。这种烟是用土耳其烟草手卷的——我说我表哥这两天怎么满身的鸡屎味儿,原来是她祸害的。我不喜欢听到这些事,这可能是因为银行不雇数学家。但我也不是冷酷无情之辈:听到她说话声发抖,我几次想把大衣脱下来替她披上,但马上又变了主意——她又说到那家银行是外资的,有不少外籍职员,也许有天嫁个外国人,就可以出国,不住公寓了。我不喜欢听到这些话,也许是因为我是个男人,不做变性手术没人肯娶我。到后来,我听到她牙齿在打架,已经在解大衣的纽扣,但这时班车开来了,这个善举就没有做成。班车紧贴着马路牙子停下,前门打开,戴太阳镜的司机低头看看外面,说道:啊哈,有人送啊。403马上就振作起来,一面往班车上爬,一面说道:可不是吗,我们管理员的表弟,在我们这里打工——那辆班车方头方脑,所有的窗口都钉了铁条,叫人想起了运生猪的车——在车门关上之前,她对我说:晚上早点来接我,别忘了。我答应了一声,心里却在想:我要是能把这事忘了才好呢。 

我想把接403房客的事忘掉,但没有成功:我才22岁,忘不掉上课,忘不掉交作业,也忘不掉去考试,单把这件事忘掉,有点说不过去。但我磨磨蹭蹭,迟了二十分钟出门,我想这是说得过去的。走在路上我又在想心事,这就不可能走快。总而言之,走到天桥底下,天都快黑了。远远看到她抱着铁柱子站在那里。我表哥说:这种铐人的方式叫做恋人式,取人柱相亲相爱之意。但这种方式很不好,没给房客留任何的颜面:挺体面的人,当街搂根大柱子,算干什么的嘛。有些房客会想:你既不仁,我也不义——假如他身手敏捷,就会设法爬上柱子,从柱顶逃掉。当然他也没地方可去,最后还得回公寓,但先让你着一宿的急。403室的房客当然没有能力从柱顶逃掉,但这么铐着她也不好:天气这么冷,铁柱又没什么暖意。我赶紧脱掉大衣,走过去披在她背上,一面说:阿姨,我来晚了,对不起对不起。一面在各个口袋里搜索公文箱的钥匙。此时天色已暗,桥底下更黑,看不到她的脸——能看见我也不敢看。她低声说道:你能帮我擦擦鼻子吗?我当然能。她鼻子下面有好长一溜清水鼻涕,三层手绢都挡不住寒意。我说:鼻涕够凉的。她哼了一声,听不清楚是哭还是笑。 

晚上我陪403的房客回公寓,我走在她的身后。这也是表哥关照的:他说,你刚得罪了房客,千万别走在她的前面。在苍茫暮色中,她显得瘦小了很多,按说披上了一件棉大衣应该显得高大一些。走着走着,我觉得心里热辣辣的,禁不住说:刚才你碰到性骚扰了吗?她说道:刚才没有——从声调里听不出什么来。我又问:刚才没有什么时候有?她说:白天,在银行里。我说:那就不该怪人家民工。她叹口气说:是啊是啊。声音没精打采的。这可是少见的事,在所有的房客里,就属她总是精神抖擞。后来她跺起脚来,带着哭声说道:坏小子,还不快来暖暖我!她想让我钻进大衣,搂着她让她暖和一点。这件事也是我的日常工作。但我不肯去,还说:阿姨,这可是性骚扰。她终于哭了起来,说道:你干吗这么和我过不去?我不过是爱慕虚荣,没做什么坏事呀! 



四 

我表哥终于买到了中意的房客,但不是在市场上买的。但这件事说起来话就长了,暂时不必提起。寒假里,有一天下了雪。我表哥没在公寓里,他带房客散步去了。这本该是我的事情,但我回学校去听报告了。那天下午他在办公室里喝茶,看到401号的红灯亮了起来。红灯连闪了两下才熄灭了,这表示住户想要出去散步。此时办公室里只有他一个人。他把脚从桌子上拿下来,穿上大头靴子,套上他的黑皮茄克,从办公室里出去,走到401门前,看到里面的女孩已经准备停当:她把头发束成了马尾辫,脸上化了淡妆,穿着白色的衬衣,黑色的紧身裤,脚上穿着长统皮靴——看来她已经知道外面在下雪。她手里拿了一个白信封。这件管理员是个秃顶的彪形大汉,他从皮带上提起钥匙串,把铁门打开。此时那个女孩把信封塞到他上衣口袋里——信封里是小费。管理员说:用不着这样——然后又改口道:用不着现在给。但是钱已经给了。管理员看了一下这间房子:这里的每一样家具都是黑色的,黑我的矮床,床上罩着黑色的床罩,黑色的钢管椅子,黑色的终端台上,放着黑色的PC机——机器是关着的。一切都收拾得井井有条,用不着他心督促、管理之责。正如他平时常说的,401的房客最让人省心。桌面上还有一个黑色的磁杯子,里面盛着冒气的热咖啡。管理员建议道:先把咖啡喝了吧。那个女孩没有回答,只是面露不耐烦之色——这位房客虽让人省心,但是很高傲。于是他走向那张几乎看不见的黑皮沙发,叉开双腿坐了下来,然后那个女孩走到他面前,站到他两腿之间,然后转过身去,跪在地板上,把双手背到身后。管理员在牙缝里出了一口气,俯下身去,用手按住她的后脑,让她把头低得更低,直至面颊贴到冷冰冰的地板,然后从袖筒里掏出一根麂皮绳索,很熟练地把她的双手反绑在身后——我说的这件事发生在黑铁时代,黑铁时代的人有很多怪癖。这位管理员像一位熟练的理发师在给女顾客洗发,一面缠绕着绳子,一面说:紧了说话啊。但那个女孩没有说话——看来松紧适中。等到捆绑完毕,他把她扶了起来,转过她的身子,左右端详了一番,看到脸上没有沾到土,头发也没有散乱,就从衣架上拿起黑色的斗篷,给她围在身上,系好了带子。随后他又看到墙上还挂有一顶黑色的女帽,就把它拿到手里,想要戴到她的头上。但那女孩摇了摇头,于是他又把帽子挂在墙上,然后打开了铁门,让她走在前面,两个人一起到漫天的大雪里去散步。 

我在表哥的办公室里坐着时,桌面上的红灯也会亮起来。他已经告诉过我,红灯亮是房客要散步,还告诉了我应该怎样做。我站起身来说:表哥,我去。我表哥犹豫了一阵,在扶手椅里艰难地侧过了身子,从腰上解下了钥匙串,和袖筒里拿出的皮绳绕在一起扔给我说:对人家客气一点——最好叫声阿姨。这种关照是多余的,虽然她比我大不了几岁,我乐意叫她阿姨。我走到401室门外,里面的女孩瞪大了双眼看着我,大概没想会是我。我开了铁门,走到她的面前说:阿姨,我表哥叫我替他。她又发了一会儿愣,然后叹了口气说:讨厌啊,你。就转过身来,把双手并在一起。我坐在终端椅上,用那根皮绳把她的手反绑起来。平时我的手是挺巧的,但那一回却变得笨手笨脚,捆了个乱七八糟,而且累得两只手都抽了筋。办好了这件事,我站起来,拿了斗篷,笨手笨脚地要给她围上,又被好喝斥了一句:笨蛋!你先把我的衣领竖起来!后来我把斗篷给她披上了,带她出了门,到外面的小公园里去散步——那是在初冬的早晨,天气干冷干冷的。大风把地面上吹得干干净净。至于天上,就不能这么说。每个树枝上都挂着一个被风撕碎了的白色塑料袋,看起来简直有点恶心。 

401的房客想让我表哥带她去散步,不想让我带她去,我以为她是爱慕虚荣。对于女人来说,爱慕虚荣不算个毛病。我不会爱任何一个不爱慕虚荣的女人。那天晚上,403的房客,女士们银行的职员,检讨说自己爱慕虚荣,我听了以后钻进了那件棉大衣,抱住她说:别哭了,阿姨。我喜欢你。她听了马上就破涕为笑,说道:坏小子,别撒谎了。我知道你喜欢谁。401的房客神态傲慢,姿势挺拔,我当然喜欢她,这是明摆着的事。403告诉我说,她是刚进来的,所以这个样子,过上一段时间就和大家一样了,但我不信。403知道我说喜欢她是撒谎,还是叫我搂着她,走完了到公寓的路。我对她没什么意思,但也喜欢搂着她。看来这个谎言很甜蜜。过去皇宫里宫女和太监谈恋爱,大概就是这样的吧。 

我和401室的女孩在公园里,她在长椅上坐下来不走了,我站在她面前,搓着手——我穿得单薄,感觉到冷了,尤其是耳朵上。就这么过了一会儿,她忽然说道:你在这里干什么?我告诉她说:我在这里打工。她说:到哪儿打工不行,偏偏要来这里——真讨厌啊你。我说我在上大学四年级。她说:那又怎么样——口气很噎人。我说:照你看,我应该看都不来看看,径直就住进来?她说这是你的一口我怎么能知道什么应该什么不应该。我说:你不喜欢我,所以就说我讨厌。要是我表哥你就不讨厌了。听了这话,她皱起眉头来说:混账!然后又说:谁告诉你的?这不是明摆着的事吗,还用人告诉。她发了一会愣,然后对我说:你坐下吧,我在她身边坐下来。她接着发愣。又过了一会儿,她说:要是你乐意,不妨搂着我。我就搂着她,过了一会儿才说:这不算性骚扰吧。好笑了起来,说道:油嘴滑舌,讨厌啊你。然后把头放在我肩上了。 

我在表哥这里打工,他给我一本公寓员工守则。那上面第一条就是:对房客进行任何形式的性骚扰。但所有的人都没把这一条当回事。人都被看起来了,还说什么不准骚扰,简直是胡扯。要是公寓里换两个女的来看管,这些房客肯定要造反,因为她们不是同性恋者。这个小公园本是管理员和房客散步的场所,她不把头靠在我肩上,反倒显得不自然。她在我肩上伸直了脖子,说了一声:不准讨厌啊!就把眼睛闭上了。以后我就成了她打盹的枕头。因为我喜欢她,就心甘情愿地被枕着,肩膀压麻了也没说什么。 

黑铁公寓的管理员终身生活在皮革的臭味里,他们必须赤膊穿皮衣,请不要以为这是种好受的滋味。我就不肯这样穿衣服——到了热天要起痱子,冬天衣服里又是冷冰冰的。假如他是男人,就必须是条彪形大汉,脸相还要凶恶。像这样一位管理员在雪天带着401小姐在公园里散步,此时天上降落的雪和米粒相似,有时大块的雪还会从杉树枝上跌落下来。公园里空无一人,他跟在小姐身后从松软的雪层中走过,同时在心疼脚上的皮鞋。小姐在一棵树前站住了,他也趁机从口袋里掏出一盒烟来。就在此时,她转过身来,径直走到他面前说,我也想吸一支烟。此时他面临着抉择:他可以说,不要吸烟,吸烟对身体没好处。他还可以不回答径直走开,这些都是管理员对待房客的方法。但他从烟盒里取出一支揉皱的骆驼牌香烟递了过去。小姐笑了一下,说道:谢谢,我想抽自己的,在斗篷里面的口袋里。管理员把自己的烟收了起来,俯身撩开她的斗篷到里面找香烟。这件斗篷的里面异常的深,他在里面翻来覆去,终于找到了一盒红色的硬壳坤烟,从中取出一支放进嘴里,然后把烟盒放回口袋里,为小姐整理好斗篷,系好颈下的带子。把一切都整理好之后,他取出自己的打火机,点燃了这支香烟,吸进了一口带有荷花苦涩味的烟——这种味道使他联想到女人阴部的气味,所以他不喜欢这种烟。他把这口烟全都喷了出来,然后很熟练地把香烟掉过头去,放到小姐嘴里——此时他细心地关照了一声:用牙咬住,不然会掉的。而小姐也闷声说了声谢谢。她转过身去,在公园里继续漫步,直到天色变暗她感到心满意足时,才回到黑铁公寓她很喜欢今天的雪——可惜的是,不是每天都下雪。管理员跟在她的身后,他的时间也在一分一秒地过去。在内心深处,他感到无奈。但他知道,必须理解房客,尤其是在这天地一色的天气里。外面一片洁白,你却呆在漆黑的屋子里,这种处境让人想到失去了的自由,因而变得心痒难熬。你不能光想着收房钱,有时也要迁就一下房客的心境——管理员就是这么想的。他还想道:好在不是每天都下雪。这件事发生在雪天,这个管理员是我的表哥。 



五 

从前,有位二十三岁的女孩子,一个有才华的音乐家,收到一纸通敌说她已被判定为专门人才,是国家的宝贵财富。因此她必须搬入一家领有执照的公寓,享受保护性的居住。乍一拿到这纸通知,她像别人一样感到天旋地转,还觉得世界末日已经来临;或者说,像从医生那里知道自己得了癌。但她很快又镇定了下来。她也像别人一样,注意到通知末尾那一行字:在二十天之内,她拥有选择住入哪家公寓的权利;过了二十天,当局就要替她行使这种权利,代她指派一个公寓,这样的公寓必然又贵又不好。所以她也像别人一样匆忙地利用了这个权利——把京城里每一家公寓都看了一个遍。实际上,要选择一个终生居住的地方,二十天是根本不够的。但她也和别人一样,对自己最后选定的地方深感满意——这主要是因为,她不满意也搬不出去,除非她住的公寓赔钱,把她卖给别的公寓。她住的这家公寓实际上只有一个垂吊此人同时又是经理、主要股东、法人代表,等等;中等身材,长得很结实,头顶光秃秃,粗糙的脸上有很多面疱留下的疤痕。真实她很害怕此人的模样,后来就不可避免地爱上了他——但也不一定是真的爱上了。到天,她要请他带出去散步……如你所知,这个女孩住在我表哥公寓的401室里,这个管理员就是我的表哥。他身上有股鱼腥味,脸相凶恶,主要是因为他的眉毛很浓。我和我表哥都是自由的,但他将要自由下到,我却自由不了多久了。这是很大的区别。想起了这件事,我就会觉得万念俱灰,找个借口不去上班。下雪那天我该在公寓里,但我扯谎说学校里有事,就没有去。 

除了我们学校对面的公寓和我表哥这样的管理员,黑铁公寓和管理员还有别的模样。比方说,有这公寓:从正面的大铁门进来时,身后照进来灰色的天光,你可以看清眼前是一大片四四方方的空场,地上满是尘土、旧玻璃、陈年发黄的废纸,还有大片干涸了的水渍,靠墙的地方堆放着拆成了木板的包装木箱,靠墙的地方有些粗铁条焊成的小笼子,看起来和马戏团用来搬运狮子老虎的笼子同什么两样。隔着铁栅栏,可以看见里面放着大大小小的包装木箱,有些小木箱上放着棉垫子,这就是椅子,有些中等木箱上放着蛇形管工作台灯,这就是桌子。有人坐在这样的椅子上,从嘴里呵出热气,去温暖手上的冻疮。还有个大木箱铺着肮脏的棉门帘子,在门帘下面露出发黄的旧报纸,这就是你睡觉的床。被推进一间空置的笼子里时,假如发现角落里有干硬的陈年老屎,你千万不要感到诧异。等到电动的大铁门隆隆关上时,头顶那些蒙满了尘土的天窗下班继续透入半透明的光线,这地方原来是旧车间,现在是黑铁公寓。所以这个故事又可以重新讲述如下: 

当办公室里的红灯亮起来时,管理员把腿从桌子上拿了下来。她拿出一面小镜子照照自己的脸,这张脸的上半部盖着一层绿色的刘海,嘴唇涂得乌黑。她对自己的样子感到满意,就放下小镜子,披上黑皮上衣,从办公室里走了出去。她在走廊上歪歪斜斜地走着,弄出很大的声音,来到401室的门外,哗啦啦地打开铁门,大声大气地门道:要干什么?这就使呆在里面的人几乎不敢说自己要干什么。此人是个肤色苍白的秃顶的大汉,低头看着自己的鼻子,唯唯诺诺地说道,想出去散步。那女孩说道:讨厌。从自己腰带上解下一副手铐放在桌子上说。自己戴上,然后就一头闯到卫生间里去了。于是他就像戴手表一样,很仔细地自己把手铐戴在手腕上,然后瞪着大眼看卫生间的大门——门里伸出两只穿着皮靴的脚,还能听到一种湍急细流的响声。这个男人按捺着心跳,等着他的管理员。在黑铁公寓里,管理员总是人们关注的中心,哪怕她正坐在马桶上撒尿……她从卫生间里走出来,一面系黑色皮裤上的腰带,一面喘着粗气,端详着面前的男人。后来,她从衣架上拿下一件黑色的长袍,像用包装袋套住一台高大的仪器,把他罩在袍里(这件长袍没有袖子,只有两个伸出手来的口子,但已经缝死了),用黑布的头罩把他的头套住,只留下一双眼睛在外,就像伊斯兰国家的妇女,这样带他出去散步。上述两个故事发生在同一时间,但地点稍有不同——黑铁时代有不止一所黑铁公寓。有些人必须住在黑铁公寓里,因为他们太聪明。这个男人像一个会行走的黑布口袋一样跟在绿头发的管理员身后。他爱她,依恋她,因为她是自由的。 

我们学校对面原来是一片工业区,现在破败了,长满了荒草。有很多厂房、仓库,现在都空着。原来人们也没发现这些房子有什么用场,后来他们发现这里可以办公寓。短短几个月,有好几家黑铁公寓搬了进来,眼看这里要成为一个公寓区。下午时分,我从窗口往外看,看到有两对人从不同的大门出来。一对是我表哥,带着401的房客,他们往西面走了。穿过一片平房区,走过一座久已废弃的铁路桥,运河对面有个小公园。还有一对往东面走,这条路的尽头有条竖着的街,那条街叫做市场街,街上有个农贸市场——往那个方向走比较热闹。那个绿头发管理员我认识,最早时她在我们学校食堂里卖饭,后来有阵子她在农贸市场上摆烟摊;连账都算不清楚,而且喜欢说个“操”字。我也认识那个秃头——他在市场街上修过手表。和别的修手表的不同,他不是浙江人,而是本地人。这个人说话文质彬彬,不像个手艺人。他还托我到学校书店里买过书,买的什么我已给忘了。401的女孩走在我表哥前面,姿势挺拔;秃头跟在绿头发的身后,弓着腰。我从窗内看着,不停地擦去窗上的呵气。下班上有一大片水,后来留下了一片白蒙蒙的污渍,和白内障病人的眼珠很相似。 



六 

绿头发的女管理员总用手指挖鼻孔,除了其状不雅,还会使手指甲开裂。她走起路来就像一个醉汉一样东歪西倒,说话声音粗哑,但是她很温柔。401的房客,那条秃顶大汉和她出去散步,在街道上走了一会,就说:咱们到啤酒馆去坐一会吧——我请你。那个女孩想了想说:好吧——下回我请你——其实不管谁要请谁,都没有下一次了。于是他们来到一家熟识的啤酒馆,在一个僻静的车厢座里并肩坐下,要了两升啤酒,把头发染绿的管理员抬头看了看,没有人在注意他们,就撩起他的风帽,把啤酒杯端到他嘴前喂给他喝。桌子上有一碟花生米她,她一粒粒地拣给他吃,还说:小心点,别咬了我的手。假如驯兽员养了一只海狮,她就会这样喂它东西吃,也会关照海狮别咬她的手——驯兽员对海狮就是这样温柔。此时啤酒馆里静悄悄,好像没有几个人,但这只是一种假相。啤酒馆里其实有很多人。 

忽然之间,一伙大汉好像从地里冒了出来,拥到了桌前,用一要裹着脱皮的钢筋棍子把染绿了头发的管理员打晕,架起了穿黑袍的房客就走。后者是一条彪形大汉,但因为双手被铐住,无力抵抗。他能做的只是努力回头看倒在地上的女孩,但架住他的那些人说:快走吧,没你的事——她死不了的。他轻声答道:我知道。但又问了一句:你们不会把她打坏吧?她会不会得脑震荡?对后一个问题,劫人的人回答说:不知道。与此同时,他在别人的挟持之下飞奔着——这地方和黑铁公寓很近,被人撵上可不是闹着玩的。当天晚上,他就被卖掉了——请不要从字面上理解这件事。办公寓的希望有房客,而假如没有什么政策上的变化,房客就不会增多。所以就有了这样的事:有些人把某有公寓的房客劫走,介绍给另外一家——当然,这是要收钱的。这些人被叫做房客贩子。菜贩是蔬菜的来源,正如房客贩子是房客的来源。买卖房客只是改变他的住址,这和买卖人口是两回事。 

劫走了秃头的房客贩子们把他拖到农贸市场附近,塞进一辆小四轮拖拉机的拖车里,在他身上盖了一床肮脏的棉门帘——这样这辆拖拉机就像一辆运菜的车,而他就像一堆容易冻坏、必须盖上的蔬菜。在拖拉机开走之前,人家又把棉被撩开,很客气地问道:先生先生(大家都知道,住公寓的都是有文化的人),嘴里要不要塞东西?秃头想了一下,皱起眉头来说:不用塞——我不叫唤。就把头缩回棉被之下了。棉被下面虽然暖和,但有一大堆白菜。房客贩子们尊重被劫者的意见,就没有塞的嘴。贩子们只对管理员坏,对房客是很好的。与此同时,绿头发的管理员在地上醒了过来,感到头很晕,她看到自己的房客不见了,就赶紧回去叫人,去追那些房客贩子。此时她的样子不大好看,满头满脸都是血。后来才知道,她的后脑勺上打了一个大包,很久都不能平躺着睡觉。 

我说过,我请这个秃头修过表,他还托我买过书。后来才发现,他还是我的老校友。他读的也是数学系,只比我高六级。但他没有念到毕业,念到大三,说是得了神经衰弱跟不上功课,就何尝了,躲在市场街上修手表。和他同年的学生一个个都进了黑铁公寓,他还在修手表。看到我到市场街上来,戴着大学的校徽趾高气扬的样子,他心里免不了要暗自得意,还觉得我是望乡台上唱山歌,一个不知死的鬼。直到后来他被办事处的人堵在修表亭子里,人家拿出一纸公文,告诉他说:根据新规定,你读过三年大学,也算个知识分子,应该住进公寓里。当时他还很不虚心,对来人大叫大嚷说:不该有新规定。此人身体健壮,躲在亭子里负隅顽抗,别人拿他也同什么办法。直到那个绿头发的女孩拿出一样东西给他看,并且说道:你想跟我们走呢,还是想被它在头上敲一下,然后再被我们拖走?那东西是根铁管子,有一头套着浇花的胶皮管子,很有分量,足可以把人打晕过去。秃头被她说服,跟他们走了,来到了办事处办的公寓里。他很感激她,因为她也可以不说服,径直就来打他一下。后来就是她管着他,所以他对她百依百顺,很有感情——这些事情都是后来这秃头亲口告诉我的。 



天黑以后,401室的小姐和管理员乘电梯回到自己的楼层,他把她带进自己的办公室,为她解去斗篷,忽然把她推倒在办公桌上。如前所述,她的双手被反绑在身后,无法支撑身体,这下几乎把脸磕破。管理员一手握住她脑后的马尾辫,另一只手拉开抽屉,从里面拿出一把大剪子,嚓嚓几剪,就把她的长发剪短,剪得乱蓬蓬地像一个鸟窝。这意外的暴力早把女孩吓呆了。假如管理员的剪子停不住,就会把耳朵剪掉。她赶紧呜咽着说道:知道,我在衣服里藏了烟。管理员更加心平气和地问道:烟应该放在哪里?女孩说,应该放在办公室,要抽时出来抽。管理员说:看来你知道自己犯的错误,这就省得我费嘴了。——还有一条,你最好别抽烟。这样身体会好。说完了这些话,他把女孩带了出去,带到楼层中央的十字路口,这里有个矮矮的圆笼子,看上去像个字纸篓。管理员打开了笼子上面的锁,把女孩塞了进去。她在里面蜷着身子,就像母体里面的婴儿。管理员把笼门锁上——这是一把定时锁,和银行金库用的相仿——管理员说,等到锁开了,你自己出来,到办公室里找我,看看该拿你怎么办——说完就走了。剩下那个犯错误的女孩,在笼子里尽量坐直,等着面颊上的泪自己干掉,等着笼门上的锁自己打开。在黑铁时代,人们总是在等待着什么。 



在黑铁公寓,女孩缩在笼子里,已经睡着了,又被一阵杂沓的脚步声惊醒。一伙穿黑色皮衣的人拖来一个裹在黑布长袍里的男人。那个女孩没有看到他的脸,但是闻到了他的气味,并且嗅出了他是一个男人。住在黑铁公寓的人嗅觉都很灵敏。他们把这个人拉进了402室—— 那间房子原来是空着的,把他推倒在床上,然后出来锁上了门。此人从床上挣扎起来,追到门口来,从袍袖里伸出双手来说:你们先把我的手铐打开了啊。那伙人里为首的转了回来,看看他戴着手铐的手,态度很好地说道:你先忍忍,明天早上我们找锁匠——你还有张合同要签。然后他们都走开了。 

新来的人撩开长袍上的风帽,甩掉头发上的白菜叶子,环顾四周。这地方和他以前住的地方相仿:高高的天花板上悬着一盏水银灯,照着黑铁的笼子,惟一不同的是眼前有个圆形的笼子,其状像鸟笼,里面有个女孩,双手反剪着缩成一团。他朝她笑了笑说:Hi——这是什么地方?女孩答道:这里是黑铁公寓——你住的是402室。那男人苦笑着说:还是黑铁公寓,只是从401搬到了402——这倒不足为怪。生在黑铁时代,不住在黑铁公寓,还想住在哪里?又过了一会,那女孩忽然想表示一下礼貌,就说:Hi——我就住在401。我们是邻居。现在她有了个男人做邻居,但是并不开心。因为她觉得此人身上的气味不好,是一股铁腥气。她皱了一下鼻子,那男人马上就察觉了。他道歉说:不好意思,我身上味不好。不能怪我——我们那里几个月洗不了一次澡。女孩说:这里好多了。卫生间里可以洗淋浴。那个男人走进卫生间,发现果然如此,而且喷头里流出的还是热水。虽然如此,这里还是黑铁公寓,说不上哪儿比哪儿更好。而且他还戴着手铐,根本不能洗澡。他又走回门边,看看对面笼子里的女孩,清清嗓子说道:想不想聊聊?女孩把头扭开,轻声说道:还怕以后没的聊——别聊了吧。谁也不想被装在一个笼子里,反剪着双手和别人聊天。但她马上又改变了主意,把头转回来说:好啊,聊吧。但是,在黑铁公寓里又能聊些什么呢。 

对于以上事件,我还可以补上几句:下雪那天傍晚,有人在街东头的啤酒馆里打翻了一个管理员,劫走了一个房客,装在拖拉机上,转了一圈转到街西口,把他卖给了我表哥——此时我在场,因为房客贩子在门口用对讲机和他谈生意时,我表哥打电话叫我过去,还让我带着点家伙:和房客贩子打交道,谨慎一点可不是多余。于是我到了公寓外面,后腰上别着一把黑市上买来的钢珠手枪,站在马路对面的人行道上。我表哥见我来到,就把门打开,让那帮人进来,上了楼,把劫来的人送进房间,然后给了他们钱,让他们出去。在此期间我一直远远地跟在他们身后。这种一前一后的加热给他们一定程度的威慑。等到把这帮人打发出了门,我表哥对我说:干得不坏。我们表兄弟俩就到办公室里去喝咖啡。 

又过了不一会儿,原主,也就是那个绿头发的女孩,给我表哥打电话,说她那时丢了一个人。我表哥说,这个人在我这里,但是我花了钱。对方也就无话可说。过了一会儿,她又问:那帮劫人的家伙是什么样子?我表哥说:四个人,穿蓝色的旧工作服,开一辆“冀”字头的小四轮拖拉机,往京石路上走了。对方说:谢谢,欠你一个情。就把电话挂上了。我表哥也把电话挂上。我想这四个人要糟了。绿头发的那伙人肯定要开着卡车去追。拖拉机跑不过汽车,追上他们肯定要倒大霉——后来京石路边上就翻了一辆拖拉机,烧得转漆漆的。车厢里散放着四具黄碜碜的骨头架子,上面一点肉都没剩,像啃过了一样——也不知怎么烧得那么干净。我表哥知道了以后,对我说:该!就该这么整。让他们知道知道,在河北撒野成,北京容不得他们撒野。后来才知道,北京城里常能见到外地来的房客贩子,开着小四轮拖拉机、农用汽车,还有各种可怕的交通工具来推销他们的货色。公寓管理员、警方等有关人士完全知道他们是些贼,到京城来销赃,但只要他们不在本地犯案,就睁一只眼闭一只眼。这是因为北京是文化城,需要他们贩来的货物。把外地的知识分子贩到北京,对此地的繁荣有益。但假若他们敢在此地作案,就对他们毫不客气——一定要让他们知道,在京城作案是死路一条。那些骨头架子知道了这些没有,却没法问了。 



过了漫长的一刻,也许已经到了早晨吧,管理员来到402室,带来了一纸合同。秃顶的男人双手接住那张纸,眯起眼来凑近了瞧了一会,说道:看不见——我没戴眼镜。别人告诉他说:看不见没关系,你先签了吧,有什么问题以后还可以修改——这种话总是在骗人时说的。被骗的人知道这一点,但没说什么,乖乖地签了字。等到管理员走开时,他对笼子里的女孩说:这里好像不错——起码还肯骗骗我。那个女孩没有回答,只是歪着头。那男人关切地说:你哪里不好?女孩转过头来,想了一会儿,终于直言不讳地说道:我憋了尿!那个秃顶男人就去按了铃。管理员来了以后,问明了情况,把笼子打开,把女孩放了出来,解开她的双手,让她进了卫生间。她方便以后,重新化了妆,换了一件衣服,跪在地下,被反绑好双手,然后又钻进了那个鸟笼子——等到管理员吹着口哨走远之后,她抱怨了一句道:都是你多事——这回就不知什么时候才能出来了! 



七 

有关我就要推动自由这件事,我表哥告诫我说:你别太拿它当回事。我觉得他说得太轻巧。我表哥这么想得开,他怎么不进公寓里当个房客?听了这话,他说:我不是想住都住不进去吗?这又是一句气人的话。我听了以后不想理他,但他还要理我,说道:表弟,处在你这种地位,想把自己气死是很容易的。他说的也有道理。我想了想,强把心头的火气散去——虽然我也知道,这最后一句话也是在气我,但我只好听他的劝。与此同时,被关在鸟笼子里的女孩终于等到了那激动人心的一瞬:笼门上的定时锁咔的一声,门自己敞开了。她挪动着坐麻了的肢体,从笼子里艰难地钻了出来。能够离开这座小笼子还不是激动人心的原因——离开了小笼子还要走进大笼子——激动人心的是她总算是等到了什么。此时大概是午夜。在灰蒙蒙的水银灯光下,她朝前走去,一直来到了办公室门前。这扇是开着的,她用肩膀推开门走了进去。管理员仰坐在扶手椅上,脚跷在桌面上。这张桌子是黑色的终端台,和她自己房间里那张一模一样。这间房子里还有一些黑色的钢木家具,和她自己房间里的也是一模一样,但这里明亮一些。管理员把腿从桌上拿下来,说道:到时间了?那女孩点点头,走上前来,转过身去,让他解开捆在手腕上的麂皮绳子。如你所知,绳扣过了夜,变得异常的结实,根本解不开。管理员把女孩接近了一些,但绳扣还是解不开。他伸开了大腿,让女孩坐在他的腿上,女孩就坐下了,坐得笔直,就如一位淑女坐在抽水马桶上,身上散发着荷花的苦涩味儿。这种气味使管理员感到一定程度的兴奋,他用一只手解绳扣,另一只手绕过了她的腰,从衬衣下面伸了上去,伸向她开关精致的乳房——她的皮肤逐渐变得粗糙了,很快出现了粟米状的颗粒,不言而喻,那是一些鸡皮疙瘩。管理员把手抽了出来,问道:你讨厌我?那女孩轻声答道:不讨厌,但我害怕你。管理员说:这就好。害怕我是应该的,讨厌我就不好了。他还给她把衣服整理好。不管怎么说吧,绳扣总是解不开的。最后管理员拿起一把大剪刀,嚓的一声把绳子剪断了。女孩马上站了起来,揉着自己的手腕。管理员说道:回去吧——你的房门是开着的。进去以后把它撞上。女孩向房门走去——猛然转过身来说道:你可以去再买根绳子——记在我的账上——还有,我对新来的房客宣传过你的公寓了。 

管理员确实对房客们说过,你们都是老房客了,有新客户来时,多宣传宣传咱们这里的好处。401的女孩照他的嘱咐办了——我们说过,她告诉秃头说,这里有热水。但他不喜欢她说话的方式。“我宣传过你的公寓了”,这样太直露。他喜欢大家把房客和管理员的关系理解为一种合作关系,但是谁也不肯这样理解这种关系。他还希望房客不要说“你的公寓”,而要说“我们的公寓”。他在每个笼子里挂了一个牌子,上面写着:请勿乱抛碎纸,爱护你自己的家。但房客都把牌子扣过来挂着。我表哥虽然不高兴,拿他们也没辙。后来,他把牌子都摘掉了。 

我表哥告诉我说,他喜欢女房客,女孩管着省心。他的房客都是些女孩,管起来是省心,可惜她们收入有限:有的是教师,有的是艺术家,没人挣大钱。开公寓除了房钱,还可以按一定的比例从房客的收入里收取管理费,这一算我表哥就很亏了。后来有了这个秃头,我表哥就赚了。这家伙在网络上开了家软件公司,我表哥听了就说:在网络上开公司——很牛逼呀你。秃头很谦虚地说道:很一般——不牛逼,不牛逼。但是一查他的账,发现确实牛逼。表哥倒没收他什么管理费,只是请他做自己的合伙人,把他的全部钱、还有全部收入都拿来入了股。秃头也无话可说:反正住在公寓里,要钱也没什么用处。我表哥还说,你要钱时管我要。那秃头也没管他要过。连网络的月费都不管他要,这一点实属可疑。表哥对我说,看来秃头有私设的小金库。这也不算什么了不起的狡猾,要是我在表哥这里住,也要私设小金库。 

这个秃头最早住过的公寓设在一座放蔬菜的土库里。这座土库在北京西面的一条运河边上,那时有道高高的土岭,有人说是元大都时代遗下的土城。不管是不是吧,那土岭的土质异常的坚硬。土库挖在光秃秃的土台里,土台周围有几小片菜地,一片乱糟糟的小树林,再远处才是新建高层建筑。总而言之,那是都市里很难得的一片荒凉地方。夏天的傍晚,那位后来染绿了头发的管理员会走进土库去找那个秃头,手里拿着一根细长的铁链子,打开铁笼的门,把铁链套在他脖子上说:走,秃头,陪我去游泳。此时秃头可能在干各种各样的事情:在台灯下修手表(有一段时间他靠修手表来挣公寓的房钱),看编程序的书,或者是用最便宜的线路板拼凑一台PC机—— 不管在干什么吧,他马上要扔下手中的事情跟她走,否则就会被链子勒死。管理员身上穿着花花绿绿的尼龙游泳衣,手里拿着塑料垫子、浴巾、消闲的妇女杂志,很快她就把这些东西随地抛撒,而秃头不等东西落地都一一接住,捧在手里。这位管理员对房客性别的看法和表哥完全相反,她说:我喜欢男房客,男房客管起来放心。 

河边有片砂地,砂地中央有棵白杨树,到了这个地方,管理员取出一把将军不下马的锁来,把秃头像一只奶山羊那样锁在树上,把钥匙挂在脖子上,一头扎进河水里去。秃头呆在岸上百无聊赖,主,蹲在地下扒沙土。每逢有人偶尔骑着自行车经过,他就低下头去,用湿沙子堆筑城堡、坦克,还有一切童年堆筑过的东西。有时候那位骑车人还会从车上下来,走下斜坡,一直走到秃头面前蹲下问道:哥们儿,你丫玩的这是什么性游戏?秃头把脸转过去不回答。这位骑车人又站起身来,对河里的管理员大声说道:姐们儿!你们玩得够野的啊!管理员只顾游水,也不理他。那个人见没有人答理,只好艰难地往堤岸上面爬,嘴里还说:我行我素,目中无人,我真服了你们了。然后他就骑上自行车走了。有时候这位过路人实在磨磨蹭蹭,管理员就在水里大喝一声道:别讨厌啊!他是我们的房客!过路人听了,瞪上秃头一眼,说道:我还以为是干什么的,原来是住在公寓的!他朝秃头脸上啐了一口,然后就走掉了。 

在岸上百无聊赖时,秃头经常在把玩项上的锁链。那条链子是公寓里的人自己做的,用铁丝弯成环,再用电焊机把缺口焊住,就做成了一条铁链,做工实在是很糟,链环七大八小,焊点七扭作歪,还尽是虚焊。样子更是别提有多难看了。把这样的链子套在脖子上实在丢人,后来秃头买了一瓶黑油漆,把它油了一遍,这回好看多了。只可惜油漆是劣质货色,经常掉色,常把他脖子染得漆黑。等到秃头当了网络工程师,挣了一些钱,就买了一条尼龙链子。这东西乌黑乌黑,看上去像是铁的,但又轻又暖,而且异常坚固,永远也挣不断,但这是以后的事情。当时发生的事情是,管理员在水里游够了,爬上岸来,把系在树上的链子解开拿在手里说:你也游游。秃头打量着自己——他穿着一件无领上衣,一条肥大的裤子,是用看不出脏的黑色合成纤维面料做成的(那种布看起来油脂麻花的,表面凸起了很多线头,结实得很,但穿在身上非常不舒服),说道:我没有游泳裤。管理员往四下看了看——我说过了吧,这里比较偏僻——说:有什么关系呢?你是男的啊。他想了想,说道:是啊,我是男的啊。就把上衣脱了下来,在身上乱抓了几把,然后又解开了拦腰系着的布带子,就跳下水去。管理员坐在岸上,手里抓着那根链子,那链子有五六米长——她看上去像个放风筝的人。秃头的水性很好,一切人类游泳的姿势都能运用自如,所以他就采用了被拴住脖子时最适合的一种姿势:狗刨式,打出很多水花,把头高高地扬在水面上。 

等到他游够了爬上岸来,管理员已经给自己铺好了垫子,戴上了太阳镜,躺在垫子上打起瞌睡来。秃头想去把衣服穿上,但管理员已经把铁链绕到自己脚上,链子因而变短,够不着衣服了。他只好在管理员身边蹲下,看上去像一只很乖的狮子狗。管理员一觉醒来,看到的情形就是这样:秃头蹲在地下,双膝紧靠在肩膀上,双手抱着膝盖,阴囊下垂,阴毛披挂在阴茎周围,像个芋头,天几乎已经黑透了。此时她大叫一声道:好啊,打道回府! 

秃头过去呆过的那所公寓是办事处办的。众所周知,办事处是城市里最低一级的行政单位,什么好事都落在后面。这家公寓就办在了菜窖里,也拉不来好的房客。所以他们把自己管辖范围内一切有点文化的人都抓了起来,关在菜窖里。就说这个秃头吧,他只念过两年多师范就退了学,在街口修手表,也被抓了起来。这些乱七八糟的人被关进了菜窖,反倒奋发上进,开了不少高科技公司,公寓的收入大增,从菜窖搬进了废车库——这位秃头说得很坦白:既然修手表都免不了被抓,倒不如发点财,让自己也过得好一点。等到有了钱,秃头就给自己买了一条尼龙锁链,买了皮革的护腕和护踝,还买了一块假豹皮苫在腰间。出门时,他戴上黑皮面具,让管理员用不锈钢手铐把自己反铐住,用锁链牵住脖子,就可以理直气壮地上街了。不管被谁看到,都可以理直气壮地说自己是个性变态,不用说是见不得人的公寓房客了。管理员经常牵着他逛街,给自己买东买西;秃头也有机会到处去遛遛。这里面的道理很简单:有钱就可以买到自由。管理员牵着他走到街口的公共厕所,递给看门的三毛钱和链子的一头,说道:大娘,替我牵着点。看厕所的看看秃头,说道:带进去吧,没人见怪的。然后管理员去上厕所,他在屋角蹲着。有个小女孩走过来说:大叔,可以往你脸上撒尿吗?他还可以理直气壮地回答道:这不是我的爱好——我们在此说到的,就是自由。管理员上完厕所回来,问他道:你撒尿吗?秃头想了想,答道:撒。于是管理员把他带到抽水马桶边上,撩开那张豹皮,取出他的把把,对准了马桶说:尿吧,秃头红着脸说,你拿着我不好意思,尿不出来。管理员就说,没关系,没关系,尿吧。为房客服务,是我们的责任嘛。说得这么好听,你要是没有钱,她肯定记不得自己有这种责任。然后,秃头就在管理员手里尿了起来,他感觉自己像个小孩子,不像个男人。因为这个管理员,秃头对那个公寓很满意。但是后来他被人劫到了另一家公寓里,此后就没有这种待遇。后来我或者表哥带他上街,只管撩起豹皮,就让他尿,谁也不给他拿着,有时尿到了腿上,有时尿到豹皮上,弄得他骚烘烘的。他对这种前景很有一点感慨。假如他的邻居肯听的话,他想要说一说,但她总是不像要听的样子。如果他执意要说,她就让他说上两句,然后用一句评论来打断他:你觉得自己太重要了。听了这样的评论,秃头先是愣上一下,然后同意道:是啊,我觉得自己太重要了。然后就不说话了。 



我说过的吧,我表哥新买来的这个秃头原来是个牛逼人物,除此之外,他还是个君子,所到之处与人方便,很少给人添麻烦。他在网络上开了一家软件公司,用户经常打电话、发电传,问他软件的问题,他也不厌其烦地解释着。无奈有些用户实在太笨,怎么解释也不管用,这时他就要亲自去一趟。如果就在本市,那还好办,要是外地,就得发个特快专递,把他自己寄过去。我送他上邮局办有关手续,开着我表哥的吉普车。这辆车的特异之处是在挡风玻璃后中央有个大铁环,可以把房客的一只手铐在上面,我和秃头出去时就是这样的;还有一个特异之处在于房客的座位比驾驶座矮很多,秃头坐在我身边,比我矮了半个头,他东张西望,嘴里哼着一支不知所云的歌。 

有关我表哥的这辆吉普车,还有些需要补充的地方:它是蓝色的,既没有顶篷,又没有门,但车上总带着一块大苫布,到了地方就把它苫上。我表哥出门时总带着一个房客,他说是帮他算账——我表哥是个文盲,但只在理论上是这样。实际上他能算账,三位以下的加减乘除算得比我还快。他还有阅读的嗜好,喜欢看话本小说,床底下纸箱里有老大一堆。虽然如此,他还是老问别人:这是多少啊?或者是:这上面说些什么?用他自己的话来说就是:总得装装样子吧。当然,我表哥带房客出门,不光是要她算账——我和他出门时,也坐在那个座位上,我表哥常常下意识地把手放在我大腿上。 



我和秃头上邮局,帮他办有关手续。手续相当烦琐,除了填单子,还要打手印,照相片,留血样,万一他在邮递的过程中逃跑了,要靠这些资料把他追回来。这些手续办好后,邮局用三十天不褪色的荧光染料在他额头、手背、前胸等部位盖了章,上面写着:邮递物品,交回有奖,藏匿有罪。万一他跑掉了,别人看到这些印迹,就会把他逮送回来。他长叹一声对我说道:出门受罪啊,小老弟。在这座公寓里,只有秃头真正把我当小老弟,这让人感到亲切,又让人感到绝望。我说:你也可以不出门,没人逼着你去。他说:那怎么成?我不能让用户失望。办好了这些手续,就要把他装箱——当然是装寄人的专用集装箱。我和他在邮局后面的库房里,看着传送带上运来的三个箱子。箱子有大号写字台那么大,是深蓝色的,绘有EMS标志,顶面漆成黄色,侧面有箭头,有大字,写着此面向上。有两个巴掌大小的窗户。打开椭圆的箱门一看,里面衬有塑料衬垫,有个大箱子占了四分之一的空间,人可以坐在上面,箱里有个化学马桶;顶上有盏有碎的节能灯。里面当然不舒适也不宽敞,但若只呆48小时,看来还能坚持得住。三个箱子都是这样的,但装箱的小姐还是说道:挑一个吧。这位小姐穿着绿色的制服。戴着绿色的大檐帽,可是穿了一双雪白的运动鞋,色调不协调。秃头挑也不挑,就朝头一个箱子里钻进去了——但他被小姐制止住。这位小姐抬起腿来,用脚尖勾住了秃头的胳臂:邮局的小姐的脚像功夫师的那样灵巧,看上去真是怪怪的。她厉声喝道:穿着衣服就钻进去吗?这话不但让秃头意外,连我都感到意外:我手里提着一条黑色的塑料垃圾袋,秃头的全部衣服鞋袜都在里面,除了他身上那条破破烂烂的内裤。他直起身来,说道:连裤衩也脱?以前不是这样啊。那小姐只说了一句:衣服和人分着邮。别的就懒得再说了。他只好把裤衩也脱了下来——他那个东西真是大极了,垂在两腿之间老大的一嘟噜。小姐看了不好意思起来,飞腿去踢他的屁股,说道:还不快钻进去——他妈的,怎么能这么大。秃头的屁股上留下了一个黑色的鞋印,这使我感到不快。也不知道为什么,我竟会有这样的想法:这个人是我送来的,要踢也得踢我啊。所以我就瞪着那个小姐,把她瞪跑了。好在邮局里人多,瞪跑了这个还有别的。 

躲在箱子里,秃头领到了邮寄途中的给养:一袋饼干,一瓶矿泉水。他还要求邮局的职员给他一个坚固的塑料袋子。邮局的人给了他袋子,还说:一听就知道你是个专递油子。我想这是指他常被邮寄,颇有经验而言,所以就请教他为什么需要这个袋子。他说:首先,这个化学马桶里盛的不是专用的药剂,而是颜色相近的蓝墨水 ——这原因很简单,药剂贵,墨水便宜;用墨水来代替药剂,有关人员就能赚钱。其结果就是屎屙到马桶里还是屎。其次,集装箱外面写着顶面朝上,但在运输的过程中哪面都可能朝上。马桶里的东西全会洒出来,他可不想吃到自己的屎。至于袋子派什么用场,他还没有讲到,邮局就要发货了。秃头钻进那个箱子,别人把门关上,上了锁,打上铅封,他就被寄走了。过了几天,用户把他寄了回来,集装箱送到我们公寓里时,果然是侧倒着的。我们把箱门打开,他从里面钻了出来:此时他已变成了个蓝色的人,手里紧握着一袋自己的屎。虽然出门是如此不便,但他还是经常出门,一会儿把自己寄到海南岛,一会儿把自己寄到吐鲁番,去给用户排忧解难。他的脸上身上都盖满了戳记,就像一封到处旅行的公文。秃头就是这样的。我受他精神的感召,虽总要送他去邮局,也不觉得麻烦。 



八 

我一直等待住在404室的房客有事叫我,最后总算等到了机会。我到她门外时,她已经着装完毕,等着我带她去散步。隔着铁栅栏我对她说:我是你的学生,猜猜看我是谁?这位老师是近视眼,留一头短发,穿着无袖的长裙和绒线衫,把嘴唇涂成了褐色。她一直教我们班,从一年级的数学分析教到了现在。我认识她,在闭路电视上天天见到。她不认识我,也不知道我叫什么名字。媸眯着眼睛看了我很久,终于叫了起来:你的拓扑考了七十五分——你这个小傻冒。我的脸忽然阴沉了下来。她说得很对,我的拓扑是考了七十五,这说明我是个小傻冒。但我还是很不高兴,冷冷地说道:请你转过身去,背着手。然后我开门进去,握住她背着的手往上提,压低她的脖子,使她跪倒在地板上,然的从腰上取下手铐,冷冷地说道:对不起了,老师。我把她反铐了起来。 

我的老师已经四十六岁了,嘴角处有很深的皱纹,但远看是看不出来的。因为她生得娇小玲珑,看起来比较年轻。我带她上公园,心里想着自己在学校里的事。数学系的功课很难,而且一年比一年难,有很多人都被刷掉了。上学期我的拓扑考了七十五,还不是补考时得到的。这不仅是这门课的全班最高分,也是自我们入校以来的全班最高分。为了这门课我经常熬夜,但被老师称作傻冒。我想着这件事,隐隐听到老师在叫我。我不想答理她,就装作没有听到。后来她用肩膀撞了我一下说:喂!叫你傻冒你不高兴了?这是不言而喻的,所以我没有回答。她又说:不要生气。你还傻得过我吗?这话说得有道理。这位老师是数学博士,我们刚入学时,她是副教授,现在是正教授——这些都是她比我傻的证明。我的火气正在散去,同时也注意到,虽然年龄大了一些,老师依然是有魅力的女人。 

我和我的数学老师坐在公园的长椅上。老师披一件半长的呢子斗篷,戴一顶黑色女帽——这身装束很时髦。傍晚时分,天上飘落着零星雪花,公园里游人稀少。我把她抱了起来,放在自己身上,让斗篷搭在自己肩上,在里面抱住她的身体。老师很柔顺地躺在我身上:除了是个有魅力的女人,她还是个讨人喜欢的房客,像住402室的秃头一样。她穿着一件紧身的绒线衫,束在腰带里,双手被铐在身后。那副手铐是防弹尼龙做的,上面有一行小字:“Made in U.S.A”。我用手指捏住绒线衫,问道:“老师,可以吗?”开头她说:随你的便。这话使我感到冷淡,所以我就僵着不动。她后来又说:没什么不可以的。这话又让人感到振奋。我把她的腰带松开,把绒线衫从腰带里拽了出来,把手伸向老师赤裸的身体。虽然皮肤略显松弛,老师的身体依然美好。在我的爱抚下,起初她保持着矜持的态度,后来就哭了起来,说道:别这样对待我。我说:我爱你呀。她说道:你以为我会相信吗?我把手缩回去,同时说道:不信就算了。老师又说:别,就这样吧。我很仔细地抚摸了各个地方,然后替她束好衣服,就如一个小孩打开属于自己的糖盒子,取出一颗糖,然后把盒子仔细盖好。她使我兴奋不已,因为她不是一般的房客,她是我的老师啊。 

有关我的老师,还要补充说,在小学里我有好几位老师,在中学里我有更多的老师,但在大学里只有一位老师,每一门功课,从一年级的分析到三年级的拓扑都是她教,而且一门比一门更难。至于考试题目,简直是匪夷所思的古怪刁钻。考完之后,你会在电子信箱里收到必须补考的分数,加上一首骂人的打油诗:“你是一个无脑汉,两耳之间屎一团……”假如你有这样的老师,自然也会对她有极深的感情。后来在公园里,我把她抱在怀里时,她也承认自己是存心整我们,理由是“眼看一群小傻瓜,死命念着傻功课,就觉得气不打一处来!”既然小傻瓜里有我一份,我听了当然不高兴。然后她就安慰我说:别不高兴——你们谁也没傻过我。现在落到了你手里,想怎么弄我就弄吧。听了这样的话,我马上替她束好衣服,理好头发,整理好项上束的丝巾(在公寓里干了这些天,我做这些事已经很内行了),把她扶在我身边坐好道:老师,我怎么会弄你?我是尊敬你的。她静坐上一会儿,又把头靠在我肩上,脸上却已经潮湿了。在黑铁公寓里,尊敬就是最大的虚伪,虚伪就是最大的轻蔑。我怎么能这样对待我的老师呢?我把她抱在怀里,吻她冷冷的嘴唇,松弛的下巴。与此同时,我一点都不爱她——这也是虚伪,但比尊敬要好多了。 



九 

我表哥很早就开始歇顶,还不到三十岁,头顶就光秃秃的了。假如所有的头发都掉光还好一点,偏偏在额头上方还剩了一小撮黑毛,看上去像过去小孩子留的盖头,或者是早年间彝族人留的那种天菩萨;还可以说,他有一撮卓别林式的小胡子,可惜长得不是地方。要是一般人头秃成了这样,肯定要把这撮毛剃光,免得别人看到他时发笑。但我表哥没有这样做,他身上有股狠劲儿,叫别人笑不出。他自己也爱和别人说个笑话,别人听了也只好苦笑一下——住在黑铁公寓里,谁敢不买他的账。只有401的房客敢不买他的账,听了他的笑话,把小嘴一瘪,小声说道:无聊。我表哥听不到,就算听到了也不以为忤。虽然表面上对她严厉,但他喜欢她。这也不是什么难想象的事,假如你是公寓的管理员,又会喜欢谁呢。 

晚上我到公寓里,在办公室里看到我表哥,他正在愁眉苦脸,好像刚拔掉了牙一样。他瞪着死鱼眼睛看了我好半天,忽然解下钥匙串扔给我说:你去告诉401,让她在一号等我。一般来说,一号是指厕所,但黑铁公寓里没有一间房子是专门的厕所。看我表哥的样子,他好像无心给我详细解释。我拿了钥匙到了401室门外,对里面说道:我表哥叫你到一号等他。那女孩对此看来已经有些精神准备,因为她没在终端台前,而是坐在床上等待着。听了这话,又问了一句道:去一号,是吗?我点了点头。她往四下看了看,说道:你转过身去。然后,在我身后就响起了窸窸窣窣的衣服声。这时我问道:哪里是一号?那女孩懒洋洋地答道:你不知道,是吗?——我可不是不知道吗。 

假如你认识我,一定会说我有点呆头呆脑。这也不足为怪,假如你像我这样总在盘算着,一定也会呆头呆脑:我一面在黑铁公寓里出出进进,观察着这种生活,一面又在盘算逃开它的办法。说老实话,要逃还是有办法逃的,天涯海角,地方很大。但我逃到哪里都没有身份,怎么谋生可是个大问题。打个比方说,我可以跑到山西去,找个私人开的小煤窑,下井去背煤——窑主看到我有胳臂有腿有脊梁,肯定会满意,多半不会向我要身份证件,但是干这种事还不如住进公寓。我正在想这些事,忽然听到有人在敲身后的铁门。回头一看,401的女孩站在铁门前:她上身着一个无肩带的黑色胸罩,下身着一条黑色三角裤,脚下穿着一双塑料拖鞋——她的皮肤非常之白。她简单地化了一下妆:涂了嘴唇,还画了眉毛,手里拿了一条浴巾。我把铁门打开,她走了出来,在我肩上拍了一下说:走啊,上一号。这时我以为一号必然是桑拿浴室。此时她脸上红扑扑的,很是兴奋,但假装轻松,吹着口哨——但不大会吹,噗噗的。她带我走到一个小门前面,让我拿钥匙打开门,里面是间灰蒙蒙的房子——从地面到天花板都是裸露的水泥。我不知道还有这间房子。地中间有张木板床,是用很厚实的木板钉成的。但是这间房子不是桑拿浴室——这里面太过凉快了。她走到床前,愣了一会儿,把浴巾铺在床上,然后就趴了上去,把手脚都伸直,对我说道:来,把我的手脚都拴住。这时我发现这床上钉有一些皮带。我把她的手脚都拴住以后,她又说,把背带解开。我把她胸罩的背带解开了,然后就不知做什么好——我发现这女孩的腰很细,身材也很苗条,但这不算什么新发现。忽然之间,这间房子里呼起了我表哥的声音,但我表哥又不在房子里。这件事又让我愣了一愣,然后才想到,这间房子里必然有暗藏的对讲设备。 

实际上,这间房子里不但有对讲设备,还有暗藏的摄像机:我们的一举一动表哥都能看到。我表哥叫着我的小名:小×,给阿姨用酒精擦擦背。女孩听了哧地笑了一声,说道:原来是小×啊。而我在东张西望地找酒精。女孩说,在床底下。笨蛋,往哪儿找。床底下果然有个广口瓶,盛了半瓶酒精,还有一大包脱脂棉。我拿酒精棉球在她背上涂时,她在看自己的手,先看手心,后看手背。擦着擦着,我表哥就进来了,双手窝着一根黑色的藤条。他的脸涨得通红,不尴不尬地咳嗽着。女孩也抬头看我表哥,急促地说道:别打屁股,打了就不能坐——我还有事没做完呢。与此同时,她羞得满脸通红。看来我表哥要打这个女孩,在这种地方也不是什么不能想像的事情。但他们俩都很不好意思,既然如此,还不如不打呢。表哥走到了床前,说道:这件事不能怪我——是你自己招的祸。女孩打断他说:要打快打吧,别说教了。此时我躲到门外去,用牙咬着指节,开始盘算在这件事里我扮演的是什么角色。 

我表哥从那扇门里出来时,已经恢复了正常的样子。他看了我一眼就走开了。我走进那间房子,看到她在板床上,把身体伸直,面侧向门口,脸上红扑扑的,一副若有所思的神情。在她背上有八道血痕,排列整齐,间隔划一,但我没敢仔细看。我走向前去,解开她手脚上的皮带,同时总道:打得厉害吗?她很冷静地答道:一般。但她的牙齿在格格地响着,浑身直打哆嗦,然后她反手扣上了胸罩上的带子,慢慢地坐了起来,双脚在地面上搜索着拖鞋。此时我发现她虽表面上镇定如常,其实疼得很厉害,因为她的脚哆里哆嗦,而且在绊蒜。我建议道:我背你回去,如何?她先是皱了一下眉头,然后说道:也好。就这样我把她背回了401室。她的身体很滑腻,还有很多汗。等到她在自己床上趴好,把枕头拉到颏下时,我还在她床边站着。她说道:你走吧。等会儿我能动了,就去冲个冷个澡。我说:不行吧,会化脓的。她说不会,这里很干净,没有细菌。我还想问问这种事情是不是经常发生,但她说道:你让我安静,好吗?这件事情的始末就是这样。后来我做了一夜的梦,梦见自己背了很多女人回自己的房间,像一个龟奴。 

表哥告诉我说,他有权力责打房客。他给我一本小册子,叫我自己去看。这本书的名字叫做“公寓员管理手册”。书上确实提到了管理员可以用藤条打房客,因为这是为了房客好,但这一点在鞭打之前必须对房客说清楚。他可以把他(或她)打疼,但不能把他打坏。而且假如房客生了病,发烧在三十八度以上,白血球在一万以上,就可以免受鞭责。但在任何情况下都不能给他吃止疼药。我看了这些规定很不满意:其中并无一条规定说道,假如房客是管理员的表弟却当如何。我表哥力气大,打起人来一定很疼,我不想让他来打我。手册上还写着,一定要营造一种平静祥和的气氛,让打的人愉快,挨的人开心——但这是不可能的事情。当然,越是不可能的事情,就越要往纸上写——这件事情我们都是知道的…… 

我很想知道401女孩的脊梁后来怎么样了,所以常去看她。当天下午她就起了床,坐在终端台前工作。那些鞭痕起初是鲜红的,后来是紫色的,然后颜色越来越淡。再后来她穿起了衬衫,那些鞭痕就看不见了。我到表哥那里要来了钥匙,走进那个房间,走到那女孩身边,拿手遮住屏幕,她看到屏幕上有手,抬起头来看着我。此时我说道:阿姨,我想看看你的背。她说:讨厌。因为头上戴着耳机,说话声音很大,简直就像斥责。但她没有搞清赠我的意思。她把一只手从键盘上拿了下来,解开腰间的皮带,把衬衫的后摆从裤子里拉了出来,说道:自己看。就去做自己的事了。我撩起她的衣服,看到那些鞭痕已经变成了浅灰色的,用手去触也只能感到很轻微的下凹。看这个趋势,这些鞭痕很快会不留痕迹地消失掉。但不管怎么说吧,挨打总不是个好滋味,而且我也不能相信让我挨揍是为了我好。 

401室的女孩说:我表哥打她,完全是公事公办。首先是有关部门给我表哥打了个电话,说道:你还管得住管不住自己的房客?要是管不住就早点关门——然后就把电话挂上了。我表哥没有办法,只好叫小力巴(该力巴就是我)把她带到一号去拴上。然后他到那里去,等小力巴走后,先问明了情况,然后说:没办法,只好打你了。他先用藤条在自己手心上试了一下,确认它既不太锋利,也不太钝,然后开始抽打她的脊梁。他还是不大好意思,关照她说:要是打疼了,你不妨叫唤出来,这样会好一点。女孩说道:谢谢。你也不妨抽一下,问一声“你改不改”,这样也会好一点。对于坐着工作的人来说,打人家的屁股实属缺德。我表哥从来不往屁股上抽。当然,被抽的地方很疼,但不疼又不行。我表哥不肯在责打时逼问“改不改”,他说这不诚实:你就是说改,我也要接着抽。女孩说,我表哥很诚实,所以她爱他。这件事情的起因是这样的:人在黑铁笼子里呆久了,难免郁闷,最后就会撒起癔症,到处乱发E-mail。发到别的公寓里是没有问题的。就怕发到国外和有关部门,内容再带有歪曲性、挑逗性和污辱性。这类行为必须制止,所以要抽一顿或者打一顿。此后起码有两个月不想再干这种事情——巴甫洛夫学说对此有很好的解释。疼痛和外伤又可以增加机体的免疫力。总而言之,我不该把此事想得太坏。当然,这也不是好事——既不好,也不坏,不过是公事公办吧了。我听了还是不开心,就说:那你们就别撒癔症了,她说:胡扯,不撒癔症怎么能成!看我瞪着眼睛,她又进一步解释说:不是我们要撒癔症,而是我们已经有了癔症——但她看样子还是蛮正常的。看到我还是瞪着眼睛,她说:别这么傻冒成不成?我顺嘴说道:不是我要装傻冒,而是我本身就是傻冒——我是真心的,但听起来像一句玩笑。听了这话,她笑起来了。 

402的秃头也说,挨两下打没有什么。在他原来的公寓里,绿头发的管理员也打过他。比方说有这么一次,夏天的中午,她走进土库,对他说道:秃头,我不得不打你了。这种事情来得很突然,不由他心里不慌,急急忙忙地把桌上的东西收拾了一下,然后问道:脱裤子吗?女孩说道:脱。他就把裤子脱掉,围上一条浴巾,精赤条条地走到院子里。大槐树下放了一个板凳。秃头趴到板凳上,把胯部横担在凳布,屁股撅得高高的,把浴巾解开,好像对方是个肛门科大夫。女孩说道:用手把阴囊兜住,别打坏了;就拿起一块木头搓衣板,双手抡动,劈劈啪啪地打了起来。这个秃头身体健壮,也经打;但不是一条好汉:他怕疼。挨了几下就哼哟哼哟的,又挨了几下,就说:差不多了吧。那女孩住了手,看看他的屁股说:不行,还得打几下。过一会秃头又说:歇歇吧。女孩说:我不累。但她不问秃头疼不疼。直到把他的屁股完全打肿,红通通亮晶晶像熟透的苹果,她才把板子丢下,擦擦脸上的汗说:打完了。唉呀,手上都打了泡了。还把手伸给秃头看。当然说的是她自己的手,秃头手上不会打泡。后者哼哟哼哟地说:可以抹点红花油。她就去抹红花油,当然,是抹在自己的手上,没抹在秃头的屁股上:这个面积很大,没有那么多红花油。实际上,这座土库只有一半是公寓,另一半放着苹果。那女孩拿了一个熟透的红苹果作为样板,放在板凳边上,先把秃头的屁股打得像苹果一样,然后就把苹果吃掉了。此时秃头已经不能动弹,只好叫人把他架回去,趴在板床上。假如库里没有苹果,就得拿茄子当样板,工程也因此变得浩大,从是止打起要一直打到天黑,把屁股打得像马路一样平坦。用手指弹弹,丁当有声。401的女孩打断他说:行了行了,你别编了……但秃头说,他一点都没有编,说得完全是真的。他也说,总不挨打就要撒癔症了。我想了一下说:我知道你们撒的是什么癔症了——你们都是受虐狂!401的女孩听了说:胡扯。就转身去工作,不再理我了。402的秃头却说:我们要真是受虐狂倒好了!在这个世界上,羡慕什么人的都有,就是没有羡慕受虐狂的。他的话把我彻底搞糊涂了。 



十 

四年级的寒假我们不准离校,要受毕业教育。在这项教育里要告诉我们毕业以后会是怎样的前景,口说无凭眼见为实,所以必须请学长出场作报告。第一场报告请了我们学校最有成就的一位校友,她是计算机系毕业的,才三十五岁就得了图林奖——这是信息科学的最高奖项。我在会议室里看到了她,瘦瘦的,穿一件紫缎子的旗袍,脖子上束一条白色纱巾。人长得一般,胳臂也很细;但是手臂上肌肉的线条清晰,简直像个轻量级的拳击手。她把双手放在桌子上,手腕上套着一副铐人猿泰山都不过分、亮晶晶、黄灿灿的大手铐。据介绍,这手铐里还裹了贫化铀的芯子,这可是做穿甲弹的材料。万一钥匙丢了,用电焊气焊都打不开,用等离子束才能割开;或者到医院里去,先截肢,把手铐取下来,然后断手再植。铀的比重很大,所以那副手铐有二十公斤重。难怪她手臂肌肉发达——是练出来的。报告是照稿念的,内容都是套话。最激动人心的内容是大家排着队去看那副手铐。那上面镀的是24K金,上面镌了四个大字:“国之瑰宝”。这评价也不为过分,只是没有说清楚什么是瑰宝:是手铐呢,还是戴手铐的人。我提出这个问题,马上得到了好几个不同的答案。坐在瑰宝旁边的一个男人说:手铐是瑰宝。我身后一位同学说:人是瑰宝。一位在场的领导说:都是瑰宝。而那位手臂强壮的学长本人却说:你是瑰宝——小兔崽子,别在这里装骚鞑子了。她的意思是说:我提这种问题是存心捣蛋。但我不是的。我没有捣蛋的胆量。除此之外她的话还有一重意思:什么都不是瑰宝…… 

大字底下还有一行小字:三部一局监造。我问她这是什么意思。她说三部是公安部、人事部、劳动部,一局是技术监督局。然后顺嘴嘟囔道:“监造归监造,钱可是我自己出。旁边有人把话头接了过去,说不管谁出钱,总是国家监造。这是政治待遇,表明了国家对她的重视——别人想买还不卖给他哪。这位瑰宝把嘴闭了起来,脸上挂上了冷峭的微笑。那副手铐之中,她有一双很美丽的手。 

大学四年级时,你还会收到个人用品公司的邮购广告,推销一些稀奇古怪的东西,产品目录上注明了是“外出用具”。从名字来看,它该是牙刷、旅行包,男人用的剃须刀,女人用的唇膏。但从图片上看,和这类用品有很大距离。那些东西怎么看怎么像些脚镣、手铐,而且价格不菲。不管卖多少钱,总不是好东西。假如这些东西要给我们戴着,还要我们来出钱,简直是岂有此理。但我表哥的房客每人手里都有一大堆,而且还在不断地买。我问她们为什么要买,回答是:“闲着没事,总要买点东西”,“出门总要戴,这是个门面”或者:“这是首饰”。我表哥从来不买这种东西,他自己用不着,给别人买吗,他说是,这太肉麻了——我看他是舍不得钱。但他说得也有道理。秃头来时戴着一副不锈钢手铐,后来撬坏了,但他还保存着,说是绿头发女孩给他买的,留着作纪念:看上去是有点肉麻。报告会结束时,有人用丝带把那副大手铐拴好,挂在我们那位校友的脖子上,使她看起来像个前线下来的伤兵。这是合乎道理的,这东西太重,会砸坏东西,更会把自己砸坏。两个保镖夹住她,把她架了出去,上了一辆装甲运钞车——她住在香山公寓,那是国家级的公寓,出来一趟要国务院批。 



听完了报告,我回到公寓里,替我表哥值班。我不喜欢坐办公室,喜欢搬把椅子坐在走廊里,和房客们聊天。说起我们这位校友,房客们都知道。知道她戴着一副贫化铀手铐,知道她住在香山公寓,还知道她是个傻逼——要是谁能把诺贝尔奖得来,他才是个大傻逼。这些话也有点道理。意外的是,她们被关在笼子里哪儿都不能去,消息反而比我灵通了百倍,连我刚刚在会场上问什么叫三部一局都知道了。我问她们怎么知道的,403室的房客朝前努了努嘴。在她面前的终端台上,放着一台黑色的Roax机,和光缆连着,光缆连着网络。我们学校里也有网络的终端,但和这里的大不相同,设备水平差了两代。我们那里要受种种限制,他们这里一点限制都没有。拿电影来打比方,我们的终端是PC机,她们是X级的。这道理很明白:我们在校园里,怕我们学坏。她们被关在这里,不怕她们学坏。假如她们做了坏事,自会有人用藤条抽她们的脊梁——连我们那位学长兼国之瑰宝也不例外。当然,她有政治待遇,所以用马来西亚的藤条,请新加坡的刽子手。此人乘一架公务机从新加坡飞来,抽完以后吃两个汉堡包,又飞回新加坡去。当她被抽得惨叫时,刽子手还会用鸟语来安慰她说:小姐,你是国宝啦,别这样叫啦。待遇归待遇,所有的费用都是她自己出:请人的钱,飞机钱、藤条钱,还包括刽子手吃的两个汉堡包。 

大学四年级时有种感觉,人们好像不再像过去那样怕我们学坏了。所谓学坏,无非就是调皮捣蛋,逃学、得零分,不想进黑铁公寓。我隐隐地感到现在学坏已经晚上。千辛万苦考进了大学,千辛万苦念到毕业,都是为了进黑色公寓。现在要下个决心不进来,总是心有未甘。我禁不住多想黑色公寓的好处,尤其是那台“Roax”机。从寄来的广告和材料上,我知道那是一种技术奇怪,使我魂梦系之。想买必须先定下自己要住的公寓,这种机器只准安装在公寓里,但定公寓我还有点犹豫:别的尚在其次,挨打这一条,不管打屁股打脊梁,打得像苹果还是打到像茄子,总归是有点吓人。
