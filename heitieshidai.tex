\chapter{黑铁时代}

一 

黑铁时代的象征是那支鹅毛笔。这支笔捏在手里弯弯曲曲像条死蛇,写起来更是弯弯曲曲。因为这支鹅毛笔,那张粗糙的桌子上就免不了要插一把红锈斑斑的刀——这把刀的用途是把笔端削尖一些。桌上还有一碗氧化铁墨水,表面浮着一层五彩油膜,散发着浓烈的腥气——虽然如此,你还是不得不用这支鹅毛笔,因为用毛笔没法写算式。每个亲手计算的人都会知道,算式有多么重要。薄暮时分,草房顶的破洞有时会在风里呼啸。有些雪花从窗纸的破洞里飞进来,不知不觉在桌面的一角积起了厚厚的一层。屋子里呵气成烟,手指也冻得通红。除此之外,墨水的表面也结了一层细小冰凌。在寒风呼啸之中,那支鹅毛笔越来越短,在指间捏不住了——这是今天最后一支鹅毛。伏案演算的人不得不站起身来,搓搓手指,用搭在肩的黑斗篷裹住冻麻了的肩膀。他去把门打开,眼前一片茫茫的白色中间,是一条黑色的小路。此时他既不愿出去,在这条泥泞的小路上走,也不愿呆在黑暗的家里。但是权衡了以后,他还是出了门,用一把无聊的锁把两扇门锁住——这件事既不是发生在过去,也不是发生在现在。它发生的地点谁也说不清楚。 

戴上耳机,独自走进这个白雪皑皑的世界,过去,比尔·盖茨设想过怎样营造一个虚拟的真实:戴上液晶眼镜和立体声耳机,钻进一件厚厚的紧身衣。眼镜里传来图像,耳机里传来声音,紧身衣上数以十万计的触点让你身临其境——当然,控制一切的是计算机。现在用不着这种笨重的东西,只要戴上这副耳机就够了。虽然对电子技术有些知识,我也不知道耳机里面有些什么。我知道它效果很好,还知道这种东西很便宜。在那条黑色的小路两旁,堆着翻卷的积雪。在小路尽头出现了街道,雪地上的一道污渍接上了一条乌黑油亮的石板路……石板就如一张沾了油的饼铛。在漫天的白气中,沿着空无一人的街道,有个女孩朝他迎面走来。她披着一件短短的黑斗篷,斗篷下露出了两条洁白的腿,迈动得飞快。她脚下穿了一双厚厚的紫色木屐,但紫色不是木头的本色——的怪她的脚跟也被染得通红。这个女人走过之后,在街面上留下了一股香气,走在路上的男人在这种气味里愣住了。他转过身去,看这女孩的背影,结果看到了她屐底的铁掌留在石板上的一溜火星。那条石板路像融化的柏油一样平静,上面映着雪天翻腾的灰色云朵。这个男人面临两种选择,一是沿着黑暗的小路继续前进,到一间灰暗的铺子里买鹅毛;或则沿着相反的方向,追随那双洁白的腿,还有被染红的脚跟。因为这件事发生在一个虚拟的世界里,所以这两种可能都发生了。 



我表哥说:你是懂科技的人,替我看看房客们都在干什么。他们在干些什么,他都看到了,看不到的只是网络上的情形。我当然可以替他去看,但是需要一笔钱来买机器和付上网费。有了这笔钱之后,我到网络上漫游,看到了这些。我当然可以告诉我表哥,他的一个房客(住在402室的秃头)在网络上勾画出这样一个世界——但我又不知道从何说起。如你所见,这既不是一个帮事,也不是一个游戏…… 

秃头再次进入自己的文件时,他嗅到空气里有一股淡淡的荷花气味,空中除了呼啸的风声,还能听到隐隐的音乐声。他知道有人已经进入了自己制造的这个虚拟世界。他在北风呼啸的街头站了一会儿,努力判断方向,然后尾随荷花的气味而去,很快就追上了走在前面的女孩,和她并肩走着。他探出头去看她的脸,这个女孩的脸很白,也比较丰满,不像他认识的任何一个人。但他也知道,在虚拟的世界里,每个人都会变形,声音也会变——他也不像他自己。他们走到街道的尽头,前面又是茫茫旷野。在风把雪吹薄的地方,露出了黑色的菜畦,菜畦旁的水沟虽已被滚来的雪堆平,但沟边疏疏落落还立着枯黄的芦苇;路边立着一座孤零零的中式木楼,共计三层,但已显得非常之高。他们在楼前站住,仰头看看此楼黑色的面容——窄小的楼廊,在木柱和窗棂上,漆皮开裂,露出底下的麻絮;还有那些开裂的窗户纸。有一条铁链子穿过门上的窗洞,把两扇门锁在一起。女孩走上石阶,掏出钥匙去开门锁。这把锁是黄铜制成的,古色古香。女孩拿出的钥匙也是古色古香,和挖耳勺很相似。秃头不轻易称赞别人,但他不禁说道:这把钥匙很好。营造虚拟的世界很容易,但把一切细节都考虑到就很不容易。他本人也是个中好手,所以很欣赏这种细腻周到的设想。门呀的一声打开之后,他们走进了一间空空落落的大厅。除了四根粗大的柱子,就是漫地的方砖。迎面还有一座一人高的镜子,在这个世界里应该说是舶来品。镜面上镀层剥落,形成很多像蕨类植物似的条纹。他走向前去,寻找一块完整的镜面,以便看清自己,最后他找到了。他头发茂密,长了满脸的黑胡子和一张瘦长脸。除此之外,他还发现自己的身材是很高的,整个来说,和铜版画上的堂吉诃德很相似。秃头准备自己变成各种模样,但现在这个样子还是出乎他的想象。他不禁后退了一步,惊叹了一声。如你所知,虚拟的世界经不起感情的任何波动。于是他又退回了自己起初进入的地方——他重新坐在了终端椅上,面对着屏幕上那个像木门似的图标,图标的下角有行小字标明了“hei”。此时再去浏览这个文件,就会发现其中插有新的段落。现在已经不是一个人的故事,而是一个游戏了。他把手放在自己胸口,感到心跳得很厉害。 



401室女孩的网址上有一个文件,名字也叫做“hei”,用红黑两色的图标作标志。这是一个黑色的铁栅栏门,门上悬挂着红色的帷幕。想要跨过这个门槛有很多困难,因为这个入口是给自己留着的。当然还有别的入口,但从那些入口进去你就不可能是主人,只能是客人。有一个闯入者越过了这个门槛,对此无须做更多的解释,在网络世界里,没有去不了的地方,只有道行不够高深的人。然后他就坐在黑铁公寓401室的终端椅上,手贴着面颊——手下的感觉异常滑腻。发现401室的女孩把虚拟世界设在真实之中,闯入者会感到诧异。他走向栅栏,看看醋睡中的秃头:这张脸苍白虚胖,脸上爬满了苍蝇,看起来像死尸,但还是活着的——还有呼吸。然后他回过头来,发现这笼子里有了一样现实中没有的东西:一座穿衣镜,边框是黑铁做成的,所以几乎看不见,能够看到的部分很窄,但假如侧点身子来照,也够宽了。她的模样就如就如平日见到的那样,只是腰更细了一些,腿也更长些,穿着就如现实中所见,泛白的牛仔裤和花格子衬衫,脸也和现实中所见的一样 ——这故事开始时就是这样。然后她搬来一把椅子坐在镜子前,开始化妆、更衣。一个男人身临其境,就会感到这个过程漫长、令人哭笑不得。等到这些事做完之后,她穿上了紫色的衣衫——麂皮的无袖短衫和短短的褶裙。这种衣料贴在身上的感觉很细腻。她穿牛仔裤和花格衬衫比较性感,穿这样的衣服不性感。当她穿上牛仔裤和衬衫时,就好比一个女人未遭到男人的玷污,可以称为处女;而穿着那身紫色的服装则显得淫荡。纯洁的形象比较性感,淫荡的形象不性感,但女孩的感觉却恰恰相反。她按了两下电铃,管理员在走廊尽头出现。当这个穿黑衣服的男人走近时,她感到胸口发闷,呼吸急促,同时还觉得腿有点软。这些感觉并不能使闯入者感到愉快,但不管怎么说吧,他还是很感动:一个男人能使女人对他有这样的感觉,就叫做不虚此生。 



二 

秃头到商店里去买鹅毛,鹅毛就插在柜台上的一个磁罐子里。他先进鹅毛伸出手去,又按捺住冲动,把手按在柜台上,对老板说:买十支鹅毛——扎毽子用的。驼背的老板走过来,低头看他放在柜台上的手——指缝间还有墨水的痕迹。看过以后抬起头来看着秃头说:我问你干什么用的了吗?这位老板有一只眼睛生了白内障,惨白惨白的像一个脓包,他就用这只眼睛盯住了秃头,一直年到他的心里去。为了回避这惨白的目光,秃头抬起头来看头顶——头顶上有纵横着的梁和柁,构成一幅复杂的画面。尽管有这些不便,秃头还是买到了鹅毛。他又可以回去伏案运算:虚拟的世界比现实世界还是多一些自由。他走出这间商店,来到街上——他又回到漫天大雪里了。他正要回到自己的住处,用鹅毛笔在羊皮纸上开始他的工作——说来你也许不信,他在虚拟的十七世纪里,用鹅毛笔和羊皮纸做工具,做着网络工程师的工作。你信也好,不信也吧,事情都没有什么两样。人一定要有他需要的环境才能工作。我现在正在网络上写着自己的小说,我可能在黑铁公寓时,对着一台电脑工作着,此时我在真实里。也可能坐在棕榈树下,用芦苇做的笔往纸草卷上写着。所以,你不要问我在什么地方…… 

秃头离开了那所商店走在路上,忽然又嗅到了一股荷花的气味。他发现那个女孩走在他身旁,样子和上一次稍有不同,但还可以看出是同一个人——或者说是同一个幻象。他说道:Hi,你又来了。她答道:是啊,要不,干什么呢。说话的腔调似乎有点熟悉。他不禁问道:你是谁?对方答道:何必要问我是谁。然后她加快了脚步。他知道是追不上的:在虚拟的世界里,能不能追上一个人,总是取决于对方的意愿。但他还是禁不住去追赶,直到她消失在街道的尽头,才停下来喘粗气。在网络上你会遇到很多人,你可以问她是谁,她会告诉你,会给你名片,甚至把电话号码写在你的手上。没有人会拒绝回答她是谁,告诉了你,你也找不到她,因为这是虚拟的真实。忽然间雪又密了起来。他穿过大雪走回自己的土房,在黑暗中想了好久,得出一个结论是:在实际的世界中,这个人是自由的。既然已经想到了这一点,也就到了重返现实的时节。他把耳机从头上摘了下来。这时周围一片寂静,一片黑暗。天花板上亮着那盏遥远的灯,在隔壁的笼子里,女孩在床上睡着。此时可能是午夜,也可能不是午夜。在黑铁公寓里,分不清白天和黑夜。 



后来,那个女孩再来访问自己的文件时,发现一些异样之处。她穿上了紫色的衣衫,按动电钮召唤管理员,管理员就来到了,站在她身后。此时她发现,这位管理员不像平日那样死气沉沉,那样呆板,而是带有一些灵气。他站在她身后沉重地喘息着——过去没有这种喘息。他躬下身子,从镜子里看自己的脸,此时他的鼻息留在她后颈上。然后,他站直了身子,用手指在她脖子上按了一下:这是示意她低下头去,把双手放到背后。此时她感到这只手指的指端十分粗糙。男人的手指应该是这样的,但她以前没有想到。她还嗅到了身后的气味:汗酸味,还有一种海风似的腥味。有关气味,她以前也没有想到。总而言之,这个管理员和她以前想象出的那个不同,他是个陌生人。这种变化使她感到现在不再是一个人的故事,而是两个人的游戏了,故事远非游戏可比,她对此又没有任何思想准备。她发现有人窥视了她的内心世界,这使她蒙羞。从镜子里看到,她的脸已经通红。但她如管理员所示,深深地低下头去,同时在心里想道:蒙羞的感觉其实是非常之好。 



晚上,我呆在宿舍里。我的房间里总是黑着灯,正如它过去总是亮着灯。过去我开了灯就懒得关上,逐渐习惯了在灯光下睡觉。后来灯泡憋掉了,我也懒得换上,逐渐习惯了在黑暗中生活。现在这间房子里笼罩着一层蓝色的光,是从monitor上发出来的。等我把机器关掉,眼前还有一个灰色的方块。不知道是阴极射线管还在发光,还是我眼底的幻象。不管怎么说吧,等这层灰色褪尽,整个房间又呈现出黑白两色的轮廓,就如一篇卡夫卡的小说。应该承认,卡夫卡的小说我读不懂,或者读懂了,却不能同意。我在网络上看到的事情,就如卡夫卡的小说。我可能是不懂,也可能是不同意。我觉得他们都太过古怪。 

秃头下次进入自己的文件,一切又都发生了变化:他的茅草房里不再像冰窖那么冷了。房子里吹着一种温暖的风,这是从墙缝里吹进来的,脚下依然冒上来森森的寒气,这是因为脚下还是那么冷。房间里的一切变得井然有序:桌子还是那张木板桌子,床还是那张木板床,但已经变了一下位置,屋里就变得宽敞了不少。桌子上乱放的纸张被收拾了起来,地面也扫过了,整个房子里明亮了很多。仔细观察后会发现,窗户纸已经换过了。原来是一张不透明的塑料纸,现在变成了一张透明的塑料薄膜。在中古的场景中出现了现代的东西,虽然不协调,但秃头不想挑剔这种毛病。他只想到了这间房子有人来收拾,就像一个家的样子了。这些都不是他的设计,是别人做的。从别人做的这些事情里,他感到了一丝暖意。 

后来,他走出了房子,发现外面的世界也发生了改变。现在正值傍晚时分,天上的云正在懒洋洋地散去。天地之间吹着和煦的暖风,在西下的阳光照耀之下,从地面到天顶,这厚厚的大气里,好像都是暖和的风。地面上的雪已经变成了薄薄的一层,而且变得千疮百孔。远处的小路两旁,立着竹编的篱笆,上面爬满了紫色的牵牛花。除此之外,天上还飞着红蜻蜓。这个世界依然是他的世界,只是添上了几分暖意。虽然这不是他的本意,但他还是觉得很好。他在小路上走着,满身都是暖意。这种温暖来自别人的关心——有人关心和没人关心是很不同的。人人都渴望爱情,但只有有人关心的人才能够体会到什么叫做爱情。如你所知,我的问题就是没人关心。 

晚上我躺在宿舍里,想着401女孩的样子,想起了她下巴上有一粒粉刺。因为这个缘故,她不算非常漂亮,只能说长得还行。我说过,我这间房子里没有灯。后来我走到窗前,看看外面的街道。这条街上漆黑一片。原来这条街上不分白天黑夜总是亮着灯,后来灯都坏了,大家只好摸黑。好在住在这里的人都熟悉这条街,所以没有灯也行。现实的世界很少发生变化,晚上你睡着时世界是这样,早上醒来时还是这样。不像在网络上,几个小时之内,一切都会变得面目全非。 



晚上,401室的女孩和管理员一直出门,走在黑暗的街道上。这条街上原来没有灯,现在有了灯——黑漆的铸铁灯柱顶上,亮着仿古的街灯,十九世纪煤气灯的式样。昏黄的灯光下,墙角窄窄的草坪上那些枯萎的月季花又恢复了生气。草坪上不再有垃圾,而且也恢复了整洁。现在这条街变得适合散步了。在她自己设计的世界里也有这条街,但她从来没有想到要让它变得整洁,这是别人的主意。这就使她心存感激——虽然还不知要感激谁。管理员一声不吭地走在前面,他的样子就如在现实中所见,只是走路的姿势更加挺拔。她决定要感激他,就加快了脚步赶上去,和他并肩走着,告诉他说,她很喜欢这条街。她还说,她想起了苏格拉底的话:不加检点的生活是不值得一过的。但是他没有回答。说句实在话,我听说过这句话,但我不知道苏格拉底是谁。 

夜色中,管理员带401的女孩到离公寓不远的一个酒吧去。这所酒吧安着黑色的铁门,铁门上镶着四片厚厚的玻璃,玻璃背后挂着红天鹅绒的帷幕,门两侧有两根黑铁的灯杆。按动铁门上的门铃,就有带黑色面具的侍者来开门,脱掉她披着的斗篷,用锁链扣住她项圈上的铁环,把她带走——我想她会喜欢的。谁知她并不喜欢,拼命地挣扎了起来。如你所知,虚拟的世界不容许任何情绪激动,每个想摆脱眼前幻象的人只要大哭大闹,马上就可以退出。所以我不能够勉强她。到了外面,她看了我一眼说:我知道你是谁了——你真是讨厌啊。我不能强迫她进入我的酒吧。实际上,我不能强迫她做任何事情。我说,陪我走走可以吗?她说,这可以。于是我们就在这条虚幻的街上走了两趟,她还把头发蓬松的头靠在我的肩上。但是我们没有说什么。她身上带有荷花苦涩的香味,只可惜这种气味不能带回现实中来。 



三 

学校里不是只有我一个人。我发现楼下的水管冻裂了,就到处去找,最后在锅炉房的某个角落里找到了一个管子工。他听说水管冻裂,只是漠然地答道:知道了。看来他是不会去修的。然后他马上就问我会不会打麻将,或者是敲三家。从这句问话来看,学校里除了我和人,还有别的人,甚至有希望能凑起一桌麻将来。除此之外,我在校园偶尔也能碰到一个长头发的家伙,背着手风琴急匆匆地走过。看来他是艺术系的学生,正要赶到什么地方去上课。我想要告诉他,学艺术也不那么保险,我认识一个女音乐家,现在就住在我表哥开的公寓里。但他总是躲着我走,假如我跟着他,他就要紧跑几步。这也不足为怪,我能看出他是艺术系的学生,他也能看出我是数学系的学生,所以他躲我像躲瘟疫一样。而我想要告诉他的正是:不要以为我才是瘟疫,你自己也是瘟疫——这话当然很不中听,所以他躲我是对的。 

在那些行将住进黑铁公寓的人中,有种隔阂:有些人认为自己过得提心吊胆是受了另一些人的连累。前两年这所学校里学生还多时,别的系的人常往我们系的人身上吐吐沫。除了数学系,物理系和化学系的人也常受到这种对待。而我们这些系里的人则往无线电系和计算机系的人头上吐吐沫。这两年这种事情少了,不是因为隔阂没有了,而是因为学生们都退了学,去另谋出路。但就我所知,退了学进去得更快,住在学校里倒安全些。那些退学的同学现在都在公寓里。你说自己没爱什么,管理员是不会放你出去的。他们会说:在公寓里照样可以学习。不但现在退学不管用,你就是十年前就退了学,也免不了住公寓,就拿住在我表哥公寓402室的向着来说,他是我的一位老校友。十年前他上大学二年级时退了学,现在这股风潮一来,照样被逮进公寓朝去。我说的这种隔阂在公寓里照样存在,这位向着住在402室,总想和邻居打招呼,但别人总是不理他。直到住了一个礼拜情况才好了一些。 

在黑铁公寓里,向着和401女孩的床是并排放着的,中间只隔了一道铁篱笆,和一张双人床并无两样。向着对这张床的模样感到很不好意思,很想把它挪开。他试了又试,但总是白费力气:床是用地脚螺丝拧在地下的,而螺丝钉一头埋在水泥里,另一头又被焊死了。弄明白了这一点以后,他忽然感到如释重负,可以心安理得地和女孩并排睡下了。应该说,401的女孩表现得相当大度,她除了偶尔说上一声“我觉得你可以多洗几遍澡”之外,没有说过别的。那个向着就不停地洗着,但身上总有一股铁锈气。最后他说:我身上的味是洗不掉的。想要去掉这股味,只能把自己阉掉。那女孩听了以后,淡淡地说道:那倒不必了。这种冷淡是不公平的,因为这个秃头不是说说而已,假如他的邻居再嫌他有味儿,他真的准备把自己给阉掉。这种自我牺牲精神不是人人都有的,所以,就是拒绝这种牺牲,起码也该说声谢谢。 

住在402室的秃头原来有个绿头发的管理员,我和她很熟。当管理员以前,她在市场街上摆烟摊。再以前,她在我们学校的食堂里卖过卤菜,两只手各套一个塑料袋接我们递过来的钱,等到拿吃的时候再把塑料袋拿下来。她的手长得很漂亮,脸长得也不错,但是最好的还是身材,夏天我在河边上散步,遇见她在河岸上晒太阳。她摘掉墨镜,眯起眼睛来看着我,然后说道:我好像见过你。——这说明她的记性也不错。我赶紧掏出学生证来给她看,说明我还没有毕业,以免她把我捉去住公寓。看完了证件以后,她用手拍拍身边的地面说:坐。这女孩是个自来熟。 

然后她又指指水里的秃头说:我们的房客。向着正被一条细长的链子牵着,在水里游着很小的圈子——那条河的水总是不大流动,绿油油的像一塘死水,秃头在水里游动时像一只小狗。后来他爬上岸来,伸手去拿裤子。女孩说道:别穿裤子了,把屁股也晒晒。他答应一声,趴在了地上。此时我注意到,点人从脸相到身材的确极像我表哥,但神情很不像。神情不像,那就什么却不像了。那女孩还告诉我说:这个人很不错。秃头听到这种称赞,满脸涨得通红。下一句话他听了就不那么高兴—— “他是我们的摇钱树!”但他还是受到了鼓励,努力去挣钱,最后居然成了个小富翁。像这么胡扯下去就不会有个完,我现在要说的是:这个秃头的为人非常老实。后来他住进我表哥的公寓,说要把自己阉掉,可不是瞎说的,在黑铁公寓里,他把秘书洗 了又洗,才撩开被子,准备上床了。这时睡在他身边的女孩说道:该去买条新内裤——身上穿的都露毛的。说完她翻了一个身,把脸转到自己那一侧去。向着又站了一会儿,没有再听到什么。他就钻到自己被子里去。又过了会儿,听到周围没有动静,他从枕头下面摸出一幅耳机来,偷偷地戴在头上了。 



我在河边碰上那个秃头,除了发现他很像我表哥之外,还发现了些别的。此人的阳具甚为伟岸,而我表哥是什么样子我却没有见过。此人甚至比我表哥还要健壮,胸膛像一个木桶,胸口都长了黑毛。我对他的管理员说:这人的毛真多,她听了哈哈大笑了一阵说:男子汉大丈夫,哪能没有毛。我又说:他是不是你的面首?那女孩愣了一阵,然后笑得发滚,用脚跟蹬秃头的头顶说:说,你是不是我的面首?后者闷声答道:不是——是也不能告诉你。管理员听了很高兴,对我说道:听见了吧,我说他不错,他就是不错。后来她把两只脚都放在他的头顶上,而秃头则用秃顶去摩挲她的脚心,这个情景让人看了很不舒服——虽然那绿头发的女孩说这很舒服。我看着身上直发冷,赶紧走了。在他营造的虚幻世界里,他应该用秃头去亲近哪个女孩的脚心,但是他没有,他只是伏在一张桌子上不停地演算,探讨世界的奥秘—— 这就是秃头的可敬之处。
