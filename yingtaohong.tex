\chapter{樱桃红}

六十年代中期的某一天,深秋时节,楚楚走在海德堡的街道上。这个季节德国阴云密布,落
叶飘零。楚楚比以前更成熟,更自信,因而也就更美丽。她穿着黄呢子军装,足蹬马靴,武
装带上挂着马鞭,大檐帽上有一颗红星。挟二战苏联红军横扫欧陆的余威--这身装束就如一
记耳光,抽在了健忘的德国小市民脸上。她从宫堡下狭窄的石板路上走过,走上了内卡河上
著名的老石桥,路上的行人畏畏缩缩地给她让路。在桥上,她向一位中年男子走去,那人惊
恐万状地举起了双手,几乎落入水中。等到知道楚楚只是问路时,又庆幸自己拣回了一条命
,略带几分谄媚地指着方向,甚至陪她走了几步。但楚楚不理他,只顾大步走开,马刺在铺
街石上打着火星。

后来,她走到一条偏僻的街道上,手里拿着一个信封,逐个对照着门牌。最后,她终于找到
了,大踏步冲上了台阶,在她身后,那些畏缩不前的行人找到了机会,赶紧像耗子一样溜着
墙根通过。楚楚按门铃,用马鞭的柄敲门,用皮靴去踢门,用俄文大声呐喊着。在此需要申
明,我们的女主人公不是没有教养的人。但在此时此地,一切繁文缛节都可以忘记--她是一
位复仇女神,向德国人讨还良心债。

门开了,一个头发灰白的老人躲在门后,缩在睡衣里,看到门前站了个苏联大兵,连忙奋力
要把门关上。但是,楚楚的马鞭已经插到门里去。她的力气也比这老头要大。她推开了门,
闯进了门廊,而那个男人则向后退却,本能地把双手举过肩头,面露惊恐之状,嘴里嘟哝着
什么。。。。。。大概是“我投降,请饶命。”后来,他认出了楚楚,就垂下手来,谦卑地说道
:我终于等到了您-您终于来了.楚楚脸色阴沉,把门用力关上,咬牙切齿地说:你这魔鬼,果然
没有死……他答道:这是上帝的意志.稍停片刻他又说:请随我来.

在客厅里,这个德国老人解释着一切:他曾经想用手枪自杀过,但枪卡壳了-他在监狱里度过了
很多年,现在因为有病被放了出来.这个老家伙满脸皱纹,牙齿被咖啡染黑,穿着一件蓝色睡衣
,赤着脚,穿一双长毛绒的地板拖鞋.楚楚坐在沙发里,用马鞭扫着自己的靴筒,而他则坐在对
面的圆凳上,状如受审.

他说道:他已到了风烛残年,地狱正在向他招手.此时楚楚截断他道:但是你还没有死……
她同一道:是.这是上帝的意志.楚楚说:你那位上帝是不是让我可怜可怜你?这句话像鞭子一
样抽在这个前战犯的身上.他因此直起腰来,眼睛里闪着火花,大声说道:不!不要对我用“可怜
”这个词!楚楚一也站了起来,厉声喝道:喊什么,你还没喊够吗!于是他又低下头来,小声说道
:是,是,我错了.我想说的是,您没有明白我的意思……楚楚进一步喊道:你什么意思?我会不
明白你的意思!?-她郁积已久的愤怒像火山一样爆发了.他继续说道:但被楚楚的喊声所淹没
,一点都听不见.直到楚楚喊完了,才听到他说:我不是请您可怜我.我不配啊…… 

听清了这两句话,楚楚又爆发了怒气,再次痛斥德国鬼子说:混账,那你叫我来干什么?......
等到她力竭,客厅里又响起了他的低语:请您惩罚我……如是者再三.楚楚终于语塞,二二乎乎
地问道:惩罚你?怎么惩罚你?他就暧昧地一笑,说道:这就要请您来吩咐了……楚楚终于陷入
了迷惘,跷起腿来,用手支着她的脸腮,小声嘀咕道:这是什么意思呢?.......那德国军官答道
:我没有意思,一切都要听您的意思.这使楚楚更加困惑了…….

趁楚楚沉思的机会,他偷偷打量她,终于幽幽地说道:您可真美啊-楚楚为之一惊.如前所述,楚
楚比在《红樱桃》那部戏里时更美丽,理应得到赞誉;但来自魔鬼的称赞决不是什么好事-如何
针锋相对地反击.实在有点困难.如果说,我丑得很!这是灭我方威风,长敌方志气,如果说:我
就是美!也是助长了敌方的气焰.她终于找到了一句恰如其分的话:狗东西,我美不美干你屁事
!而他又低下头去说:您说得对.所以要请您惩罚我…….

然后,楚楚又在屋里来回踱步,终于说道:你写信叫我来干什么?他舔舔嘴唇,抬起头来说道:“我正要告诉您。我欠别人的都已还清。我只欠您的。”

楚楚:你什么意思?

他说:我只欠您的。这就是说,我是您的了。

随着这句话,他向楚楚低下了头,暴露了他满头的花白头发。。。。。。。
处处瞠目结舌地看着他,终于高叫道:我要你这糟老头子干什么?而那德国鬼子说:不要我
这糟老头子(这句俗话又舌头不会打弯的洋人说出来,声调十分有趣),您干吗要来呢?楚
楚因此震怒,想要斥骂他,但话到了嘴边又噎住。她终于说道:他妈的,你说得也对。她退
回沙发上坐下,开始沉思起来。。。。。。。

后来,他回避着楚楚的目光说:您要不要喝点咖啡?她想了一下,骤然想到自己跟眼前的男
人势不两立,就喊道:魔鬼!谁喝你的咖啡!但他又幽幽地说:您错了。您是这里的主人。
所以,不是我的咖啡,是您的咖啡。这使楚楚更加糊涂了,她终于减低了声音,说道:那就
喝一点吧。于是,他走到厨房里去。。。。。。楚楚一个人在客厅里。她终于可以充分表现
自己的困惑:她不知那德国人要搞什么鬼。

他端着咖啡回来,把托盘放在茶几上,退回自己的座位。她拿起杯子,喝了一口咖啡。然后
她说:苦兮兮的,有什么好喝!我们知道,德国人最讲究喝咖啡,这话使他难以忍受,瞪起
眼来大喝一声:这是最上等的巴西咖啡-我自己都舍不得喝,给您留的!楚楚一惊,双手捧
住了杯子--但也马上又领悟到自己的不对,小声说道:我错了,我不该夸耀我的咖啡。楚楚
也明白了,她伸出手来,把杯子里的咖啡倒在地毯上。可以看得出来,那德国鬼子尽了最大
的努力才克制了自己,没有向楚楚扑去-客厅里铺着波斯的手织地毯,非常值钱。要是楚楚
知道地毯的价值,也不会把咖啡往上倒:应该珍惜伊朗人民的劳动成果-等到最后一滴咖啡
落到地摊上,他才颓然落座道:您做的一切都是对的。。。。。。最后,楚楚把杯子在地毯
边上的地板上摔成了碎片。这老东西禁不住嘟囔了一句:这可是明朝的杯子呀-当然,您做
的都是对的。

楚楚终于按捺不住自己高尚的愤怒,朝那德国人扑去,左右开弓,痛打他的嘴巴。令人诧异
的是,他离开了凳子,跪在地板上,用脸去迎楚楚的手,并用暧昧的声音说道:打得好,请
珍惜你的手!打得好,请珍惜你的手!这使楚楚有点诧异,停下手来问道:怎么个珍惜法?
那德国人征得了许可,爬着取来了一双黑皮手套,让楚楚戴上。后来,楚楚又去砸他的家具
,把一切都砸坏。最后砸的是那德国人坐的凳子,这是个厚重的琴凳,怎么都摔不坏。德国
人说道:缅甸柚木的,我去拿把斧子来。楚楚在愤怒中一脚把他蹬倒,说道:老狗!我不是
给你劈柴来的!但过了一会儿,她又觉得疑惑,停下手来问道:你是不是有病(与此同时,
  她用手指指脑子)?德国人却恢复了普鲁士贵族的自尊,在地毯上跪得笔直,傲然答道:我
没有病!我只是很坏!。。。。。。需要提醒读者的是,阶级敌人是不会彻底坦白的。这个
老纳粹不仅是坏,还有满肚子各种各样的变态心理。我们要彻底把他揭发出来。。。。。。

再后来,楚楚在客厅里踱步,而他在地毯的中央跪好,低着头。周围现在是一片月球景色。
楚楚趾高气扬地说道:老东西,这回你心疼了吧.他心不在焉地答道:是,是。很心疼。但
是。。。。。。楚楚痛恨这个“但是”,厉声喝道:什么,“但是”?德国人就答道:是。是。
没有但是。您说的都是对的。楚楚更高声地喝道:有话就说,有屁就放,兜什么圈子!德国
人说:是的,是的。首先我想告诉您,您生气的样子可真动人(楚楚要不要因此动怒,要不
  要打那德国鬼子,打几下等等,由导演来决定),其次,您为什么不来惩罚我呢?这使楚楚
为难:还怎么惩罚你?他说:我能不能提个建议?楚楚倒吃了一惊:你来提建议?新鲜哪。
。。。。。。。。后来又说:好吧,听听你的主意。他就站了起来,说道,请随我来。
在到后厅的路上,他说道:您还是那样纯真。上帝啊,当年我犯的是什么样的罪孽啊。。。
。。。然后他打开了房门。这里光线幽暗,在这间房子的中央,有一张手术床,与寻常手术
床不同的是,床上钉有一些黑色的皮带:可以看出是用来把手术者的四肢、脖子拴在床上之
用。而这个房间也半像手术室,半像刑讯室。楚楚见了这景象,不禁后退。他说道:我准备
了这些。我一直在等您来-


那德国人走向手术床,他把床边台子上的白布单揭开。台上放着纹身的用具。。。。。。他
脱掉了睡衣,俯卧在床上。奇怪的是,此人的脸虽苍老,身体却像是少年,又白又嫩。要是
又老又皱,就不够刺激-不是谁都配为艺术作牺牲的!他又说:现在,来惩罚我吧。说着,
他就闭上了眼睛。楚楚犹豫了片刻,终于走上前去,用床头的皮带把他的脖子扎住-她已经
被这种景象魇住了。等到皮带全部扎紧,那德国鬼子绷紧了身躯,发出难以形容的呻吟声-
这种声音是楚楚连针都拿不住了。。。。。。楚楚触摸着她的背部,觉得这日耳曼人白玉般
的皮肤简直是艺术品,她有点难以下手。但那德国鬼子说道:请不要怜惜我。。。。。。。
楚楚终于在他背上文出了一只衔着橄榄枝的鸽子。那德国人通过床前墙上的镜子,看到了这
一切,用异样的声音说道:您终于原谅我了。楚楚伏下身去,在上面轻轻一吻。请注意,她
吻的不是德国鬼子,而是吻了这只象征和平的鸟-在此之前,他一直在痛苦地呻吟,挣扎,
至此发出了一声满意的叹息,躺倒不动了。。。。。。再往下就不用我来写,都由导演来安
排。楚楚还在他身上干了些什么,德国鬼子又说了些什么,都由导演来安排。导演是内行-
让我们言归正传,等到纹身结束之后,楚楚松开了绑住他的皮带,翻过他的身体,发现那德
国人已经死掉了,令楚楚不胜诧异的是,他脸上竟带着幸福的微笑。此时,悠扬的乐曲声渐
起,银幕上出现了中英两种文字的字幕:To be continued和“待续”。
