\chapter{2010}

\section{一 老大哥}

1 

每天早上,王二都要在床上从一数到十。这件事具有决定一天行止的意义。假如数出来是一个自然数列,那就是说,他还得上班,必须马上起床。假如数出的好带有随机的性质,他就不上班了,在订上舒舒服服地睡下去。假如你年龄不小并且曾在技术部工作多年,可能也会这样干。因为过去你遇到过这种情况:早上到班时,忽然某个同事没来。下班时大家去看他,他也不在家。问遍了他的亲戚朋友,都不知他上哪儿去了。在这种怀况下,你作为部里的老大哥,就会提心吊胆,生怕他从河里浮出来,脑盖被打得粉碎——这种情况时有发生。过些日子你收到一张通知:某同志积劳成疾,患了数盲症,正在疗养。这时你只好叹口气,从花名册上勾去你的名字,找人作见证,砸他的柜子,撬他的抽屉,取出他的技术文件,把他手上的活分给大家;再过些日子,他就出来了,但不是从河里出来——简而言之,上了电视,登上报纸,走上了领导岗位,见了面也不认识你。这一切的契机就是数盲症。这种病使你愤愤不已、心理不平衡,但是始终不肯来光顾你,你恨好盲症,又怕得数盲症,所以就猜测并且试探它发作起来是何种情形。未离婚时,我前妻见到我这种五迷三道的样子,就说:你简直像女孩子怕强奸一样。我认为这是个有益的启示,遗憾的是我没当过女孩子,不知道是怎样一种情形;问她她也不肯讲。她甚到不肯告诉我数盲症是像个男人呢,还是像男人的那个东西。 

2010年我住在北戴河,住在一片柴油燃烧的烟云之下。冬天的太阳出来以后,我看到的是一片棕色的风景。这种风景你在照片和电视上都看不到,因为现在每一个镜头的前面都加了蓝色的滤光片。这是上级规定的。这种风景只能用肉眼看见。假如将来有一天,上级规定每个人都必须戴蓝色眼镜的话,就再没有人能看到这样的风景。天会像上个世纪一样的蓝。领导上很可能会做这样的规定,因为这样一来,困扰我们的污染问题就不存在了。在我过四十八岁生日那一天早上,我像往日一样去上班。这一天就像我这一辈子度过的每一天一样,并不特别好,也不特别坏。我选择这一天开始我的日记,起初也没有什么特别的寓意。只是在时隔半年,我在整理这些日记时,才发现它是一系列变化的开始。所以我在这一天开始记日记,恐怕也不全是无意的了。 

有关数盲症,我还知道这样一些事:它只在壮年男子身上发作,而且患这种病的人都是做技术工作的。官方对它的解释是:这是一种职业病,是过度劳累造成的,所以数盲症患者总能得到很好的待遇。这一点叫人垂涎欲滴,而且心服口服。数盲者不能按行阅读,只能听汇报;不能辩向,只能乘专车;除了当领导还能当什么?这是正面的说法。反面的说法是:官方宣布的症状谁知是真是假。数盲清正廉洁,从来没有数盲贪赃枉法(不识数的人不可能贪),更没有人以权谋私,任何人都服气。这也是正面说法。反面的说法是他们用不着贪赃枉法,只要拿领导分内的就够多了。正面的说法是领导上的待遇并不超过工作需要,反面的说法是超过了好几百倍;所以应该算算账。为此要有一种计数法、一种记账法、一种逻辑,对数盲和非数盲通用,但又不可能。有位外国的学者说,数盲实质上是不进位,只要是工作用了无穷进制计数法。这种算法我们学不会。假如你就这一点对数盲发牢骚,他就笑眯眯地安慰你说:你们用的二进制、十进制我们也不会嘛。大家各有所长,都是工作需要。 



现在要说明的是,北戴河是华北一座新兴的科技城市,它之所以是科技城市,是因为技术部设在这里。王二是技术部的老大哥,也就是常务副部长。这是未患数盲症的人所能担任的最高职务,是一种类似工头的角色。有时他把自己叫做“王二”,有时把自己叫做“我”;但从来不把自己叫做“老大哥”,这个称呼是专供别人使用的。 

我总是从反面理解世界。早上起来时,我数数,同时也是把灵魂注入了肉体。我爬起来,从侧屋里推出摩托车,从山上驶下来,驶到一片黑烟和噪声里去。这种声音和黑烟是从过往车辆上安着的柴油机上喷出来的,黑烟散发着一种燃烧卫生球的气味,而噪声和你的脑子发生共振。这种情形可惜以往那些描写地狱的诗人——比方说但丁——没有见过,所以他们的诗显得想象力不够。 

只要你到了大街上,睾丸都会感到这种震荡(对于这件事,有一个对策,就是用一个泡沫塑料外壳把睾丸包装起来,此物商店有售,但是用了以后小便时有困难),而黑烟会使你的鼻涕变得像墨汁一样(你也可以用棉花塞住鼻子,用嘴呼吸,然后整个舌头都变黑,变得像脏羊肚一样)。早几年,还可以用我设计的防毒面具,后来吓死过小孩子,不让用了。当然,假如你坐在偶尔驶过的日产轿车里,感觉会有不同。日本人对出口中国的车辆都做了特殊设计,隔音性能极好,而且有空气滤清器。当然,日本人很少得数盲症,故而这些车的售价都到了天文数字,只有得了数盲症的领导才不觉得贵。因为这些原故,乘日本车的人极少,大多数人乘坐在吼声如雷的国产柴油车辆上。驾车的家伙们还表现出了破罐破摔的气概,十之八九把消声器拆了下来,让黑烟横扫街道,让噪声震破玻璃。因此街上的行人都打伞,见了黑烟过来,就把伞横过来挡挡,而临街的窗户都贴了米形纸条,好像本市在遭空袭。这都是因为有人拆了消声器。假如你逮住一个问他为什么这么干,他就说,消声器降低马力增加油耗,而且装上以后还是黑,还是吵,只不过稍好一点,实属不值。当然,你还可以说,取下消声器,省了你的油,吵了大家,所以应该安上。他则认为安上消声器,大家安静,却费了他的油,所以应当取下来。归根结底,假如消声器能省油,谁也不会不安它。如果说到了这儿,所有的人都会同意:也不知是哪个王八蛋设计的这种破机器。只有我不同意,因为这个王八蛋就是我。所有街上跑的,家里安的柴油机,只要是黑烟滚滚,吼声如雷,就是我设计的,假如既不吵,也不黑,那就是进口的,而且售价达到了天文数字,具体数字是多少是国家机密,我们不该知道,而知道这些数字的人,又根本不知是多少。 



2 

每个当了老大哥的人,都有这样一种特殊的品行,就合我来说,有时候我就是我,有时候是王二,他是一个随时随地就在眼前的四十八岁的男人。在后一种情况下, “我”却不知到哪里去了。小徐没有摩托车,必须有人去接他上班。好吧,王二就在眼前,那么王二就去接他吧——这时根本就没有“我”这种东西。等到“我”回来时,就会发现这样做消耗了我的汽油,毁了我的车——这种小摩托设计载重是八十公斤,王二一个就有八十公斤。除此之外,他像个鸡奸者一样趴在我身上。小徐这东西占了你的便宜也不说你好。这都是责任心过强带来的害处。 

责任心过重常常使我大受伤害,每次部里有人失踪了,我都到处去找:去公安局,去医院,甚至低声下气去问保安(他们对我最不友好,摩托车在他们门前停片刻,车胎就会瘪)。到处都找不到之后,坐在技术部里长吁短叹道:假如某某能回来,咱们就开party庆祝——我贡献一百美元。事们说:算了吧老大哥,这小子准是得了数盲症。但我不爱听这话。我从来不相信哪个事,我都有被欺骗、遭遗弃的感觉,一屁股坐在凳子上,叫道:给我拿救心丹来! 

其实我根本不像表面上那样天真。我已经四十八岁了,我认识的人发数盲的,多到我记不住。这就是说,我完全知道谁会发数盲——我见过的太多了。就以目前为例,我可以打赌,技术部有一个数盲,就是趴在我背上的这个姓徐的。早上他提着塑料水桶,里面只有点底子,或者底子都没有(你要知道班上不供应饮水,自己不带水就是想喝别人的);头上戴顶二战时期飞行员的帽子,哆哆嗦嗦地站在路边上,拖着两截清鼻涕,长得尖嘴猴腮。就是把他行将发数盲这一点撇去,也足够不讨人喜欢。我不知道有谁喜欢他,不论是男人女人。但是他现在没有发数盲,他是我的人。他没有钱可以找我借,当然事后准不还;没水喝可以找我要,但是我的水也不多。这就是说,我必须爱他,因为我是老大哥。 

二十年前我来过北戴河,这地方东西两端各有一座小山,山上树木葱茏,中间是一片马鞍形的地带,有海滩,海滩背后没有了树,那些别墅还在那里,但都大大地变了样。所有的门窗都不见了,换上了草帘子、包装箱上拆下的木板、瓦楞纸箱,里面住着施工队。保安员、小商小贩,总之,各种进城打工的人,门窗都被他们运回家了。他们在院子里用砖头垒起了一些类似猪圈的东西,那是他们的厕所。烟囱里冒出漆黑的烟,因为烧着废轮胎。海滩上一片污黑,全被废油污染了。海面上漂满了塑料袋,白花花的看不到海水。废轮胎、废油、塑料袋我们大量地拥有,而且全世界正源源不断地往这里送。简言之,海滩变成了一片黑烟和废油的沼泽地,如果山上很脏的话,这里就是个粪坑。而小徐却偏愿意住在这里——这就是说,我不得不弯下来接他。假比不是这样,饯愿永远不上这里来。出于过去的职业训练,我见了丑陋的东西就难受。 



技术部的房子在东山边上,三面环有走廊,这说明这座房子有年头了,过去是某位达官贵人的避暑别墅。前几年站在走廊上可以望见大海,现在在刮大风的日子里还可以看见,在其他的日子里只能看到一片黑烟。走廊用玻璃窗封上了,这些玻璃原来是无色的,现在变成了茶色。这些变化的原因当然是柴油机冒出的黑为,现在这所房子顶上有一根铁管烟囱也在突突地冒这种黑烟。但这也是没有办法的事,因为这间房子也需要取暖、需要照明,取暖就需要柴油机冷却水来供给暖气,照明则需要柴油机带动地下室里的发电机。这个嘣嘣乱响的鬼东西是我十年前的作品,代表我那时的能力。现在我应当能设计出一种柴油机,起码像泰国的产品,那种机器发出蚕吃桑叶的沙沙声;或者我设计出瑞典柴油机,那种东西你就是把屁股坐在上面,也不知开动了没有。但是应当是应当,实际上我就会造这种鬼东西——开动起来像打夯机和烟雾弹的东西。世界上其他地方不像我们这样,人家甚至很少用柴油机,这是因为那里能找到足够多的未患数盲症的人,来设主、制造、维修那些清洁、有效的集中供电系统。虽然现在已经证明了数盲不传染,但是要请这种人到中国来做技术顾问,却没人应聘;因为人们怀疑它与环境有关系。人们还说,数盲是二十一世纪的艾滋病,在未搞清病因、发现防护措施之前,科技人员绝不敢拿自己的前途冒险——事实上,的确有几位到中国服务的科技人员在这里发了数盲症,后来成为伟大的国际主义战士,享受中国政府的终身养老金。此后有人敢来冒险,但各国政府又禁止科技人员到中国来——科技人员是种宝贵的资源。来的和平队都是些信教青年,所学专业都是艺术、人文学科。就算在来中国前学习一点科学技术的突击课程,顶多只能胜任科技翻译的工作,而希望全在未患数盲症的中国人身上。这些人在早上八点钟以前到了这间房子里,满怀使命感开始工作。 



王二来上班的时候,已经是最后一个。他从摩托车座位下面的工具箱里出一个塑料水箱,走进那间房子,有一个大号的洋铁壶放在小小的门厅里,旁边放了一个量杯。王二从水箱里量出一升水,倒进水壶里,然后旋紧盖子,把水箱放到一个架子上——那上面已经放了四十多个水箱,每个水箱上都有一块橡皮膏,写着名字。然后他脱掉大衣,走到水池子前面,拧开水管子,里面就流出一种棕色的流体——这种东西就叫做自来水。王二从水池边拿起一条试纸试了,发现它是中性的,就在里面洗了手。不管它是不是中性,都没人敢在里面洗脸。因此他拿出了一块湿式的卫生纸巾,先擦了脸,又擦了手,然后走进大厅。这是一种精细的作风,和数盲作风形成了鲜明的对比。在开大会时,你常能看到领导在主席台上倒一塑料杯矿泉水,喝上几口,把剩下的扔在那里,过一会再去倒一杯。等开完了会,满桌子都是盛不的杯子。这就叫领导风度。好在这些水也不会浪费,我们当然不肯喝,想喝也喝不着。保安员都喝了,他们也渴。水这种东西,可不止是H2O而已。 

因为每人每天只有五公升的饮水,所以烧茶的开水都要大家平摊。在这种情况下,我们当然想利用一下自来水——这种水是直接从河里抽上来的,没有经过处理——就算不能达到饮水用的标准,能洗澡也成。有时候咸的,这不要紧,因为不管怎么说,它总比海水淡,甚至可以考虑用电渗析。有时含酸,有时含碱,这可以用碱或酸来中和。有时候水里含有大量的苯、废油,多到可以用离心机分离出来当燃料,有时候又什么都不含。有时它是红的,有时它是绿的,有时是黄的——水管里竟会流出屎汤子——这就要看上游的小工厂往河里倒什么了。有时候他们倒酸,有时倒碱,有时倒有机毒物,有时倒大粪。要净化这种水,就要造出一个无所不能的净化系统,能从酸、碱、有机毒物甚至屎里提取饮用水。这对于科班出身的工程师也不是件容易的事,更何况我们四十一人里有四十个是半路出家。除此之外,还有两个办法可以解决洗澡问题,其一是在夏天到海里去游泳,上岸后用砂子把身上的柴油渍擦去,然后用毛巾蘸饮水擦,因为柴油渍总不能擦得很干净,故而洗了以的像匹梅花鹿;另一个办法是在冬天用蒸馏水来洗澡——我们有利用柴油机废热制蒸馏水的设备。蒸馏水虽然无色透明,但也不干净。洗这种澡鼻子一定要灵,闻见汽油味不要大惊小怪;酚味也不坏,这是一种消毒剂;闻见骚味也不怕,有人说尿对头发好。假如闻见了苯味,就要毫不犹豫地从喷头下逃开,躲开一切热蒸汽,赤向裸体逃到寒风里去。苯中毒是无药可医的毛病,死以前还会肿成一个大水泡,像少时的水母一样半透明。同事们说,洗澡这件事要量力而行,并且要有措施。跑得慢的手边要有防毒面具,女孩子要穿三点式,但是老大哥和有病的不准洗。他们坚决劝阻我在冬天洗澡,虽然我自己说,老夫四十有八不为夭寿,但他们还是不让我在干净和肺炎之间一搏。,并且说,现在我们需要你,等你得了数盲症,干什么我们都不管。所以我只好脏兮兮地忍着。 

我到现在还在设计净水器,一想就是七八个小时,把脑子都想疼了。一种可能是我终于造出了巧夺天工的净水器,从此可以得到无限的干净水,这当然美妙无比。但我也知道遥遥无期。另一种可能是我没有造出这样的净水器就死掉了,死了就不再需要水,问题也解决了;但也是遥遥无期。最好的一种可能性是我得了数盲症,从此也没了水的问题。 



3 

王二坐在绘图桌前的高脚凳上,手里拿了一把飞鱼形的刀子在削铅笔。那刀子有一斤多重,本身是一件工艺品,除了削铅笔,还可以用来削苹果、切菜、杀人。现在的每一把 刀子都是这样笨重,这是因为每把刀子都是铸铁做的,虽然是优质的球墨铸铁,但毕竟不像钢材那樟树做得轻巧。他在考虑图板上的柴油机时,心里想的也全是球墨铸铁,不到万不得已,不能考虑像金子一样贵重的进口钢材。除此之外,钢是危险品,要特批,报告打上去,一年也批不回来。在这种情况下,当然只能设计出些粗笨、低效的东西,这是可以原谅的。只不过他的设计比合理的粗笨还要粗笨,比合理的低效还要低效,这就是不能原谅的了。他只能在另一个领域施展想象力:把柴油机做成巧夺天工的形状,有些像老虎,有些像鲤鱼,有些什么都不像,但是看上去尚属顺眼。不管做成什么样子,粗笨和低效都不能改变,而且像这样稀奇古怪的东西根本不能大批生产,每种只能造个三五台,然后就被世界各国的艺术馆买了去,和贝宁的乌木雕、尼泊尔的手织地毯陈列在一起。如今全世界所有的艺术经纪人都知道中国有个“Wang Two”,但是不知道他是个工程师,只知道他是个结合了后工业社会和民族艺术的雕塑家。这样他的设计给国家挣了一些外汇,但是到底有多少,他自己不知道。这是国家机密。 

有一件事我们尚未提到,就是王二和他技术部的绝大多数同仁一样,喝然现在做着技术工作,但是他们的生活并不是在工学院时开始的。王二本人从工艺美术学院毕业,同事则来自音乐学院、美术学院、中文系、哲学系、歌剧院等等;是一锅偏向艺术和人文学科的大杂烩,但是这锅杂烩在这一点上是一致的:每个人的档案里,在最后学历一条上,都有“速校二年”一条。这是因为随着数盲症的蔓延,所有未患这种病的人都有义务改行,到“速成学校”突击学习技术学科,然后走上新的岗位。还有一个奇怪的现象。就是原来的工程师患起数盲症来很快,改行的工程师却比较耐久。他们是科技精英,虽然假如没有数盲症这件事的话就够不上精英,只能叫做蹩脚货。就以我自己来说,就曾找领导谈过多次,说明自己在速校把数学老师气得吐血的事实。领导上听了以后只给了这样的指示:加强业务学习——水平低是好事,还有提高的余地,所以我们不怕水平低。我说我快五十了,没法提高。他却说五十很年轻。我问多少岁不年轻,他说是二十,同时伸出三个指头,几乎把我气死。和数盲辩理行不通。顺便说一句,数学老师吐血是真的,但他有三期肺痨;而且不是气的,而是笑的。上课时他讲不动了,就让大家讲故事。我讲了个下流笑话,他吐了血,后来就死掉了。 

除了这技术部里坐着一些蹩脚货,还有一些更蹩脚的在钢铁厂里,指挥冶炼球墨铸铁,另一些在炼油厂指挥炼劣质柴油,所到之处都是一团糟,但是离了他们也不行。不管怎么说,王二在这群人里还算出类拔萃。他削好了铅笔,忽然大厅时响起了小号声,还有一个压倒卡罗索的雄浑嗓音领唱道:“Happy birthday to you!” 他在一片欢声笑语时伸直了脖子,想看看这位寿星是谁。但是一把纸花撒到了他头上。这个寿星老原来就是自己。然后他就接受了别人的生日祝贺,包括了两个女实习生的亲吻,并且宣布说,等你们结婚时,一人送一件毛衣。这是因为当时她们每个人都穿了毛衣——一件蓝毛衣和一件红毛衣,当然都是机织毛衣,看起来像些毡片,穿在漂亮姑娘身上不适宜。而王二的手织毛衣都是工艺品,比之刀子更送得出手。这些毛衣需要些想象力才能看出是毛衣,需要更多的想象力才能看出怎么穿。但是穿上以后总是很好看。但是这两记亲吻带来了麻烦——他上衣的口袋里出现了两张纸条。这肯定是她们塞进来的,但是各是谁塞的,却是问题。有一个规定说,禁止把未患数盲症的人调离技术岗位,这就是说,技术部门实在缺人。还有一个规定说,女人不在此列。这就是说,领导机关也要些不是数盲的人,来担任秘书工作。还有一条并不是最不重要,那就是秘书必须长得顺眼,不能长得像王二一样。因此女孩子最好的出路是在十八岁时考上工学院(工学院考分高得很,而且不招男生),二十二岁毕业,到技术部实习一年,然后到上级部门当秘书。此后很快就成了首长夫人。这是一条铁的规律,甚至不是孩子的人都不例外,只要漂亮就可以。因为这个原故,工学院挑相貌,挑来挑去,简直招不上生来。现在听说条件放宽了,但是要签合同,保证接受整容手术。我觉得以后可能会接受肯变性的男生。当然,这种货色,就如艺术家改行的我们,是二等品。 

有关艺术家改行的事,还可以补充几句,我们改行后,原来的位子就被数盲同志们接替了。所以现在简直没有可以看得进去的小说、念得上嘴的诗歌,看得入眼的画;没有一段音乐不走调,假如它原来有调的话。与此同时,艺术家的待遇也提高到了令人垂涎欲滴的程度。但是这也叫人心服口服——你总得叫人家有事可干嘛。而且艺术现在算是危险性工作了,它教化于民,负有提升大家灵魂的责任,是“灵魂的工程师”。万一把别人的灵魂做坏了,你得负责任;这种危险还是让数盲来承担。假如大家都去当领导,领导就会多得受不了,假如不让人家当领导,人家又劳苦功高。所以就让他们当特级作家、特级画家,这还是亏待人家了。 



4 

我有个哥哥,已经六十多岁了,现在住在美国。1970年左右,他在乡下当过知青。我那时只有七八岁,也知道他当时苦得很,因为每次回家来,他都像只猪一样能吃。他告诉我,他坐车不用买票,而且表演给我看。有一回被售票员逮住,他就说:老子是知青!售票员大姐听了连忙说:我弟弟也是知青。就把他放了。他还告诉我说,他们在乡下很快活,成天偷鸡摸狗不干活也没有人管。这件事告诉我,为非作歹是倒霉蛋的一种特权。我们就是一批倒霉蛋。所以拥有这种特权。举例来说,假如我看中了一间空房子,就可以撬开门搬进去住,不管它贴着什么封条。过几天房管局的人找到我,无非是让我把原来房子的钥匙交出来,再补办个换房手续。但是不管我搬到哪里,房子都没有空调,没有干净的供水,没有高高的院墙,门口也没有人守卫,所以搬不搬也差不多。现比方说,我们和哪个女孩子好,就可以不办任何手续地同居,假如风纪警察请去谈话,无非是说:你们双方都没有结婚,何不办个结婚手续?只是过不了几天,这位女孩子调到机关去,就会和我们离婚。然后就是傍肩,天天吵吵闹闹。据我所知,大家都有点烦这个。但这种生活方式是不能改变的,除非得了数盲症。 



我简直想患数盲症,主要是因为现在的工作不能胜任。今天早上搞电力的小赵递给我一张纸,说道:对不起老大哥,遇到了问题。我拿起来一看,是道偏微分方程。我就知道这一点,别的一概不知。我举起手来说:大家把手上的事放一放,开会了。于是我们这些前演奏家、前男高音、过去的美术编辑、摄影记者等等,搬着凳子围成个圈子,面对着黑板上的微分方程,各自发表宏论。假如此时姓徐的不在,那也好些。他在场只会增加我们的痛苦。我说过,我们这间屋子里的人几乎都是蹩脚货,这孙子是个例外。他是个工科硕士(很多年以前得的学位),像这种人不是发了数盲症,就是到了国外,这孙子又是个例外。他听了某些人的意见,面露微笑。听了另一些人的意见,捧腹大笑。听了我的意见之后,站在椅子牚上,双手掩住肚子,状如怀孕的母猴,在那里扭来扭去。坐在他旁边的人想把他拖出去,他拼命地挣扎道:让我听听嘛!一个月就这么点乐子……这使大家的面子都挂不住了。大胖子男高音跳起来引吭高歌,还有人吹喇叭给他伴奏。在音乐的伴奏下,有些人动手拧他——怀着艺术家那种行业性的妒贤嫉能,以及对卑鄙小人的仇恨。这家伙是个贱骨头,挨拧很受用。等到乱完了之后,我就宣布散会。偏微分方程不解了,因为解不出来,改用近似算法。这个例子说明我们设计的东本为什么这么蹩脚——用了太多的近似算法。而在近似算法方面,我们都是天才。我们已经发明了一整套新的数学,覆盖了整个应用数学的领域,出版了一个手册,一流装帧,一流插图,诗歌的正文,散文家的注释,但是内容蹩脚之极。手册的读者,我人下级单位的同行经常给我们寄子弹头,说再把书写得这样不着边际,就要把我们都杀掉。其实我们不是故作高深,而是要掩饰痛脚。 

不光数学是我人瓣痛脚,还有各种力学、热力学、化学、电工学等等。事实上,我们的痛脚包括了一切科学部门。我知道,美国有个《天才科学家》杂志(这个天才当然是带引号的),专门刊载我们的这些发明,而有一些汉奸卖国贼给他们写稿,还把我们的照片传出去,以此来挣美元稿费,其中就包括了这个姓徐的。因为他的努力,我已经有两次上该刊的中心页,三次上了封面,还当选过一次年度“天才数学家”。据说正经搞理工的读了那本刊特,不仅是捧腹大笑,还能起性,所以我经常接到英文求爱信和裸体照片,有男有女,其中有些还不错,但多数很糟糕;危险部位全被炭笔涂掉了。我一封信都不回。对于某些搞同性恋的数学家,我比《花花公子》的玩伴女郎还性感。为此我不止一次起了宰掉小徐的心。但是我也是明白,就是倒霉蛋也不能杀人。 

我觉得外国的科学家缺少同情心——假如他们和工程师都傻掉,只剩下一些艺术家,我倒想看看他们那里会发生什么样的事。假如毕加索活着,马蒂斯活着,高更和莫奈都活着,我也想看看他们画起柴油机是否比我高明。但是最没有同情心的是小徐这种人。我曾经把炭笔塞到他手里,强迫他画一张画,哪怕是画个鸡蛋也行。但是他就是不接,还笑嘻嘻地说:我不成,我有自知之明。这话又是暗讽,说我们都没有自知之明。 

在马蒂斯决定复活,替我来画柴油机,我还有一件事要提醒他:他休想得到一点顶用的技术资料。有件事和他死前大不一样:国外所有的技术书刊都以光盘、磁盘的形式出版,而这类东西是禁止进口的,以防夹带了反动或者下流的信息。至于想用计算机终端从国外查点什么,连门都没有。这是因为一切信息,尤其是外国来的信息都是危险的。的电话可以,必须说中文,因为有人监听,听见一句外文就掐线。我不知马蒂斯中文说得怎么样,假如说得不好,就得准备当个哑巴。除此之外,什么材料都是危险品:易燃的、易爆的、坚硬的。危险这个词现在真是太广义了。在这种条件下,让马蒂斯来试试,看他能搞出些什么! 



会后小徐对我说:你把你的贝宁木雕给我,我就给你算这道题。我说你妈逼你想什么呢你,又不是我要算这道题。那时候我的脸色大概很难看,吓得他连连后退,过了老半天才敢来找我解释:“老大哥,要是你要算这道题我马上就算,要你什么我是你孙子!” 

这时我已经恢复了老大哥的风度,心平气和地说:我不要算这道题,是公家要算这道题。我尽心尽力要把它算出来,这是我的责任,但它毕竟不是我的题。小徐说:只要是公家的题他就不算,这是他的原则。但是他不愿为此得罪老大哥。我说:我怎么会?坚持原则是好事。为了表示我不记恨他,我和他拥抱,吻了他的面颊,这让我觉得有点恶心——这家伙有点娘娘腔。但我既然是老大哥,对所有的人就必须一视同仁。 

有关那件木雕,有必要说明几句,那是上大学时非洲同学送我的,底座上刻着歪歪斜斜的中国字:老大哥留念——我们是有色人种。这是个纪念品,其一,这说明我上大学时就是老大哥;其二,它说明有个黑人把我当成黑人。一般来说,我们黄种人总是被黑人当成白人,被白人当成黑人,被自己人不当人,处处不落好。我能被黑人当黑人,足以说明我的品行。这姓徐的竟想把它要走,拿到黑市上卖。只此一举,就说明他要得数盲症了。 



开完了数学讨论会后,我坐到绘图桌前,那个穿红毛衣的实习生搬凳子坐在我身边,假装要帮我削铅笔,削了几下又放下了。说实在的,削铅笔不那么容易,刀子钝笔芯糟,假如她只是心里有话要说,那就是糟蹋东西。那孩子悄声对我说:王老师,我会算这道偏微分题。我也悄声说道:别管我们的事——辅导老师没关照你吗?她说:关照过的,但是我的确会算。我不理她(我还要命哪),她还是不走,这叫我心里一动——于是我压低了声音说:读过《1984》?她脸色绯红,低着头不说话。这就是说,读过了。 

我们过去都是艺术家,艺术家的品行就是:自己明明很笨,却不肯承认。明明学不会解偏微分方程(我们中间最伟大的天才也只会解几种常微分方程),却总妄想有一天在睡梦中把它解开,然后天不亮就跑到班上来,激动地走来走去,搓手指,把粉笔头碾成粉;好容易等到大家来齐了,才宣布说:亲爱的老大哥,亲爱的同事们,这道题我解出来了!!然后就在黑板上写出证明,大体上和数学教科书上写的一样,只是在讲解时杂有一些比喻,和譬如“操他妈“之类的语气助词,这能使大家都能理解。有了这些比喻和“操他妈”,证明就属于我们了。讲解者在这种时候十分激动并且能得到极大的快感,有一位天才的指挥家在给大家讲解“拉格朗日极值” 时倒下去了,发了心肌梗塞,就此一命呜呼。这种死法人人羡慕。因为这个缘故,我们才不容易得数盲症。也是因为这个缘故,我们不喜欢女人来帮助我们。当然,有些少数丧失了自尊心的人也会这么干,那就是另一个故事了。 

关于艺术家不得数盲症的机理,有必要讲得更明确:我们在科技方面十足低能,弄不懂偏微分,所以偏微分才能吸引住我们。假如能弄懂,就会觉得没有意思了。这就是说,我们不能太聪明,并且要保持艺术家的狂傲的性情,才能在世界上坚持住。 

另一个故事是这样的:以前我有一位同事,是吹萨克管的,是个美男子。因为在十几岁时玩过一阵子无线电,速校毕业后负责电子工程。此人钻研业务到了走火入魔的程度,发誓不把概率论里的大数定理搞明白死不瞑目。因此他就丧失了自尊心。有一回,我们部里来了个小眼镜,她说能证明大数定理,也不知用了什么手段,居然让美男子听懂了证明。然后他就完全惟小眼镜马首是瞻。听说他们在家里玩一种性游戏:小眼镜穿着黑皮短裙,骑在美男子脖子上。后来她实习期满要调到上级单位时,两人就双双殉情而死——这当然又是小眼镜的主意。刚毕业的女孩子总是对殉情自杀特别感兴趣(她们最爱说的一句话就是——让我们一块死吧!仿佛只剩下电死吊死还是淹死这样一些问题),但是不能听她们的,都死了谁来干活?我就接到过多次同死的邀证,都拒绝了,是这么说的:你能调到上面去很好呀,别为这个内疚;我们大男人,不和女孩子争,等等。讲完了,挨个耳光,事情就过去了。这是因为我从来不请教女人数学问题。假如证教过,知道了她们有多聪明——她们的美丽已经是明摆的了——多半就没有勇气拒绝死亡邀请。这是活下去的诀窍。 

有关这个诀窍,必须再说明一遍,因为它很严重。不能问女人科学问题,因为你已经四十多岁了,做了多年科技工作,不懂大数定理、不会解偏微分方程,而且得不了数盲症,又有何面目活着?我们都在危险中,所以就不要让一个二十岁的女孩子告诉你,你不会的她都会。这是因为你是男高音、画家、诗人,她要得到你。活下去的氓窍是,保持愚蠢,又不能知道自己有多蠢。有一句话,我要与大家共勉:好死不如恶活。我的兄弟们,我已经四十八岁了,还有一身病,但还在坚持。 

5 

今天是星期四,也是我四十八岁的生日。这一天的一切,都有必要好好总结一下。我像往常一样上班去,天像往常一样黄,自来水像往常一样臭,像往常一样,有人遇到了一道数学题,我们开会讨论,并且像往常一样没有解出来。这都是表面现象。实际上,我比往常老了一岁,天比往常更黄了一点,自来水比往常更臭了一点,没有解出的数学题比往常多了一道,一切都比往常更糟糕。我在制止这个恶化的趋势方面竭尽了心力:力图忘掉今天是我生日,力图改进我的柴油机想让它少冒点烟,力图想出一种净水器,力图解出那道数学题,但是全都没有结果。我们技术部里每个人都在力图解决这些问量(只有第一个问题除外),但是都没有结果,因为他们都比我还笨。只有一个人除外。首先,他可以解出那道数学题,其次,他是学化工的,在水处理方面肯定有办法;最后,他是管燃料的,假如能给我纯净一点的燃料,柴油机就可以少冒一点烟。但是他什么都不干,到班上打一晃,看完了我们的洋相后,就溜出去了,而且是借了我的摩托车。我有确实的情报,他是跑到上级那里去打小报告去了——虽然他自己说是去医院看病——此种情形说明他很快会发数盲症。我应该不借他车,但是我不能。他说,他要去看病。而且我是老大哥。

\section{二 红毛衣&老左}

1 

红毛衣说,她看过《1984》。这是乔治·奥威尔的作品,是一本禁书(现在有很多禁书),因此没有铅印本,但是有无数手抄本,到了工学院的女生人手一本的地步。我的外号就是从书里有个情节,女主人公往男主人公兜里塞了一张条子,昨天就出了这种事,我兜里出现了一张纸条,上面写着“I love you!”连写法都和书上一模一样,足见看《1984》入了迷。只有一点和书上不同:作为男主人公,我不知是谁塞的。在此这前,我过生日,每个实习生都要吻我,这是一种礼仪。一共两个女孩子。有一个很奔放,简直是在咬我,另一个很不好意思。那个不好意思的脸红扑扑,嘴唇很硬,这种情形说明她从未有过性经验,所以应该把她排除在外,但其实真凶就是她。我总算找到我需要的人了。 

王二把红毛衣请到家里来喝咖啡——我这样写,是因为当时我正在大公无私的状态——王二有真正的哥伦比亚咖啡,是他哥哥寄来的,不过有年头了,没有香味。但毕竟是真正的咖啡。现在他还给王二寄咖啡,但是总也收不到,因为邮政系统也是一团糟。好在还可以打越洋电话,否则就会和哥哥断掉联系。打越洋电话比国内电话容易得多,拿起听筒摇上几下(现在电话都是人力驳接的了),说:你给我接美国,然后喀喀乱响一阵,就换了声音,“ATNT operator……”,你告诉她对方付款、电话号码,马上就会通。当然,有时也不顺利,接线员朝你大吼一声:美国,美国在哪儿?你只好告诉他往上找,左边第一个,有时他还是找不到,此时就只好骑车奔往电话局,自己来接线,不过这种现象不多。哥哥要给王二打电话就麻烦得多,先接中国,再接河北,再接秦皇岛,再接北戴河;这就要三个钟头。接到北戴河就不能接了,好在此地人人认识王二,半个电话局的人都会出来找他。但是他跑去接电话时,十回里有九回不是他的电话。电话里的人再三道歉说,他想找某人,但是电话局的人不认识某人,并且建议他找王二,王二谁都认识,所以只好找王二传话。这些话越扯越远,就此打住——红毛衣对王二说:味真怪。这说明她没喝过真咖啡。喝完了以后,她还是一副手足无措的样子,连杯子也不知往哪里放。这是因为她以前没到单身男人家里做过客——这孩子长着一个圆圆的娃娃脸,很可爱。王二说,把杯子放在桌子上,她就把杯子放到桌子上,与此同时,提醒自己一定要勇敢一些。这屋子里很暖和,墙上挂着挂毯,茶几上有一件乌木雕,但是看不出雕的是什么。她把手放上去,问王二这是什么。王二说是阳具。她赶紧放开手(好像握到了蛇),定了定神,又握住它说:很好玩。此时王二招了招手说,你坐过来。她就坐到王二身边,心里怦怦地跳,但也觉得自己很勇敢。王二抚摸了她的头发,吻了吻她的额头,说道:你很可爱。然后又用一根手指触触她毛衣底下凸起的乳房,然后说:说吧,找我有什么事。那孩子把脸伏在他胸前说:我爱你——我有点恋父情结等等。语不成声。王二哈哈地笑了起来:真奇怪,你们个个都有恋父情结。别逗了,看我能为你做点什么。于是她坐直了身子,看着王二的脸。王二的眼睛时,全是慈爱。于是她不再扭捏,坦言道,她喜欢大胖子。王二说,大胖子有傍肩了,是和平队里的一个金发女郎。后来她又说,喜欢小赵。王二摇摇头说,你对他不合适。再说,他也不需要你。小孙就要到湖边去砸碱了,你肯不肯押他去?她马上就答应了。这说明小说真是有危险的,《1984》就能让一个女孩子情愿担当看守这样危险的工作。只有数盲才能写出毫无危险的小说——那种小说谁都不看,故而无危险。 

有关这件事,我还有点需要补充的地方。我当然爱听女孩子对我说:我爱你。但恋父情结之类的话一点都不爱听。她们这样说,当然有她们的道理,但我不爱听也有我的道理。我还什么都没有做呢,怎么就被人看成了个老头子呢? 



我就在湖边砸过碱,那是一片大得不得了的碱地,好似一片冰雪世界。这个比方年轻人未必能听懂,因为有十几年冬天不下雪了。由于缺乏电力,所有的碱厂都停了产,纯碱却是工业不可或缺之物。所幸有些玉米地里会长出碱来,虽然含有很多的盐,但也不是不能用。当然,地里出产碱的话,就不长玉米,这叫做鱼与熊掌不可兼得。那里是不折不扣的地狱,但是犯了错误的话,就不能不去。小孙设计的锅炉爆炸了——这多半不是他的错,谁知那锅炉是怎么烧的。现在的锅炉工都是农民,技术员都是锅炉工,工程师都是艺术家,艺术家都有数盲症,操蛋的可不只是我们——但是锅炉工也炸死了,死无对证。故此他得到湖边砸上两年碱。这件事本身并不是那么坏——只要你砸过碱,什么也不怕了——但是因为锅炉炸死了人,他情绪低落,十之八九会在湖边自杀。我得找个女人和他一道去,这样他就能活过来。我当年去的时候,双手铐在一起拎着行李。我前妻跟在后面,手里摆弄着一把手枪,说着:别做蠢事——否则一枪崩了你!走着走着一声枪响,把我的帽子打了一个大洞。她很不好意思地说:走火了。我说:不怪你。国产枪总是走火,球墨铸铁就是不行。她又板起脸一说:往前走!球墨铸铁一样打死你! 

有关球墨铸铁的事,需要补充几句。这种材料是非常之好的,可加工性好,熔点低,而且钢铁厂那些笨蛋就炼得出来,就是太笨重。拿它造出来的柴油机像犰狳,方头方脑怪得很;造出的手枪像中世纪的火铳,最小号的也有十五公斤。我前妻端了一阵,就累出了腱鞘炎。后一上我拿泡沫塑料做了一个,和真的一样,而且不会走火,不重要的场合拿它充数。只是用它时要小心,别被风吹走了。 

有关碱场的风光,还有必要补充几句。那里一片白茫茫,中间是一片洼地。洼地里有一些小木棚,犯人和管教就住在里面。那地方有很多好处:因为水里含碱,洗衣服不用肥皂,当然衣服也很快就糟。因为风很大,可以放风筝,但是冬天也特别冷。伙食有利于健康,但是热量也不够。在那里除了干活,还要伺候管教。假如你是男的管教是女的,或者你是女的管教是男的,就得陪管教睡觉。这是因 为晚上实在实在没事可干,一人睡一个被窝又太冷了。 

我设计的柴油机没有爆炸过——这种东西不会爆炸,除非你在气缸里放上雷管,而那种爆炸就不是我的责任了——我去砸碱是另有理由。大概是在十年以前吧,就像天外来客一样,技术部里来了一个归国留学生,学工程的博士。当然了,在他看来我们都是垃圾,我们的设计都是犯罪,我们听了也都服气。然后他就当了老大哥,我下台了。这使我很高兴。就是现在,谁要肯替我当这个老大哥,就是我的大恩人。他一到部里来,大家都觉得自己活着纯属多余,当然也不肯干活;因此就把他累得要死。 

除了设计工作,他还给我们开课,从普通物理到数字电路全讲。听课的寥寥无几,但我总是去听的。我从他那里学了不少东西,所以才能设计柴油机,速校里学的东西只够设计蒸汽机——过去我设计的动力机械就是蒸汽机,装到汽车上,把道路轧出深深的车辙——后来我和他发生了技术路线上的争论——他主张大胆借鉴新技术,一步跨入二十一世纪;我主张主要借鉴二十世纪前期的技术,先走进二十世纪再说,理由如下:你别看我们这些人是垃圾,底下的人更是垃圾。提高技术水平要一步步一米这本是两个非数盲之间的争论,争着争着,数盲就介入了,把我定为右倾机会主义路线头子,送到湖边去砸碱。有个女孩子毅然站了出来——她就是我前妻。砸了两年,提前被接了回来。这是因为好多人得了数盲症(包括那位留学生),部里缺人,又把我调回来当老大哥。这位留学生当了我们部长,隔三差五到部里来转转,见了我就放些臭屁:老大哥,以前的事你要正确对待呀!我就说:正确对待!部长,我爱你!搂住就给他个kiss。其实不是kiss,而是借机把鼻涕抹到他脸上。他一转身我就伸脚钩他的腿。谁要是被碱水泡过两年,准会和我一样。 

有关砸碱的事,需要补充一下。当你用十字镐敲到厚厚的碱层上时,碱渣飞溅,必须注意别让它迸进眼睛里。这是因为碱的烧伤有渗透性,会把眼睛烧瞎。你最好戴保护眼镜,但是谁也不会给你这种眼镜(你只能自己做),也不会告诉你这件事(你只能自己知道),所以有好多人把眼睛烧瞎了——有人瞎一只眼,有人瞎两只眼。瞎了两只眼的人就可放心大胆地不戴眼镜砸碱,因为再没有眼睛可瞎了。 



红毛衣的事后来是这样的:小孙判下来之后,我们部里该派个人看守他——这种事一般是轮班去的,而且总是我排第一班。这一回她站了出来,自告奋勇去基层锻炼。我前妻当年也是这样的,开完了宣判会,大义凛然地走到我面前,喝道:王犯,把手伸出来!就把我铐上了。那副大铐子差点把我腕骨砸断,因为是铸铁的,有七八公斤,里面还有毛边,能把皮肉全割破。我人这种铐子,是因为铸铁没有危险性。后来我做了一副铝的,供自己用——这铐子还在,我把它找了出来,让红毛衣拿去铐小孙——当时我垂头丧气,灰头土脸,拎着行李走上囚车,她在后面又推又搡,连踢带打。事后她解释说,不这样数盲们会觉得她立场不稳而换别人。红毛衣把小孙押走时,也凶得很。总而言之,一直把我押到碱场的小木棚里,我前妻才把我放开,说道:现在,和我做爱。这就是所谓的浪漫爱情。根据我的经验,浪漫的结果是男方第一夜阳痿。我是这么对我前妻解释的:瞧,你把它吓坏了。但是红毛衣后来从碱场打电话来说,小孙没吓坏。他现在情绪很好,,吃得下睡得着,夜不虚度。一开始总是这样的,后来就开始吵架。不过等吵起来时,也该回来了。 



2 

我前妻是学工的,三十岁时被调到市政府当秘书,就和我离婚,成了市长夫人。她告诉我说,她很爱我;但是她非嫁给市长不可,因为我是个混蛋。这件事使我着实恼火(虽然我也承认混蛋这个评价恰如其分),但是下班以后,我又不得不去找她。这是因为我需要些进口的东西——我的摩托车快没油了。除了找她要汽油之外,还可以用工业用的粗苯兑上少许柴油来当汽油,去年我用了一阵这种油,尿里就出现两个加号,这说明我已经开始苯中毒,很快就要肿成个大水泡。另一个办法是把我这辆娇小玲珑的日本摩托车卖掉,换辆柴油摩托。后者的样子和二十世纪大量生产的手扶拖拉机很相似,结构也很像,说实在的,根本就是一种东西;这样就用不着汽油。这样做又有个克服不了的困难——我现在有点外强中干,要在冬天把柴油机摇起来,肯定不能回回成功。最后一条路就是不要摩托,走路或骑车来上班。这也肯定不行,路上的黑烟能把我呛死。除了这些原因,还有一个最重要的原因:这辆日本摩托是件漂亮东西,我不能放弃它。所以不管愿意不愿意,我都得去要汽油。而且这件事本身没什么不道德,因为我们部里几乎每个人都和一个以上的女秘书“傍着肩”(换言之,有女秘书、首长夫人做情妇),并且有时向她们要点进口货,而这些女秘书都在我们这里实习过。假如没有实习制度,全部的人都要像我一样留胡子(铸铁刀刮不了胡子,只能把脸皮刮下来,非用进口刀片不可),但是留胡子的人没几个。这件事的卑鄙这处在于我有半年没有去找她了,每次她打电话来,我都对接电话的人喊一声:告诉她我不在。第一次去找她就是要东西,我又算个什么东西。但是我还是决定去找她,并把这件事载入日记。像这样的事应该向数盲汇报。最好市长能知道我搞他老婆了。 

我去找她之前,心里别扭了好久。为了证明我对她有感情,我给她织了一件长毛衣。其实我用不着织毛衣,只要在部里说一声,自然会有人给我去要汽油。但这马上就会在全市的女秘书中传开,对我前妻是个致命的羞辱(说明她的傍肩吹了)。我很不想这样。我带着毛衣去找她,但是没好意思拿出来——我老觉得这有点像贿赂。她给了我汽油加上一大堆的调侃,这些我都泰然接受了。直到她看到了我那块车牌子,哈哈大笑了一阵,说道:原来你是个诚实的人!我以前怎么没想到。好哇好哇……我就暴怒起一火跑到院子里,发动了车子想要跑掉,这时忽然想到工具箱里有件毛衣,就把它拿出来朝她劈面掷去,说道:拿去,我不欠你什么。然后就奔回家里来了。 

有关那块车牌子应该说明一下。我想地之我有可能突然死掉——比方说,在街上被汽车撞死,或者中了风——总之,不是顾影自怜或在伤感,而真有这种可能性,因此要对自己做些总结。所以我做了个车牌,上面写着“我是诚实的人”。这牌子挂了好几天,没有人注意。我当然不是说自己从来没说过谎——这种人就算有也不在中国——与此相反,我要承认自己真话不多。我是说我在总体上是诚实的。这就是说,我做任何事都尽可能偏向诚实。这一点谁也不能提了反驳。但是我前妻见了这牌子,就像见了天大的笑话一样,这大大挫伤了我的自尊心。 



有关汽油和毛衣的账是这样算的:汽油是进口的特供物资,而且又是危险品,一般人搞不到。假如你是有汽车的人,那就要多少有多少,假如你不是,汽油就是无价之宝;而毛衣是王二手织的工艺品,假如你是五二,那就要多少有多少,假如你不是王二,那也是无价之宝。以上算法是以对人民币而言,假如拿到港口附近的美元黑市上去,毛衣值得还要多一些,因为王二是科班出身的工艺美术家,本人又有些名气。 

用美元来算,劣质柴油和机织毛衣就是一文不值的垃圾——除了某一种特定牌号的柴油可以卖给流浪汉,因为可以当毒品吸——但是到黑市上买卖东西是犯法的,所以这种算法不能考虑。在可以考虑的算法内,毛衣和汽油等值。顺便说一句,柴油是各种东西兑成的,成分复杂而不稳定,有时能创造了贱些奇迹。有些柴油可以炒菜 ——这就是说,菜籽没掺多了;有些柴油可以刷墙——这就是说,桐油掺多了;有些柴油可以救火——乡镇企业的产品常中这样,当然是水掺多了。只要不是最后一种情况,都可以加入我设计的柴油机。我的设计就如一口中国猪,可以吃各种东本,甚至吃屎。奇迹归奇迹,它们还是一堆破烂,一文不值——因为它能把你的生活变成垃圾。 

这件事给我的启示是有两种办法可以创造真正的价值。一种是用工业的精巧,另一种是用手和心。用其他方式造出的,均属大粪。但是大粪没有危险性。我住在山上一座木板房子里,地板上铺着自己做的手织地毯,墙上挂的挂毯也是自己做的。我还有一台Fisher牌的音响设备,这是用挂毯跟小徐换的。我的房子里很温暖,很舒适,环境也安静。晚上我躺在地毯上听美国的乡村音乐,身上一点都没有发痒。这是因白天在她家里洗了个热水澡。这件事很不光彩,但是我没法抵挡这种诱惑。在那个白瓷卫生间里,我还喝了几口喷头里出来的热水——是甜的,比发给我们的饮水都要好。当时我渴极了。在此之前,她给我可乐,我没喝。这似乎证明了我前妻的话:只要我能克服违拗心理,一切都会好。我前妻住在一个小院子里,房子很漂亮,安着茶色玻璃窗子。院子里有几棵矮矮的罗汉松,铺着很好看的地砖——第一次看到时我入了迷,后来就讨厌这种地砖、这个院子。她还问我为什么老不来,我说市长就住隔壁,这当然是托辞。真正的原因是我没有这样的院子。但是假如这样说了的话,她就会嚷起来:你跟我计较有什么用?这世道又不是我安排的呀! 



也许是因为白天洗了澡,也许是因为屋里太暖和,我身上的那个东西又变得很违拗。那东西直起来以后,朝上有一个弧度。因为它的样子,所以是我前妻调侃的对象。事实上这样子帅得很,所有表现工艺品全是这样的。就在这个时候,有人来敲我的窗子——原来是我前妻。她把自己套在一个透明的塑料斗篷里——现在女人出门都要套这种东西,否则就会与烟炱同色。在这件斗篷下面,是我送她的毛线外套——我把它织得像件莲花做成的鱼鳞甲,长度刚好超过大腿——再下面什么都没穿,除了脚上的长统靴子和密密麻麻的鸡皮疙瘩。她是走着来的,大概走了一个半小时吧,但她还是强笑着说:我来谢谢你送我毛衣。焐了老半天她才暖过来。我们俩做了爱,她在我这里过夜。她说:你的确是个诚实的人。和诚实的人做爱有快感,和不诚实的人做爱什么也感不到——就这点区别。 

我前妻已经三十五岁了,依然很漂亮。她想留下来和我过几天,但是我没答应。第二天早上起了个大早,用摩托车把她送了回去,然后再去接小徐。这一次她不肯穿那件毛衣,怕把它搞脏了,就把自己裹在一条毯子里,在后座上裸露出光洁的两条腿,让半城的人大开眼界。在我年轻时,这准会引起一场轩然大波。但是现在什么也引不起。假如风纪警察把我逮了去,我就说我是技术部的。假如他还是不放我,我就说我有点毛病——为什么只准别人有毛病,不准我有毛病?事实上技术部的人只要不杀人放火,并且别被保安逮到,干什么都没问题。 

有一点需要说明的是:假如我被判定得了数盲症,就不会和领导的夫人乱搞。得数盲的人不乱搞,假如组织上不安排,连自己老婆也不搞。我想这一点应该让上级知道。 

3 

我是中国年龄最大的工程师,这是我前妻告诉我的。我做技术工作有很多年了。我前妻还说,假如我患了数盲症,给我重新安排工作时,要计算我的分数,在算法公式里数盲前年龄和数盲前工龄占很大比重。她给我算了一遍,发现已经到了天文数字。我一旦数盲,就能当个省级干部。这就是我们破镜重圆之时,到了那时,市长会接到一份录音文件——某发某号冒号自即日起逗号某同志括号起女括号终不再担任你秘书和夫人句号她括号起女字旁括号终的工作由某某某接替句号完句号。然后她就拿一份红头文件来找我,说道:王二,咱俩复婚了。你在这文件上画个圈。此时我就会问:往哪儿画?而且画出个锯齿形的阿米巴。考虑到现在画二十厘米以下的圈不用圆规,实在难以想象但这是真的,假如我得了数盲症的话。这一切都明蝗白白,不明白的只有是谁来安排这些。我前妻说:我们呗。说着挺起了乳房,但是假如我得了数盲症,就会看不出她挺的是乳房。数盲在这方面表现极差,据说只会说一句话:今天机关布置和家属过夫妻生活,你安排一下。你给他安排了,他又分不出前胸和后背。 

有关夫妻生活的故事,我是知道的。据说数盲都是这样进行的:看着女人的肉体,傻头傻脑地说一句:“夫妻生活要重视呀”,然后流一点口水就开始干了;一边干,一边还要说些“不会休息就不会工作”之类的中外格言。女方一致认为,那些中外格言全是老生常谈,她们管这件事叫做“被肚皮拱了一下”。我的问题是没有能拱人的肚皮,肚脐眼倒是凸出的,但是那一点东西太小了。我的骨头架子很大,但是人太瘦了。我前妻的话不是认真说的,而是想挑逗我。据说尚不是数盲的人一想到未来,就会性欲勃发,而得了数盲症的人不管你说些什么都不勃发。谁都知道,我举得数盲症,要是能得早得了。但我也不是那么容易挑逗的——我已经四十八岁了。到了这个岁数,人不得不一本正经。 

有关拿肚皮拱人的事,还有些补充的地方。我们都知道,在二十一世纪,最具危险性的是信息。做爱这件一口除了纯生物的成分,就是交流些信息。爱抚之类全是堕落的信息,带有危险性。中外格言则是些好的信息,但对勃发没有助益。好在他们的肚子不管勃发不勃发,老是挺着的。 

我前妻对我说,你又吓坏了?因为这时说服工作(马上就要谈到,不是针对我的)也不管用了。自从要了一回汽油,我们就和好了,她天天都要来。这是候我们都赤身裸体,躺在我家的地毯上。我告诉她,我不再是年轻人了,不能要求得那么多。事实却不是这样的,我想起了红毛衣就魂不守舍。那个小姑娘清纯俏丽,乳房紧凑,最主要的是傻乎乎的,一勾就能上手。从一个方面说,年轻人属于年轻人,不属于我。从另一个方面说,我觉得我是个傻瓜。像这样的事决不能告诉我前妻,否则她会敲着我的脑袋说:送上门来的都不搞!你真是不可救药了! 

我不可救药了,这一点领导上早就知道。主要的问题是谁是领导。一方面,领导是一些全秃或半秃顶的大肚子数盲,负责人报告和接见外宾,这些人谁都不知道我。另一方面,领导是一些女秘书,负责接电话、批计划,这些人都知道我。因为每天都要打交道。今天早上我给省物资处摇电话,催问我们的铸铁和铜材,摇着了一个陌生的女秘书。我马上自报家门:我是北戴河王二,眼看过年了,今年的铸铁怎么还没到?对方应声答道:知道你!你是寂寞,是乡愁,是忧郁的老大哥……这就发生了一件常常发生的事,给上级机关的电话,必须忍受调戏。她说的那些鬼话和我的照片都登在这甚的妇女杂志上。假如你不顺着子说几句,以后永远另想和她谈铸铁问题。结果一扯就是一个半钟头,一直扯到“你还和老左好?真是不可救药”。为了工作,不得不做点牺牲。我主产:我正在考虑改变一下呢,告诉我你的三围好吗?电话就断了,再摇也摇不通了,真叫人恼米。我原准备谈完了三围,就谈铸铁哩。这是电话之一。另一个电话打给供应处,要绘图纸。一通了对方马上就说:上次告诉你的三围,记住了吗?你答:记住了——34、22、34.你是玛丽莲·梦露。快给我纸。这样答是不行的,对方勃然大怒:怎么?就这态度?纸没了!你必须像接色情电话那样哼哼着说:34啊啊22啊啊34,我的心肝梦露,人还记得我的一骂?这样就能得到合理的回答:记着呢。三箱子纸。你派某某来拿(某某是她的傍肩)。其实她对你一点意思也没有,这种调戏是因为她在首长身边工作,烦得要命,非说点带危险性的话不可。最怕一通了电话,是个男声:你哪里?一整天就泡上了。你决不敢挂,否则他叫公安局追查。然后就从纸的问题讲开去,咿咿啊啊说个不停。这叫做“被电话粘上了”,只能打手势叫人给你搬躺椅,躺下以后再叫人给你围上毯子,最后打手势叫他们把茶杯拿来,与此同时,嘴里应着“是的是的”。所有的女秘书都是满嘴胡说八道,因为在首长身边工作可不容易啊,连女人都被逼得要发疯。我前妻也疯得很。说实在的,近二十年,我没见过一个正常的人。 

今天是星期五,明天是星期六,后天就是星期天。有一句话最不该说,但我禁不住要把它说出来,我就是有这种毛病。星期六要去会老左。说出来以后,我前妻翻身就爬起来穿衣服,说道:你真让我恶心!我赶紧把她的外套压在身子底下,但她半裸着身子跳出屋子,扔下一句:留着你的外套送给鼻涕虫吧!然后外面就响起了汽车发动的声音。她是开着市长的丰田轿车来的,我的小摩托追也追不上,所以我根本就没去追。我只是躺在地毯上,和我前妻的外套以及无限的懊悔躺在一起。 

我爱我前妻,这种爱从她给我打开手铐那时开始从未改变。所以我几乎做到了平生不二色。我前妻也爱我,所以假如我被哪个女孩子勾引,一时糊涂犯了错误,我想她能原谅我。现在她还巴不得我犯这种错误,这说明我那种过于老实的天性已经有所改变。但事实上我是不能改变的。所以到了星期六下午,我着意地打扮了一下—— 修剪了胡子,脱下黑茄克,换上一件黑西服上衣,打上黑领带,带上一束纸做的花(现在根本找不到鲜花),骑车到市府小区的北门外面等着。天冷得很,穿得又单薄,等了十分钟,我就开始发抖。今天没有风,好处是不太冷,坏处是天上开始落烟炱。这种东西落到领子上你千万不要掸,而是要用气把它吹开,否则就会沾到衣服上,用任何溶剂都洗不掉。因为它是柴油不完全燃烧形成的碳,既不溶于任何溶剂,化学性质又无比稳定。除了往头上、领子上掉,它还会往毛孔和鼻孔里钻,使你咳出焦油似的黑痰。这种情景和我设计的蹩脚柴油机大有关系,所以使我两眼发直,考虑如何让它们不那么蹩脚的问题。有一个办法是在排气孔附近放些粘蝇纸,把烟炱粘住,但是粘蝇纸太贵了。还有一个办法是雇些农村孩子,手拿纱网,把烟炱都逮住。这亲是便宜,只是看起来有点古怪。就在这时,有人挽住了我的手臂,把我手上的纸花抢了过去,把我手背都抓破了。这个女人又瘦又高,手比我的手还大,而且永远不剪指甲,嗓音粗哑。虽我不想抱怨,但是她让我在寒风里等了十五分钟——这也太过分了。 

星期天我到碱场去看小孙和红毛衣,带去了铁百宝囊和大家捎的东西。一切都是老样子——一望无际的大碱滩,小铁道,还有人推的铁矿车。他们俩在单独一个地方,这也是老规矩。我们是政治犯、责任事故犯和刑事犯隔离。老远我就年见他们俩了,红毛衣在砸碱,小孙披着大衣蹲在地上。我一驶过去,他们债就换了位置红毛在后面吆喝,小孙在前面挥着十字镐,他脚上还带着大铁镣,足有二十公斤。这说明他们俩是傻瓜,把规定、定额等等还当回事。你要知剩,碱场的主要任务是折磨人,出多少碱无关紧要。不过一个星,他们债都瘦了,样子惨得很,但偏说是很幸福,还说碱滩上空气好——这就叫嘴硬。空气好是好,西北风的风力也不小。碱场发的大衣里全是再生毛,一点不挡风,我问他是不是饿惨了。红毛衣说饿点没什么。但是听说我带来了吃的东西,又非得马上看看可。后来我们在碱滩上野餐了一顿。我说小孙的镣太重了,红毛衣说都挑遍了,这是最轻的。于是我拿出一副假脚镣来。这东西是铝合金的,又轻以不磨脚,是技术部的无价宝——有一半人已经用过,另一半也会用到。我再三关照红毛衣,可别叫别人偷走了。还有假鞭子假警棍,看上去像真的,打着又不疼,我建议她常在大庭广众下修理小孙,这样显得立场坚定(其实是一种性游戏,但她现在体会不到)。还有一把手枪,和上级发的一模一样,只是轻飘飘的,但是同样的容易走火(这样不露破绽),只是打不列人。这亲她就可以立场坚定地用手枪以准小孙的胸膛。我问他们晚上冷不冷。红毛衣说两个人不冷,小孙又说也不暖和。我说我带的全是急用的东西,下礼拜小赵会来在他们的木棚里安上各种偷电的电器,那时家才有家的样子。红毛衣说:这儿是天堂嘛——不回去了。但我知道是过甚其辞。最后我给了小孙一大把特供的condom,——顺便说说,特供是指带有危险性,只有领导才能接触的东西,比方说,丙烷气打火机,只有领导用。我们用煤油的楼机,打一百下才能打着。数盲用钢刀子,我们用铁刀子。但是condom有什么危险 实在难以理解——他赶紧红着脸接过去。红毛衣问明了是什么,却很大方地吻了我一下,说:谢谢老大哥雪里送炭。然后把condom都收了去,说道:我掌握。这些日子他们都用国产工具凑合。那种东西是再生橡胶制的,像半截浇花的管子,有人叫它皮靴,这是指其厚,但是当鞋穿稍嫌薄了点。又有人叫它“穿甲弹”,这是指其硬,打坦克又嫌稍软。用以前要煮半小时,但是年轻人未必能等。假如他们不堪忍受,什么都不用,红毛衣就会怀孕。在碱场怀孕是一等一的丑闻,我作为老大哥,绝不能让这种事发生。 

现在我想到,condom的危险一定在于其物理性能,太薄太软,容易破;而穿甲弹就无这种危险。要不然就是因为戴上它感觉太好,使人喜欢多干,故而有害于健康;穿甲弹也无这种危险。从数盲一方想问题,总是乱糟糟。能避免还是不要这样想为好。 

我和我前妻在碱滩上服过两年刑,也用过穿甲弹。我不愿意这亲的事也发生在他们身上。这是因为我喜欢红毛衣,做梦总梦见她的裸体。学美术的人在这方面最具想象力。当然,想是想,真正干起来会有困难——就是和我前妻干也有困难。看着那些鲜嫩的肌肤、紧凑的乳房,我就会想到我已经老了,这不是我该干的事。百得面对老左那种又黑又皱的向躯体,才会勃起如坚铁。我前妻说我恶心,大概是指这一点吧。 



4 

星期六下午,老左早就看到了老大哥,但是别人还没看见呢。在这段时间里,她躲在暖暖和和的传达室里,看着那个大个子男人在寒风中里,手里拿着花站着等她,心里暖洋洋的。她说这是个动人的景象。但是在我看来一点也不动人。我倒希望看到她拿了花在街上等我,当希那个景象也不动人。更正确的说法是吓人,但是我不敢说。说出来以后她会更吓人。 

我们俩在小区里走,她用右手挽着我,用左手擦鼻子下边的清鼻涕。经过一番内心的痛苦挣扎,我把 手绢掏出来给了她,但是她给揣到兜里了。我并没说把手绢送给她,所以这是偷。手绢没有什么,有时她连我的内裤都偷。偷去以后给别的女人看,证明她也有傍肩。这件事使我沦为大家的笑柄。但这只是她很多不讨人喜欢的素质中最不重要的一种。王二认为,她最不讨人喜欢的素质是认为别人有的东西她都该有一份;而且她懒得要命,什么都不肯干。简言之,这种毛病就叫做等天上掉大饼,在等待时嘴里还不干不净。几年前她在技术部工作时,每天只管给自己织毛衣,并且骂所有的女人是骚货,男人都不是好东西。因为这个缘故,所有的人都不理她;于是她就服了三十片安眠药,打算自杀。因为是在班上服的药,所以大家不能坐视,就把她送进了医院,并且分班到医院去看护她,以防她再次自杀。等到轮到王二时,她对他说:老大哥,难道我真的那么不讨男人喜欢吗?在这种情况下他只能答道:才不是呢,你很可爱嘛。她主这样把王二搞到手里了。我现在一想自己过她可爱“,就要毛发直立,恨不得把自己阉掉。但是现在阉已经晚了。 

我实在想不出老左有什么用处:在技术部没有用,调到上级机关也没用。至今她还是个科员,没当上首长秘书,所以对部里一点贡献都没有。连首长都不要她——这说明首长对女人还有点鉴赏力——我就更不该要她。但是作为老大哥,我不能让她没人要。 

老左的套间里有一股馊味,她自己大概也能闻到,所以点上了卫生香。她的窗帘、沙发套、床单等等都是黑的——这对一个讨厌洗衣服的女人是人好主意。她进了一次卫生间,拿了一大卷卫生纸了崃,然后就干净利索地脱了衣服,钻进被子,在那里不断地撕纸,擦鼻涕。被子上面马上就堆满了。这个女人心情一紧张就流鼻涕,所以有鼻涕虫的外号。在身体方面,她还有很多奇异之处,其中包括体温只有35度,所以服安眠药那一回在医院里住了三个多月,直到大夫发现她的正常体温就是这样才出来。我从口袋里掏出香烟,她就说:你要是抽烟,就把窗子打开。所以我就把窗子打开。抽守一支烟,她又说,把窗子关上。我又把窗子关上,把小碟子里的烟蒂倒掉,洗净了盛子,就脱掉衣服上床去,和她做爱。做这件事从来没有发生过困难,虽然她丝毫不配合,只管擦鼻涕和提要求。除此之外,她还像僵尸一样硬,使我觉得自己像个奸尸犯——然后她忽然两眼一翻,尖叫起来。与此同时,邻居就敲暖气管。这是因为单身女秘书的房子建筑标准很低,一点不隔音。这也是因为她很想叫邻居知道她在干什么事情。这样等我回去时,邻居就在走廊上等着,对我说:老大哥你真行。我只好说:不是我行,是老左行。这种感觉一点都不好。 

人们说,领导有数盲症,老左有性盲症。我认为这种说法是对的。领导上不识数,但是做报告时总有大量的数目字和百分比——其实他根本不知是多少。老左不懂性,但她最喜欢谈论。在她开口说话前,先要流一会鼻涕——她心里一紧张就这样。然后说:我的傍肩王二,阳具伟岸。她的同事对打着哈哈,主把我的老底盘了出来:24公分长,直径40毫米——老左学过机械,会使卡钳——我要是知道她这么无聊,就绝不让她量。然后那些骚娘们就拿我寻开心。见了我就伸出右手做V字形,伸出左手,并齐四根手指做铁沙掌之势,合起来是24,就是我的尺寸。我要是说此时我不恨老左,就是伪善。 



等到事情一完,她伸手到床头柜上拿日历,找到两周后的星期六,在上面画个红圆圈。这说明我已经尽到了义务,可以回家去了。这件事我丝毫都不喜欢。但是到了画圈的日子我必须来。如果我不来,她就会服安眠药。她丝毫也不爱我,甚至丝毫也不喜欢性生活,但是动坚信女人每两周应该有一次性生活,因为报纸上是这么说。假如不过性生活就会早衰——顺便说说,我觉得她老一点更顺眼——为此需要一个傍肩。对此我没有不同意见,惟一的问题是,为什么非得是我呢? 



5 

我讲给小孙他们送东西的事,还有到老左那里的事,讲得七颠八倒。这说明我该要发数盲症了。数盲既不懂什么叫顺序,也没有时间观念,星期一上午听报告,报告人就是这样七颠八倒。其中还停下来几次问大家:今天的题目是什么?引起了哄堂大彩。大家鼓掌的时候,报告人站起来笑着点头,大概把我们笑什么也忘了。我坐在第一排,看到他一根接一根地吸万宝路,馋得要命。吸烟是我惟一的嗜好,咱们国产烟其实也很好,就是烟叶里什么都有,有时吸出螺丝钉,有时吸出电影票。有时候不起火,有时一声爆响,把头发全燎着——里面有黑色火药,烟厂的人也有幽默感。我前妻给了我一盒烟,同时劝我戒烟(她总是这样的)。我想,应该戒,健康要紧。所以我狠狠心送给小孙了。但是红毛衣马上该夺了去,说是抽烟时管我要。这个女孩子有控制人的品行,和我前妻一模一样。 

有关这个报告会,还有些要补充的地方。这个报告人原来是我们部里的,现在则是我们部长。他是正部长,这就意味着不再是我们的人了。他现在很白很胖,秃了的头顶又长出一层黄毛来。不仅头发是黄的,眉毛和睫毛全是黄的。不管你信不信,所有得了数盲症的人都要变成白种人——这是因为吃的好,穿的好,又不见阳光。而我们正在变成黑种人,假如我的贝宁同学现在送我木雕,底座上准写着:我们是黑人。这是因为我们喝的小里有苦咸味,这就是说,有大量的钙镁离子。钙镁离子到了体内会催化迈拉德反应——也就是造酱油的反应,这在速校里学过,以致大家肤色黝黑,像酱油一样。除了肤色黑,头发眉毛也打卷。这我就不知是为什么了。我们的体质太怪了,体内不光有酱油,还有苯、酚、萘、茚、茆、芘等等古怪的东西,含量都高,而且都能点着。所以死了以后到火葬厂非常好烧。他们说,我们进了炉子,给火就着。烧着烧着还会爆炸,这一点不好,但也炸不坏什么。烧出来的骨灰是造上等玻璃的好原料,因为骨灰里铅多钙少。这,是说,我们像上个世纪的猪一样,浑身是宝。这是因为上个世纪生产的全部铅酸电池都到了中国,不仅不要钱,还倒给些钱。同时到达的还有大量化工废料。数盲认为这很好,因为能挣外汇;而我们认为妈的逼非常不好,会把大家都害死(除了数盲,因为他们不接触这些东西)。数盲听了这样的汇报,就笑嘻嘻地说:有污染不怕,慢慢治理嘛。我操你妈,要是能治理,人家会大老远给你送来吗? 

除了白白净净,数盲还有件怪诞之处,死掉后极难烧,不管你怎么喷柴油,都是不起火光冒泡。你别看那么大的肚子,光是水没有油。这就是说,庞大的身躯像三岁的女孩那么嫩,大概是因为吃得太好吧。这种情况使火葬厂极头疼,因为只要死两个数盲,就能把全年的柴油都用掉。火葬厂的老大哥问计于我,我让他做台压榨机,先把水榨榨再烧,不知他照办了没有。 



我小的时候,我哥哥给我讲过他们插队的事。当时有一种情形和今天很相似,那就是与种负筛选的机制。我哥哥年轻时,每一个身心健康的年轻人都要下乡去插队,而有病的人却能得到照顾,在城里工厂工作。这两种处境有很大的差距,下乡的人吃不饱,穿不暖,而在城里就可以吃得很饱,穿得很暖。现在则是有数盲症的人可以做领导,在机关工作,得到“特供“商品;而没有数盲症的人必须做技术工作,待遇差不是大问题,真正的问题是要负各种责任。小孙砸碱去了,工业锅炉那一摊就没人敢接。我也收到一大堆群众来信,骂我的柴油机嗓声大效率低。领导上只管大方向,不问具体工作,所以也不负一点责任。我不知在这种情况下应该得到何种结论,反正我哥哥那时候的结论是装病。在这方面有很丰富多彩的知识。他产中间有些人给自己用了肾上腺素,就得了血压高的毛病;有人在胸部透视时在衣袋里放上撕碎的火柴盒上的磷皮,就得了肺结核。肝炎也能装出来,只要请一位真正的肝炎患者吃顿饭,然后让他替你到化验室抽血。其中最为简便的是装肾病,不冒任何风险,也不用请人,只要一个新鲜鸡蛋。在验尿时往尿样里滴几滴蛋白,就得了肾炎——当然,急性肾炎还要刺破指尖,往里滴几滴血。不过谁也不愿得急性的,怕被留下住院。后来领导上发现得肾炎的太多,就规定了必须在化验室里取尿样。但是知青们把蛋清事先抹在龟头上,也就解决了这个问题,陆续病退回城。事实上有病的人不能装成没病,没病的人要装有病谁也挡不住。 

但是这些知识对我没有用——我现在尿里就有两个加号,肝功能也不正常。我们部里人人都有点病(因为环境是那么的脏),所以不能照顾。只有数盲没有身体上的病 ——他们住的地方有干净的水、滤清了的空气。但是他们病得最厉害,连数都不识了,所以不能不照顾。这种情形真让人无话可讲。我现在要考虑的是让谁来做工业锅炉的设计——当然,最合适的人选是小徐。这小子是学化工的,有点靠谱。但是他绝不肯干。别人又都不在行。算来算去只能我接下来。但我一点也不懂锅炉,我只懂柴油机。现在谁想要锅炉,就会得到一台柴油机,用汽缸烧水,用废气烧蒸汽,而且还会嘣嘣响。可以想见下面那些需要锅炉的工厂——纸厂、印染厂等等,见了这种东西一定会气疯。但我也没办法。让他们去疯吧。 



6 

今天是星期一,我的生日过去四天了。在这四天里,发生了很多事情。现在我不能把它们全记下来,因为我的脑袋被打了一个大洞,脑子里昏昏沉沉——除此之外,夜也深了。所以把到今天早上以前的事做个总结就睡觉。我和我前妻和好,后来又把好气跑了。这件事(把她气跑)从表面看来是因为我和老左睡了觉,其实不是的。因为我完全可以不去和老左睡觉,所以真实的原因是我很违拗。我受不了她比我强。假如她听到这些话,就会说:王犯,我们又何必分出个彼此呢?我就会答道:是!管教。——做出个恭顺的样子。其实我想:凭什么我是王犯你是管教?

\section{三 蓝毛衣\&我前妻}

1 

有关老大哥王二这个人,还有好多需要补充的地方。这个人像白痴一样笨,像天才一样聪明,在这两方面都是无与伦比的。他设计过上百种柴油机,除了几种早期作品,都是莫名名状的怪物,这是他鲁钝的地方;但是每一种都能正常转动,这又是他天才的地方。他还设计过一种公共汽车。接到设计任务,他就去对数盲说:他刚刚参加了一个仿生学的学习班,仿生学是二十一世纪的技术,故而这辆公共汽车如果是普通外形的轮动车辆,就未免落伍。他要把它设计成步行机械,并且有某种动物的外形。数盲一听说二十一世纪的技术,登时表示支持。过了半年,一架生铁造成的老母猪就蹒跚走过大街,喷着浓烟,发出巨响,肚子底下悬着十几个假乳房,里面是乘客席。这辆公共汽车后来被日本人买了去,放到一个游乐场里了。这种奇妙的设计能力是年轻同事模仿的对象,但是谁都比不上。因为他不是存心要出洋相,他这个人本来就是这样。 

据他前妻说,王二的身体也有很多奇异之处,这其中就包括他的阳具。那东西总是懒洋洋的,和它主人那种勤奋的天性很不一样。要使它活跃起来,还得做一番说服工作。你对它说:同志,你振作起来!它就能直起身子。你对它说:立正!它就能直挺挺。在干那件事时,你说一声:同志,你走错了路。它还能改变方向。当然,最后还要对它说:稍息,解散。这个东西就被叫做二等兵王二,而那件事则被叫做出操。因为这些缘故,王二对女孩子来说很有魅力。但是这些事他自己一点都不知道。他一点也不觉得自己有什么与众不同的地方。 



不管数盲怎样看我,我觉得自己仍是个艺术家。作为艺术家必须要有幽默感,而幽默感有现个传统来源:宗教(在我们这里是数盲)和性器官。这是因为在中世纪,只有宗教和性在影响人的思维。由此产生了一些笑话,比方说,领工资时,拿到了那些微不足道的钱,就闭上眼睛说:我要是数盲多好。但是这个笑话于点都不逗,因为数盲不领工资,人家是供给制——换言之,共产主义对他们早就实现了。还有一个笑话说,我得了数盲症以后,每天都要洗澡,还要抽十支万宝路。这个笑话比较短,因为数盲不知道每天洗澡,要到你安排了才洗,抽烟也根本没数。 

有关共产主义,也是个很有意思的问题。教科书上说,到了那时各尽所能各取所需,连数都不用数。根据这个道理,那时候的人就该都是数盲。假如不是从小数钱、数冰棍,谁会识数。但是到那时都不识数了,谁来算题?假如没人算题,就没科学技术,又怎能各取所需?对这个问题我有个天才的答案:到了共产主义也会有人犯错误。对于有错误的人,就不让他各取所需。然后他就会识数。然后就可以让他算题。这只是个笑话,不能当真。因为不识数的人不可能犯错误,错误就是识数,由此堕入了循环定义。 

我做梦都想患数盲症,就像我哥哥当年下乡时做梦都想患一种重病一样。假如我成了数盲,就能躲开柴油机,重新获得我的雕刻刀、画室、彩色毛线等等,要知道我天生就是这么一块料。我哥哥当年想得一种病,则是因为他在乡下吃不饱——要知道他天和是一个饭桶,粗茶淡饭吃多少都不饱,非吃肉不可。我现在就落到了他当年的困境里。我们哥俩都只有一种方法来脱困,就是真的得上这种病。他的病是夏天睡潮地、大冬天只穿运动短裤得上的,虽然有往龟头上抹蛋清等等绝妙的手段,他却不敢尝试。所以他就得了风湿性关节炎,一辈子都好不了,现在住在得克萨斯的沙漠里。而我则只能朝数盲的方向努力改造自己。凭良心说,我一点不想争当数盲,只要能做原来的工作就完全满意了。这一点数盲一定能知道。我个人以为,一个人设计的公共汽车是一口老母猪,足以说明他已经无可救药,不一定非要主他完全不识数。但是我也知道,什么人是无可救药,什么人不是,只有数盲才知道。 

我说我做梦都想得数盲症,但是梦醒后会为这些梦感到羞愧。假如我们都得了数盲症,一切都要完蛋。老人们怎么办?孩子们怎么办?他们要饿饭了,至于我们自己一,也就是中年男人们,倒是不值得同情。因为我们都有数盲症,没饭吃,可以吃鸡鸭鱼肉毒蛇王八。女人们又怎么办?假如所有的男人都浑浑噩噩,世界上就会没有爱情,她们怎么活呀。但是我们自己又没问题——我们按组织上的安排和家属过家庭生活就够了。 

我和我前妻是在速校认识的,速校是一片雪地上三座小楼房。其实那不是雪,而是一片盐碱地。当时的土地盐碱化已经很严重了。楼房前面有几棵杨树,所有的叶子全都卷着。当时的污染也已经很严重了。我在班上又是老大哥(班长),上课时坐在第一排。第一课是扫盲课,我们都是科盲。老师进来我喊起立,发现她是我所见过的最漂亮的女孩子,但是穿了一件极难看的列宁服。所以坐下之后就举手发言道:报告老师,你的衣服很难看——我给你打件毛衣吧。那时候她工学院还没毕业,在速校实习,一看学生都有胡子,心里已经发慌,我的发言又有调戏之嫌,登时面红耳赤。后来她就专拣我来提问,比方说:在黑板上画个根号,问道:老大哥,你看它像个什么?我看了半天,它像个有电的警告符号,故而答道:伸手就死,老师!她又画个积分号,这回不用她问,我就说:这像一泡屎!在她看来,我像个存心捣蛋的混蛋(其实我不是的,不管什么时候我都很真诚)同时我又是她生命里的第一个男人。她决心迎接这种挑战。 

礼拜一早上,接到我前妻的电话。她先问老左床上如何——这话一早上听了十遍了,我听了着实恼火,吼了起来:你们不要这样墙倒众人推!老左怎么了?再怎么她还有点同情心!(其实她是没有的,否则就不会主我摸她那干瘪的乳房,那东西像抹布一样,能够摸透,握在手里成一束,虎口以上溢出掊分还算有点模样)……我前妻听了以后,叹了口气说:是嘛,我没同情心——告诉你,你的事有希望了。这几天你自己当点心。我听了面红耳赤,因为我一直在托她给我办出国手续。这件事难于上青天,但她居然办出了眉目。我觍着脸问,是怎么个情形?她说,电话里不能讲,下班她过来。但是下了班她过来,我既不在家,也不在部里,我坐在个小黑屋里,脑袋上满是血。 



2 

对于一个识数的人来说,自己存在是惟一确定无疑的事。这可以叫做实事求是,。可以叫做无可奈何。假如肯定了有自己,就能肯定还有一个叫做世界的东西,你得和它打交道。承认了这些事,就承认了有所谓无可奈何。你识数,这就是无可奈何。有的声音好听,有的声音不好听;有的东西好看,有的东西不好看;这些都不能随心所欲。因为你是如此的明白,只好无可奈何地去上班干你该干的事。但假如你不明白的话,就可以随心所欲。一般人到了这种境地,就能想到当个领导,但我有另外的主意。我想去美国,和我哥哥、嫂子、我年过八旬的母亲生活在一起。除此之外,还想弄个画室重操旧业。我哥哥隔段时间就托人带一份文件,让我办出国手续。但这是不可能的事:技术人员了崭,因公因私都不可能。我哥哥在电话里说:你干吗非识数不可?这是一种暗示——他一定记得好多年有给我讲过知青装病的事,所以知道我能听懂。但是,现在你也知道了,数盲这种病不能装,只能真的去得。而真的去得这种病,我还下不了决心。 

有关不准技术人员出国的事,还有一些需要补充的地方。前几年还是让我们出国的,但是大家出去了就不回来,简直无一例外。现在的规定是出国前要体检,没有数盲症的男性一概禁止出国。但这是内部规定,明明是没得数盲症,体检证上偏写成三期梅毒,不但出不了国,还要被关进医青霉素。那种青霉素是进口的,却是兽用药,杂质很多,打在屁股上浑身都疼,而且发高烧。自从打过了那种针,我就老有点黄疸。因为这个原故,我再也不敢打这种主意。患了数盲症的领导可以出国访问,这方面大家都服气,人家没有不回来的。这也说明数盲在外国也治不好,得吃救济——外国人抠得很,不肯救济我们的人。女人可以出国,内部也有掌握——年轻漂亮的不成。洋鬼子精着哪,见了年轻漂亮的就娶去做老婆。老左就出过国,但是大家都服气,因为她回来了,并且在床上对我说:还是祖国好。这个女人觉悟高,明明是我对她好,她却记在祖国账上,让人没话讲。我前妻也可以出国,但是要到六十岁以后。不管怎么说,她总是有个盼头,我却是一点盼头也没有。 

我前妻说,我有张卑鄙的嘴,这蝓身上下最恶劣的东西。好在还有一件好东本,那就是二等兵王二。她帮我的忙,全是看它的面子。但这话打击不了我。别人有困难都去求傍肩,傍肩也帮助,你说是看谁的面子?只是没有求帮出国的,这事太难。我前妻办出了眉目,不知是怎么办的。这件事她始终不告诉我,后来这事失败了,她也不说当初的眉目是什么? 

现在可以说说“眉目”是怎么没的。接完了这个电话我就去听报告。要是推个事不去,就好了。“数盲症可不是装的”——报告人又一次引起哄堂大笑时,小徐对我说:装得真像!我就这样回答他。假如不理他就好了。就在这时,在我们身后巡逻的保安员用警棍在我脑袋上敲了一下,引起了短暂的昏迷。这些农村来的小伙子工作很认真,但是下手不知轻重。他们看到我们老笑,已经很气愤了——会场秩序不好要扣他们薪水。小徐也挨了一下,不肯吃哑巴亏,回头就和他们打了起来,登时演成群殴的场面。他们手里有警棍,我们身上也有东西,有的是铁链子,有的是半截水管子,有的是发射橡皮棍的气动手枪,有的是喷射阿摩尼亚的气罐——听大报告时大家都有准备,而且我们的人也不少,除了各机关的技术人员,大企业的人都来了。坐在我们边上的是玻璃公司,那帮家伙对打群架兴趣极大,早就把板凳腿拆下来了。一动手就有人递给我一根板凳腿,我也瞎挥了几下,打倒了几个保安员,自己也挨了几下警棍——年纪大了,身手不灵活——而会计部的小姑娘则是假装劝架时朝保安员的裆下施以偷袭。转瞬之间,就把保安员打得落花流水,大家溃退而出,一哄而散。当然,也得有几条好汉留下来顶缸,否则会有大麻烦。今天的事是因我而起,我留下来。等保安的大队人马来了后,我就带头扔下板凳腿,举手投降。人家看我血流满面,也不好意思再打我。别的投降者,不是真伤员,就是体质单薄者,还在脸上涂了红药水。这正是我们的狡滑处,你要是审问,就说:什么都没干,只是挨了打。所以人家问都不问,直接押去关小号,半平米的地方塞两人人,是聊大天的好地方。我和一个穿黑茄克的小伙子塞在一起,我看他很面熟,。进去以后才知道,是那个穿蓝毛衣的姑娘。等我前妻来放我时,她正坐在我腿上,但这是因为没地方坐。那孩子连忙解释说:大姐,我们是清白的,信不信由你。而我前妻摸了她脸一把说:当然是清白的,可怜有小家伙——快点回去睡觉吧! 

考虑到礼拜一的群架里有人伤得很重,还破了相,想让保安把我放了可不容易。这件事要劳动市长亲自打电话:“你们那时有个王二,是我家属的前夫,如果没什么严重问题就放了吧。”除此之外还有好多治安方面的指示,把保安的头烦得要死。他来开锁时还念念叨叨:什么叫“家属的前夫”。我要承认,这种关系实在古怪。但这还是直截了当的说法,还有人是某数盲的“家属的前小叔子的哥哥”,有人是“小姨子的前姐夫”,不得数盲也搞不清楚。不过这无关紧要,数词上只要知道是和自己有关系就够了。具体是什么,人家并不想弄清楚。对于我们来说,这种关系很明白,我们是绿帽子的发放者,他们是绿帽子的接受者。好多人认为这种暧昧的关系,有助于和傍肩间性生活的和谐。我个人不这样想。因为这个缘故,我前妻说我笨。 

我前妻把我放出后,就朝我冷笑。她看我愣愣怔怔的样子,就递给我一面小镜子——那样子很难看,我早知道头破了,但不知流了那么多血。但我还能挺住。她说,你那件事吹了。我听了就晃起来,幸亏她从我兜里摸出了救心丹,塞在我嘴里。后来子带我到医院去处理伤口,出来时更难看了——剃了个阴阳头。我一直觉得昏昏沉沉,回到家就睡了。躺下时,我前妻睡在我身边,醒来时天已大亮,我身上有张纸条,上面写着:1.接着睡;2.今后少惹事,还有希望。希望是指出国的事,我知道原来的希望是打架打没的。我就接着睡了。 



有关保安的情况,需要补充如下:那些人在现在这样的天气里穿着蓝色的棉大衣,戴着藤帽,手持木棍,戴红色袖标,在街上维持秩序。上级说,现在城市治安混乱,警力不够用了,从农村征调保安员进城,是个好办法。但是这帮人来了以后,秩序就更加糟糕,因为他们上了班什么都不管,下班以的什么都偷。除此之外,他们最感兴起的事就是揍我们——当然,我们也不是那么无辜。你要以为北戴河是新兴科技城市,大家都是知识分子,故而只有挨打的份,那就太天真了。我们挨揍多年,早就懂得怎么还手了。 

而我和蓝毛衣的事是这样的:小号里面像个电话亭,架着一块木板,可以坐一个人,另一个只能站着。保安的头总问,要不要单间。我说,你给我个人做伴吧。这时候黑皮茄克就钻了过来,站在我身边。保安把我们塞了进去,隔着门和我说了会儿话先说他很公道,是他的人先动手打我,这是他们的不对,明天就的根那小子回家种地。我说你用不着和那孩子为难,等等。他说这事你不用管,打了别人我不管,可不能打你,什么时候都得敬老 ——我没理他,知道自己在外人看来已经老了,没有什么如感觉。后来他又说,你们的人用了手扣子,把我的人脸打坏了,你看怎么办。——这是真的,我看见他们的人有脸上受伤的。回去以后要说说:打架不准用利器。但是不能嘴软——我说你公事公办呗,我们都在人里。送我们去砸碱好了,我们又不是没砸过。——我知道他想让我帮他把使手扣子的找出来,但是我不能这么干。任何时候都不能把自己人交了崃。我还说:我脑袋也被的破了,这也得有个说法。他说,送人砸碱是公安的事,但是告诉你的人小心点,别再东到我们手里吧。这就是说,谁要是东了单被他们逮住,就会被打得稀烂。我说,我会告诉大家的,不过你们也要小心点,有人知道人都住在承德棒槌山,全村出来干保安,家里只有老人孩子,别以为我们找不到——我这是唬人,其实我们远没有那么坏。他就悻悻地走了。 

这时我才觉得头疼,还有骑在我腿上的这家伙不对劲。那里像地狱一样黑,但是气味不大对。他拉着我的手往皮茄克底下伸时,我以为他是个homo。知道他光板穿着皮茄克时,我说了一句:你不冷吗?后来手伸到胸前,摸着两个圆滚滚的东西,我才大吃一惊:这是什么?你怎么长了这种东西?她吃吃地笑,我听出是蓝毛衣,马上关照她不要高声。一个女孩子到了这里是很危险的。保安员可不是些太监。后来她又拿一个冷冰冰的东西让我摸——是个带锯齿的手扣子。原来就是她用了手扣子!这下把我气坏了,骂道:混账!谁叫你整这东西!她轻描淡写地说:怕啥。我说:你是不怕,今后谁落到保安手里,怕也没用了。她说:哪个乡巴佬敢犯坏,咱们就到村里去抄他的老窝,烧他的房子,这不是你的主意吗。——听着真可怕。这一位可不您红毛衣,不是纯情少女,伸手就拉我的裤子拉锁。我说:学校里就教了你这个?她就说,老生常变。老大哥,你太老派。后来她又说,有一种传闻,说我是个gay,看来是真的。我说放屁,我要不是后脑勺正在流血,准能表现出男儿本色。后来她拿手绢给我捂着伤口,就这样聊起天来,直到我前妻知道了消息,赶来把我们都放出来。她把我腿都坐麻了,半天不能走路。要是个男的,还可以轮轮班。下回关小号可不能挑女的。昨天的事就是这样。 



3 

有关和保安员打架的事,还有些可以补充的地方。从某种意义上讲,我们和保安员都是诚实的人,都在尽自己的本分。我们在诚实地劳动:设计各种东西;他们也在诚实地劳动——监视我们。我们觉得他们的监视十足可恨,他们觉得我们不老实十足可恨,所以就经常打架。结果是双方都常有人受伤住院。数盲十分公允地决定:不管谁受了伤,不能报销医药费,不能上班算旷工,结果是越打越厉害。这一回保安有好几个人被打断了鼻梁,他们肯定不甘心,想要从我们身上捞回来。作为老大哥,我要时时刻刻提防在心。假如蓝毛衣是男的,我会毫不客气地揍他一顿。但是对女孩子不能这样办。再说,她不归我管。她在我们这里是客人。 

在聊天的时候,有人说假如没有保安就好了。世界上只剩下了三种人:我们、数盲、傍肩,生活会愉快得多——我们干我们的工作,数盲发他们的昏,傍肩居间调和。这种建议当然是居心叵测——没有保安,我们会把数盲都吃下去,连骨头渣都不剩。如果把傍肩们划掉,那就不成个世界。如果世界上没有数盲,我们就会和保安爆发战争——要知道他们恨的就是我们。这场战争胜负难以预料,我们狡滑,会制造各种武器,保安人多,他们在村里有大量的预备队。就算我们获胜,中国人口也是百不存一。算来算去,只有我们可以划去。勾去我们,顶多中国倒回中世纪。那时的技术水平可以养活三亿人——这也不可怕,饿死一些就是了。 

我秃着脑袋去上班时,别人问我是不是和蓝毛衣出过操。我想说没有,但是蓝毛衣面红耳赤地看产我,露出一点乞求的样子——这就是说,她已经夸下了海口,说和我出操了。但我又不会扯谎,于是就说:这种事可是讲得的吗?大伙就起哄,让我请大家吃雪花梨。我出了钱,蓝毛衣就去买了半筐来。今年的雪花梨可真怪,有苯酚味,吃起来像药皂。人吃下大量的苯酚会有什么结果,是个极复杂的医学问题。我现在知道的只是我打嗝是股药皂味。后来我偷偷问蓝毛衣,是不是真想和我出操,她说其实并不想,只不过和别人打了赌。好还说,我太老了,恐怕满足不了她。现在的女孩子越来越坏了,不但拿我打赌,还要打击我的自尊心。 

后来我和我前妻说起这件事,她说我是个笨蛋,人家不是这个意思。假如她是我的话,就会说:那也不一定——这是针对蓝毛衣的“满足不了她”说的。这样就能听到更多挑逗性的话。我听了这些话,就开始乱琢磨起来。忽然之间,听见我前妻厉声喝道:混账东西,站好了!没让你稍息!听了这话,我马上就要站起来,但是她扯着我说:别乱动——没说你。我又老老实实地趴着不动,她又掐我:混账东本,动起来,这回是说你。你们简直要气死我。事情完了她想起这件事,笑得打滚,还说我装起傻来像真的一样。我说我没装傻,她就开始不高兴,说,再装就不逗了。最后我只好违心地承认自己在装傻。这也是出于十年来的积习。 

我说现在的女孩子越来越坏,是认真说的。过去的女孩子,比方说,我前妻,有很重的责任心。当我们犯下错误去砸碱时,她们当管教,我们不砸碱时,她们调到上级单位当秘书,不管干什么,都是为了庇护我们。假如她们不庇护,我们就都会完蛋。她们从来不参与打架。而现在的女孩子就不然,她们对生活的理解就是傍肩和打架,所以不能帮忙只能捣乱。但是也不能一概而论,还有像红毛衣那样比较好的孩子,现在对我们有用。将来就更有用。 



我前妻还说,她一直盼着我再犯下砸碱的罪过——到那时她就扔下市长秘书不干,再当一回管教。说实在的,我对那件事从来就不喜欢。在碱场里她问我:王犯,喜欢不喜欢砸碱?我就得答道:报告管教,喜欢!国家需要碱! 

当年我去砸碱时,我前妻把我押到木棚里,然后命令道:现在,和我做爱。因为她路上差一点把我打死,我犹豫起来,过了一会才答道:报告管教,犯人王二正在服刑!坚决服从命令!就朝她猛扑过去,但是劳而无功。这原因我已经说过,路上吓得着实不轻。她摸着我的阳具,说道:可怜的小家伙,吓坏了。也不知为什么,那东西弹动了一下。她嗖的一下坐起来,说:这家伙懂人话!我也嗖的一下坐了起来,说道:你别拿我寻开心了——士可杀不可辱!她板着脸一指手提包(我们拿它当枕头用),说道:躺下!不然我给你上铐子!我只好老老实实躺着,让她对它轻声细语。过了一会,那东西就精神抖擞挺在那里,她又躺下来说道:开始吧——它比你乖。你当然能够明白,这件事使我感到很难堪。它是我的东西,却听别人的命令,是个叛徒和奸细。以后发生的一哼更让我难堪,每天下工回了棚子,她就说:脱裤子,我要和它说会话。人不准偷听。我躺在那里,又冷,又寂寞。但有什么办法——她是管教吗。 

老大哥王二在碱场是模范犯人,这个荣誉称号很有分量。这说明他在思想改造、劳动、服从管教方面取得了很大成就。假如数盲必须信任一个非数盲的男人,而候选的人里有一个先进生产者、一个模范设计师,还有王二,他就是首要的人选。理由是明摆着的:先进生产者、模范设计师都可能是假的,模范犯人总是货真价实。他肯定能经住考验,因为所有的模范犯人都曾自愿放弃减刑。当年狱领导来问我:王犯,想不想早回家?我就答道:不想,国家需要碱。但这不说明我觉悟高,而是我前妻事先告诉我,这是个圈套,要求减刑的一律加刑。领导上问我:王犯,我们认为你的案子可能判错了,你写个申诉吧。我就答道:我申请加刑——我要为国家的碱业贡献青春!这也是我前妻教我的。结果就被减了刑说实在的,一开头我不大敢听她的,我怕她万一搞错,真被加了刑——国家真的需要碱。但是她又说,加刑怕啥,不还有我陪你吗;与此同时,圆睁杏眼,露出要发火的样子,我就不敢和她争,只敢服从。如其不然,就会被罚,天不亮时手执木棍,到广场上走正步,高唱各国国歌。二百多首国歌可不那么容易记住。走着走着——“报告管教,忘了词!”“就地趴下,五个俯卧撑!”或者是:“王犯,先去喝口胖大海——我对你怎么样?”“报告管教,恩比天高,情比海深!”“知道就好!从马赛曲接着唱吧。”她的心真狠,我都唱到了“上帝保佑女皇”(U.K.),又让折回去唱法国国歌——我们是按字母顺序。最后各国国歌都被我唱成了一个调,和数盲唱得差不多了。我前妻说,只要你事事听我的,就能得数盲症。我估计是真的,但是我不肯听她的,起码是出了碱场就不肯。这是因为在恭顺的外貌下,我还有一颗男儿的心。 

等我被放出来以后,我们就结了婚。我们的事迹上了报纸的头版。报道的题目是:女管教和男犯人——一条成功的经验。我老婆文章的题目是:心慈手狠——改造王二经验谈。我文章的题目是:为国家服一辈子刑,砸一辈子碱。又过了一阵子,我们俩就离了婚。除了别的原因(老左),还有一个原因是她老把我当两个人,使我险些精神分裂。 



4 

我从碱场回到技术部工作时,被我前妻和教得甚好,早上一到班,就跑到部长面前报告:报告管教,犯人王二身体良好,今天早上尚未大便!假如是我前妻,就会答道:稍息!先去大便,回来上镣。发现痔疮,及时报告。我答道:是!就跑去蹲茅坑。但是部长不这么回答,在全体同事的哄笔中,他扭扭捏捏地说:老大哥,对我有意见,可以单独谈,别出洋相。我说:是!可以去大便吗?他却不理我,扭头就跑。这套仪式就进行不下去了。你要知道,在释放的仪式上各级领导都说,要我们把碱场的好思想好作风带回原单位发扬光大。不知为什么,回来就行不通。部长还一再托人和我说:过去的事是他不对。要知道,就是他把我送去砸碱的。那时候他还没有数盲症,听我报告怪不好意思的。等到他得了数盲症,就不是这样了,听着报告就会笑眯眯地说:身体好就好呀!按时大便也很重要——同志们都要重视这个问题,当然还有别的问题——就这样一点两点说下去,不扯到天黑不算完。到了这个时候,我再也不敢找他汇报,躲他还来不及。这主要是因为我真的要大便,不能老陪着他。总而言之,拿这一套对付他是不行的。 

我前妻听我报告时,常常忽然用手遮住嘴,额头上暴起青筋——那就是她憋不住笑了。报告完了,她押我去砸碱。到了地方,我挥起十字镐来。我喜欢砸碱。砸着砸着,忽然她厉声喝道:够了,省点劲别人来时再用。开了镣,陪我走走。我就打开脚镣联她溜达,走到一个土丘上,只听到她长叹一声:天苍苍野茫茫呀!我连忙答道:是!管教!她嗔怪地说:老大哥!现在边上没人吗!我低下头去,过一会才说:报告管教,我脑子里只有一根筋,你最好别把我搞糊涂。他伸出小手来,拍拍我的脸,说道:我是不是对你太狠了?你是不是记恨了?这一瞬间我身体都有了反应——换言之,这时用她对那东西悄声细语,也能干成。我心里觉得有些委屈,想和她说说话——比方说,我原是个很有前途的艺术家,名字都上了若干艺术殿堂的收藏名录,怎么搞到了这个样子,靠女人庇护,等等;但是没等我开始说,她就转过脸去,说道:天苍苍野茫茫呀,王犯,你有何看法?我只好答道:是,管教!如果能风吹草低见牛羊就好了。她说:王犯,牛羊能让你想起什么?我就答道:诗曰,马牛其风,和教。她说:大天白日的,咱们俩总不好真像牛羊一样吧。我就答道:报告管教,我看见那边有辆废矿车。她说:很好,王犯,你很能领会领导意图。咱们就到那里去。开步走,一二一!一二一!我很爱我前妻,但是始终没有爱成。她也很爱我,但也没爱成。我们俩之间始终有堵墙。 

把时光推到我初做技术工作时,我三十刚出头,英俊潇洒。那时候我前妻就看上了我,但是我却看上她。说实在的,我谁也看不上,心里想的只是我是个艺术家。那个时候搞技术的艺术家很少,别人都是些退休返聘的老家伙,我在女实习生那里极红,所以狂妄之极,朝秦暮楚,害得她几乎自杀。这种事当然应该遭到报应,所以她就押我去砸碱。等到我报应遭够了,她要和我认真谈谈时,我已经改不过口了——“是,管教!”假如你有个丈夫是这样的,也会觉得离婚较好。另外一方面,虽然我前妻的身体很美丽,但是和她干的感觉还没有和老左好,所以我也想离婚。这件事总的来说是命中注定。有一件事也是命里注定:我这一辈子谁也不佩服,包括毕加索(艺术家都不肯佩服别人),只佩服我们部长(工程师必须佩服比自己强的人)。这家伙简直什么都会,声光电热、有机无机高分子,加上全部数学,虽然他是个混蛋。等我砸碱归来时,他的样子很悲惨,得了溃疡病,只有九十多斤。这是因为他的事业全都失败了,大规模集成电路厂成品率为零,化工厂天天爆炸,电厂一送电就会电死人。一切和我预料的一样,他的高技术路线不符合国情。所以他找我谈话:老剞,以前是我的错,咱们合作吧——重新来过!但是我却向他报告说,要大便。当然,我也可以报告说,可以,咱们合作。以他的能力,加我的经验,事情会有改观,但我觉得不到火候。结果是他顶不住,傻掉了,现在胃病好了,变成了个大胖子。我却成了技术部的实际负责人,顶他的差事,这种事就叫命里定。刚出碱场时,我是一条黑大汉(窝窝头养人),现在瘦得很,也得了溃疡病,一天到晚盼着傻掉。现在的问题是,没有我前妻的指导,想傻也傻不掉。 



5 

我前妻说过,想要有前途,就得表现好。“表现好”这句话我是懂的,就是要坦白交待。头上的伤刚刚不疼,我就把她约到家里来,开始坦白。从写条子的事开始,蓝毛衣比我写的条子是中文,内容挺下流,用了一个“玩”字,还用在了自己身上。然后又交待在小黑屋里的事:那孩子才是真有恋父情结,她说她喜欢老一点的,有胡子更好。口臭都不反对,只是要用胶纸把嘴粘上。但是绝不能是数盲。老、嘴臭的人有的是,但全是数盲。所以就不好找。我一时色迷心窍,说了些挑逗的话,什么自己比数盲强点有限,等等。她说她是想做爱,又不是想解数学题,只是要点气氛。我就说等出去好好聊聊,我搞过舞美,会做气氛等等。其实她要是真找上门,我还得躲出去,当时无非是胡扯八道,以度长夜罢了。 

我坦白了之后,我前妻冷笑一声说:我以前说你浑身最坏的部分是嘴,现在知道错了。你最坏的部分是良心。我说:是管教!她说,是什么呀你,是。人家小姑娘的话,你怎么能告诉我?我听了直发愣,觉得自己是坏了良心。她又说,你自己想想吧,为什么和我说这个。我说,是想让你帮助我。她瞪起眼来说,你真浑!我什么时候不帮你?一边骂一边哭。我赶紧找块干净手绢给她。等哭够了,她才说,可怜的家伙,你是真急了——要不然也不这样。你不是这样的。 

我到底是怎样的,她根本就不知道。连我自己都不知道。除了不能发数盲症,我什么卑鄙的事都能干出来,因为我已经受够了。我讨厌爱,讨厌关怀,讨厌女人的幽默感。假如生活里还有别的,这些东西并不坏。但是只有这些,就真让人受不了。我不是工程师,我是艺术家。难道我生出来就是为了当一辈子的老大哥吗? 

我现在在日记里坦白我的卑鄙思想,而这本日记除非我死了,她绝看不到。在表面上,我是个善良、坦白、责任心强的人。其实不是的,我很卑鄙。昨天晚上我前妻在我这里睡,等她睡了之后,我爬志来看她的裸体。她的裸体绝美,作为一个学美术的人,对女人的身体不会大惊小怪,我再说一遍,她的裸体绝美。她真正具有危险性。只要她睡着,我看到她的裸体,就会勃起,欲念丛生,但她永远看不到。我不相信有什么男人可以抵挡她的魅力,哪怕他是数盲。所以她一定能把我送出国去。我一定要让她做到这一点,而且我自己还不肯得数盲症。这是因为我恨透了她——她把我撇下,去嫁了个数盲——但是恨透了首先是因为我爱她爱得要死。这一点她也永远休想知道。

\section{四 Party}

1 

星期四早上我在图板上画一台柴油机,画着画着把笔一摔,吼道:不干了!开party!于是惹出一场大祸来。这件事告诉我,每个人都可以从正反两面来看。从正面来理解,我是个小人物,连柴油机都画不好,简直屁都不是;从反面来理解,我可以惹出一场大祸,把北戴河的本山变成剥落坑(出自《浮士德》),招来好几万人在上面又唱又舞并且乱搞——顺便说一句,“乱搞”使不止一位数盲得了心脏病死掉,我把火葬厂的老大哥害惨了——这说明我是摩非斯特菲里斯,在世界七大魔鬼中名列第四。我倒想知道一下,其余六位藏在哪里。这个两面性使这篇日记相当难写,我还是像数盲做报告,先正面后反面,然后回到正面上去。顺便说一句,数盲做报告时,眼前有个提示器,上面有两盏灯,一会闪绿,这就是说不能我讲正面的,也要谈点反面的;一会闪红,这就是说要以正面为主,负面不要说得太多。提示器还显示讲稿,但是数盲决不照念——嫌它太短。我没有这种东西,反面很可能会谈得过多。先说我没画完的柴油机,这是个大家伙,是矿山抽水用的;既不能画成狮子,也不能画成鲤鱼,而是要正经八百地画,因为这东西坏了就会把井下的矿工都淹死。我把它画成方头方脑的样子,十二缸V形,马力够了,看来不会有什么问题。假如我把它画完了,世界上就会再多一个嘣嘣乱响的蠢东西。假如给它纯粹的烷烃,就能发出八百匹马力,虽然它是球墨铸铁做的,也能长久地工作。但是给它的是水面上捞起来的废油,所以连二百马力都不会有,而且肯定老坏。所以它还有一桩奇异之处,配有一个锅炉,假如柴油机坏了,烧起火来,就是台蒸汽机,能够发出一百匹马力,并且往四面八方漏蒸汽。一百马力能使矿工有机会逃生,但是矿井还是要被淹掉。至于它的外形,完全是一堆屎。对我来说,正面的东西就是一堆屎,连我自己在内。 

虽然我能把柴油机画好,但是我根本就不想画它。我情愿画点别的,哪怕去画大粪。在一泡大粪面前,我能表现得像个画家,而在柴油机的图板面前,我永远是一泡大粪。假如我想变成个人,就得做自己能做好的事,否则就是大粪。为此我要出国,或者得数盲症。这件事别人能做成,但就是我做不成。 



在那个星期四早上,因为工作让我很头疼,所以我就把铅笔一摔,吼道:不干了!开party!去把你们的傍肩都证来!大厅里哄的一声,大家都往外屋拥,去抢电话,通知他们的人。这以后的事就是反面的了。只有我和小徐坐着不动。他是考勤员,问我:今天怎么算?我说:所有的人都病了。他说:那得派人去医院搞假条。我说:你去。这个混蛋斗胆要借我的车,我一时糊涂就借给他了。结果他骑着到处兜风,不光耗尽了油,连挡泥板也撞瘪了。最糟的是被保安逮了去,挨了两一之后,就信口胡招,把我们在医院里的关系出卖了,口袋里的一沓病假条就是罪证。他这个人干出了这种事,我倒是不意外。只可惜我们的大夫去砸碱了。但这些都是后话了。 

小徐一去就没回来,他死掉了。在这件事上,数盲肯定经我要悲痛。进了局子我才发现,我们的一举一动上面都知道:知道我们拿柴油换白薯,拿铸铁换雪花梨。这些事都是我领头干的,因为工资不够花;当然更知道谁在和谁乱搞,不过数盲表示这些事不必深究,他们教育家属的工作也没做好。我个人认为这些事双方不提最好,省得大家都不好意思。但是不提不等于不重要。 

我的问题主要是经济问题:有人管我要储藏室钥匙,我连问都不问就给了他。储藏室里就是些铸铁、柴油,捣腾没了可以找别人借。过一会又有人管我要地下室钥匙,说要动用战略储备,去开动部里那辆旧北京吉普。我问他要干什么,他说去把小孙接回来。为此还要开介绍信,说咱们这里提审他。我批准了——我是常务副部长,手里有介绍信。然后大胖子进屋来,高声唱道:有螃蟹——要两桶柴油换。我也批准了。但是要他少带几个人去——搞得那么沸沸扬扬不好。他答应了,但是他们把那辆柴油车开走时,车上至少有十个人。过一会又走了一辆车,说是去拉雪花梨,去的人也不少,拉走了不少铸铁,拿去换梨;但是又有好几辆车开进来,上面是外单位的人。我跑出去要把他们撵走,party是晚上的事,白天来干什么?我不想有人来帮我们折腾。但是我发现是玻璃公司的人,前几天人家刚帮我们打了一架,交情非比寻常。更何况人家也不是空手,带来了几箱鲅鱼,还有好多铁棍。鲅鱼是吃的,铁棍干什么用,我都没敢打听。然后会计部的人也来了,都是女孩子,撵人家就更不恰当了。有个小姑妨撞到我屋里来,管我要铁筷子,要烫头发——我没理她。她就跑出去说,屋里坐了个人,一声不吭,好可怕!别人告诉她说,这是我们老大哥,他总这样。其实我不是总这样,人来得太多,我心情坏。 

有关战略储备,也是个严重问题。我存了两桶八十升汽油,这是违法的。汽油是危险品,可以造燃烧弹,威胁到数盲的安全。但是可以造燃烧弹的东西多着哪,比方说,苯,自来水里有的是,只是领导要它没有用。搜我家时又发现了一把钢制的水果刀,这也是危险品,钢刀可以杀人。假如哪位数盲乐意试试,我能用铸铁刀把他杀死。根据以上事实,我认为汽油和钢的危险性并不表现在它可以伤人。主要的问题是它们对数盲有用。凡对他们有用之物,则危险。我还存了一件最大的危险品 ——吉普车。这东本开得很快,当然有危险。我存它的目的是万一有人得了重病,可以在几个小时内把他送进天津或北京的医院,救他一命。然而我们谁都不是高速车辆驾驶员(政审通不过),开车上高速公路当然要威胁到数盲的安全。 



以前开party没来过这么多人。我前妻打个电话来,说你那里好像来了很多人,是怎么回事,我说到了年终,和关系单位联欢。她说你小心点,我们这里有反应。这使我想到了小徐借了我的车去医院,肯定是先到了她们那里。没想到的是再过一会他就要死掉了。我想证她来,但在市府的电话上不敢乱讲,就没有说。中午时分我就开始和大家打招呼,让他们少招人,但是不管用。到了午后,不知哪来这么多人,连保安的人都吓跑了,怕我们找他们报仇——我们的人太多了。然后我就豁出去不想后果了—— 要玩就让大家玩高兴。 



2 

那天早上我还想到这样一些事:其实我过去是有数盲症的,上小学时连四则运算都算不好——当时我就画得很好了,所以觉得算不好没有关系。上中学时物理化学全是一塌糊涂。几何学得还可以,代数不及格。高考之前觉得数学吃零蛋太难看,找我哥哥恶补了一下,在一百五十分里得到了三十来分,就把老师和同学吓了一大跳。假如他们知道我现在是工程师,一定要吓死了。 

那天早上我想到国十年前,我们家住在北京的一个四合院里。冬天我哥哥从乡下回来,和我住在一间房子里。那房子中间有一个蜂窝煤炉子,我把全部身心投入了炉子 ——这时因为我怕冷,还有一个原因是我喜欢摆弄炉子。我哥哥歪在靠窗户的床上看书。那个窗户下面是玻璃,上面糊着纸。当时的情形就是这样的。 



我哥哥正在插队,每年冬天都带着一肚子的疑惑回来——比方说,当时的年轻人只有少数能上大学,按理说应该选最聪明的人上学才对,但实际情况是选了一些连字都不大识的傻瓜。另一个例子是:大家都在田里出苦力时,要选几个聪明人去县里开会,住舒适的招待所吃很好的伙食;但实际情况又是选了一些不可却药的傻子,到那里讲些老母猪听了也要狂笑起来的傻话。顺便说一句,那种傻话叫做“讲用”,我当时只有八岁,已经觉得它愚不可及。他老在唠叨说,这世道也不知是怎么了。考虑到我当时的年龄,当然回答不出来。 

我哥哥年轻时经历的事,现在也存在。当然,现在男孩子不用去插队了,在高中毕业时,大家都要去兵营里军训。然后根据教官的意见,把最聪明的孩子送到技工学校受训,比较傻的却送到各种管理、外交、艺术院校里去。后者假如没有数盲症的话,在那里念上几年,肯定就不识数了。当然,这两类孩子将来的待遇会有天渊之别。教官在鉴别孩子的智能方面比任何心理学家都在行——众所周知,聪明的男孩子会调皮捣蛋,而说什么信什么的,肯定是笨蛋。 

我们俩住在一间平房里时,我哥哥总在读书,先读各种“选读”、“选集”之类,因为那些书里有读不懂的地方,所以他又开始涉猎思辨哲学和中国传统哲学,黑格尔和《朱子语类》、《曾国藩家书》等等;不读书时就坐在窗边疯狂地咬手指。我哥哥非常聪明,根据他后来的表现,他是百万人里挑一的数学天才。 

有关我哥哥读的书,有一点需要补充,现在各种男孩上的学院里还在用它们当教本。而技工学校的教本,说来惭愧,都是我们编的。这是因为我们都要到技工学校任教,高深的教本我们教不动。当我为误人子弟而内疚时,就在工程课上教几节素描,还有人在数学课上教美声唱法,在物理课上教唐诗宋词,所有的学生都被我们教得乌七八糟,将来想发数盲症都发不了。 

有关我自己的智力情况,我还没有提到过。在碱场里,我前妻对我有个评价:王犯,在工程上你是个天才,但你十足外行。这都是因为你先灌了一脑子艺术才来学工。你应该去搞艺术,这方面虽然我不懂,但我觉得你一定非常人可比。我听了这话,心里很舒服,马上说道:报告管教!今天晚上我睡门口,给你挡着风!她说:混账东西,你想感冒得肺炎吗?我又说:那我睡你脚下,给你焐脚!她又说:身上冷怎么办?最后还是睡在老地方,和她并着头,哪儿冷焐哪儿。 

我前妻从来不拍我的马屁,她也用不着这么做。所以她说的话一定属实。假如我也算个聪明人的话,家兄聪明到什么程度就难以定论了,因为他比我聪明了一百倍都不止。但是读了一冬思辨哲学以后,出了一件古怪的事:有一天早上他对我说,我有五块钱,花了三块,怎么只剩了两块了?出于对他智力的尊敬,我犹豫着回答道:你说你有五块钱?对呀。花了三块?对呀。那么应该剩几块呢?他这才哈哈大笑起来,说哲学书都把人读笨了。这当然是从反面来讲,要从正面来讲,就不能说是读笨,应该说读聪明了才对。 



有一件事必须说清楚,我努力做了近十几年的技术工作,水平毫无进境,甚至可以说是越做越笨。我们周围的情形也越来越糟了,凭我的笨脑子什么办法也本不出来。看看我的同事,和我一样。假如谁看上去比较聪明,比较有前途,就会得数记美。只有我这样的笨蛋不得数盲症。 

假如我哥哥的一生被“文革”毁掉了的话,我的一生就被数盲症毁掉了。他现在是个数学教授,不是数学家。我现在是工程师,不是艺术家。假如时局有利的话,我们是可以做成后一种人的。这些事情使我很烦闷。这些事当然是从反面讲的,从正面讲,就根本没有烦闷这回事。 

我哥哥当然是个反面人物,他拿短期护照出国,逾期不归。现在他转到了正面上了:拿了绿卡,也了美籍华人。按领导上的布置,我早就通知他了——“我们的政策是既往不咎”——但他还是不回来,并且说,借我十个胆子也不敢回来。在此我要坦白一点,假如给我个短期护照,我要干出反面的事来。不给我护照,我也要干反面的事——开party——我们就是想干反面的事,故而我们才是危险的。 



3 

现在可以说说为什么要开party——因为好久没开party了,大家都烦躁得很。比方说大胖子,画着图忽然就会引吭高歌,震得玻璃嗡嗡响;还有人会冷不防用小号吹个花腔,能把人吓出一头冷汗。还有位抒情诗人会冷不防跳起来朗诵一首抒情短诗,但是本钱不够,尚不足以把人吓出神经病。不会制造噪音的人吃了烤白薯,皮都不扔,留着打他们。我们这屋里很热,所以老有股馊白薯味。所有的设计工作都没有正进展,有的还有负进展,这就是说,无缘无故把好好的图纸撕掉。我自己也有点不正常,时时在图板上画出裸体女人来。这就是说,再不开party就要出事:和保安打架,和傍肩殉情自杀,或者把摩托车开到别人轮子底下去。前几天和保安在会场上打了一架,就是个危险的信号。如果听之任之,架就会越打越大。 

除此之外,上级机关也越来越难打交道了,秘书们说话都带有进攻性、挑逗性,而且她们还常常擅离职守,上班时间跑出来会情人;我们一打电话就会被数盲粘上。数盲在办公室里也越来越坐不住了,经常开大会做大报告;会场秩序也越来越不好,保安员也越来越混蛋。除此之外,还该谈到有好几个礼拜没刮风了,天上的烟越来越黄,像小孩子屙的屎;整个城市一天到晚嘣嘣乱响,像个弹棉花的工厂。这种情形早晚要把人逼疯掉。 

数盲同志们对我的辩解反驳道:你说天是黄的?我怎么没看见?对他们来说,玻璃是蓝的,不论家里、办公室的玻璃,一是汽车的窗玻璃都是蓝的。这种玻璃表面有层有机硅透光膜,都是进口的。假如我们能读到些国外科技书刊,没准也能造出这种涂料,但是那给书刊里常夹有半裸女郎的广告,所以有危险,不让我们看。我们看到的全是正面的,没有危险的东西,所以心情烦闷,走向反面。 



开party就一定要在上班时间折腾,消耗大量的公款公物;否则就等于没有开。然后就折腾一夜,傍肩也不回家。数盲问起来,就说回原单位联欢了。不要以为数盲蠢,所有的 “家属”都不见了,还不知道是怎么回事,于是整天气呼呼,碍着数盲风度不便发作,而我们(非数盲男人和傍肩们)见了这种景象就十分开心。这些行径最起码是犯错误,有些还是犯法的,但是开party就是要犯错误和犯法,否则就没有效果。等到犯过错误和犯过法后,大家都能正常一段时间。当然,作为老大哥我要承担责任,去砸一段时间的碱,或者关一段时间的小号——这要看犯的事多大而定。但是这对我不是大问题:哪里我都熟。等把这给道理都讲够了以后,还有一点我明白:各单位都可以开party,各单位都有老大哥可以承担责任,干吗非是我不可呢? 

外单位的老大哥总给我打电话,问你们什么时候开party。我听了当然气愤,反问道:你们为什么不开?这班混蛋说:我们不行——我们没有号召力。再说,你们都是文艺单位下来的人,开出的party有趣。不管怎么说,你们是老大哥单位。这话听来有理,但却是混账逻辑。这种逻辑要把我害死。 

不等公安局的人来找我问星期四party的一口我就知道此事决不能善了;我叫大家搜集硬币,越多越好。这个party上了电视新闻,是近年来罕见的集体旷工事件,光这一条就得去砸碱,更何况浪费了大量的宝贵物资。光电石就用了十来桶,但我们没有动气焊,只是用来点乙炔灯,给广场照明——由,你就可知那party有多大。硬币是铝的,准备熔了造假脚镣。那东西一文不值,只是有点不好找。蓝毛衣说她要押我去碱场,不准别人争。我告诉她,她也得准备去砸碱了。因为这回上面和我们算总账,连打架用手扣子的一叩发了。挨打的保安举报说,凶手是女的。我与在抵赖,但未必能赖掉。蓝毛衣知道以后分毫不惧,反而到处去吹吹嘘:姐们要去砸碱了——照我看她也该砸碱,她把人家的鼻梁都打断,彻底破了相。我们要赔出人家买老婆的钱。但是最后谁也没去砸碱,而是比那更糟。我们国家学习新加坡二十世纪的先进经验,改用鞭刑,数盲决定,拿我们这桩试点。这就是说,要在背上挨几鞭子了。以前没挨过,挺他妈的新鲜。也许就因为这个,蓝毛衣主动坦白了(她很想挨几鞭,大大地出个风头),还把手扣子交了出来,就被公安局的请走了,再没回来。 

有关蓝毛衣闯的祸,还有补充的必要。我说过,那一天小徐借了我的车去拿病假条。拿到了病假条,在回来的路上和别人撞了车,与对方驾驶员口角,被保安请去了。人家一查他的证件,发现是技术部的人,除此之外,对方发现他很面熟,星期一下午打架就是他先动的手。在这种情况下,对方当然对他发生了很大的兴趣,当他把一切可交待的事都交待了以后,这种兴趣还是有增无减,这一下就闯了个大祸,小徐进医院后也没醒过来,径直死掉了。尸检时发现肝脏碎了,而且是连同好几根肋骨一道被打碎的。身上还有很多伤,但已没有深究的必要,因为这一处就足以让他死了。这种事当然不能由着它发生,所以保安方面一个死刑,三个无期徒刑。保安方面当然也有些要申诉的一中:在星期一的斗殴中有人用了手扣子,把他们的人破了相,所以他们的人才会下狠手打人。因此数盲要把使手扣子的人找出来,抽上一顿,以示公允。 

讲了这么多反面的事,也该讲点正面的了。星期四我开了party,等到过完了周末,公安局的人就到部里来,客客气气地主产:请问谁是王二?您有麻烦了,要跟我们去一下。说完给我戴上铐子,这个铐子是不锈钢的,有两个顶针那么大,套在两个大拇指上。我认为假如我是摩菲斯特的话,设计这个铐子的就是撒旦本人了——用这么点钢就把人扣住了,怎么能想出来。那位警察听了,摘下大檐帽说:您高抬我——没法子,就给那么点钢。当时我吓得够呛,他就是这种拇指铐的发明者。骂人家是魔鬼,这事怎么得了。谁知他在我对面坐了下来,说道:你这儿挺暖和,我多坐一会。你有什么事要办就先办办。别怕,我没数盲症。我赶紧说:看得出来,看得出来!我以前以为你们都有那种病哩。他说:这就是外界对我们的工作不理解了。这说得很对,别人对我们也不理解,说我设计的机器是大粪,还要求枪毙我。所以,理解万岁。 



假如可以和数盲们说理的话(其实和他们没法说理),我可以辩解道:星期四我只说了一句“开party”,此外什么都没干。这句话只是振动了一下空气而已——当然它和后来发生的事有一种极牵强的关系。数盲就顺着这种关系找到了我,让我挨鞭子。除此之外,蓝毛衣与保安打死小徐然后又偿命一事的关系也很可疑。假如保安该给小徐偿命,毙了他活该。假如不该偿命,把他放了也没什么不可以,这么胡搅蛮缠干什么。再说一遍,我知道说理是不许可的。但是我觉得他们实在不讲理。刚进局子,警察就告诉我说,我的案子上面要直接抓,让我做最坏的准备。事实上没有那么坏。 

我的案子数盲们很重视,所以警察一直劝我交待出个把别人来,但是我不肯。我倒不是皮肉痒痒想挨鞭子,而是身不由己——身为老大哥,如果让别人去挨鞭子,今后没法做人。这件事一连拖了半个多月,其间还被带到公安医院查了几次体。最后人家说,你年纪大了,心脏也不行,有生命危险——你可要本明白。我听了也有点犹豫,要知道我挺怕死。后来弄明白生存率有百分之七十(后来知道实际上是五十)就鼓起了勇气,签了认罪书,住进了公安医院。这里和活还蛮好的,睡单间,一流伙食,每天看病吃药。住医院有两个好处,一是先把我身上的病控制住,鞭刑后的生存率就能比50%高。二是假如让我信在家里,鞭刑前准会服止疼药,打吗啡针,这样鞭刑的意义就失掉了。 

后来我知道,我是命里注定要挨鞭子的,公安局的同志问我那么多,是觉得两个人太少,想多拉几个。他们后来说,人多了热闹,也显得不疼。但我不这么想。他们又说,你这个案子上面动了真怒,多报几个人好,少了可能毙了你。这可让我够害怕的,但我挺住了。这样好,万一后来知道不招也能活就会后悔。宁可当场死,也不吃后悔药。 

有关小徐,有必要补充几句。首先,他已经死了,我不说死人的坏话,所以本日记里一切他的坏话都取消。其次,虽然他死了,我还是不喜欢他;因为他什么都不肯干,和老左简直是一样,而且公开宣称他想得数盲症。最后,他已经死了,至死都没患数盲症,所以他是我的人;故此上面说的那句话也取消。而且这件事我也有责任,假如早发现他不见了,就可派出人去找他。发现他被保安逮走了,我可以率大队人马去救他——玻璃公司的哥们带来了铁棍,就是为这样的事预备的。荡平保安总部,冲到地下室把他救出来,在这个过程中,肯定要出人命。假如我干了这样的事,等待我的就不是鞭刑——额头上要吃子弹了。 



4 

星期四我去参加那个party——现在我是从反面来说,坐的是技术部开最后一辆车。当时天已经黑了,但是我也能看出来,这车不是往东山上开——东山上有好多疗养院,现在也都空着,但西山是禁区。这里是中央的地方。自从海里满是柴油,人家就不来了,连警卫部队都撤走了,但别人还是不敢进去。最可怕的是它离市府小区极近,肯定会让数盲们发现。不过,我既然已经豁了出去,也就不问了。车进了西山的围墙,空气登时变得很好闻,因为这里有很多的树,甚至可以说,整个西山就是座大树林。现在树很少见,城里的树都被农民偷走了,所以有好多年没闻见这么好闻的松树味。出于一种朴素的敬畏之心,农民还没到这里来偷。连小偷都不敢来的地方,我们来了,这件事不怎么好。 

等到车开到广场上,看到那里黑压压的人群,我脑子里又嗡的一声。整个北戴河,整个秦皇岛没得数盲症的人都在这里,甚至还有天津和北京来的人,开来了各种柴油车、烧焦炭的煤气车、电石车,以各种垃圾为燃料,这些是各单位的公务车,一个个千奇百怪;还有新式的日本车、德国车、美国车、瑞典车,烧高级燃料,还有用电池的无污染车,每年要到日本去充一次电,然后就可以开一年,都是首长专车。这两种车的区别在于前一种开起来地动山摇,后一种寂静无声;前一种跑得慢,后一种开得快;前一种车上没有玻璃,驾驶员暴露在外,跨在各种怪模怪样的机件上,一不小心就会摔出来,后一种很严密;前一种车上有各种管道、铸铁手柄、传动皮带等等,后一种同有这些东西,倒有录相机加彩电、小酒吧、电子游戏机、卫星天线、全球定位系统等等;前一种很难开,后一种是人就能开,除了数盲本人,但他也不是真不能开,只是觉得开车失了身份。除了这些之外,还有别的区别。前一种车是我的人开来的,后一种是傍肩们开来的。现在他们正在广场上换车开,三五十辆结成一个车队,浩浩荡荡开出去,到山道上赛车;剩下的人在广场上,有五六千人,有个骡马在集的气概。这么大的集会,假如我不是头儿就好了。但是我们这辆车开来时,所有的人都对我们鼓掌,并且有人在扩音器里说:老大哥王二来了,可以开始了。这就是说,这本烂账又记在我头上。我觉得有股要虚脱的感觉,但是挺住了,站在车头上,大声问道:吃的东西够吗?底下人就哄我:老大哥,闭嘴!俗气!车还没停稳,就有些女人叫我们车上的人:喂!陈犯!我在这里!刘犯,快滚过来!这是弟兄们的傍肩在打招呼,都是砸碱时傍上的。但是没有叫王犯的——我忘了通知她了。 

在医院里我又见到了蓝毛衣,她和我一样穿上了白底蓝条的睡袍,跷着二郎腿,坐在走廊里的沙发上和小护士吹牛,说这一回她肯定上吉尼斯大全。假如先抽她,她就是二十一世纪第一个受鞭刑的人。假如先抽我,她就是二十一世纪第一个受鞭刑的女人。这孩子身材不高,有一点横宽,体质极佳,十之八九打不死。我们俩在医院里大吃大喝,鸡鸭鱼肉不在话下,还吃王八喝鹿血。原来定的是我八下,她六下。上级的指示有两条:1.一定要抽得狠,抽得疼,把歪风邪气打下去;2.一定不能把我们俩打死,以免国际上的人权组织起哄。说实在的,这两条指示自相矛盾,乱七八糟。可以想象有一条是首长的意图,还有一条是秘书加上去的。但是都要执行。所以就把我加到十二下,把她加到了八下,给我们俩吃王八,还请了些五迷三道的大气功师给我们发气。除了这些措施,别的医疗保障方案还很多,但是都怕负责任,让我们自己定夺。这些方案都是胡说八道——试举一例,让我练铁裆功健体,在睾丸上挂砖头——只有一条有道理,我们接纳了。那就是在受鞭刑前灌肠导尿。大庭广众下,被打出屎来可不好。 

现在我知道这件事正在紧锣密鼓的筹备中——国家花了宝贵的外汇从新加坡的历史博物馆买来了藤鞭,那种东西浸了药物,打一下疼得发疯,事后又不感染——只是对我来说,有没有“整后”大成问题;从外省调来了武警,以防那天出乱子;与此同时,海滨路正在搭台子。这些事和我没有关系,我应该在日记里多写点我的问题。 



星期四晚上,有人运来了一台很大的音响设备,有他妈的逼好几十千瓦,对着话筒吹口气,山海关都能听到。先有人说,上星期是我们技术部老大哥生日!我们的老大哥王二,万岁!万岁!万万岁!我乍听时几乎晕过去,一切不受惩罚的幻想都破灭了。到了这个地步,心里挺平静。在我看来,僭称万岁的事最严重,一有人提就死定了。但是居然就没人问。现在看来是有关心我的人把这事按下了。 

有关万岁的事我要补充几句:我们部里有好几位浪漫诗人(我不能举出名字,以免他们也受鞭刑),但我认为,诗人的定义就是措辞不当的人。当然,数盲诗人不在此列。他们的问题不是措辞不当,而是诗写得太长而且永不分行。我个人的意见是措辞不当相对好一些。上星期有位数盲诗人在广播里朗诵诗篇,从早九点到晚八点,连题目都没念完,是否过分了一点? 

那天晚上的餐桌上有各种好东西:香槟、茅台、鱼子酱,我们预备的东西全扔掉了。等到party散了以后,桌上还剩了大量的食品,全是特供。后来数盲让我招出这些东西是怎么来的。说实在的,我不知道。他们又让各特供点清点,仿佛我犯下了抢劫罪。我认为他们应当回家清点。但是局子里的人说,不能这样报上去,否则会说我偷到他们家里去了。 

从正面来说,我已经体会到鱼子酱为什么是特供(危险品)了:这种东西太好吃,足以使人为之厮打起来。而在数盲那里就没有危险,他们好吃的东西多极了,犯不着为它打架。 



后来大胖子要露一手美声唱法,不幸的是话筒有毛病,他嗓门又大,故而完全失都,满山满海都是驴鸣;别人就把他撵下台去。上来一个乐队,玩的又是重金属,好在我及时用棉花把耳朵塞住了。后来有人建议让砸过碱的大哥大姐们跳迪斯科,我就没有听见,糊里糊涂地被人放倒上了镣铐,这回可是铸铁的真家伙。爬起来以后看见大家跳,我也跳。别人是一对一对的,我是一个人瞎扭,自得其乐。忽然有人在我背上点了一指,回头一看,是我前妻。穿着套装,很合体,脸上浅笑着,妩媚之极。我赶紧把棉花掏出来,这会儿不是乐队吵,而是铁链子哗哗地吵。因为所有跳舞的男人都戴镣。我说:报告管教,忘了通知你。她说:没有关系。我说:又要劳动你送我去砸碱了。她说:大概吧。你是有意的吗?我想了本说:对我来说,没有什么事是有意,也没有什么事是无意。她凑过来,贴住我的脸说:你很诚实。这时候有人宣布说,各房间都有热水,可以洗澡,也可以喝。这就是说,早有人把深井启动了。深层地下水是特供的,它的危险性在于可以洗澡,洗澡很舒服,洗了还想洗,就会把水用光;我们用当然犯法,这是因为假如我们抽走了深层地下水,表层带有盐碱的水就会渗下去——数盲抽才没有问题,虽然他们抽了地下水,表层水也会渗下去。这件事我负完全责任——听到这条通知,她就带我去出操。进了房间才发现镣铐都打不开——后来是用手锯打开的——所以只好戴着干。那天晚上她没有发口令,我自己就行——事后她说:这样的情形是第一次吧。我说:是。她又说:这说明,你爱我?我说:大概吧。她一听,眼睛里全是泪,因为这回答不能让她满意。她又问道:那你可爱过别人?我斩钉截铁地说:没有。她就说:那我死了也不亏。后来又干了两三次,都是我主动。然后我们开着她的车回我的小屋,喝了很多酒,又干了很多回。后来就睡了,再以后我醒来,我前妻已经走了,到现在还没见着。 

假如我在受鞭刑的时候死掉的话(这一段是我受刑前写的,现在知道我并没有死),希望领导上能把这个日记本交给我前妻。这个笔记本里有好几处说到我爱她,希望她看了能够满意。我一直不肯告诉她,是因为她是我的管教,我是她的“王犯”,这种关系比爱不爱的神圣得多。而那天晚上我告诉她,我大概爱她,情形和现在差不多,我觉得自己快完蛋了。当时我们那间屋里点着床头灯,挂着窗帘,但还是一会红、一会绿。这是因为有些混蛋带来了船上用的救生火箭,正在不停地燃放,而且火箭朝小区飞去。还有人在喇叭里说些放肆的话,恶意攻击——我没有说过这些话,但要对此负责任。窗帘上火光熊熊,不知烧了什么东西,很有可能在烧房子;后来才知道是烧木板箱。在这个地方开这种party,罪在不赦,因此我觉得自己很有可能被枪毙。当时我还本过,假如要枪毙我,千万别遇上球墨铸铁的枪。那种枪虽然不危险,但是拉好了架式等它不响,响的时候又没准备,死都死不明白。在这种情况下和她肌肤相亲,一切禁忌都不存在了。 

除了告诉她这给,我还要告诉她,在小木屋的地板一面,有个木箱子,里面有点贵重的东西。有一套雕刻的工具、钢制的小刀等等,这些东西别人见了就会抄走。谁知道呢?也许她的下一个傍肩也是艺术家,这样就能派些用场。有些旧版的图书画册,还有我过去全部作品的幻灯片,给她留作纪念。还有几千美元,是我哥哥托人带来的,决不是黑市上换的,送给她——当然,假如要没收,我也没意见。有意见也没用——我已经死了。

\section{五 鞭刑}

1 

我住进医院时,脊梁还完整。中间出来一次,是到广场上挨鞭子。后来里面住了很久。初进去时,还要交待问题。每个新见面的警察都先递个小本子过来,说道:老大哥,先给我签个字,然后咱们再谈。我成了明星了,虽然我什么都没干。就说市府小区断电的事吧,我事先一点都不知道。那天晚上我刚下了车,整个西山忽然灯火通明,我倒大吃一惊:这儿怎么有电哪?顺便说一句,电也是危险品,可以电死人。早就没有电了,自己发的不算。领导那里当然有电,他们勇于承担风险。正好电业局的老大哥在我身边,告诉我说:西山一直接着小区的电网,日本机组,好使着哪。我又问:会不会超负荷?他就哈哈大笑:这边一接通,那边就断掉了。所以那天晚上市府小区一团漆黑。本来一团漆黑时还有件可干(拿肚皮拱人),但是夫人们也都不见了——跑到我们这里来了,来之前还洗劫了家里的电冰箱、贮藏室。既然没有电,暖气也就停了,数盲们在黑暗中,又寂寞,又冷,还没人给他们做晚饭,生生饿了三天,只吃了些饼干。因此这个祸就惹得很大。公安局的老大哥后来说:你也该挨抽——第一、那天晚上不请我们;第二,我替你挨了多少骂!电话都炸了窝,让我派人上山拿人,都是我按住了。顺便说说,老大哥是常务副职的俗称,另一个意思是非数盲,各单位都有。我回答说:第一,以前我真的不知道公安局也有老大哥和弟兄们(他当即反驳说:屁话,数盲能办案吗?)假如挨了鞭子不死,一定补过。第二,我认为不值得感谢,因为那天夜里我们人多,你们敢来,恐怕是走着来,爬着回去。他听了哈哈大笑,说:你这张臭嘴——但是说得对。所以我们没上山拿人。我尽量安排,不让你死,但是万一有个三长两短,也别怪我。 

根据从国外买来的卫星图片分析,星期四晚上有上万人、成千辆车到过西山,但这还不算多。星期五和星期六还有人从各地赶来,星期一party才散。高峰期是星期天,西山上有三万多人,在每个房间里都留下了用过的避孕套,搜集起来装了半垃圾车。但是我早就下山了,没有看到这种盛况。在这三天里,数盲们遇到了很大的困难:既没有秘书,也没有专车,既不能工作,也没有家庭生活,所以感到很失落。西山上扩音器地动山摇地响,又有些信号火箭飞过来,市里的数盲就从小区里跑掉,去了山海关空军机场,等party完了才回来。后来他们到现场去看,看到半垃圾车的实物,又觉得心里酸溜溜的,一致认为对王二要严惩不贷。在此我要郑重声明,这件事和我无关。我没有这等身手,一人造出半车货来。 

有关party的事,我还最后有些要补充的地方。那几天我们成了数盲——吃数盲的饭,喝数盲的水,用数盲的电,和数盲的老婆睡觉;数盲成了我们——没了吃的、饮水、电、老婆,一切都要自己想办法。他们本可以像我们一样,到自由市场买块烤白薯、到饮水站要点饮水、点一盏电石灯,或者到地下室启动应急发电机,然后自己去找个傍肩,但是这样做证明他没有数盲症,所以他们不肯。假如不是我星期四在西山上开那个party,那么就会有别人在别的时间、别的地方开这种party。这是因为在此之前,我们,各种工厂的技术员、工程师,以及各种科技机构里的男人,还有所有的女秘书、夫人等等,觉得生活很压抑、需要发泄。这件事不能怪王二一个人。那半垃圾车的货就是证明。只有数盲才不觉得压抑,也不觉得有什么要发泄的,所以这个道理和他们说不通,他们认为这些事都怪我一个人。除此之外,他们也没有数量的概念,认为我一个人射出半垃圾车精液完全可能,并且不肯想想,射出半垃圾车后,我还能剩下什么。等到这件事过后,大家都发泄过了,感觉良好;但数盲们却觉得受了压抑,也需要发泄,要抽我的脊梁。我没有数盲症,只是个小人物,所以脊梁就保不住了。当然,这件事也不那么简单。听说有不少夫人旗帜鲜明地对丈夫表示:要是王二有个三长两短,我就和你一刀两断!但是在大是大非面前,数盲总能站稳脚跟的。所以她们的努力也就能保住我一条命。除此之外,听说各机关都增加了夫妻生活的次数。这说明数盲们也会接受教训。虽然数量增加了,质量还是没改进。根据可靠情报,他们现在还是废话连篇,而且还是在拿肚皮拱人。 

我现在可以坦率地说出一切,就如那位希腊勇士——当被带到暴君面前,被问到“你凭什么反对我”时,他坦然答道:老年。我现在的样子和老人差不多,但是问题还不在这里。我现在已经做好了死去的准备,这是最主要的。在我看来,数盲最讨厌的一点是废话连篇,假如你不制止他,可以说上一百年。除此之外,他讲的每一句话,我们都听过一千遍。当然,在这一点上,双方见仁见智,永远谈不拢。数盲们说,这话我讲了一千遍,你还是没有听进去;我们说,你讲了一千遍我还是听不进,可见就是听不进。数盲又说,一千遍没听进,那就讲一千零一遍。但是他根本不知道一千遍是多少遍,更不知这么多遍可以让人发疯。除此之外,我还有点善意的劝告,在干那事时,要把注意力从废话上转到女人身上,这样肚皮和阳具就能有点区别。当然,他们的绿帽子绝不是我一人给戴上的——只要有数量的概念就能明白,我一个人戴不上那么多绿帽子,但他们是没有数量概念的——讲出了这些话,我就可以挨鞭子和死掉了。 



2 

受刑日早上五点我就起来了,到手术室里接受处理——情况和手术前备皮差不多。然后穿上我自己挑的衣服,经过消毒的中山装,从手术室里出来,有位年轻的警察给我戴上铐子。那铐子看上去是不锈钢的,但戴上才知道,它又轻又暖,是某种工程塑料。我就开始琢磨它,想方设法把它往硬东西上蹭,发现它的表面比钢还要硬。问它是什么做的,押送的警察也不知道,只知道是进口的。看来世界上的技术正在日新月异地进步,不学习就会落伍。走到医院门口,遇上蓝毛衣,她穿着黑皮茄克,黑皮短裙,黑色长袜,高跟鞋,也戴着那种高级手铐,几位女警押着她。我吻她时,别人都扭过头去,然后我们就上了一辆囚车,这是一辆装甲车,也是特供,因为装甲不像球墨铸铁。她坐在我身边,然后就把脑袋倚在我肩上,说,起得早,困了。然后就睡了。这孩子长了张大宽脸,厚嘴唇,脸上有雀斑,但是相当耐看。她在睡梦里一再咂嘴。她用了一种法国香水,非常动人。这是特供。今天也有给我用的特供,那就是进口强心针。虽然还没用,但肯定能用上。 

她睡了一小会儿,起来说道:老大哥,和你商量件事。呆会儿我先上。我说:你要破吉尼斯纪录吗?她说不是的,把你打个血淋糊拉,我看了害怕。听到了“血淋糊拉”这四个字,我背上开始刺痒,说:难道我就不怕?她愣了一下才说:好,你先上就你先上。我闭上眼睛——说着就使劲闭眼。我说:算了,和你逗着玩,让你先上。于是我就开始想象她挨打会是什么样,这些想法都很刺激。她说害怕,我就能懂了。这就是说,她和别的女人是一样的。 

我前妻也说过害怕,那是在砸碱的时候,晚上她要上厕所,让我陪着去。到了地方,她进去了,我在外面遇上巡逻队,就有麻烦。 

——黑更半夜,你怎么出来了? 

——报告,是管教拿枪押出来的! 

——那就不同了。怎么枪在你手里? 

——报告,她拿着嫌累! 

——那又不同了。她不拿枪,你跑了怎么办? 

——报告,我逃跑时先把枪还她。 

——你要是不还她怎么办? 

——报告,不还是犯错误,我不敢不还。 

——那你就在这里等着吧。你都把我绕糊涂了! 

我前妻在里面都听见了,出来时就说:王犯,对答甚为得体!我回答说:是管教教导有方。她说:真他妈的冷!把枪还我。快点回去暖着我。向后转!跑步走!一二一!一二一! 



在那辆东摇西晃的囚车里,和我蓝毛衣聊了一会。我问她爱看什么书。她睁大双眼,连雀斑都放出光彩来:《塔拉斯·布尔巴》!!!这是果戈理的书,里面有战争、酷刑、处决等等,是一本关于英雄的书。这比我想象的好得多,但这决不是说这书不危险(它也是禁书),而是我心里有更不祥的猜测——Story of O。当然,是我猜错了。 

后来蓝毛衣就又睡着了。又把头歪在我身上,十分沉重。在受鞭刑的早上,前往刑场的中途,我想一个人消停一会,看来也是不可能。这个女孩子我都是猜不透。。本来挨鞭子是我们的事情——首先是我的事,因为我是老大哥——莫不成她也本来当老大哥?但是她硬要来插一杠子。首先,根本没人请她来帮我们打架;其次,更没人请她去把保安的鼻梁打断。要知道我们和保安的关系并不像表面上那么坏,在她插一杠之前,保安打我们,我们也打保安,双方都留有分寸;至多打到头破血流,从来不把骨头打断。这甚至可以说是一种游戏。她插一杠以后,双方都死了人(我们的人被打死,他们的人被枪毙),以后就再没法算一种游戏了。这件事实在让人痛心。 



3 

我既胆小又怕疼,原本宁可自杀也不会去挨鞭子。这一点在我坐在囚车上前往刑场的路上已经充分表现出来:我出了一路的冷汗,服了三片救心丹,虽然早上导过尿,弹力护身里还有点潮湿的意思。最可怕的是到了刑场上多半还要出乖露丑,让大家都看到我是孬种。我在鞋底里藏了一片保险刀片,随时可以拿来割脉。但是我挺着没用,主要是今天这么大的场面,假如主角畏罪自杀,数盲恼羞成怒,谁知会出些什么可怕的事。可以想象的后果是:1.随便揪另外一个人抽一顿;2.把该我挨的鞭子加在蓝毛衣背上。不管发生了什么事,别人都要看不起我。我不能让这样的事发生。我的责任心极强,这就是我总是当老大哥的原因。 

我哥哥也是个负责的人。他得了关节炎从乡下回了城,进了一家小工厂,每天拐着腿去上班,哪怕是天阴下雨腿疼时也按时前往,夜里往往还要加班。我问他为什么,他说:你还看不出来吗?假如大家都不好好干,国破民穷百业凋零之时,我们就会有另一次“文化革命”,或者和外国开战,或者调军队进城来军管。总而言之,领导上想要破罐破摔,有好多种摔法,你想象都想象不出来。想要避免被摔碎,我们必须要表现得像个好罐子。在我看来,像他那样负责的人还是挺多的,在青少年时期,我只见过一两次摔罐子的情形。到了中年,该我负责任时,我想我是尽心尽力了,人家要抽我的脊梁,我都让抽了。 

我哥哥王大和我极相像。下乡插队时,他是集体户的户长,除了干活,还要管大伙的吃喝。进工厂以后,他是班组长,上班总是早来晚走,还不敢拿加班费。后来他又当过学生班长、工会小组长、各种会议的召集人等等,直到他当得不胜其烦,逃到美国再也不敢回来。有个老美一见了他就说:你在军队里呆过,当过二十年军曹!当然,这是想当福尔摩斯的老美。其实我哥哥一天兵都没当过。现在王大一想起自己干过的各种不伦不类的差事就做噩梦。我和他的经历大体上差不多,但是不做噩梦,因为我还在噩梦中。我们俩在遗传上一定有点古怪。假如我死了,应该有人解剖一下我的尸体,找出毛病的所在,最好还能找出矫正的办法来。 



在乘车前往刑场的途中,我一直在想今天的要点。第一,我不能被人抽出屁滚尿流的样子。这是因为在场的会有大批我们的人,假如我屁滚尿流,会伤大家的心。虽然按我的体质和性格一定会显出屁滚尿流的样子,但我要拼命顶住。第二,今天我不能死掉。假如我死掉,就会出天大的乱子。其实作为受刑人,死活不是我该考虑的问题,但是作为老大哥,必须把不该考虑的事全考虑到。数盲对出乱子的年法是:不怕,不就是死几个人吗。听起来没什么,但你要想到他们不识数,根本不知几个是多少——也许大伙都死光,他还觉得只有几个。但是这两个要点又是自相矛盾的。公安局的老大哥告诉我说:今天抽人的是保安的人,他管不着,氢“你要是挺不住千万别硬挺,装出个屁滚尿流的样子,我就能插手了”。这就是说,假如我要保命就要屁滚尿流,不屁滚尿流就不能保命。这两方面都要顾及,事先难以拿主意,只有等鞭子抽上再作定夺了。 



到了地方,看到海滨广场上黑压压的一大片人,少说有一万。一半是我们的人,别一半是警察,手里拿着小巧玲珑的冲锋枪。那东西做得真是精巧,我一看就入了迷。我们还真有不少好东西,不光有球墨铸铁,只是平时不肯拿出来。有关球墨铸铁的枪,我有一些补充说明。那种枪放的时候“嗵”的一响,冒出一股浓烟来。假如那枪对着你放,有一定的危险性,看见浓烟后,就有一颗半斤重的铅弹发出蜣螂飞翔的声音朝你飞来。这种子弹中在身上必死无疑,但是赶紧躲的话,还能躲开,或者拼命逃跑,那个铅屎克螂未必能追上你;假如是你拿枪朝别人放,危险就更大,沉重的枪身要猛烈地往后撞,所以在开枪前最好在胸口垫个包装纸箱。我和我前妻在碱滩上打野兔子时,放过铸铁枪,像这样精巧的冲锋枪却没放过——大概是进口的吧。对于这种枪,我也有点要补充的地方:它完全是危险品做的,所以真是好看。故而它当然是特供。见到了这种东西,说明我闯的祸真是不小。 

广场上有一座木板搭的台子,上面有桌子、麦克风、数盲等等。台子后面有座X型的木架子,看来要把我们拴在上面。我们从囚车上下来时,遇到了山呼海啸般的掌声,还有人高呼欢迎老大哥,然后这掌声又被更大的声响压下去了。会场周围的武警齐声喝道:“不——准——乱——动!”那种嗓门和保安是一个类型的。蓝毛衣听了,禁不住往后一缩,撞在我身上。我却推了她一把,说:别怕,不是冲咱们来的。然后我们就进到台子背后的棚子里等候。这个棚子是铝合金和玻璃做的,里面就我们两个人。隔着玻璃往外看,到处是戴钢盔的武警,我们好像进了笼子一样。瘆人的是这棚子很隔音,所以很静,这里有一把长条椅子,太阳晒得很暖和。我指指椅子说:请坐。蓝毛衣坐下来,我隔着玻璃往外看,看见数盲在作大报告。平时我对报告不感兴趣,今天倒想听听,但是听不到。棚里有台黑白电视机,放着外面会场的实况,但是无伴音。我给了它几巴掌,想把伴音打出来,但是不成功。不仅没打出伴音,倒打出大片的雪花。反正闲着没事,我又打了它一顿,把雪花打掉,还打出一点彩——原来不是黑白电视,是彩色的,但伴音还是打不出来。今天见到的都是特供,只有这台电视例外。这使我想起了数盲常说的话:好钢要用在刀刃上!我今天就是到了刀刃上。 



4 

我和蓝毛衣住在医院里养伤,讨论受刑那一天的感受,一致认为在铝合金棚子里等待的时候最难熬。那一天市长和四个副市长都发了言,一共讲了五个多钟头,只有寥寥数语提我们的事,大多数时间却在谈精神文明问题、生产问题、污染问题、计划生育。考虑到我们就要血肉横飞,闲扯这些淡话真是有点奇怪,所幸我们在棚子里一点都听不到(这是后来知道的),只看见武警在打呵欠。隔了老半天,才有人把玻璃门打开一条缝。蓝毛衣“刷”地站了起来,正要走出去,那人却说:不要着急,还早。我就是告诉你们这个。然后关上门走掉了。蓝毛衣就在棚里来回走,我却坐了下来,想打瞌睡,但是睡不着。要知道也许再过一会我就要死了。所以我就琢磨那个手铐——那东本是那么像不锈钢,仔细看却能看出,它的颜色有点灰暗。真的钢比它亮。这也许是因为镀了一层无光膜。后来我又把手铐举到玻璃边轻轻地敲,声音很脆,但是有点轻飘飘。敲着敲着来了一个哨兵,对着我蹲下来。我还是继续敲,他就张大嘴巴,让我看嘴形——干—什—么?我也这么答道:不干什么。不干什么。他说:不—准—敲。我说:就—敲。他端起枪来,对准我的胸口。我把胸膛往上一挺。他就笑了。然后回头看了看,走开了。 

后来人家告诉我们说,那一天电视在向全国转播,大家等着看北戴河抽人,等来等去不见动手,不是数盲的人都熬不住,睡着了。数盲倒是瞪着大眼在看着,但也早忘了等着干什么。电视镜头一会照照这个,一会照照那个,终于照到了一个人在啃面包,就转播了那个面包消失的全过程。然后他像眼镜蛇一样张开大嘴,让全国人民兵他的扁桃腺,还用舌头舔面包渣,这使我好一阵见了面包就恶心。然后又转播了一个人抽烟,抽了一口,憋住了气,用左眼看看烟头,再用右眼看看烟头,最后用两只眼看烟头,把自己看成了对眼,足足憋了四分钟,才把那口淡淡的烟呼出来。后来知道,那个人原来是个蹼泳运动员,肺活量大得惊人。我要是有那么好的肺,就绝不吸烟。还转播了一个小会计给自己化妆,先是他开细细给自己画眼晕,画完了,照照镜子,用纸擦掉,再画一遍。忽然之间,她拿出口红,给自己画了个大花脸,然后吐着舌头给另人看。我要是像她那么年轻漂亮,就绝不在电视上糟蹋自己。后来才知道,电视摄像机位置很隐蔽(同样很隐蔽的还有一大批狙击手),会场上的人一点也不知自己上了镜头。后来这些上镜头的人都倒了霉。然后有人嘘起来,等到嘘声很大的时候,武警朝天鸣枪,大家都趴下,数盲往天上看。但是我们在棚子里看不到武警鸣枪,也听不见。只看到大家趴下数盲朝天上看,所以一点也看不明白。假如不是在屏幕上见到了熟人,我还以为放错了频道,这是个电视剧哪。看见这种情形,蓝毛衣就哭起来了。 

我在受鞭刑之前,在一个玻璃亭子里关了很久才去挨抽。当时我以为自己很可能马上就会死掉,但是没有,虽然挨第八鞭后死了一会,吸了氧气,打了强心针。醒过来以后,有人要把我解下来送医院——余下的下回再打。我坚决不同意,并且抱着柱子不撒手,说自己没问题。我可不乐意再被关在棚子里。数盲们尊重我的意见,又打了我四下,然后七手八脚地把我解了下来,要架我上担架。但经过现场抢救,我还能自己站住,就把搀扶的手都推开,从台上走下去。这时候会场上已经乱了,到处都在和武警扭打,还有枪响。只有台前一片人端坐不动——都是我的同事。不动是对的,动就会有伤亡,而且伤了谁都不好。我不在,也不知是谁在为头。我朝那边走了几步,又被人架住。公安局的老大哥凑着耳朵说,你还是快走为好。我点点头。就在这里有个女人站了起来,她戴着墨镜,穿一件薄呢子大衣,高跟靴子,径直走过来,原来是我前妻。原来她回到部里,掌握着这帮人,这我就放心了。我对她说:你给我根烟。她拿出烟来,吸着了放到我嘴上。我抽了一口,猛烈地咳呛起来,同时眼前阵阵发黑,赶紧取下烟来又递给她,说道:给别人吧,别糟蹋了。然后我就人事不知了。 

我说过,蓝毛衣在棚子里哭过,当时她说:你看看,他们都在干什么?一点都不尊重我们。我赶紧安慰她说:会尊重的——不尊重我们,也得尊重国家的鞭刑。人是我心里却在想:看来他们把我们安排成了个会尾巴。所谓会尾巴,就是很不重要的议题,万一来不及进行,就推到下次会。看情形,我们要被押回医院。以后还要五点起来,灌肠导尿——导尿这件事最可怕,因为二等兵王二不经折腾,动不动就直起来,那些小护士面面相觑,然后说:老大哥,可喜可贺。她们的意思是说我这把岁数了还这样可喜可贺,但我觉得自己为老不尊,难堪得很——这些一可以忍受,还要在这个棚子里等候,不知会等到什么时候。因为有了这些细节,所以真被绑到X形架上时,我倒感到如释重负。 

现在我知道,其实数盲们很重视我们,那天惟一的议题就是揍我们,但是不管揍谁,唧怕是揍自己,数盲们都要讲两句,两句并不多,在这方面我们没话可讲;不幸的是数盲根本就不知两句是几句,讲起来就没完。因为这个原故,蓝毛衣哭得很伤心。我让她把头倚在我肩上,因为我是老大哥,比她大二十多岁,我显得既端庄又体贴。其实我也一阵阵的想撒癔症。在一个静静的玻璃棚子里,看着外面的浑浑噩噩,再加上生死未卜,我心情坏得很,但我能控制得住。这一切得益于我前妻对我的训练。当年在碱场里她训练我走正步,喊了“一!”后,就这样对我说:王犯,你脾气很坏!而我保持着金鸡独立的姿势,朗声答道:报告管教,一定改!她压低了声音说:看着点人。然后凑过来吻我一下说:告诉你,不准改,改了就没意思了。你只要控制住自己就行了。因为有这样的训练,所以我不但能控制自己,而且能眼观六路耳听八方。一下在电视上看到了玻璃棚子,透过玻璃还看见我和蓝毛衣拥在一起,就说,咱俩上电视了。蓝毛衣转过身来,把哭哭啼啼的样子暴露在大庭广众之下。等到看明白这一点,她暴跳如雷,原地跳了好几下高。后来又对我说:老大哥,你得为我作证,我可不是怕了才哭的。我说:当然,但也得我能活着才成。 

后来电视调了一下焦距,棚子、玻璃都不见了,只剩下我们两个,坐在椅子上。我们俩笑着朝电视招了好多次手,但是没什么反应。我眯着眼睛,想把摄像机找出来,但是阳光正从那个方向来,所以什么都看不到。蓝毛衣倚着我说,她有个好主意,假如我挨了鞭子不死,我们俩就傍起肩来。我说,这是老生常谈。我有个更好的主意。她振作起来,说道,结婚?生个孩子?我说,不是的。我想认你当我干女儿。她勃然大怒,跳起来用并在一起的手打我。后来她说,你们这些混蛋,都不和我好,都让我当女儿!我的便宜这么好占吗!这种说法引起了我的注意,这孩子的脾气、体态、相貌无一不是当女儿的料。但是她的亲生父母怎么了?假如有父母的话,谁也不敢来挨鞭子。这个问题的答案是这样的:用不着你操心!他们是数盲!早不认我了!然后她问我,当你女儿也可以商量。你爱我吗?我说,爱。与此同时,双眼平视着她,用交叉在一起的食指指向她的胸膛。她的胸脯很大,对她那个年龄的女孩子来说,实在是太大了。 



5 

我前妻训练过我怎么说“爱”,这一手在受鞭刑之前,面对蓝毛衣的时候用上了。这种训练是这样的:在走正步时,她喊二(如前所述,“一”的内容是有关我的脾气),我换了一条腿站着,她问道:王犯,说“爱”的要领是什么?我就答道:报告!双眼平视对方,平静,缓和,深情地,用胸音!她说,转过来,做一遍!我保持着“二”的姿势,单腿转过身来——这一手就是少林寺武僧也要佩服的——对她说了爱。她问:周围有人吗?我说:没有(当时是清晨四点半,天还不大亮)。她说:很好。还有叫?我说:报告。你得先说稍息才成。她说:稍息。我就放下腿,走过去吻她,做得和热恋的情人一样。这时候她说:你要是能“情不自禁”就好了。我说:是!管教!请指示要领!她勃然大怒,说:混账!我要揍你!我就喀嚓一转身,面对我们的木棚做好了跑步的准备姿势,朗声答道:是!管教!拿鞭子还是拿棍子?我以为能把她气疯,但是没有。她叹了口气,说道:不和你怄气。现在——解散!我受训的事就是这样的。等到开party那晚上,我们俩躺在双人床上,我用胳臂揽着她。她问我,在碱场干吗这样怄她。我憋了一口气,好半天才吐出来,什么也没说。她猛地翻身起来,扑在我身上,用手指划着我的胸膛说:等你死了,我要把你的心扒出来,吃下去!我说:用不着等那么久,现在就吃吧。于是她在那时咬了一口,留下一个牙印。后来在医院里,一个女医生也看见了,她问我谁咬的,我问她问这个干什么。她说没什么,这个女人的牙很好呀。但是这又扯远了。 

我在玻璃棚子里对蓝毛衣主闻爱,就照要领行事,但因为两个人都戴了铐子,所以我往左扭,她往右扭,就这样往一块凑,从头顶往下看,一定像个太极图。就在这里棚子的拉门哗一声拉开了。我们俩站了起来,站得笔直。门口站了一大群人。公安局的老大哥说:你们俩谁是头一个?我看了一眼蓝毛衣,发现她脸色苍白,就朝前跨了一小步——但是蓝毛衣已经大步走了过去,我就退回来坐下。她把手伸过去,人家给她开了铐,她就往外走,但是被好几只手推了回来。拉门又关到只剩一条缝,那位老大哥在外面说:别着急,还要等等。这下连我都沉不住气了,跳起来问道:等到什么时候?他说:这我也不知道,和我急没用。他对蓝毛衣说:再叫你就脱掉外衣,快一点,大家都少受罪。然后拉上门,上了锁,走了。所有的人都走了。蓝毛衣转地头来,说:现在干什么?声音发抖。我知道她怕了,就说:活动活动。语气平缓,一如平日。这可不是我前妻训练出来的,而是我的本性。当初她在身后一枪打穿了我的帽子,我还是不急不慌。而不急不慌的原因是我极傲,甚至极狂。我已经说过,狂妄是艺术家的本性。这种品行深为我前妻所不喜,所以她常拿着手枪以准我的脑袋,说道:王犯,我要一枪崩了你,然后自杀。我真的吓得要命 ——谁知枪里有子儿没有——但我还是挺得住,说道:报告管教,崩完以后,您就说走了火,不用自杀。她把枪口拿开,说道:王犯,你是瘦驴屙硬屎,你承认了吧。我真的是瘦驴屙硬屎,但我就是不承认。哪怕她真的崩,我也认了。 



我在玻璃棚子里老想起我前妻,而眼前的事却是蓝毛衣在伸臂,下腰,踢腿。一活动起来,她的胆子就大了。后来她在屋里翻了一个跟头,然后走到屋角,脱下高跟鞋,倚墙倒立起来,于是茄克、裙子都溜了下来,露出了肚皮、内裤、吊袜带,大腿,等等。要知道,我们现在正上电视,我就朝她摇头道:不好看。她又正过来,穿上鞋,搓着手上的土,走到我身边来,说道:我的腿不好看?我说其实是好看的,但是咱们在上电视,你别毒害青少年。 

有件事必须解释一下,我们的电视没声音,于是我就以为电视是无声的。其实不对,电视有声音,所有的地方都被人下了微型话筒。氢我们在棚子里说话,全中国,乃至全世界都能听见。这是因为全世界的电视台都买了转播权。这句话就被上级数盲听见了,发出指示道:我们的好多同志,觉悟还不如一个犯人!乱七八糟的镜头怎能上电视!这个指示就往前方(这是电视行业术语,指转播现场)传,但是怎么也传不到,电话一会儿打到新疆,一会儿打到西藏,当地的数盲就大慌大乱,打听他们觉悟为什么不如犯人,不如哪个犯人。平时乱七八糟的事也有,都不如那天糟糕,但是这件事当时我们并不知道。我们在等待,太阳逐渐不那么厉害了,棚子里也没有刚才热。我们都冷静下来,并肩坐着看电视,电视里就是我们自己。只要心平气和,就能觉得活着是好的,不管是怎么活着。 



\section{六 认识}

1 

受过鞭刑后,我的头发都白了,还多了一种咳嗽的毛病。这不能怪别人,尤其是不能怪受刑,要知道我身体一直不好,还有吸烟的恶习。现在我戒烟,但是我的肺民经被烟熏了三十年,病根深入每个肺泡。上级通知我,可以办出国,但是我拒绝了。这是因为我的手抖了起来——这不是病,而是年老。挨过了鞭子,我已经不止四十八岁了。不管怎么说吧,作为一个艺术家,我已经完蛋了,虽然还能凑合画几下。我现在能做的事,就是写作,而写作只要能说话就可以。我一点也不想出国,因为我生在这里,就死在这里也好。按我的身体外表来看,已经有七十岁,该做这方面的考虑了。在此我申明自己的态度——鞭刑是新鲜事物。作为一个受刑人,我认为它对我有好处。当然,它对身体有点损害,但是皮肉之苦可以陶冶情操;另一方面,假如犯了法就送去砸碱,我国的识数人口就会不够用了。人力资源是我国最伟大的资源,有二十亿之多。惟一的问题是识数的人太少了。在这种情况下,让不识数的受徒刑,识数的受鞭刑,实属英明之举。另外,正如数盲们已经指出的那样,我们需要疼痛。疼痛可以把我们这些坏蛋改造成新人。 

有关鞭刑,还可以从其他方面来认识。它可以使社会上有关方面心理上得到平衡。我们心情烦了就开party,数盲们也会烦,特别是感到戴了绿帽时。这时候就该找个人抽一顿。当然,要把全体绿帽子的发送者都抽一顿是不可能的,人力物力都不许可。所以就来抽我。这是应该的——我是老大哥。 

而蓝毛衣挨抽也有道理:保安同志最恨城里人。我们吃得好(其实也不好,只是相对他们而言),住得好(同前),干活也轻松,这是凭什么?无非是凭了脑子聪明。这一点他们真比不上,所以心里有气。有气了就来打架,在斗殴中又总是吃亏。好容易逮着一个落单的,又把他打死了,自己贴进一条人命。他们需要有个机会,既安全又有效地抽我们一顿。蓝毛衣就给了他们这样的机会。事后保安同志们一致认为抽蓝毛衣过瘾,但是数盲们不这样看。 



蓝毛衣经过治疗,身体完全恢复了。她现在常来看我,提到我们之间的事,我就说:现在不行了,我认你做干孙女吧。她勃然大怒,摔了我的茶杯,还说:混账,你真是占便宜没够!——这是因为我们一起受刑,我很爱她。假如受刑日我和蓝毛衣在棚子里的举动有什么不妥,我愿负全部责任,并愿受鞭刑。上次抽了我的背,把我抽老了二十岁,这回请抽我胸口,没准能把我抽回来。 

至于那些不妥的举动是这样的:我和蓝毛衣在棚子里坐着,直到日暮时分。忽然听见有人在敲玻璃门。回头一看,是公安局的老大哥,他往台上比了个手势,蓝毛衣点点头,回过身来,拿出条黑丝带,在脖子上打了个蝴蝶结,问我怎么样。我说:好看。她站起来,俯身吻了我的脸,笑笑说:老大哥,和你在一起真好!我走了。我说:你走吧。然后低下头来,不去看她。因为她笑起来很好看所以我已经爱上了她——按我现在的情形来看,这种爱有乱伦之嫌。 

后来她就走到一边。听见她嗖嗖地拉拉锁,我禁不住扭头看了她一眼——我的卑鄙动机是这样的,没准我就要死了,不看白不看——看见她只穿了黑三角裤,长袜,高跟鞋;脖子上系着黑蝴蝶结,皮肤白皙,很可爱。后来所有的人(数盲有在内)都要交待,那天看见了没有,承认看见的要办学习班。我什么都看见了,而且在极近的距离内,所以早该去学习班。她的乳房又大又圆,一边长了一个,总共是两个。然后她朝我露齿一笑,走到我面前说:摸摸。我往直里坐了坐,捧起那两个东西,用嘴唇轻轻触她的乳头,两边都触过了,然后把她推开,拍拍她屁股,说:你去吧。这当然是危险动作,但是我当时生死未卜,不怕危险-她就往六口走。门已经开了,进来不少人。我没有回头,在看电视。从电视上看见有两位警察奋勇摘下大檐帽,遮住她胸前。还有些人揪住她的头发,扭住她的胳臂,拿个黑布口袋要往她头上套,她在奋力挣扎。后来同志们又把她放开了。考虑到国际影响,我认为这是对的。对外宣传的口径,是我们俩犯下罪行的,天良发现,自愿挨一顿鞭子,用皮肉抵偿国家财产的损失和别人的鼻梁,这样说很好,但惟一的问题是国家要我们的皮肉干什么。我和蓝毛衣就是按这个口径进行。过了一会,她走到台上,朝四面招手、飞吻,但是身前总是有两个人,举着大檐帽。然后就被带去挨鞭子。我知道上级对我们很重视,从外省请来了好几个赶大车老把式,还反复操练过,所以一鞭子就把她打得像猫一样悲鸣。这个过程相对比较快,因为蓝行身体很棒,只晕了一次,而且用水一泼就过来。后来她破口大骂,和宣传口径配合不上了,这样一来只好速战速决,赶紧把她解决掉。等到抽我时,架子上还热乎着哪。我挨打时紧贴在她的体温上,这种体温在某种程度上抵消了疼痛。假如没有这种抵消作用,我就是死定了。 

对于我们挨鞭子的事,有必要补充一点:作为一个前美术工作者——或者按南方的说法,作为一个美术从业员,我认为自己在受鞭刑时很难看。假如不是看录相,我还不知自己上身长下身短,更不知自己手臂是那样的长;在台上举起双手向观众致意进,简直像双鹤齐唳。除此这外,我身上没什么肉,却有极复杂的线条:肋骨、锁骨、胸骨等等,从正面看,就如从底下看一只土鳖虫。应该有人举着大檐帽遮在我胸前,但偏不来遮。挨打时我就如土鳖受到炙烤,越打背越弓,最后简直缩成了一个球。而且我一声也没吭。而蓝毛衣受刑就很好看,她肉体丰满,挣扎有力,惨呼声声,给人以精神上的震撼。受刑后,信件从全世界飞来,堆在医院的门厅里。不管是男是女,都是向她求爱,让我离好远点。因此,谁可爱谁不可爱,谁表现好谁表现坏,昭然若揭。但是数盲们认为我的表现比蓝毛衣还好,真是糊涂油蒙了心了。 



2 

我以为挨了鞭子之后所有的事就算结了呢,现在知道没这么简单。上级让我谈受鞭刑的认识,谈好了再出院。我觉得这事很古怪,住院是因为我有伤,现在我拄着棍能走路了,还住在医院里干什么。有什么要谈的,等我上了班再谈也可。上级说:这里条件很好嘛,你为什么要出院?我说我想上班。他们就说:我们认为你不必上班,就住在医院里吧。因为他认为这是对你好,所以就不听你申辩,只有对你坏时才让你申辩,但是申辩又没有用。 

我说我要出院想上班是真的,虽然听上去有点难以想象。我听说我前妻辞掉了市府的位子,回技术部工作了,像这样的事近十年不曾有过一起,但是现在每天都有好几十起。虽然领导上没让她当常务副部长,但是部里人叫她老大姐。这使我发了疯地想出院回到部里去。这个鬼医院不准探视,也逃不出去,比监狱还监狱。我对数盲说,你们是不是想等我养好了再抽几鞭子?不要拖拖拉拉,现在就抽好了。他们说绝不是的,只是要请我谈谈认识。我已经谈过了(上一节就是),以为他们看到那样的认识会把我放出去。数盲说,那样认识是不行的,还要再进一步。他妈的,不知往哪里进。说实在的,挨了一顿鞭子,我对世界的认识是进了一步,但是我知道把它来不是很恰当,尤其是谈给数盲去听。 

数盲们一会儿说我受刑表现很好,一会儿又说,应该再抽我几鞭子才好,简直把人搞糊涂了。他们说我表现好,我就说:谢谢。他们说要再抽我,我就问:什么时候抽?他们目瞪口呆,接不下话茬。这说明这些话都不是认真说的,换言之,是废话。至于蓝毛衣是表现,他们一致认为是恶劣之极,但是谁也不说要抽她。据说有几位数盲看抽她时发了心脏病,这是她裸露身体受鞭之过。这件事不足为奇,他们想看到的是抽我。蓝毛衣是另一个节目,不是给他们看的——放错频道了。 

女之鞭刑时必须要露出肉体,但是电视上不能有女人的肉体,这是个两难命题。所以听说现在有了这样一种做法:在受刑前,先在她身上涂一层迷彩,涂得哪是乳房,哪是屁股,全都看不出来。但是这又引出了另一个问题:涂了迷彩后,她在哪里也看不大清。所以现在进口了热像仪供掌鞭人使用。但是还有一个问题,就是在热像仪上看不清谁是谁,除此之外,车把式也不够聪明,操作不了热像仪,所以经常把警卫打着,但是这个问题已经很小了。 

至于我表现不好的地方,是当众亲吻了蓝毛衣的乳房。我的态度是,反正亲都亲过了,你看怎么办吧。我的认识就是这样的。顺便再说一句:数盲们把我除名了,我现在不是老大哥了。现在让我谈认识,谈好了放我出国。但我一点也不想出国。既不在技术部工作,也不是老大哥,我还出国干什么。 

他们让蓝毛衣出院了,理由是她表现不好,还比了她延长实习期的处分。这对她没有什么,我看她乐意在技术部里干,但是对我就很严重。我现在被转到一个单间里,除了送饭的老太太,谁也不让进来。假如蓝毛衣在,她会打进来,我还能有人说说话。现在除了拿录音机来听我认识的数盲,我谁也见不着了。我这一辈子从来没有这种经历,所以我脾气变得很坏。 

有位数盲对我说:你想想,你为什么会住在这里?我们又为什么把你从技术部除名?我说实话:我不知道。和我拐弯抹角地说话是没有用的,除非你是想露一手幽默感。但是众所周知,数盲没有幽默感。等他走了以后,我想:他这不是暗示我得了数盲症吧?假如是这样,这小子就有了幽默感啦。 



3 

在X架上,最能感觉自己是个造型艺术家,有丰富的空间想象力。比方说,有一鞭是斜着下来的,你马上变成两块硬面锅盔,或者是cheese cake,对接在一起。假如有鞭横着抽在腰眼上,就会觉得上半身冲天而起,自己有四米多高。假如鞭子是竖直地抽下来,你就会觉得自己像含露的芙蓉,冉冉开放。每一鞭的感觉都不一样,这是因为每一鞭都换个把式,每个把式鞭打的概念都不一样——一样的是他们都是农村来的,痛恨我们,说我们在城里吃的好住的好,不好好干活还闹事,就是该揍——疼痛也在变化,一开始像个硕大的章鱼,紧紧吸在胸前,后来就变得轻飘飘,像个幽灵,像一缕黑烟。到了这个程度,就快不行了。我这样说,数盲们本该很高兴。但是他们不高兴——这些比方他们听不懂。 

蓝毛衣挨抽的感觉肯定和我大不一样。本来该抽脊梁,却常常歪到屁股上。因为这个缘故,受刑之后刚把她放下来,她就冲到车把式面前,挨个儿啐人家,一连啐了三个人,才晕死过去,被人抬走了。年轻人就是身体好。我被放开时,像水银一样往地上出溜,就地抢救了一阵,才能爬起来。这就是我挨抽的认识,可以断言,不是数盲爱听的那种。 

以下的认识,数盲们大概也不爱听。而我这样谈,是因为我已经烦透了。当我露出一身骨头,站到台上向大家致意时,有一种投错胎转错世的感觉。假设有位数盲光着脊梁腆着大肚子到了台上,低头找不到肚脐眼,也会有这种感觉,因为谁生下来也不是为了挨鞭子呀。后来人家用皮绳捆着我手腕往架子上吊(那帮家伙手真狠,把我下巴颏撞破了)让我的光板胸膛体会到X形架的厚重和蓝毛衣的体温,这时候我抬着头看到头顶棕黄色的烟云——万籁无声。此时在我视野里,只有一个血迹斑斑的X开架的上半部,还有楔形黄色的天空,万籁无声,还有背上冷嗖嗖的,时间停住了。你说这是在干吗呢?我不知别人会怎么想,反正我此生体验到的一切荒诞,在此时达到了顶峰。 

数盲们说,我们花了宝贵的外汇进口了鞭子,开了万人大会向全国转播,市长副市长都讲了话,难首,就是为了让你体会到这些?这是巧妙的发问,但也属于我此生体会到的长吁短叹荒诞中的一种。所谓外汇、万人大会等等,都是为了铺垫数盲们的殷切期望和拳拳爱心,而我,渺小的王二,怎敢不感动?我的回答是:你不妨把我想象得更渺小,就说我是个分子,物理学证明,分子有分子的轨道——假如说我不配,那就说我是个原子,原子也有轨道,更小的东西更有轨道,凡是东西必有轨道 ——你去把你的期望和爱心投到分子上面,看看可能把它从轨道上移动分毫?不管怎么说,挨鞭子的是我,认识是我的事。我的认识还没说完呢。我主,此生体会到的一切荒诞,都在鞭刑架上达到了顶峰。这就是说,我觉得一切都不对头。不是一般的不对头,而是彻头彻尾的不对头。 

数盲们要我说明什么叫不对头,我能想到的一切比方都和数学有关,比方说你在证一道数学题,证出了一些触目惊心的结论:三角形内角和有720度、四方形是圆的等等;此时就会觉得不以头。但是数盲早把数学全忘了,所以就说不明白。这件事说明会讲话不等于会思维。数盲们作大报告,就如坐在马桶上放松括约肌,思维根本来不及。事实上思维就是分辨对头和不对头,而数盲就是学会了如何作报告而忘记了如何思维。我的这些认识都是说给会思维的人听的。我认为我们该做的事是把一切已知的事都想明白,然后再去解偏微分方程不迟。现在我就能想出件不对头的事,是有关蓝毛衣的:又要抽人家,又不让人家露肉,这对不对。假如这样想,就会发现世界上根本没有两难命题,只有从根上就不对的事而已。我被绑在架子上等着挨鞭子时,就觉得从根上都不对。假如这事不是发生在我身上,我就不会这样想。 



4 

小的时候,我哥哥告诉我,这世界上有种东西叫做“荒唐”,它就像关节疼,有时厉害,有时轻微,但是始终不可断绝。但是我八九岁时哪儿都不疼,所以就耸耸肩,表示不能想象。现在我身上疼的地方可多了,所以认为它是个很好的比方。我小的时候,听到“形势一片大好”、“前途是光明的,道路是曲折的”之类的话,只觉得它是一些话而已,绝不会像我哥哥那样笑得打跌。人是不会在八岁时就体会到什么是荒唐的,但像我这样一直到了四十八岁,挨了一顿鞭子才明白,就实在是太晚了。 

我在X形架上感到的荒唐是这样的:眼前这个世界不真实,它没有上点土方像是真的,倒像是谁编出来的故事——一个乌托邦。刚这样想了,背上就挨了一鞭子,疼得发疯 ——假如你想知道什么是疼得发疯,就找个电钻在牙上钻一下子——这时候我不禁口出怨言:妈妈的,你让我怎样理解才对!在冥冥中得到了回音:你怎么理解都不对,这就叫荒唐!像这样的鬼话,数盲们看了以后一定气得要死。假如真是这样,我的目的就达到了。他们抽了我一顿,还让我谈认识。谈了很多次,却说不懂我的意思。 

我对荒唐的理解是这样的:它和疼痛大有关系。我们的生活一直在疼痛之中,但在一般条件下疼得不厉害,不足于以发人深省。就以我哥哥来说,去插队(挨饿),得了关节炎,他都不觉得有什么。直到关节炎发展到了心脏病,做手术,可巧那一回上级要求做个针刺麻醉的手术给外宾看,就把他挑上了。领导上要求始终面带微笑,他做到了。但是事的告诉我:针刺一点作用没有,完全是干拉。拉到要翻白眼时,大夫说:病人挺不住了,上麻药吧。领导上却说:念段毛主席语录给他听—— 这是“文化革命”里的事。我哥哥始终微笑着,是怕领导说:这小子做怪相,甭给他做手术了——就这样开着胸晾在手术台上,肯定比疼还糟。做完手术后,他告诉我,有荒唐这种事,但我不懂。砸过碱、关过小号、被保安开过瓢后,还是不懂。等到吊上了架子,挨了一鞭子才懂了。那是一种直接威胁生命的剧痛,根本挺不住的,可是我被吊在那里还有十一鞭子等着你哪,你说往哪里跑吧。由此就得到了疼痛的真意:你的生命受到了威胁;轻度的疼痛是威胁的开始,中度的是威胁严重,等到要命的疼时,已经无路可逃了。 

我住医院的时候,他们发现了我的日记本,拿去研究了一番,又还给我了,还问我以前的日记哪里去了。以前我是不记日记的,原因就是怕数盲们看见。现在我变了主意,不但记日记,还把它放在数盲们能看见的地方。如果有话不说,就是帮助他们掩饰荒唐。在这本日记上,数盲在幽默感有两个传统来源(见“三、蓝毛衣\&我前妻”)“数盲和生殖器”处打了个大问号。据我所知,保安员同志们的幽默感也有两个来源,“眼镜”和生殖器。眼镜就是我们,这提示了幽默感从何而来。当你发现有什么人和东西比你聪明,你莫奈他(它)何时,就会开怀大笑。我们比保安员聪明,这是不争的事实,而生殖器经我们聪明也是不争的事实。它最知道自己要什么,一次都不会搞错。当然,最聪明的是数盲,,他们不但知道自己要什么,而且都能得到。数盲最伟大的地方,就是能够理解,并且利用荒唐。因为他们如此聪明,就觉得那东西蠢得很,一点都不逗了。 

有位数盲警告我说,我的认识很危险。这就是说,我已经和易燃易爆品列入一类了。危险的东西应该由上级来掌握,这就是说,我再也别想从医院里出去了。 

有关“危险”这件事,我现在是这么看的:假如有什么东西对他们有用处的话,数盲就说:这有危险!说了以后,它果真就有了危险——谁敢来拿就会挨顿揍——当然,这种危险是对我们而言。我不明白,我对他们有什么用处。我一个糟老头子,一条腿也被打坏了,走路都得拄拐。就算他们是同性恋,这个玩笑也开得过分了。 

除此之外,我什么都能够解释通了。当然,危险的定义还要拓宽一些。队了对他们有用的东西,对他们危险的东西也在内。比方说,有魅力的女人,比方说,我前妻,其实对他们毫无用处,但是我们和她们在一起时什么却敢干,所以对他们有危险,要赶紧从我们这里调走。鬼聪明的男人,比方说小徐,也有危险,假如不吸收他入伙,就会把什么都揭穿。最重要的一点是:没有什么对我们是有危险的,甚至连鞭刑都不危险。活到这个份上,还有什么可怕的呢。 

我还要说,数盲把一切有危险的东西都拿走了,也就拿去了活下去的理由。等到明白了这一点,我们就会有最大的危险性——这是对他们而言。这就是说,干什么事都要有个限度——物极必反。 



5 

有一件事我始终不明白,就是女人为什么不得数盲症。 

他们把我从医院里放出来了。我也不知为什么。回到技术部一看,一个人都没有。有种直觉叫我到海滨广场去看看。这不是什么好兆头……

\section{七 结局}

老大哥王二在我受鞭刑时死掉了——我是他的前妻,这本日记现在在我手里。他住院时,领导上说他得了数盲症,但是我不信。他不会得数盲症,因为他是天生的老大哥,永远不会改变。我要说,他对危险的态度过于乐观了——他以为受过鞭刑之后,这世上再没有对他危险的事了——他就是因此死掉了。女人不得数盲症的原因很简单——得了没有好处,所以很少有人得,得了也只会受人耻笑。老左就有数盲症,她跑到我这里来,我们三个女人:我、蓝毛衣、老左,哭了一顿,纪念这个男人。对于爱上他这一点,我从来就没有后悔过,今后也不会爱上别的人了。当一个人爱另外一个人时,后者受鞭刑,鞭子就会打到前者心上。我是这样,他也是这样。惟一的区别是,我的心脏比他的好。现在我人活着,心已死。这是一件好事,我可以平静地干我该干的事了。 


(编者注:窝子录入完成于2004年8月15日。未校对稿。)

