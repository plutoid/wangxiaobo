\chapter{万寿寺(后面部分)}

\section{第五章}

第一节 

早上我来上班时,看到我的办公室门敞开着。在我的办公桌──也就是那张香案──上,放着我的工作计划。除此之外,还有一股马尿的气味──这是领导身上的味,他总抽最便宜的烟卷,把这种气味留在一切他到过的地方。我记得自己把计划认真地修改过,交上去了,现在它又跑了回来,使我大吃一惊,生怕现存不多的记忆也出了问题。打开那个白纸册子,看到我在那页上打的补丁还在,这是个好现象。但有一个更坏的现象:我精心拟定、体现了高尚情操的三个题目上,被人打上了大红叉子。这三个题目是:《老佛爷性事考》、《历史脐带考》、《万寿寺考》。在这三个大叉子边上,还有四个字的批语:“一派胡言!”这使我感到莫名的委屈。虽然这三个题目可能还不够崇高,但已是我能想到的最崇高的题目了。再说,就是这样的题目我也可能做不了。我真不知道领导的意图是什么,也许,他们想要我的命?我尽量达观地看待这件事,但还是难免愤恨。整整一上午都在愤恨中过去了。 

将近中午时,白衣女人走进我的房子,见到我的样子,就把眉头挑了起来:怎么了你?我尽量心平气和地答道:没怎么。没怎么。她掏出个小镜来,说道:自己照照吧。镜子里是一张愤怒的灰色人脸,除了咬牙切齿,还是斗鸡眼──我还不知道自己有内斜视的毛病,在心情不好时尤为显著。这下可糟了,别人可以一目了然地看到我的内心──看来我该戴副墨镜。然后她在屋里走动,看到了桌上的表格,就大笑起来:原来是因为这个!你这家伙呀,没气性就不要耍无赖,气不了别人,老是气着你自己。现在我知道了自己是个鼠肚鸡肠的人,这使我很伤心,但又感到冤枉。我拟这三个题目不是想耍无赖、气领导,而是一本正经的。 

我的故事重新开始时,一切如前所述。那个小妓女的房前,是一片绿色的世界。绿竹封锁了天空,门前长满了绿草,就是那片空地上,也长满了青苔。时而有般落的笋壳、枯萎的竹叶飘落在地,在地上破碎地陈列着,老妓女马上就把它们扫掉。因为这个缘故,天黑以后,门前就会变成一片纯蓝色的世界,这个女孩讨厌蓝色。她常在空地上走来走去,把每棵竹子都摇一摇,不但摇下了枯萎的叶子,连半枯萎的也摇了下来。她觉得这没有什么,叶子可以在地下继续枯萎。但等她刚一走回房子,拉上拉门,老妓女就走了出来,提着木板钉成的簸箕,拿着竹枝编成的短条帚,在空地上走上一圈,把所有的叶子(包括全枯萎的和半枯萎的)通通扫掉,然后嘟嘟囔囔地走回去。在做这件事时,老妓女赤裸着身体、躬着腰,在绿色之中留下白色的反差,所以像一只四肢着地的北极熊。然后,小妓女又跑出去摇竹子,老妓女又跑出去扫地,并且嘟囔得越来越厉害。这个小妓女因为年轻,而且天性快乐,所以把这当做一种游戏,没有想到这会给自己招来杀身之祸。 

在我新写的故事里,也有一帮刺客受老妓女的雇佣,来到了凤凰寨里。但老妓女请他们来,不是要杀薛嵩,而是要杀死红线。这个故事的正确之处在于:同性相斥,异性相吸。老妓女既是女人,就不该要杀男人,应该是想杀女人才对。她给刺客先生们的任务是:红线必须杀死,薛嵩务必生擒。假如你说,刺客先生是男人不是女人,他们有自己的主见,会以为薛嵩必须杀死,红线务必生擒;那么你就是站在了正确的一面。更正确的意见是:老妓女请人杀红线,应该请女人来杀,女人更可靠。你说得对。老妓女这样干了一次,那个正确的刺客的脑袋已经被挂起来了。这说明请刺客时,不仅要找可靠的人,还要注意对方的业务水平。起初,老妓女想请一个可靠的人,就请来了那位漂亮的女刺客,但她业务水平低,没有杀着红线,只砍掉了薛嵩半个耳朵,还把自己的命送掉了。后来,她又请来了声誉最高的刺客,但这些人却很不可靠。 

因为这个缘故,等到漫长的一天过去,蓝色降临时,就会有一个纯蓝色的男人从空地上走过。此人头很大,还打着缠头,像一个深海里的水母,飘飘摇摇地过去,走进老妓女的屋子。从门缝里看到这个景象以后,那女孩明白了老妓女为什么要扫地──倘若地上有枯枝败叶,人脚踩上就会有很大的响动,小妓女听到之后,就知道隔壁来了不明身份的男人,而老妓女不愿意让人知道──这是女孩的理解。实际上来的不是嫖客而是刺客头子,来和老妓女商讨杀薛嵩的事;所以这是一个很大的误解。因为老去摇叶子,老太太觉得她是薛嵩的眼线,所以决定在杀薛嵩的同时把她也杀掉。因为这个缘故,这个小妓女也落到了死定了的地步,这使她感觉很坏。 

那天晚上她睡在门口,把拉门留了一个缝,把一只眼睛留在门缝里。这样,就是睡着了也能看见。夜里她在睡梦中看到有二十多个蓝色的人经过,醒来时很是吃惊,自己扳指头算了一遍,不禁脱口惊叹道:我的妈呀,这老太太不要命了!她爬起来,想去看看热闹,就溜出了门,溜上了人家的走廊。在她面前的是一个从里面被照亮的纸拉门。当她伸出舌头,想要舔破窗户纸时,被一只大手捂住了嘴,另有一只大手,箍住了她的脖子,更多的手正在她身上摸着,这些手又冷又湿,掌心似有些粘液。这女孩最怕这个。虽然如此,她还挣扎着回了一下头,看清了身后那些蓝色的人影,小声嘀咕了一句:全是那老东西害的!,才无可奈何地晕过去了。 

2 

中午吃饭时,我对那白衣女人发起了牢骚:领导在我新拟的题目上打叉,叉掉《老佛爷性事考》我无话可说;为什么把《历史脐带考》也叉掉?他根本就不知我在说什么!前面所引的旧稿里已经提到,历史的脐带是一条软掉的鸡巴,这是很隐晦的暗语,从字面上看不出来的……那白衣女人沉下脸来说:这就要怪你自己长了一张驴嘴,什么话都到处去说!这话让我机灵:原来我这么没城府,与直肠子驴相仿。我连忙压低嗓音问:我对领导也说了历史的脐带啦?她哼了一声说:还用和他说!别人就不会打小报告了?说起来就该咬你一口,只要能招女孩笑一笑,你能把自己家祖坟都揭开……此时我出了一身冷汗:我不但是直肠子驴,还是好色之徒!等我问起是谁出卖了我时,她却不肯说:我不来挑拨离间,你自己打听去吧……我不需要去打听了,因为我已经下定了决心,今后除她之外,什么女人我都绝不多看一眼,更不会和她们说话。但我还有一个问题:《万寿寺考》是我顺笔写上的,写时觉得挺逗,但不知逗在哪里。我把这问题也提了出来,那白衣女人不回答,只是用筷子敲碗,厉声喝道:讨厌!讨厌!我在吃饭!我也不敢再问了。但我知道“万寿寺”也是个典故,这典故是我发明的,人人都知道,只有我不知道。 

在我新写的故事里,我决心把线索集中在那小妓女的身上。从外表看,她和红线很像,都长着棕色的身体,远看带点绿色,近看才不绿;但从内心来看就很不一样。主要的区别是,她还没被某一个男人盘算住,天真烂漫,心在所有的男人身上;当然,蓝色的男人例外。这种颜色的人她都送给了老妓女。这就是说,除了反对蓝色,她的内心是一片空白。 

这个女孩子最怕冷和粘,因为她害怕蛇和青蛙。但是红线却不怕冷血动物,她常用左手拿住青蛙的腿,右手捏住一条蛇的脖子,让右手的蛇吞掉左手的青蛙。再把蛇嘴捏开,把青蛙拖出来。这样折腾上十几次,再把他们放开。以后蛇一见青蛙就倒胃;而青蛙见到了蛇,就狂怒起来,跳到它头上去撒尿。所以,假如用冷冰冰的手去摸红线,不仅不能吓晕红线,还会被她在睾丸上踢上一脚。但红线也并非无懈可击:她最怕耗子。用热烘烘、毛扎扎的手去摸她,就能把她吓晕。但小妓女却不怕耗子。她把耗子视为一种美味,尤其是活着的。她养了一箱小白鼠,常常抓出一只,用蜜抹遍它的全身,然后拎着尾巴把这可怜的小动物放到嘴里,作为每餐前的开胃菜。假如用热烘烘的手去摸小妓女,她不仅不怕,还会转身咬掉你的鼻子。这两个女孩有时拿同性恋作为一种游戏,但她们互相不信任。红线总要问:你今天吃没吃耗子?小妓女撒谎道:好久没吃了,我的嘴是干净的。她也问红线:你今天有没有用手去拿蛇?红线说:拿过,可我洗手了。我的手也是干净的。其实她根本就没洗手。她们互相欺骗,像一对真正的恋人一样。不知为什么,那些刺客做好了一切准备,要用凉手去摸小妓(已经得逞了),还要用热手去摸红线(尚未得逞)。这就是说,他们在寨子里有内线,知道些内幕消息。 

每个女孩都有弱点,当男人不知道这个弱点时,她才是安全的。但假如她的弱点为男人所知,必是因为另一个女人的出卖。小妓女在晕过去之前,认为自己是被老妓女出卖了。这种想法当然是很有道理。被人摸晕以后,她就被人捆了起来,嘴里塞了一只臭袜子,抬进莱妓女的屋里。醒来以后,她就在心里唠叨道:妈的,怎么会死在她手里?真是讨厌死了! 

在我的记忆中,夜有不同的颜色。有些夜是紫色的,星星和月亮就变得惨白。有的夜是透明的淡绿色,星星和月亮都是玫瑰色的。最惨不忍睹的夜才是如烟的蓝色,星星和月亮像一些涂上去的黄油漆。在这样的夜里摸上别人家的走廊去偷听,本身就是个荒唐的主意;因此丧命更是荒诞不经。自从到了湘西,小妓女就没有穿过衣服。现在她觉得穿着衣服死掉比较有尊严。她有一件白色的晨衣,长度只及大腿,镶着红边,还配有一条细细的红腰带,她要穿着这件衣服死去。她还有一个干净的木棉枕头,从来没有用过,她想要被这个枕头闷死。具体的方法是这样的:由一个强壮的男人躺在地上,她再躺在此人身上。此人紧紧抱住她,箍住她的双手,另一人手持枕头来闷死她,而且这两个男人都不能是蓝的。就是这样的死法,她也不觉得太有意思。 

3 

在我自己的故事里,我刚刚遭人出卖,被领导用红笔打了三个大叉子,虽然没有被人捆倒,没有被人往嘴里塞上臭袜子,更谈不到死的问题,但心情很沮丧。按那白衣女人的说法,我是被女孩出卖的。这使我更加痛苦。这种痛苦不在小妓女的痛苦之下。逮住了小妓女,那些刺客就出发去杀红线。在他们出发前,老妓女特别提醒他们,这个小贼婆很有点厉害。那些人听了哈哈大笑,说道:一个小贼婆有什么了不起?嘻嘻哈哈地走了出去,未加注意,结果是吃了大亏。此后,只剩下小妓女和老妓女呆在一个房子里,那个女孩就开始起鸡皮疙瘩,心里想着:糟了,这回落到贞节女人的手里啦。 

妓女这种职业似乎谈不上贞节,这种看法只在一般情况下是对的。有些妓女最讲贞节,老妓女就是这种妓女中的一个。她从来不看着男人的眼睛说话,总是看着他的脚说话;而且在他面前总是四肢着地的爬。据她自己说,干了这么多年,从来没见过男人的生殖器官。当然,她也承认,有时免不了用手去拿。但她还说:用手拿和用眼看,就是贞节不贞节的区别。老妓女说,她有一位师姐,因为看到了那个东西,就上吊自杀了。上吊之前还把自己的眼睛挖掉了。有眼睛的人在拿东西时总禁不住要看看,但拿这样东西时又要扼杀这种冲动。所以还不如戴个墨镜。顺便说一句,老妓女就有这么一副墨镜,是烟水晶制成的,镶在银框子上。假如把镜片磨磨就好了,但是没有磨,因为水晶太硬,难以加工。所谓镜片,只是两块六棱的晶体。这墨镜戴在鼻子上,整个人看上去像穿山甲。当然,她本人的修为很深,已经用不着这副眼睛,所以也不用再装成穿山甲了。 

另一件重要的事是决不要吃豆子,也不要喝凉水,以免在男人面前放屁。她还有一位师妹,在男人面前放了一次响屁,也上吊而死,上吊之前还用个木塞子把自己钉住。总而言之,老妓女有很多师姐妹,都已经上吊自杀了。她有很多经验教训,还有很多规矩,执行起来坚定不移。按照她的说法,妓女这个行业,简直像毕达哥拉斯学派一样,有很多清规戒律。顺便说一句,毕达哥拉斯学派也不准吃豆子,也不知是不是为了防止放屁。但我必须补充说,只要没有男人在场,老妓女就任何规矩都不遵循。她赤身裸体,打响嗝、放响屁;用长长的指甲抓搔自己的身体来解痒,与此同时,侧着头,闭着眼,从下面的嘴角流出口水──也就是俗称哈喇子的那种东西。更难看的是她拿把剃头刀,岔开腿坐在走廊上,看似要剖腹自杀,其实在刮阴毛。那女孩把这些事讲给男人们听,自然招致那老妓女最深的仇恨。其实她本心是善良的,也尊敬前辈,只是想和老太太开个玩笑。但从结果来看,这个玩笑不开更好。 

综上所述不难看出,在唐朝,妓女这个行业分为两派。老妓女所属的那一派是学院派,严谨、认真,有很多清规戒律,努力追求着真善美。这不是什么坏事,人生在世,不管做着什么事,总该有所追求。另一派则是小妓女所属的自由派,主张自由奔放、回归自然,率性而行。我觉得回归自然也不是坏事。身为作者,对笔下的人物应该做到不偏不倚。但我偏向自由派,假如有自由派的史学,一定会认为,《老佛爷性事考》、《历史脐带考》都是史学成就。不管怎么说吧,这段说明总算解释了老妓女为什么要收拾小妓女──这是一种门派之争。那位白衣女人看到这里,微微一笑道:瞎扯什么呀!就把稿子放下来,说道:走吧,你表弟在等我们呢。对这些故事,她没说好,也没说不好,我也不知该因此而满意呢,还是该失望。 

白衣女人后来指出,我有措辞不当的毛病。凡我指为学院派者,都是一些很不像我的人。凡我指为自由派者,都是气质上像我的人。她说得很有道理,但对我毫无帮助。因为我对自己的气质一无所知。古人虽说人贵有自知之明,但这种要求对一个只保有两天记忆的人来说,未免太过分。所以,我只好请求读者原谅我辞不达意的毛病。 

在谈我表弟的事之前,我想把小妓女的故事讲完。如前所述,小妓女在男人面前很随便。她属于那种没有贞节的自由派妓女,和有贞节的学院派妓女住在一起多有不便。她和薛嵩说了好几次,想要搬家。但薛嵩总说:凑合凑合罢,没时间给你造房子。 

那个老妓女也说过,她不想看到小妓女,要薛嵩在两座房子之间造个板障。薛嵩也说,凑合凑合吧,我忙不过来呀!以前薛嵩可不是这个样子,根本不需要别人说话,他自己就会找上门去,问对方有什么活要做;他会精心地给小妓女设计新家,陶土和木头造成模型,几经修改,直到用户满意,然后动工制作;他还会用上等的楠木造出老妓女要的板障,再用腻子勾缝,打磨得精光,在上面用彩色绘出树木和风景,使人在撞上以前根本看不出有板障。不但是妓女,寨子里每一个人都发现少了一台永动机,整个寨子少了心脏──因为薛嵩迷上了红线,不再工作,所以没有人建造住房、修筑水道、建造运送柴火的索道。作为没有贞节的女人,小妓女还能凑合着过;而老妓女则活得一点体面都没有了。原来薛嵩造了一台抓痒痒的机器,用风力驱动四十个木头牙轮,背上痒了可以往上蹭蹭,现在坏了,薛嵩也不来修。原来薛嵩造了一架可以自由转动的聚光灯,灯架上还有一面镜子,供老妓女在室内修饰自己之用。现在也转不动了,老妓女的一切隐私活动只好到光天化日下来进行。这就使老妓女的贞节几乎沦为笑柄。 

假如不赶紧想点办法,那就只有自杀一途了。 

寨子里没有了薛嵩的服务,就显出学院派的不利之处。这个妓女流派只擅长琴棋书画,对于谋生的知识一向少学。举例来说,风力搔痒机坏了,那个小妓女就全不顾体面,拿擦脚的浮石去擦背。这种不优雅的举动把老妓女几乎气到两眼翻白;而她自己也痒得要发疯,却找不到地方蹭。供水的管道坏了,小妓女自会去提水,而那个老妓女则只会把水桶放在屋檐下面,然后默默祈祷,指望天上下雨,送下一些水来。至于送柴的索道损坏,对小妓女毫无影响。随便拣些枯枝败叶就是柴火。就是这样的事,老妓女也不会,她只会从园子里割下一棵新鲜蔬菜,拿到走廊上去,希望能把一头到处游荡的老水牛招来。把它招来不是目的,目的是希望它在门前屙屎。牛粪在干燥之后,是一种绝妙的燃料。很不幸的是,那些水牛中有良心的不多,往往吃了菜却不肯屙屎。当老妓女指着水牛屁股破口大骂时,小妓女就在走廊上笑得打滚──像这样幸灾乐祸,自然会招来杀身之祸…… 

4 

我和我表弟媳是初次见面。那女孩长得圆头圆脸,鼻子上也有几粒斑点。和我说话时,她一刻不停地扭着身体。这是一种异域风情,并不讨厌。她很可能属于不拘小节的自由派。她不会说中国话,我不会说泰国话,互相讲了几句英文。她和我表弟讲潮汕话,而我表弟却不是潮汕人。她自己也不是潮汕人,但泰国潮汕人多,大家都会讲几句潮汕话。小妓女和薛嵩相识之处,也遇到了这个问题。他不会讲广东话,她不会讲陕西话。于是大家都去学习苗语,以便沟通。虽然会说英语,我也想学几句潮汕话。只可惜这种语言除了和表弟媳攀谈,再没有什么用处了。 

我表弟现在很有钱,衣冠楚楚,隐隐透着点暴发户的气焰。从表面上看,他很尊敬我,站在饭店门口等我们,还短着舌头叫道:表嫂,很漂亮啦!接下来的话就招人讨厌:他问我们怎么来的。混帐东西,我们当然是挤公共汽车来的!我觉得自己身为表哥,有骂表弟的资格。但白衣女人不等我开口就说:BUS上不挤,很快就到了。我表弟对我们很客气,但对我的表弟媳就很坏,朝她大吼大叫,那女孩静静地听着,不和他吵。我能理解她的心情:今天请你的亲戚,只好让你一些,让你作一回一家之主。等把我们往包厢里让时,我表弟却管不住自己的肛门,放了个响屁。那女孩朝我伸伸舌头,微微一笑。我很喜欢她的这个笑容,但又怕她因此招来杀身之祸。 

在凤凰寨里,等到刺客们走远,那个老妓女想要动手杀掉小妓女。所以等到现在,是因为她觉得不在男人面前杀人,似乎也是贞节的一部分。她要除掉本行里的一个败类,妓女队伍中的一个害群之马。干这件事时,她没有一丝一毫的犹豫,只是有点不在行。她找出了自己的匕首,笨手笨脚地在人家身上比划开了。她虽不常杀人,对此事也有点概念,知道应该一刀捅进对方心窝里。问题是:哪儿是心窝。开头她以为胸口的正中是心窝,拿手指按了以后,才知道那里是胸骨,恐怕扎不动。后来她想到心脏是长在左边,用手去推女孩的左乳房;把它按到一边去,发现下面是肋骨。这骨头虽然软些,但她也怕扎不动。然后她又想从肚子上下手,从下面挑近心脏的所在。就这样摸摸弄弄,女孩的皮肤上小米似的斑点越来越密了。后来,她猛地坐了起来,把臭袜子吐了出来,说道:别摸好吗!我肠子里都长鸡皮疙瘩了!老妓女吃了一惊,匕首掉在地上。过了很久,才问了一句:肠子里能起鸡皮疙瘩吗?那女孩毅然答道:当然能!等我屙出屎来你就看到了!老妓女闻言又吃一惊,暗自说道:好粗鄙的语言啊。这小婊子看来真是不能不杀。她的决心很大,而且是越来越大。但怎么杀始终是个问题。 

别的不说,怎么把臭袜子塞回女孩嘴里就是个很大的难题。她试了好几次,每次都被对方咬了手。那女孩还说:慢着,我有话问你。为什么要杀我?老妓女说道:因为你不守妇道,是我们这行的败类。女孩沉吟道:果然是为这个。但是你呢?勾结男人杀害同行姐妹,难道你不是败类?这话很有力量,足以使老妓女瞠目结舌。但那老女人及时地丢下刀子,把耳朵堵上了。 

我知道把老妓女要杀小妓女的事和我表弟请我们吃饭的事混在一起讲不够妥当,但又没有别的办法,因为这些故事是我在餐桌上想出来的。小妓女的样子就像我的表弟媳,老妓女就像我表弟。那个老妓女和一切道德卫道士一样,惯于训斥人,但不惯于和人说理。我表弟就常对表弟媳嚷嚷。而那女孩和一切反道德的人一样,惯于和人说理,却不惯于训斥别人。表弟媳总是和颜悦色地回答表弟的喝斥。 

老妓女和小妓女常有冲突,每次都是老妓女发起,却无法收场。举例来说,只要她们同时出现在两个不同的回廊上,那老妓女就会注视着地面,用宏亮的嗓音漫声吟哦道:阴毛该刮刮了,在男人面前,总要像个样子啊。老妓女就这样挑起了道德争论,她却不知如何来收场。那女孩马上反唇相讥道:请教大姐,为什么刮掉阴毛就像样子?她马上就无话可答。其实明路就在眼前,只消说,这是讲卫生啊,小妓女就会被折服;除非她愿意承认自己就是不讲卫生。但老妓女只是想:这小婊子竟敢反驳我!就此气得发抖,转身就回屋去了。相反,假如是小妓女在走廊上说:别刮那些毛,在男人面前总要像个样子啊;那老妓女也会收起剃刀、蓄起阴毛。她们之间的冲突其实与阴毛无关,只与对待道德训诫的态度有关。顺便说一句,我表弟和表弟媳在争些什么,我一句也没听懂,好像不是争论阴毛的问题。但从表弟的样子来看,只要我们一走,他就要把表弟媳杀死。 

5 

不管怎么说吧,老妓女已经决定杀小妓女,而且决心不可动摇。但小妓女还不甘心,她把反驳老妓女的话说了好几遍,还故意一字一字,鼓唇作势,想让她听不见也能看见。但老妓女只做没听见也没看见,心里却在想反驳的道理,终于想好了,就把手从耳朵上放下来,说道:小婊子;你既是败类,就不是同行姐妹。我杀你也不是败类。说毕,把刀抢到手里,上前来杀小妓女。要不是小妓女嘴快,就被她杀掉了。她马上想到一句反驳的话:不对,不对,我既不是同行姐妹,就和你不是一类,如何能算是败类。所以和你还是一类。老妓女一听话头不对,赶紧丢下刀子,把耳朵又捂上了。我老婆后来评论道,这一段像金庸小说里的某种俗套,但我不这样想。学院派总是拘泥于俗套,这是他们的弱点,可供利用。可惜自由派和学院派斗嘴,虽然可以占到一些口舌上的便宜,但无法改善自己的地位,因为刀把子捏在人家的手里。 

这故事还有另一种讲法,没有这么复杂。在这种讲法里,老妓女没有和小妓女废话,小妓女也没有把臭袜子吐出来。前者只想把后者拖出房子去杀,以防血污了地板;她可没想到这件事办起来这么难。起初她想从小妓女上半身下手来拖,没想到那女孩像条刚钓出水面的鱼一样狂翻乱滚,一头撞在她鼻子上;撞得她觉得油盐酱醋一起从口鼻里往外淌──这当然是个比方,她嘴里没有淌出酱油和醋,实际上,淌出来的是血。后来,她又打算从脚的方向下手。这回女孩比较文静,仰卧在地板上,把脚往天上举,等老妓女走近了,猛一脚把她从房间里蹬出去。天明时,刺客们吃了败仗从薛嵩那里回来时,发现老妓女的房子外观有很大的改变;纸窗、纸门、纸墙壁上,到处留下人形的窟窿。说话之间,老妓女又一次从房子里摔了出来,栽倒在地下。这使那些刺客很是惊讶,赞叹道:你这是干嘛呢?她答道:我要把那小婊子拖出去杀掉;他们就说:是吗?看不出是你拖她呀。那些人都被土蜂螫得红肿,在蓝颜色的烘托下,变成紫色的了。 

我应该从头说起这个小妓女。在我心中,这个女孩是这个样子:在她棕色的脸中央,鼻头上有几粒细碎的斑点,眼睛大得惊人。当你见到她时,心情会很好,分手后很快就会忘记了。如果你说像这样的人很适合被杀死,我就要声明,这不是我的本意。总而言之,她和老妓女一起跟薛嵩来到湘西,同为凤凰寨的创始人,地位没有尊卑之分。从老妓女的立场出发,杀掉一位创始人,逮住另一位创始人,剩下一个创始人,就是她自己。此后她就是凤凰寨的当然主人。现在这种写法比前无疑更为正确。 

天明时分,小妓女被老妓女和一群蓝色的刺客围在凤凰寨的中心。那些人既没杀掉红线,也没逮住薛嵩,就想把她杀掉充数。那女孩听到了他们的打算,叹了一口气说:好吧,我同意。看来我想不同意也不行了。可你们也该让我知道知道,薛嵩和红线到底怎么样了。从昨天晚上开始,她既没有见到红线,又没见到薛嵩;而前者是她的朋友,后者是她的恋人。关心他们的下落,是理所当然的事情。连老妓女带刺客头子,都以为这种要求是合情合理的。但他们也不知红线和薛嵩到底怎样了。既然不知道,也就不能杀掉她。 

现在可以说说那个女孩为什么讨厌蓝色。在湘西的草地上,蓝色如烟,往事也如烟。清晨时分,被露水打湿的草地是一片殷蓝,直伸到天际;此时天空是灰蒙蒙的。这种蓝色和薄暮时寨子上空悬挂的炊烟相仿。诚然,正午时的天空也是蓝色,此时平静的水面上反光也是蓝色,但这两种蓝色就没有人注意。因此就造成了这样的局面:只有如烟的殷蓝色才叫作蓝色,别的颜色都不叫蓝色。每天早上,小妓女双手环抱于胸,走到蓝色的草地上,此时往事在她心里交织着。因为她讨厌往事,所以也讨厌蓝色。既然她讨厌回忆往事,又何必到草地上来──这一点我也无法解释。我能够解释的只是蓝色为什么可鄙:我们领导总穿蓝色制服。后来,她躺在老妓女家里的地板上时,就是这样想的:既然被蓝色如烟的人逮住,就会得到一个蓝色如烟的死。具体的说,可能是这样:她被带到门外,浑身涂满了蓝颜色,头朝下地栽进一个铁皮桶,里面盛着蓝墨水。此后她就从现在消失,回到往事…… 

按照以前留下的线索,那些刺客和老妓女要杀掉这个小妓女,她以一种就范的态度对他们说:好吧,随你们的便罢;但你们得告诉我,薛嵩和红线怎样了;但她又摆出了个不肯就范的姿势,整个身体呈S形。在S形的顶端是她捆在一处的两只脚,然后是她的小腿和蜷着的膝盖。大腿和屁股朝反方向折了回来。这个S形的底部是她的整个躯体。她拿出这个姿势来,是准备用脚蹬人。当然,这个姿势有点不够优雅,因为羞处露在外面,朝向她想蹬的那个人。老妓女训斥她说:怎么能这样!在男人面前总要像个样子!但那小妓女毅然答道:我就不像样子了,你能怎么样吧!不告诉我薛嵩怎样了,我就不让你们杀!当然,那些刺客可以一拥而上,把这小妓女揪住,像对付一条鳝鱼一样,把她蜷着的身体拉开,一刀砍掉她的脑袋。但那些刺客觉得这样做不够得体:大家都是有教养的人,人家不让杀怎么能杀呢──除此之外,刺客都是男人,对女人总要让着一些。但要告诉她薛嵩怎样了,又是不可能的事,因为他们也不知道。当然,他们也可以撒句谎,说:他们俩都被我们杀掉了;但这又是不可能的事,大家都是有教养的人,怎么能说慌呢。刺客头子不好意思的笑了一下说:好吧,那就暂时不杀你。小妓女很高兴,说道:谢谢!就放下腿,翻身坐了起来。当然,现在是杀掉她的大好时机,可以猛冲过去,把她一刀杀死。但那刺客头子又觉得这样做不够得体。所以,他们就没杀掉那个小妓女。 



第二节 

我该把和表弟吃饭的事做一了结。吃饭时他把手放在桌子上。这只右手很小,又肥又厚,靠近手掌的指节上长了一些毛。人家说,长这样的手是有福的。这种福分表现在他戴的金戒指上:他有四根手指戴有又宽又厚的金戒指,我毫不怀疑戒指是真金的,只怀疑假如我们不来,他会不会把这些戒指全戴上──当小姐给他斟酒时,他用手指在桌面上敲着。饭后,我开始犹豫:既然我是表哥,是不是该我付账……但我表弟毫不犹豫,掏出一张信用卡来。是VISA卡,卡上是美元。后来,我们走到马路上,表弟和他太太要回王府饭店,我开始盘算他们该坐哪路车──要知道,路径繁多,既可以乘地铁,也可以乘电车、公共汽车、双层巴士(特一路),假如不怕绕路的话,还可以乘市郊车。但我表弟毫不犹豫,拦住了一辆黄色的出租车,递给司机一张百元大票,大声大气地说:送我表哥表嫂到学院路。我对他的果决由衷佩服。回到家里,我们并排坐在床上。我老婆也堕入了沉思之中。后来,她拥抱了我,在我耳畔说道:我只喜欢你。然后她凉凉的小手就向下搜索过来。 

那天夜里,那个自称是我老婆的女人在床上陈列她白色、修长的身躯。起初,是我环绕着这个身躯,后来则是这个身躯在环绕我。对于一位自己不了解的女士,只能说这么多。我始终在犹豫之中,好像在下一局棋。她说,我只喜欢你。这就是说,她不喜欢我表弟。但是似乎存在着喜欢我表弟的可能性。也许,他们以前认识?或者我表弟追求过她?在这方面存在着无穷多种可能性。这么多可能性马上就把我绕糊涂了。 

因为写到了一些邪恶的人:老妓女、刺客头子,现在我觉得薛嵩比较可爱了。白衣女人再次重申她只爱我,我的心情也好多了。薛嵩留着可爱的板寸头,手很小,而且手背上很有肉。这是过去的薛嵩。照小妓女的记忆,那时候他像个可爱的小老鼠,不知什么时候就会从地缝里钻出来,出现在她的面前,兴高采烈地说道:我要和你做爱!就把她扑倒在地,带来一种热烘烘的亲切感觉。他的男根呈深棕色,好像涂了油一样有光泽。这种事情不应被视为苟合,而应视为同派学兄学妹之间的切磋技艺。小妓女对这种切磋感到幸福,唯一使她不满的是:薛嵩老到老妓女那里去。每当她撅起嘴来时,薛嵩就热情洋溢地说道:我们要作大事,要团结,不要有门户之见嘛!此后就更加热情地把她扑倒在地,使她忘掉心中的不满……以后她就忘掉了门派分歧,主动叫老妓女为大姐;在此之前她称对方为老婊子,老破鞋,还有一个称呼,用了个很粗俗的字眼,和逼迫的逼同音不同字。只可惜老妓女已经恨了她,还是要把她杀死。所以,在被捆倒在地下时,小妓女暗暗后悔,觉得多叫了几声大姐,少叫了几次老逼,自己吃了大亏。 

过去的薛嵩和现在的薛嵩很不一样,现在的薛嵩长了一头长发,乱蓬蓬地绞结着,肤色灰暗,颧骨突出,眼睛又大又凸出,茫然地瞪着。他的手又大又粗糙,身上很凉,心事重重;但一点都不是傻呵呵的;他的男根呈死灰色,毫无光泽,好像一条死蛇。照小妓女的看法,他变成这样,完全要怪红线。但红线是她的朋友,她不好意思和她翻脸。 

在凤凰寨里,薛嵩发生了很多变化,小妓女却始终如一,总是笑嘻嘻地走来走去。见到男人,就屈起右手的中指,随手一弹,弹到他的龟头上,就算打过了招呼。这一指弹到了薛嵩的龟头上,他才会猛醒,注视着那小妓女,说道:晚上我去看你。那女孩就赶回家去,收拾房子,准备茶水,用一块橘子皮把牙齿擦得洁白如玉。然后就坐下等待薛嵩,但薛嵩总是不来。一直要等到过了一个星期才会来,坐在走廊说:我红线答应过前天晚上来看你。要是别的女人,准会用脏水泼他,但小妓女不会。只要薛嵩来了,她就满足了。 

过去的薛嵩还有种傻呵呵的劲头,一心要在湘西做一番事业。在旅途中,他一直在设计未来的凤凰城,做了很多模型。有一个是铜的,他假设当地多铜,所以以为凤凰寨要用铜来制作。假如纯用铜太耗费,就用石块建造墙壁,用铜水来勾缝。另一个模型是铁的。有一些凤凰寨是一组高高的塔楼,这些塔楼要用花岗石建造。另一些凤凰寨是一组四方形的碉楼,这些碉楼要用石灰岩来建造。最平淡无奇的设计是一片楠木的楼房,所有的木料都要在明矾水里泡过,可以防火。到了地方一看,这里只是一片瘠薄的红土地,什么都不出产,还在闹白蚁。凤凰寨未经建造时是一片杂树和竹子的林子,建造之后仍是这样的林子。但这没有扫薛嵩的兴,他说:好啊,好啊。我们有了一座生态城市了。他拿出工具,给大家建造生态房屋。这种工作也让他心满意足。棕色皮肤,小手小脚,这是我表弟小时的模样。至于他的男根什么样子,我却没有见过。这该去问我的表弟媳。 

过去的薛嵩还有种傻呵呵的劲头,一心要在湘西做一番事业。在旅途中,他一直在设计未来的凤凰城,做了很多模型。有一个是铜的,他假设当地多铜,所以以为凤凰寨要用铜来制作。假如纯用铜太耗费,就用石块建造墙壁,用铜水来勾缝。另一个模型是铁的。有一些凤凰寨是一组高高的塔楼,这些塔楼要用花岗石建造。另一些凤凰寨是一组四方形的碉楼,这些碉楼要用石灰岩来建造。最平淡无奇的设计是一片楠木的楼房,所有的木料都要在明矾水里泡过,可以防火。到了地方一看,这里只是一片瘠薄的红土地,什么都不出产,还在闹白蚁。凤凰寨未经建造时是一片杂树和竹子的林子,建造之后仍是这样的林子。但这没有扫薛嵩的兴,他说:好啊,好啊。我们有了一座生态城市了。他拿出工具,给大家建造生态房屋。这种工作也让他心满意足。棕色皮肤,小手小脚,这是我表弟小时的模样。至于他的男根什么样子,我却没有见过。这该去问我的表弟媳。 

2 

到现在为止,我还没有说到那些蓝色的刺客怎样行刺──这些刺客都属于学院派。在一个蓝色的夜里,趁着黄色的月光,他们摸进薛嵩的院子;也就是说,走进了一位自由派能工巧匠的内心。开头,他们走在铺着黄色砂石的小径上,两面是黑色的树林。后来就看到一堵厚木板钉成的墙。这些木板都刨过、打磨过,用榫头连接,在月光下像一堵磨砖对缝的墙。这本是一种工艺上的奇迹,但是出于自由派之手,就不值得赞美。中间是一两扇木头门。在这座门前,刺客们屏住了呼吸。他们排成两排,握紧了手中的兵器,让一位有专长的同伙从中过去,去撬那扇门。对付这种门有很多方法,一种是用刀尖从门缝里插进去,把门闸拨开。但这个方法不能用,两个门扇对得很紧,简直没有缝。另一种是用铁棍把门扇从框上摘下来。这一手也不能用,因为门安得很结实。第三种办法要用千斤顶,但没有带。第四种方法是用火烧,但会惊动薛嵩。这位刺客因此花了些时间……后来他低声叫道:他妈的。因为这门既没有锁,也没有反插住,一推就开了。 

在这座门里,是一道厚木板铺成的小径,小径像栈道一样有双桁架支撑。那些刺客就像一队夜间在水边觅食的鹭鸶,行走在小径上。在小径尽头,又是一道竹篱笆墙,有一座竹板门。吸取了上回的教训,走在前面的刺客径直去推门。那门“呀”的一声开了。有感于这个声音,刺客头子发出一道口令:“往后传,悄声”。这句话就朝后传去,越传声音越大,到最后简直就像叫喊。如果复述头头的声音不大,就显不出头头的威严。刺客头子对手下人的喧嚣不满,就又传出一道口令:“谁敢高声就宰了他!”但手下人有感于这道命令的威严,就更大声地复述着,把半个凤凰寨的人都吵起来了。刺客头子在狂怒中吼道:操你妈,都闭嘴!这句骂人话被数十人同声复述,隆隆地滚过了夜空。然后,这些小人物又因为辱骂了领导而自行掌嘴。学院派可能不是这样粗鄙,但我只能这样来写。因为如你所知,我没当过学院派。 

后来他们又走过了圆竹子扎成的小径,这条路就像一道乡间的小桥。小桥的尽头是一道草扎的墙,像草房的屋顶一样;有草排做成的门。门后的小路用芦花和草穗铺成,走在上面很舒服。然后又出现了木头墙和木头门……有一位刺客抱怨道:娘的,这么多的门。对此,我有一种解释:作为一位能工巧匠,薛嵩喜欢造门,而且常常忘记自己已经造了多少门,铺设了多少小径,所以他家里有无数的门和小径。还有一种解释是:薛嵩的院子里一共只有三道门,三条小径。一条是进来的路,一条是家里的路,还有一条是出去的路。这些刺客没有走对,正在他院里转圈子。按照前一种解释,那些刺客应该耐着性子穿过所有的门,走完全部小径;这些刺客就在做这件事──这样的夜间漫步很有趣,但迷了路就不好了。现在的情形就很像迷了路,所以他们也怀疑后一种解释可能成真;所以一面走,一面在路边上搜索,终于在黑暗的林间看到了一座房子的轮廓。 

有一件事情必须提到,那就是月光比日光短命得多。他们出来时,到处是黄色的月光,现在一点也没有了,蓝色的夜变成了黑色的。还有一件事必须提到:在夜里,路上比别的地方明亮,所以一定要走路。总而言之,那些刺客发现了路边有座房子,就把它团团围住,冲了进去,然后就惊呆了。只见在黑暗中有一对眼睛,发着蓝色的晶光;眼睛中间的距离足有一尺多。那间房子里充满了腐草的气味。有人不禁赞叹道:我的妈,红线原来是这样。但是刺客头子很镇定,他说了一声:我们走,就领头退了出去。他手下的人问道:怎么回事?怎么回事?难道我们不杀红线了?他就感到很气愤,还觉得手下人太笨。他是对的。大家早就该明白,刚才冲进了牛棚,所看到的是水牛的眼睛。假如红线的眼睛是这个样子,那就难以匹敌;照人的尺寸来衡量,长这样眼睛的人身高大概有三丈八尺,眼珠子有碗口大;还不知是谁杀谁呢。后来他们又冲进了猪圈、鸡窝和鸭棚,到处都找不到红线,也找不到薛嵩。后来冲进了土蜂窝,被螫了一顿,就这样回来了。这就产生了一个问题,薛嵩和红线到哪里去了。有一种解释是这样的:他们哪里都没去,就住在大家的头顶上。薛嵩造了一座高脚房子,支撑在一些柱子上。那条竹子小径就从高脚房底下蜿蜒通过。那些刺客倒是发现了一些柱子,但是以为它们是树。这房子在白天很容易看到,到了夜里就看不到了。 

3 

按照这种说法,薛嵩和红线住在离地很远的、木板构成的平面上。在白天,爬上一道梯子,从一个四方的窟窿里穿过四寸厚的木板,就能到达薛嵩所住的地方。这里有一座空中花园,有四个四方形的花坛,呈田字形排列。每天早上。薛嵩都到花坛中央去迎接林间的雾气,同时发现,树林变矮了。参天的巨木变成了灌木,修长的竹子变成了芦苇丛,就连漫天的迷雾也变成了只及膝盖的低雾。薛嵩对此很是满意,就拿起工具开始工作。首先,他要给所有的木头打一遍蜡。这些木头既要防水,又要防虫,既要防腐,又要防蛀;这可不大容易;打一遍蜡要三个小时,然后还要腰疼。如果你说薛嵩花了很大功夫给自己找罪来受,我倒没有什么意见,一面给木板打蜡,一面他还在想,给这片平台再加上一层,这一层要像剧院的包厢环绕花园,中间留下一个天井,不要挡住花园所需的阳光,假如你据此以为薛嵩的罪还没有受够,我也没有不同意见。 

在花园的左前方,也就是来宾入口附近,有一座水车,像一个巨大的车轮矗立在那里,薛嵩用它往平台上汲水。遗憾的是这水车转起来很重,这倒不是因为它造得不好,而是因为汲程很高。薛嵩在水车边贴了张标语,用水车的口吻写着“顺手转我一下”,这就是说,他想利用来宾的劳动力。他自己住在花园后面一座小小的和式房子里,睡在硬木板上,铺着一张薄薄的草席,枕一个四方形的硬木枕。只有过最简朴的生活,才能保持工作的动力。他喝的是清水,吃芭蕉叶里包着的小包米饭。而红线则住在右面一个大亭子里。这个亭子同时又是一个升降平台,红线的柚木笼子就放在平台上。她坐在笼子中央磕瓜子,从一个黑色的釉罐里取出瓜子,把瓜子皮磕在一个白罐子里。后来她叫道:薛嵩!薛嵩!薛嵩就奔了过来,手里还拿着修剪花草的剪子。他把盛瓜子皮的罐子取出来,又放进去一个空罐。与此同时,红线坐在棕垫子上磕瓜子,偏着头看薛嵩,终于忍不住说道:你进不进来?薛嵩眯着眼看红线(因为总做精细的工作,他已经得了近视眼),看遍了她棕色、有光泽的身体,觉得她真漂亮。他感到性的冲动,但又抑制了自己,说道:等忙完了就进来。红线叹了一口气,说道:好吧,你把我放下去。于是薛嵩搬动了把手,把红线和她的笼子放下去,降落在车座上。然后他又去忙自己的事。他的大手上满是松香和焊锡的烫伤,因为他总在焊东西。比方说,焊铁皮灯罩,或是白铁烟筒。这座平台上有一个小小的厨房,他想把炊烟排到远远的地方,不要污染眼前的环境。他还以为红线乘着车子在下面菜园里工作,其实远不是这样。她从笼子下面的活门里钻了出去,找小妓女去聊大天。对此不宜横加责备,因为她还是个孩子嘛──假如这故事是这样的,就可以解释夜里那些刺客走进薛嵩家以后,为什么会觉得那么黑。这是因为他们走在人家的地基底下。不要说是黑夜,就是在白天,那地方也相当的黑。 

这故事还有另一种讲法。那些刺客在薛嵩家里乱闯,访问过牛栏、猪圈之后,忽然听见一个女孩的声音在说:“大叔,大叔!你门找谁?”他们瞪大了眼睛往四下看,但什么也看不见,因为实在太黑。后来,那女孩用责备的口气说:你们点个亮嘛。但刺客们却犯起了犹豫。众所周知,刺客不喜欢明火执杖。刺客头子想了一下,猛地拍了一下大腿,说道:对!早就该点火!我们人多。这就是说,既然人多,就该喜欢明火执杖。我很喜欢这个刺客头子,因为他有较高的智力──学院派的人一贯如此。 

4 

那天夜里,刺客头子让手下人点上火──他们随身携带着盛在竹筒里的火煤,还有小巧的松脂火把,这是走夜路的人必备之物──看到就在他们身边有一个很大的木笼子,简直伸手可及,但在没有亮的时候,他们以为这是一垛柴火。在笼子中央坐着一个小姑娘。她的项上、手上和脚上,各带了一个木枷。假如仔细观察,就会发现这三个木枷都是心形的。脖子上的那一个非常小巧,就如一件饰物,手上和足上的都非常平滑,是爱情的象征。这些东西是胡桃木做的,打了蜡。薛嵩之所以不用柚木,是因为柚木不多,已经不够用了。刺客头子看得没有那么仔细,他觉得很气愤:把一个女孩子关在笼子里,还把她锁住,这太过分了;也没问问她是谁,就下令道:把她放出来! 

他手下的人扑向笼边的栅栏,用手去摇撼。正如这位小姑娘(她就是红线)微笑着指出的那样:这没用,结实着呢。于是,他们决定用刀。红线一看到刀,就说:别动!不准砍!这是我的东西!但有人已经砍了一下,留下了一道刀痕。不管柚木怎么硬,都硬不过刀。还不等他砍第二刀,红线就撮唇打了一个唿哨。然后,随着一阵不详的嗡嗡声,无数黄蜂从空而降。这一点和前一个故事讲的一样。所不同的是:这个黄蜂窝就在这伙刺客的头上,只是因为高,他们看不到。红线叫他们点起火来,黄蜂受到火光和烟雾的扰动,全都很气愤,围着球形的蜂窝团团乱转,有些已经飞了起来;但那些刺客也没看见。这也不怪他们,谁没事老往天上看。等到红线打个唿哨,黄蜂就一起下来螫人。这一回倒是看到了,但已经有点晚了。那些黄蜂专螫刺客,不螫红线,因为她身上亮闪闪的涂了一层蜜蜡。涂这种东西有两种好处,第一:涂了皮肤好。第二,黄蜂遇到她时,以为是自己的表弟蜜蜂,对她就特别友好。在这个故事里,红线相当狡猾。她让刺客大叔们点火,完全是有意的。她看到这伙人在黑地里鬼鬼祟祟,就知道他们不怀好意。同时又嗅出他们身上没涂蜜蜡,就想到要让黄蜂去叮他们。虽然如此,也不能说她做得不对。因为他们是来杀她的,让想杀自己的人吃点苦头,难道不是天经地义吗? 

有关薛嵩的家,另有一种说法是这样的:它是一片柚木的大陆,可以在八根木柱上升降──当然,是通过一套极复杂的机构,有滑轮、缆绳、连杆、齿轮,还有蜗轮、蜗杆等等组成。薛嵩在自己门前转动一个轮子,轮子带动整套机构,他的花园和房子,连同地基,就缓缓地升起来。当然,速度极慢,绝不是人眼可以看出的。要连转三天三夜,才能把整个院子升到离地三丈的柱顶。把它降下来相对要容易得多,但薛嵩轻易不肯把它降下来,怕再升起来太困难。根据这个说法,那天晚上,刺客们摸进薛嵩的家,马上就发现在平地上有个孤零零的笼子,红线睡在里面。他们点亮了灯笼火把,把笼子团团围住,但找不到入口,就问红线说:你是怎么进去的?这个小女孩回答得很干脆:不告诉你们。她坐在笼子中央的蒲团上磕瓜子,离每一边都很远,这样,想从栅栏缝里用刀来砍她就是徒劳的了。那些刺客互相抱怨,为什么不带条长枪来,以便用枪从栅栏缝里刺她;与此同时,他们还抓住栅栏使劲摇撼。红线则轻描淡写地说道:省点劲罢。柚木的,结实着哪。那些刺客看到要杀的对象近在咫尺却杀不到,全都气坏了。有人就用刀去砍柚木栅栏,才砍了一下,红线就变了脸色。打了一个唿哨。砍到第二下,红线尖叫了起来:薛嵩!薛嵩!有人在他们头顶上应道:干什么?红线叫道:把房子放下来!于是随着一阵可怕的嘎嘎声,刺客们头顶上的天就平拍了下来。反应快的刺客及时侧了一下头,被砸得头破血流,摔倒在地。反应慢的继续直愣愣地站着,脑袋就被拍进腔子里,腔子又被拍到胯下,只剩下下半身,继续直愣愣地站着。 

对于这件事,必须补充说,房子从头顶上砸下来,对红线却是安全的,因为那柚木房基上有个四方的洞,正好是严丝合缝嵌在笼子上。按照红线的设想,这房子应该一直降到地面上,把所有的刺客都拍进地里。但实际上,它降到齐腰高的地方就停住了。红线喝道:怎么回事?薛嵩不好意思地说:卡住了。滑轨有毛病,总是这样……红线说:真没用!她纵身跃起,甩开了身上的枷锁(假如有的话),从笼顶上一个暗口钻了出去,赶去帮薛嵩修理机器。那些倒在地上未死的刺客就叹息道:原来入口是在顶上的啊。 

根据这种说法,那些刺客回到老妓女门前时,头上也是红肿着的,但不是蜂螫的,而是砸的了。根据这种说法,刺客头子不是刺客里最聪明的人。他手下有个人比他还要聪明,当他们倒在地下时,那个人拉了头子一下说:咱们就这样躺着,等人家修好机器来砸死我们吗?刺客头子很不满意这个说法,但也找不出反驳的理由,就下了撤退的命令。他们从地基和地面之间爬出来以后,那人又出了个很好的主意:咱们现在摸回去,谅他没有第二层房子来砸我们。刺客头子不喜欢别人再给他出主意,就朝他呲出了满嘴雪白的牙。于是这些人就这样退走了。 

假如这队刺客照这人的主意摸回去,就会看到薛嵩和红线打着火把,全神贯注地修理那些复杂的机器,这故事后来的发展也很不一样了。认真地想一想,我认为那些刺客会悄悄地摸上去,把红线抓住一刀杀掉,把薛嵩抓走,交给老妓女,让他在老妓女的监督之下,给凤凰寨造房子,修上下水道。这种说法我虽然不喜欢,但它也是一种待穷尽的可能。 

第三节 

第二天早上,我们又来上班。把上面提到的故事写在纸上之后,我又开始冥思苦想起来。昨天的事情说明,在暴躁、易怒的外表下,我心柔弱,多愁善感,就像那个小妓女。说起来难听,但我对此并无不满。本着这种态度,我开始为领导考虑,有我这样的下属真够他一呛:报上来的研究题目尽在那些部位,怎么向上级交待呢。我现在想了起来,我住院时他来医院看过我,提来了一袋去年的红香蕉苹果。那种水果拿在手里轻飘飘的,倒像是胖大海。这种果子我当然不吃,送给了一位农村来的病友,叫他拿回去喂猪──不知猪对这些苹果有何评价。但不管怎么说罢,他来看过我,还带来了礼物……现在我是真心要拟个过得去的研究题目,但怎么也拟不出。我觉得自己可以原谅:我刚被车撞过。所以,我把题目放下,又去写故事了。 

塞万提斯说,堂吉诃德所爱的达辛尼亚,是托波索地方腌猪肉的第一把好手。薛嵩也是湘西地方烧玻璃的第一把好手。假如他想在第二年春天烧玻璃,头年秋天就到山上去割一大车蓑草,晾干以后,交给寨子里一个女人,叫她拿草当柴来烧,还给她一些坛子。这样她就有了一车白来的干草,但她只能把它烧掉,不能派别的用场──虽然蓑草还可以用来作蓑衣,还要把烧成的灰都收集起来。这样,经过一冬,薛嵩就得到很多洁白如玉的灰,都盛在坛子里。这种灰有很大的碱性──他得到了烧玻璃的第一种原料,就是碱。他还到河滩上采来最洁白的砂子,这是第二种原料,到山上采集最好的长石,这是第三种原料,还有第四和第五种原料,恕我不一一尽数,搜集齐了一起放到坩锅里去烧;然后把烧融的玻璃液倒到熔化的锡上冷却──一块平板玻璃就这样制好了。这块玻璃有时厚,有时薄,这是因为薛嵩虽然很注意原料的配比,却总忘掉它的总量。分量多了,玻璃液就多,浇出的玻璃就厚,反之则薄。假如太薄,玻璃上会有星星点点的圆洞,就如擀面擀薄了的景象。这种玻璃使薛嵩大为欢喜。等到玻璃凉了,他把它拿起来,看着这些洞哈哈大笑。这种玻璃没楞没角,像块面饼。多数是方形,也有梯形和三角形的。薛嵩自会给玻璃配上窗框,给窗框配上房子,这些房子有些是三角形,有些是梯形,依玻璃的形状而定。这种玻璃蓝里透绿,透过它往外看,就如置身于深水里。 

薛嵩还是打造铜器的第一把高手,他把铜皮放在木头上,用木榔头敲。随着这些敲击,铜皮弯曲起来,逐渐成形。他再用铁榔头砸出边来,用锡焊好,一个铜夜壶就造好了。他还是制造陶器、浇铸铁器、编造竹器的高手,最优秀的皮匠和厨师。至于作木匠,他到湘西才开始学,也已成了高手。总而言之,他有无数手艺,多到他自己也记不清,像这样的人当然很有用,只是要把他盯紧一些,否则他会胡闹。在烧制玻璃时,他发现粘稠的玻璃液可以拉出丝来,就五迷三道地想用这种丝来造衣服。这样平板玻璃就造不成──全被他拉成了丝。而这种衣服是透明的,穿上以后伤风败俗。让他造夜壶也要小心,稍不留神,夜壶就不见了,变成一个铜人。铜皮下面有猾轮,有肠衣做的弦牵动,还有一颗发条心脏,这样就可以到处乱跑,还能说几句简单的话。虽然还有夜壶的功能,但很讨人嫌。黑更半夜的,它每隔一小时就跑到你面前来滴滴嘟嘟地说:请撒尿。根本不管你想不想尿。老妓女就有这样一把夜壶,她很不喜欢,把它放在柜子里,它就在柜子里乱转,在柜子里滴滴嘟嘟地说,请撒尿。好在他还有从善如流的好处,你不喜欢这把夜壶,他马上就去打另一把,直到你满意为止。不过,这都是他迷上红线以前的事。现在你再找他做事,他总是说:我忙,等下回吧。 

根据现在这种说法,老妓女迷恋薛嵩,不只是迷恋他巧夺天工的手艺,还迷恋他勤勤恳恳的态度。以前,他来看老妓女,看到她因年迈走了形的身体,就说:大妈,你要是信得过我,就让我给你做个整形手术。拉拉脸皮,垫垫乳房,我觉得没什么难的。老妓女不肯,这是因为她觉得人活到什么年龄就该有什么样子,不想做手术;还因为学院派不喜欢这类雕虫小技;但最本质的原因是:薛嵩没做过这种手术。这家伙胆子大得很,只在猫屁眼上练了两次,就敢给人割痔疮。后来,他一面和老妓女做爱,一面拨弄她瘪水袋似的乳房,说道:越看我越觉得有把握。要是别人胆敢这样不敬,老妓女就要用大嘴巴抽他。但是薛嵩就不同了。有一阵子,老妓女真的考虑要做这个手术。这是因为薛嵩小手小脚,长着棕色发亮的皮肤。头上留着短发,脑后还有一络长发。老妓女喜欢他。既然喜欢,就该把身体交给他练练手。 

有关这位老妓女,我们已经说过,她总把阴毛剃得精光。她嘴上有些黄色的胡子,因为太软,用刀剃不掉。薛嵩给她做过一个拔毛器,原理是用一盏灯,加热一些松香,把胡子粘住,然后使松香冷凝,就可以拔下毛来(据我所知,屠宰厂就用这个原理给猪头退毛,直到发现松香有毒),现在坏了(确切地说,是没有松香了,也不知怎么往里加),老妓女只好用粉把胡子遮住,看上去像腿毛很重的人穿上了长统丝袜。有关这个拔毛器,还要补充说,薛嵩的一起作品都有太过复杂、难于操纵的毛病。如果不繁复,就不能体现自己是个能工巧匠。繁复本身却是个负担──我现在就陷入了这种困境…… 

2 

后来,透明把薛嵩逮住,给他套上枷锁,押着他去干活。因为薛嵩已有两年多不务正业,积压的工作很多。但只要押着他的人稍不注意,薛嵩就会脱开枷锁跑掉,跑到坟头上去凭吊红线,因为根据这种说法,红线已经死掉了。薛嵩经常跑掉,使老妓女很不高兴,虽然他不会跑远,而且总能在坟头上逮到,但老妓女害怕他在这段路上又会遇上一个小姑娘,从此再变得五迷三道。所以她就命令薛嵩造出更复杂的锁,把他自己锁住。造锁对能工巧匠来说,是一种挑战。薛嵩全心全意地投入这项工作。他造出了十二位数码锁,定时锁,还有用钥匙的锁,那钥匙有两寸宽,上面有无数的沟槽,完全无法复制。这些锁的图纸任何人看了都要头晕,它们还坚固无比,用巨斧都砍不开。但用来对付他自己,却毫无用处。他可以用铁丝捅开,也可以用竹棍捅开,甚至用草棍捅开这些锁。假如你让他得不到任何棍子,他还能用气把它吹开。老妓女以为他在耍花招,就直截了当地命令道:去造一把你自己打不开的锁。薛嵩接受了这个任务,他思考了三天三夜,既没有画图纸,也没有动手做。最后,他对老妓女说:大妈,这种锁我造不出来。老妓女说:胡扯!我不信你这么笨!此时她指的是薛嵩不会缺少造锁的聪明。后来她又说:我不信你有这么聪明!此时指的是薛嵩开锁的聪明。最后她说:我不信你这么刚好!这就是说,她不信薛嵩开锁的聪明正好胜过了造锁的聪明。实际上,聪明只有一种,用于开锁,就是开锁的聪明;用于造锁,就是造锁的聪明。薛嵩叹了一口气,摇了摇头,走开去做别的工作了。 

希腊先哲曾说:上坡和下坡是同一条路,善恶同体;上坡路反过来就是下坡,善反过来就是恶。薛嵩所拥有的,也是这样一种智慧。他设计一种机构时,同时也就设计了破解这种机构的方法──只消把这机构反过来想就得到了这种方法。在他那里,造一把自己打不开的锁,成了哲学问题。经过长时间的冥思苦索,他有了一个答案,但一直不想把它告诉老妓女。那就是:确实存在着一种锁,他能把它造出来,又让自己打不开,那就是实心的铁疙瘩。这种锁一旦锁上了,就再不能打开。作为一个能工巧匠,我痛恨这种设计。作为一个爱智慧的人,我痛恨这种智慧。因为它脱离了设计和智慧的范畴,属于另一个世界。 

后来,薛嵩把这个方案交给了老妓女,老妓女虽然毫无智慧,但马上就相信此案可行。此后,薛嵩又亲手做了一个铁壳,把锁铤装上,用坩锅烧开一锅铁水,在老妓女的监督下,把它浇在铁壳里。他就这样造了一把打不开的锁,完成了老妓女交给他的任务。锁是铁链的中枢,扣住了他自己的手脚。这样他迈不开腿,也抡不开手,既不能跑掉,也不能反抗,只能干活。对这个故事无须解释:自从红线死了以后,薛嵩已经心丧如死,巴不得像行尸走肉一样的活着。但作为讲故事的人,也就是我,尚须加以解释:这故事有一种特别的讨厌之处,那就是它有了寓意。而故事就是故事,不该有寓意。坦白地说,我犯了一个错误,违背了我自己的本意。既然如此,就该谈谈我有何寓意。这很明显,我是修历史的。我的寓意只能是历史。 

我现在想,在我写的小说定稿时,要把这一段删掉──既已有了这种打算,就可以肆无忌惮地写。在我看来,整个历史可以浓缩成一个场景:一位贤者坐在君王面前,君王问道:有没有一种方法,可以控制天下苍生?这位智者、夫子,或者叫作傻逼,为了炫耀他的聪明,就答道:有的。这就是控制大家的意志。说他是智者,是因为他确实有这种鬼聪明。说他是傻逼,是因为他忘记了自己也是天下苍生的一分子,自己害起自己来了。从那一天开始,不仅天下苍生尽被控制,连智慧也被控制。有意志的智慧坚挺着,既有用,又有趣,可以给人带来极大的快感;没有意志的智慧软塌塌的,除了充当历史的脐带,别无用场了……所谓学院派,就是被历史的脐带缠住的流派……照这个样子写下去,这篇小说会成为学术论文,充其量成为学院派的小说。幸亏在我的故事里,红线没有被刺客杀死,薛嵩也没有被老妓女逮住。我还有其它的可能性。这篇小说我还是作得了主的,作为自由派的坚定分子,我不容许本节这种可能发生。请相信,已经写到的一切足以使我惭愧。我远不是薛嵩那样勤勉工作的人。 

午后,万寿寺里升起了一片炎热的薄雾,响起了吵人的蝉鸣。我把写着的故事放到一边,又拿起了那份白色的表格,对着那三个红色的叉子想了半天;终于相信这三个题目里毫无崇高,根本就是个恶意的玩笑。假如我努力想出三个更崇高的题目,它们会是更恶毒的玩笑。总而言之,我所有崇高的努力都会导致最恶毒的玩笑。也许我该往相反的方向去想。于是我又撕了一张黄纸片,在上面写下三个最恶毒的玩笑:《唐代之精神文明建设考》、《宋代之精神文明建设考》、《元代之精神文明建设考》。所以说它们是最恶毒的玩笑,是因为我根本就不知道它们是怎样的东西,而且这世界上也不会有人知道。 

我把这张纸片贴到表格上,拿着它出了门。到对面配殿里找我们的领导,也就是那个戴蓝布制帽、穿蓝布制服、带有马尿气味的人,把这张表格交给他,与此同时,心中忐忑不安。生怕他会翻了脸打我……谁知他看了以后,把表格往抽屉里一锁,对我说道:早就该这样写!虽然已经对这个结果有一点预感,但我还是被惊呆了……顺便说一句,我以为最恶毒的玩笑是《当代之精神文明建设考》,因为它是最没有人懂得的陈词滥调,也许你能告诉我,这是否就是最崇高的题目?假如是的话,那么,最恶毒的努力带来的反而是崇高。这是怎么回事,我真的不懂了。 

3 

我终于从领导那里得到了一句赞许的话。但这话在我心中激起了最恶毒的仇恨。怀着这种心情,我把刺客们行刺薛嵩的经过重写了一遍:从前,有一群刺客去袭击薛嵩。午夜时分,他们摸进了薛嵩的家,摸进了这位能工巧匠的内心。他们的目的是杀死红线,把薛嵩抓走,交给雇主,就算是完成了任务。但是这个任务没有完成。这是这个故事不可改变的梗概。在这个梗概之下,对那些刺客来说,依然存在着种种可能性。 

举例来说,有一重可能是这样的:那些刺客摸到薛嵩家门口。那里有座木头门楼。打起火来一照,看到门楼上方挂了一块柚木的匾,上面用红油漆写了两个谦虚的隶字:“薛宅”。门的左侧钉了一块木牌,上面用红油漆歪歪斜斜地写着:“红线客居于此”,底下是一段苗文。据我所知,当时的苗文是一种象形文字。那段文字的第一个符号是一只鸟,仿佛是一只鸽子。第二个符号肯定是一条蛇。再后面是颗牛头。但你若说它是颗羊头,我也无法反对;随后是颗骷髅头,但也可能是个湖泊、一个茄子或是别的瓜果,或者是别的任何一种东西。底下还有些别的符号,因为太潦草,就完全无法形容,更不要说是辨认。据说苗文就是这样,头几个符号只要能读懂,后面就可以猜到,用不着写得太仔细。刺客里有一位饱学之士,他在火光下咬着手指,开始解读这些文字。很显然,这段苗文是红线所书。这第一个符号,也就是鸽子,是指她自己。按照汉族的读法,应该读作“奴家”、“贱妾”,或者“小女子”、“小贱人”之类。第二个字,也就是那条蛇,该刺客认为是男性生殖器的象征。虽然还不知怎么解释,但肯定不是个好意思。再往下怎么读,就很成问题。假如是牛头,就是好意思。要是羊头就是坏意思。总而言之,虽然是饱学之士,也没读懂红线写了些什么。这只能怪她写得太潦草了。这些刺客气壮山河地来杀人,却在门前被一片潦草的苗文难住,这很使他们气馁。很显然,这些刺客也属学院派。学院派的妓女请来的刺客,当然也是学院派。 

后来,那些刺客说道:不管她写的是什么,咱们冲进去。这种干净利落的态度虽然带有自由派的作风,却正是刺客们需要的……于是一脚踹开了门,呐喊一声杀进了薛嵩家里。随即就发现,好像是到了一个木板桥上,桥面下凹,这桥还有点飘飘忽忽的不甚牢靠——好像是座悬索桥,只是看不到悬索在哪里。那些刺客停了下来,经过简短的商议,认为既然身处险地,只有向前冲杀才是出路。于是大家呐喊一声向前冲去,冲了一阵,停下来一看,还在那座木桥上,而且还在桥面的最低点上。于是停下来商量,这一回得到的结论是:既然身在险地,还是速退为妙。于是呐喊一声,朝后冲去。又冲了许久,发现还在原地。然后又一次合计,又往前冲;停下来再合计,又往后冲。其实,他们根本不在桥上,而是在一个大木桶里。这只桶由一根轴担在空中,他们往前冲,桶就往前滚;往后冲就往后滚。前滚后滚的动力就是这些刺客本身的移动。薛嵩和红线远远看到了那只桶在滚,也不来干涉,只是觉得有趣。直到天明,桶缝里透进光来,刺客们才觉得不对,用刀把桶壁砍破钻了出来。此时大家的嗓子也喊哑了,腿也跑软了,自然没有兴趣继续前进,去杀红线、捉薛嵩,而是退了回去。按照这种说法,刺客们去杀红线,却冲进了一只木桶。如你所知,这只是众多可能中比较简单的一种。 

还有更复杂的可能性:薛嵩的家里是一座精心设计的迷宫,到处是十字路口、丁字路口、环形路口、立体交叉的路口,假如不是路口,就是死胡同。到处是墙壁,墙上却没有门。好不容易看到一扇门,呐喊一声冲进去,却落进了茅坑里。他们在里面瞎摸了一夜,终于从原路退了回来。总而言之,刺客们在薛嵩家里没有找到薛嵩,也没有找到红线,只带回了一大堆的感叹:这个薛嵩,简直是有毛病! 

薛嵩的家里还可能是一片湖泊,在水边停了几只小船。那些刺客上了船,顺着两边都是芦苇的水道撑起船来。从午夜到天明,从天明又撑到午夜,每个人都精疲力尽,饥肠辘辘。最后总算是回到了原来上船的地方。出于某种恶意,船上的篙、桨等等,全都难用得要命;后来才发现这些船具里都灌了铅,而且都灌在最不凑手的地方。那些水道的水也很浅,他们在烂泥里撑船——甚至可以说是在陆地上行船。有很多地方的芦苇是假的,水也是假的——是涂在地上的清漆,但在朦胧中看不出真假,就把船撑上了山,又撑了下来;连设计这个圈套的薛嵩也不得不佩服这些刺客的蛮力。在陆地上行舟当然很累,撑了这一圈船之后,每个人的手上都起了燎浆大泡,并且感到腰酸腿疼。在这种情况下,他们也没兴趣继续前进,去杀红线、逮薛嵩。总而言之,薛嵩是如此的诡计多端,假如没有一些他那些机关的情报,就没法把他逮住。所以,他们就回去拷问小妓女,想要问出些有价值的口供。我已经说过,这些刺客是不可靠的。所以他们还想拷问老妓女。如果可能,他们还想拷问一切人。作为这篇小说的作者,我知道一切情报。所以,我才是他们最想拷问的人。 

考虑各种可能性时,不应该把红线扣除在外。如前所述,她和各种各样的冷血动物都很有交情,养了很多青蛙、蜥蜴、毒蛇,还有癞蛤蟆。她让这些爬虫互相通婚,生出了各种千奇百怪的变种。当那些刺客冲到她面前时,她打开了一个竹篓,放出她的虾兵蟹将来:有没有脚的蜥蜴,长的像大头鱼,全靠身体的力量在地下一跳一蹦;有硕大无比的蟾蜍,腿却短得要命,长着三角脑袋,看上去有点像鳄鱼;有身材肥胖的眼镜蛇,长了一百条腿,所有的腿都在飞快地挪动,但因为腿太多,互相妨碍,身体移动得却不快;还有有毒的青蛙,嘴上长着角质的凸起,张开蜻蜓般的翅膀飞在空中。这种诡计决非学院派所为。很显然,红线也是自由派。假如一个深山里的苗族女孩也是学院派,只能说明学院派根本就不存在。所有这些妖魔鬼怪一起朝刺客们扑来,呲出了毒牙、喷射着毒液;吓得他们转身就跑。现在,他们很想找人打听一下,这个红线到底是个会妖术的女巫,还是仅仅患有精神病。假如是前者,他们就不想再去杀她;有妖术的人死掉以后会变成更加难缠的恶鬼,还不如不杀。假如是后者,就非杀她不可,因为他们这么多大男人,总不能被一个女疯子吓跑了。总而言之,最后的结果是,如果没有知情人领路,就找不到红线,也找不到薛嵩。我的故事再次开始就是这样的。而那位白亿女人则朝我厉声喝道:越编越不像样子了,你! 

\section{第六章}

第一节 

用不着睁开眼睛,我就知道来到了清晨;清晨的宁静和午夜不同。有个软软的东西触着我的身体,从喉头到胸膛,一路触下来;我想,这是她的双唇。还有些发丝沙沙地拂着身体的两侧。与此同时,我嗅到她的体味,就如苦涩的荷花;还能感到她在我腹部呼气,好像一团温暖的雾。我虽然喜欢,也感到恐惧,因为再往下的部位生得十分不雅。我害怕她去亲近那里。也许就是因为恐惧,那东西猛地竖起来了。她在上面拍了一下,喝道:讨厌!快起来!我翻身坐了起来,甩着沉重的脑袋,搞不清楚谁讨厌,是我还是它。 

在睁开眼睛之前,我知道自己发生了一种深刻的变化,但不是又一次失去记忆:昨天做的事情和写的稿子还保存在我心里,但我对自己的所作所为不满,觉得太过粗俗。从今以后,我要变得高雅些。一面下着这样的决心,一面我也觉得,自己有点做作。 

因为老婆这个字眼十分庸俗,我决定把她称作白衣女人。因为她总穿白印花布的连衣裙,那布料又总是很软,好像洗过很多遍。所以她紧紧地裹在那种布料里,非常赏心悦目。她从我身边走过时,我顺手一抄,在裙子上捻了一把。她马上说道:别乱来啊──快起来,要迟到了。我立刻把手收了回来,放在嘴里咬着,用这种方式惩办这只手,心里想着:看来,这个举动格调不高……我该克服这种病态的爱好。我现在经常把手放在嘴里咬,但这不再使我焦虑。因为现在我已经悟到了,人要有高尚的情操,这就是说,我知善明恶,不再是浑沌未凿。别的问题很快就会迎刃而解了。 

对这位白衣女人,需要补充说,她骑自行车的样子也十分优雅;因为她挺直了脖子,姿式挺拔,小腿在裙子下从容不迫地起落;行驶在灰色的雾里──就如一只高傲的白天鹅,巡游在朝雾初升的湖里。……我一不小心闯了红灯,然后一面看着路口的民警,一面讪讪地推着车子转了回来,回到路口的白线之内。这时她满脸都是笑意,说:你是不是又想被汽车撞一下?我认真地想了想,想到病房里龌龊的空气,还有别人在我耳畔撤尿的声音,由衷地答到:不想。我不想被汽车再撞一下,会撞坏的。她笑了起来,拉住我肩头的衣服,伸过头在我面颊上吻了一下,还说,真逗。我还想听到她再说什么,但是绿灯亮了。我们又骑上自行车,驶往万寿寺。 

现在重读我的手稿,有些地方不能使我满意。比方说,那个老妓女奶袋尖尖,长了一嘴黄胡子,定起路来像一只摇摇晃晃的北极熊,全无可取之处。这不是我的本意。作为失去记忆的人,我的本意总是隐藏着。按照这种本意,故事里不该有全不可取的人──即使她是学院派的妓女。更何况这位白衣女人,如果不说她是一位学院派,就不足以形容她的气质。我对学院派怀有极大的善意,但因为本意是隐藏着的,所以把我也瞒过了。 

所以,很可能那个学院派的老妓女并不老,大约有四十四五岁的样子;体形依然美好,腰依然很细,四肢依然灵活,乳房虽然稍有松弛,但把它在人前袒露出来时,她并不感到羞愧。她的脸上虽有不少细碎的皱纹,但却没有黄胡子,只有一些黄色的茸毛长在手背、还有小臂的外侧上。总的来说,她的身体像个熟透的桃子,虽然柔软,但并无可厌之处,只是再熟就要烂掉了。这样描写一个中年妇女使我的良心感到平安,因为这说明我毕竟是善良的。实际上,这个女人不仅不老,心地也不坏,只是有些古怪;一旦决定了的事,就再不肯改变。假如这样考虑这个故事,与前就大不相同了。 

我的故事重新开始时,老妓女既不老,也不难看,只是有点神神叨叨的;或者说,有点二百五。这一点体现在她家的凉台上。这里有一道木栏杆,或者说是一道扶手。这道扶手有很多座子,上面安装了一些瓷罐,里面放着各种瓜子,有白瓜子、黑瓜子、葵花子、玫瑰瓜子、蛇胆瓜子等等,所以从外面看起来,这间房子里住的好像不是一个妓女,而是一群鹦鹉。她经常把男人送到凉台上,一面磕瓜子,一面歪着头上下打量他,终于吐出了瓜子皮,摇摇头,说道:难看死了。这是指他腰间蔑条吊起的龟头而言。那东西吊歪了就像个吊死鬼,是有点难看。在凉台的柱子上,挂着一束蔑条。她取下一条,拿在手里,用命令的口吻说道:解下来!这是命令那个男人把拴好的竹蔑条解下来,她要亲手来拴这根蔑条。那个男人解下腰间的蔑条时,她还把手上的蔑条揉来揉去,使之柔软;然后就像裁缝给人量腰围一样,把双手伸向他的腰间,几经周折,终于拴好了那根蔑条,吊好了那粒龟头;然后她就退后,继续磕瓜子,欣赏自己的杰作。这回它倒是不歪,只是仰着头,像一个癞蛤蟆仰头飘浮于水面上的样子。打量了好久之后,她终于得出了自己的结论,说道:更难看!就一头冲回自己屋里去,再也不出来了。别人来找她时,她也总在磕瓜子,歪着头打量他的腰间;最后终于吐出两片瓜子皮,也说:真难看──解下来罢。就自顾自进房子里去了。 

有关这位老妓女,还要补充说,她是柔软的。肚子柔软,面颊柔软,臀部柔软,乳房也柔软。柔软得到处起皱纹。虽然还能保持良好的外形,但眼看就要垮掉了。在她乳房下面,有两道弧形的皱纹,由无数细小的皱纹组成;凑近了一看,就像绳子一样。她常让薛嵩看这两条皱纹,还说:我都这样了,你还不来多陪陪我。在她肘弯外面,有两块松松的皮,有铜钱大小,颜色灰暗,好像海绵垫子一样;在这两块松皮上面,也有无数的皱纹。同样的松皮也长在了膝盖上,比肘部的还要大。她常拿这四块松皮给男人看,并且呼天抢地似地说道:你们看看,这还得了吗?我就要完蛋了!还不快陪我玩玩?小妓女和寨子里的苗族女人一致认为,情况远没有她说的这样严重,这女人用这一手拉拢男人。在这种场合,她们认为她并不老,还很年轻。在另一种场合她们就认为此人又老又丑。如此说来,她们对她有两种自相矛盾的看法:假如说又老又丑值得同情,她们就认为她不老不丑;假如说又老又丑不值得同情,她们就说她又老又丑。这样一来,她们对她的态度也就不矛盾了。 

这个女人对别人的态度也充满了矛盾。每次她看到小妓女在凉台上和别人调情,就厉声喝斥道:真下流!给男人作垫子!下流死了!轮到她自己时,又满不在乎地说:这没什么,哪个女人不给男人作垫子。这两种态度也是自相矛盾,一种用来对己,另一种用来对人。寨子里的女人都恨她恨得要死,她也恨每个女人恨到要死。这倒没什么稀奇,女人之间都是这样子的。所有的女人中她最恨红线,这倒不足为奇,因为红线抢了她的男人。 

这个女人很爱薛嵩,因为薛嵩是凤凰寨里最温柔的男人。假如他不来过夜,她就自己一个人睡,把一个木棉枕头夹在两腿之间;到了第二天早上,就到处和别人说:这个混蛋昨晚上又没来。早晚我要杀了他!人家以为她只是说说而已,但她真的干出来了。虽然不是杀薛嵩,只是杀红线,但已够惊世骇俗的了。她有几个东罗马金币,是她毕生的积蓄,闲着没事的时候经常拿来用牙咬,她觉得用牙咬比用眼睛看更开心。那些金币上满是她的牙印。后来,她就用这些钱雇了一些刺客去杀死红线,抢回薛嵩。据我所知,她马上就后悔了。一方面是因为她舍不得这些钱,另一方面她也觉得要别人的命未免太过份。后来,那个小妓女问她为什么要干这种事时,她死皮赖脸地答道:我吃醋啦。怎么啦,你就没吃过醋吗? 

2 

根据这种说法,这女人并没有说要杀掉小妓女,是那些刺客自作主张地把那女孩提了来,嘴里塞上了臭袜子,捆倒在她家的地上。那女人说:你们怎能这样!这是我的邻居啊。刺客头子说:你不懂。暗杀这种事,最怕走漏风声。他从老妓女手里接过几个金币,掂了掂那几块沾满了唾液、温暖的金子(老妓女为了告别自己的金币,又最后咬了它们几口),就说:放心吧,老太太;既然收了你的钱,一定帮你把事情办好;买卖就是这么一种做法。老妓女听了恨得牙根痒痒,因为她不觉得自己是老太太。她安慰小妓女说:别着急,等事情办好就放你。但没留神,她自己也被捆了起来,嘴里也塞上了臭袜子。然后那些刺客就在她家里搜了一阵,把她所有的金币银币都搜走了。原来这帮刺客还兼做强盗的生意。后来,那帮刺客兼强盗就出发去杀红线,他们还要杀掉薛嵩。除此之外,他们还要把薛嵩家好好搜上一搜,因为薛嵩毕竟是节度使,家里一定有些值钱的东西。用刺客头子的话来说,要做就做彻底,“买卖就是这种做法嘛”。临走时,他们把两个妓女背对背地拴在了一起,这样谁也跑不掉,等他们走后,小妓女就从鼻子里哼哼着骂老妓女,说道:老婊子,你真不是个东西。老妓女挨了一会儿骂,也从鼻子里答道:小婊子,骂两句就算了,别没完呀。咱俩以前是邻居,现在更是邻居了。又过了一会儿,她提议道:这么坐着有点累。咱们侧躺着好不好?这是个很合理的建议,小妓女虽然很生她的气,也只好同意了。 

在我新写的故事里,那个女人和那个女孩被背靠背地捆着,像一对连体双胞胎。我好像在什么地方见过这样的连体双胞胎──整个脊背长在一起,后脑勺也长在一起,泡在一个玻璃瓶子里──想必是在某个自然博物馆里。但我不想去找那个拥有一对连体双胞胎的自然博物馆。像所有的人一样,我去过不少博物馆、图书馆、电影院,所以就是找到了也没有什么意义。 

她们侧躺在地下,嘴里塞着臭袜子,但还是唠叨个不停。女孩说:老婊子,你这是干了些啥。女人说:我也不知这是干了些啥,我要是知道就好了。女孩说:他们杀了薛嵩回来,准要把咱俩都杀掉。这回好了吧?合了你的意了吧?女人答道:你少说几句罢。你不过是丢了一条命,我连我的金子都丢掉了!你有过金子吗?小妓女从来不攒钱,有了钱就花掉,她也知道这是种毛病,所以被噎住了。但她依旧心有不平,终于说道:呆会儿他们要杀,让他们先杀你。我看见你挨杀,心里也高兴一点。那女人沉吟了片刻,就答应了:好吧,我岁数也大些,就先死一会儿罢。过一会她又说:你的屁股还挺滑溜的嘛。女孩因此大怒道:滑溜不滑溜的,都要死掉了。这都怪你!老妓女感到理屈,就不说话了。 

两个妓女被背靠背地捆着,侧躺在地板上,直到天明时那些刺客们狼狈地回来。这些蓝色的人气急败坏,急于杀人泄债,就把那小妓女从老妓女背上解了下来,不顾她们之间的约定,要把她先杀掉。如前所述,她不肯引颈就戮,在地下翻翻滚滚用脚蹬人,还说,我们已经商量好了,要杀先杀她。那些刺客反正要杀一个人,杀谁都无所谓。于是就来杀老妓女。谁知她也不肯引颈就戮,也在地下翻翻滚滚,用脚来蹬人;还说:我付了钱让你们杀人,人没有杀掉,倒来杀我,真他妈的没道理!这就让那些刺客陷入了两难境地:假如小妓女不肯引颈就戮,他们可以先杀老妓女;假如老妓女不肯引颈就戮,他们可以先杀小妓女;现在两个妓女都不肯引颈就戮,他们就像不里丹的驴子不知该吃哪堆草那样,不知该杀谁好了。就在这时,白昼降临到这个地方,林间的雾气散去了,阳光照了进来,虽然阳光里还带有一点水汽…… 

在早上的阳光下,林间的空地上躺着两个女人的身体。一个很年青,充满了朝气,别人看了还能心平气和。另一个已经略见衰老,略显松弛,但依然美好,看起来就十分刺激。这是因为后一种身体时常被隐藏起来,如今被暴露在光天化日之下,就很能勾起人的邪念。前一个身体说道:老婊子!你说过让他们先杀你!后一个身体答道:他们想杀就让杀吗?没那么便宜!假如你是刺客头子,不知你会得出何种结论。我觉得这个结论应该是:前者和我们是一头的,后者不是。过了一会儿,后一个身体说道:喂,你们!好意思这么对待我吗?我可是给了你们钱的啊。前一个身体则说:好不要脸!还给他们钱……此时的结论似乎该是:后者和我们是一头的。前者不是。既然两个身体都可能和我们一头,刺客头子决定试上一试。他给她们讲了自己在薛嵩家里的不幸遭遇,然后提出一个问题:有没有一条路,或者一个方法,可以悄悄地摸进去,出其不意地逮住薛嵩和红线?这两个身体同声答道:不知道!此时的结论当然是:她们都不是和我们一头的。 

3 

如前所述,那个刺客头子也是学院派刺客,我既决定对学院派抱有善意,就不能厚此薄彼,只好对他也抱有善意。这个家伙要杀人,这一点当然不好。但反正不是杀我。他常把人看作身体,这就带有一点福科的作风──可惜我不记得福科是谁。他看起人来,总是有意地不看他(或她)脸,这样每个人就更像身体,更不像人。这个刺客头子从脸到足趾都是蓝色的,蓝得有点发紫。他的这种蓝色是天生的。假如他身上破了,还会流出蓝色的血,滴在地下好像一些蓝油漆──他手下的人虽然也是蓝的,但不是天生的,而是涂的蓝颜色,这些手下人总带着蓝墨水,一但碰破了皮,就往伤口里倒,假装蓝血──这是为了和领导保持一致。这个人的信条是:做事就要做彻底。他决定把这两个身体通通杀掉。他对身体有一种冷酷无情的态度,这样就和薛嵩有了区别。薛嵩对所有的身体都有好感,所以他就成了个老好人。在这个故事里,薛嵩就是这个样子。 

在这个故事里,薛嵩始终保持了小手小脚,是个留着寸头的、棕色皮肤的男孩子。他忙忙乱乱地在寨子里到处跑,有时跑进老妓女的视野里。后者当然不会放弃这个机会,所以就说:薛嵩,来陪我玩!薛嵩马上就答应,跑过来伏在老妓女的身上,双手捧住她的某一只乳房,把乳头放在拇指和食指之间认真地打量──那样子像个修表匠。当然,他还要打量别的地方。最后的结论是:大妈,你好漂亮啊。假如这是曲意奉承,就可以说明自由派与学院派的关系──薛嵩是自由派,老妓女是学院派,自由派要拍学院派的马屁,不漂亮也得说漂亮。可惜薛嵩根本不会曲意奉承,他真的觉得老妓女漂亮。 

后来,薛嵩跪了起来,解掉腰间的竹蔑条,还很客气地问道:可以吗?随后就和老妓女做爱,很自然,很澎湃。总而言之,他使老妓女觉得他真的爱她;然后就说:大妈,我还有别的事,一会儿再来陪你;就跑掉了。假如他根本不爱她,说一会儿来看她是谎话,这也能说明点问题。亚里士多德说:谎言自有理由,真实则无缘无故。想想这个理由吧:学院派很崇高,让人不能不巴结。除了拍马屁,还要说些甜言蜜语来讨她的好。但是,很不幸,他也真爱这个老妓女。他真想一会儿就来看他。既然是真的,就不能说是拍马屁了。 

更加不幸的是,他走着走着,别的女人也会在篱笆后面叫道:薛嵩,来陪我玩。他也会跑进去,伏在人家身上说:大姐,你好漂亮啊;过一会儿也要去解竹蔑条,并且说:可以吗?倘若对方说,不可以(这种情况很少见),他就把蔑条重新系上,并且说:真遗憾,但你的确很漂亮;然后就走掉了。在更多的情况下他要和那女人做爱,而且很自然,很澎湃;然后又说:对不起,我还有别的事,一会儿再来陪你;就走掉了。这也是实话,假如不是在别处绊住了,他真想回来看她。假如有位八十岁的老太太叫他:薛嵩,陪我玩;他也会跑进去,把玩她老态龙钟的身体,然后说:老奶奶,你真是个漂亮的老奶奶。然后不和她做爱,走掉了。他做得很对。假如是个三岁的女孩叫他,他就跑进去抱抱她,然后说:小妹妹,你真漂亮,可惜太小了,不能和你玩;然后走掉了。假如走在路上,听到一头母水中在背后“哞”地一叫,他也要回头看看,然后对它说:捣什么乱啊你,然后走掉了。这个寨子里所有的女人都喜欢薛嵩,因为他对女人的身体深具爱心,热爱一切年龄、一切体态的身体。这寨子里的一切男人都恨薛嵩,也是因为他对女人的身体深具爱心,喜欢一切年龄、一切体态的身体。作为一个男人,他还有些可赞美之处,但作为一寨之主,他简直混帐得很。像他这样处处留情的人物,当然属于邪恶的自由派。 

这个故事现在的样子使我十分满意,因为里面没有一个女人是可厌的。作为一个自由派的男人,我喜欢一切女人,不管是老的还是小的,是漂亮的还是丑的,不管她声音清丽委婉,还是又粗又哑;性情温柔还是凶猛泼辣,我都喜欢。唱过了这些高调之后,我也要承认,还是温柔漂亮一点的女人我喜欢得更多一点,不管她是自由派还是学院派。 

4 

在这个故事里,薛嵩也遇到了红线。此后他就把一切年龄、一切体态的妇女都弃之如敝履。这一下就不像自由派了。红线也无甚出奇之处,只是个子很高、腿很长,身材苗条。假如是汉族女人,长到这样高以后,就会自然地矮下去──也就是说,低着头,猫着腰,像比自己矮的人看齐。但苗族女孩不会这样。红线在林子里找了一棵老树,在树皮上刻上自己的高度,每天都去比量,巴不得再长个一寸两寸。她就这样被薛嵩看到了。后者马上就对她入了迷,开始制造各种抢婚的工具,从一个多情种子,变成了一个能工巧匠。这就使老妓女为之嫉妒、痛苦,请了人来杀她。有关这件事的前因,我觉得自己已经解释得足够清楚了。 

至于这件事的后果,就是她请来的人把她自己给逮住了,而且那些人还要拷打她,想从她那里获得薛嵩的情报──老妓女本来可以自愿说出些情报,但被捆上了就不能说,她也是有尊严的人哪──把她脸朝里地绑在一棵树上,说道:老婊子,打你了啊!她还是满不在乎地说:打吧。于是,藤条就在她背上呼啸起来了。我可以体会到这种看不见的疼痛。后来,人家把她放开,让她趴在满是青苔的地上;空出了那棵长满了青苔的老树。此时她背上满是伤痕和鲜血。那个小妓女在一边看了,恶狠狠地说了一声:“该!”但老妓女还是镇定自若,对一个样子和善的刺客说:劳驾,给我拿把瓜子来。再以后,她就趴在地上磕瓜子。虽然背上被抽开了花,她的臀部依然很美,腰也很细。小妓女看了,感到莫名的愤怒,痛恨她的身体,更恨她满不在乎的态度。像这样把痛苦和死亡置之度外,她可学不来…… 

后来,那个刺客头子对着那棵空出的树,作了一个优雅的手势,对小妓女说:小婊子,现在轮到你了。那女孩跺跺脚走了过去,抱住那棵树,伏在了老妓女的体温上,让人家把她捆在树上。她感到悲愤和委曲,就一头撞在树上,把头都撞破了。刺客头子看到这种不理性的举动,就劝止说:别这样。打你是我们的工作,不用你自己来做。于是,那小妓女觉得简直要气死了,大喊一声:你们!一个气我,一个打我!到底还让不让人活?刺客头子闻声又劝止道:别这样。让你死或让你活,是我们的事。不用你来操心。这就使小妓女完全走投无路了。 

第二节 

说到我自己,虽然不是妓女也不是刺客,但我觉得自己是自由派。这个流派层次较低,但想要改变也不是一朝一夕的事。下午,我们院里的热水锅炉坏了,原来流出滚烫的清澈液体,现在流出一种温吞吞的黄汤子。因为这种汤子和化粪池堵塞后流出的东西有可疑的近似之处,渴疯了的人也不敢尝试。在这种情况下,我跑到隔壁面馆去打了两壶开水,一壶自己喝,另一壶送给了白衣女人;这种自力更生的作法就像我写到过的自由派小妓女。但别人却不是我这样的。有好几位老先生经常跑到锅炉面前,扭开龙头,看看流出的黄汤子,再舔舔乾裂的嘴唇,说一声:后勤怎么还不来修!就痛苦地走开了;丝毫想不到隔壁有家面馆。这种逆来顺受的可爱态度,和学院派的老妓女很有点相似。但我也不敢幸灾乐祸,恐怕会招来杀身之祸…… 

对于这个热水锅炉,需要进一步的描述:它是个不锈钢制成的方盒子,通着三百八十度的三相电。我觉得只要是用电的东西,就和我有缘份。我切断了电源,围着它转了好几圈。最后得出一个结论:只要能找到管钳,卸掉水管,我就能把它修好;没有管钳,用手拧不动水管(我已经试过了),就只好望洋兴叹。下一个问题就是:到哪里去找管钳。这么大的一个单位,必定有修理工,还会有工作间,能找到那儿就好了。我可不像薛嵩,东西坏了也不去修。但我对这个院子不很熟悉,转着圈子到处打听哪里能借到工具。转来转去,终于转到了白衣女人的房间里。她听到了我的这种打算,马上叉着脖子把我撵回自己屋里;还说:你自己出洋相不要紧,别人可要笑话我了。我保证不去出洋相,但求她告诉我哪里能借到管钳。她说她不知道。看来也不像假话。然后,我在自己屋里,朝着摊开的稿纸俯下身来,心里却在想:真是不幸,连她也不理解我。看来她也是个学院派…… 

我总忘不了坏掉的锅炉在造成干渴,这种干渴就在我唇上,根本不是喝水可解。行动的欲望就像一种奇痒,深入我的内心。但每当我朝院里(那边是锅炉的方向)看时,就能看到一个白色的身影在那边晃动。看来,白衣女人已经知道我禁不住要采取行动,正在那边巡逻──她比我自己还了解我。又过了一会儿,我开始出鼻血,只好用手绢捂着鼻子跑出去,到门口的小铺买了─卷卫生纸。又过了一会儿,纸也剩得不多了。我只好捏着鼻子去找那位白衣女人。她见了我大吃一惊,说道:怎么了?又流鼻血了?我也大吃一惊:原来我常流鼻血,这可不是什么好消息……她在抽屉里乱翻了一阵说:糟了,药都放在家里。这是我意料中事,我瓮声瓮气地说道:我一个人也能回家去,但要把车也推回去,要不明早上没得骑。她倒有点发楞:你是什么意思?现在轮到我表现自由派的慎密之处:我的意思是,我自己推车走回去,但要劳你在路上捏住我的鼻子……但一出了门,我就知道还欠慎密:这个样子实在古怪,招得路上所有的人都来看我。除此之外,她还飞腿来踢我的屁股,因为鼻子在她手里,我全无还手之力,这可算是乘人之危了。她小声喝道:不准躲!不让你修锅炉你就流鼻血,你想吓我吗?……这话太没道理,鼻血也不是想流就能流得出的。何况,流鼻血和修锅炉之间关系尚未弄清,怎能连事情都没搞明白就踢我!因为她声音里带点哭腔,我也不便和她争吵。回到家里,躺在床上,用了一点白药,鼻血也就止住了。她也该回去上班。但她还抛下了一句狠话:等你好了再咬你…… 

2 

白衣女人曾说,我所用的自由派、学院派,词意很不准确。现在我有点明白了。所谓自由派,就是不能忍受现状的人,学院派则相反。我自己就是前一种,看到现状有一点不合理就急不可耐,结果造成了鼻子出血。白衣女人则是学院派,她不准我急不可耐,我鼻子出了血,她还要咬我。小妓女和老妓女也有这样的区别,当被捆在一起挨打时,这种差别最充份地凸现了出来。 

我写到的这个故事可以在古书里查到。有一本书叫作《甘泽谣》,里面有一个人物叫作薛嵩,还有一个人叫作红线。再有一个人叫作田承嗣,我觉得他就是那个浑身发蓝的刺客头子。这样说明以后,我就失掉了薛嵩、红线,也失掉了这个故事。但我觉得无关紧要。重要的是通过写作来改变自己。通过写作来改变自己,是福科的主张。这样说明了以后,我也失去了这个主张。但这也无关紧要,重要的是照此去做。通过写作,我也许能增点涵养,变成个学院派。这样鼻子也能少出点血。 

那个蓝色的刺客头子把小妓女捆在树上,一面用藤条在她背上抽出美丽的花纹,一面坦白了自己的身份。如前所述,他就是田承嗣,和薛嵩一样,也是一个节度使。这就是说,他假装是个刺客头子,拿了老妓女的钱,替她来杀红线,实际上却不是的。他有自己的目的,想要杀死薛嵩,夺取凤凰寨。我想他这样说是想打击妓女们的意志,让她们觉得一切都完了,从此俯首贴耳──这个成语叫我想到一头驴。当然,他的目的没有达到。那个小妓女听了,就尖叫道:老婊子!看你干的这些事!你这是引鬼上门!那个老妓女一声不吭,继续磕着瓜子,想着主意。后来,她站了起来,走到田承嗣的身边,说道:老田,放了她。田承嗣纳闷道:放了她干什么?那女人说:把我捆上啊。田承嗣又纳闷道:把你捆上干什么?那女人说:我替她挨几下。田承嗣说:挨打是很疼的呀。老妓女说:没有关系。我也该多挨几下。这样一来,这个老妓女就表现出崇高的精神;用自己的皮肉去保全别人的皮肉。在这个故事里,还是第一次出现了这种精神。这说明我变得崇高了。看来,通过写作来改变自己,并不是一句空话呀…… 

在这个故事里,田承嗣是卑鄙的化身──现在我已认定,田承嗣根本就不是学院派,他不配。起初我觉得,老妓女的自我牺牲会把他逼人两难的境地。假如他接受了老妓女的提议,放了小妓女去打老妓女,崇高的精神就得以实现,他所代表的邪恶就受到了打击。假如他不打老妓女,继续打小妓女,那老妓女就要少挨打。按照他邪恶的价值观,少挨打是好的。老妓女的崇高精神没有受到惩罚,对他来说是一种失败。照我看,他是没办法了。很不幸的是,田承嗣也有自己邪恶的聪明。他叫手下的人把老妓女捆在另一棵树上(很不幸的是,凤凰寨里有很多的树),同时加以拷打。小妓女还嘲笑她说:老姨子,瞧你干的这些事!你真是笨死了。她只好摇头晃脑地说:真是的,我笨死了。但是,小婊子,我可是真心要救你啊。小妓女乾脆地答道:救个屁──这其实不是一句有意义的话,只是一声感叹;然后,她就低下头去,闭上眼睛,忍受背上的疼痛。在这个故事里,我想要颂扬崇高的精神,结果却让邪恶得了胜,但我决定要原谅自己,因为我已失去了记忆,又是个操蛋鬼,对我也不能要求过高。再说,邪恶也不会老得胜…… 

3 

鼻血止住之后,我在家里到处搜索,没有找到户口本,却找到了几页残稿,写道:“盛夏时节,在长安城里,薛嵩走过金色的池塘,走上一座高塔去修理一具热水锅炉……”在我失去记忆以前,这是我写下的最后的字句。打个不恰当的比喻。这像是我前生留下的遗嘱。看来,我想修理锅炉不是头一次了。我觉得可以从此想到很多东西。可惜的是,一下子不能都想起来。 

以此为契机,我却想起了这样一件事:在大学里,有个同宿舍的同学戴一副断了腿的水晶眼镜,不管我怎么苦苦哀求,他都不肯摘下来叫我修理。这孙子说,这副眼镜是他爸爸的遗物,他要就这么戴到死……这眼镜他小心藏着,不让我碰。但我一见他用绳子接着眼镜就心痒难熬。终于有一天,我在宿舍里把他一闷棍打晕,并在他苏醒之前把镜腿换上了……然后,他就很坚决地从宿舍里搬走了。他倒没有告我打他,只是到处宣扬我有精神病。别人对他说:你可以把新装上的镜腿再拆下来,这样,你父亲的遗物还是老样子。他却说:拆了干啥?招着王二再来敲我的脑袋?我没有那么傻!从这件事里,我很意外地发现自己上过大学──我是科班出身的。现在我可以认为自己是个学院派的历史学家,这是一个好消息。还有一个坏消息:我很可能是个有修理癖的疯子。正如白衣女人指出的,我所指的自由派,就是些气质像我的人。现在我知道了自己可能是疯子,自由派这个名称就有了问题:我总不好把疯子算作一派吧。 

我对自衣女人用脚来踢我的事很是不满──就算我犯了疯病,也是为所里的器具损坏而疯,是一种高尚的疯病,踢我很不够意思──最起码应该脱了鞋在家里踢,穿着鞋在街上踢是不应该的。但细细一想,她还是对我好。继而想到,她说过,让我骑车小心,还说自己不愿意当寡妇,也是不希望我死之意。这使我从心里感到一丝暖意。说实在的,我自己也不想早早地死掉。我又回过头来写我的故事──我现在能做到的只是在故事里寻找崇高。在这个故事里,那个蓝色的刺客头子,也就是田承嗣,逮住了两个妓女,拷问她们薛嵩在哪里──在此必须重申,田承嗣不是自由派也不是学院派,他哪派都不是。 

这两个女人──一位学院派的妓女和一位现代派的妓女,表现出崇高的气节,没有告诉他。其实他根本多此一间,薛嵩就在他们身后。黎明时分,薛嵩把他的柚木院子高高地升了起来,这片浮动的土地连同上面的花园、房屋,高踞在八根柱子上,而那八根柱子又高踞在林梢顶上,在朝霞的衬托之下,好像一个庞大无比的长腿蜘蛛。薛嵩站在这个空中花园的边上,隔着十里地都能看见。而寨中心那片空地离得很近,顶多也就是一两里地。奇怪的是,那些刺客和两个妓女都没有往那边看。 

薛嵩遭人袭击之后,一直在努力升高他的院子。院子越高,离地面越远,也就越安全。他长时间地不言不语,好像怯懦已经吞食了他的内心。但到了黎明时分,他忽然呐喊一声,从地上一跃而起,奔进房子去拿他的武装。首先,他戴上一顶铜盔,这东西大体上和消防队员戴的头盔差不多,只是更高、更亮,盔顶有鱼鳍一样的冠子,用皮带扣在颏下;这样他一下子高了有一尺多。然后他又穿上护胸甲,这东西表面是一层发乌的青铜,镌有大海和海上的星辰。在青铜后面是亮闪闪的黄铜,黄铜背后是厚厚的水牛皮。最里面的一层是柔软的黄牛皮。这个结构的奥妙之处在于青铜硬而且脆,可以弹开锋利的刀锋;黄铜质地绵密,富有韧性,可以提供内层防护。至于牛皮,主要是用来缓冲甲面上的打击;这就深得现代复合装甲结构之精髓。此后他穿上护裆甲,那东西的形状就如一个龟头向上的生殖器,其作用也是保护这个重要的器官;只是那东西异常之大,把大象的家伙装进去,也未必装得满──看到红线疑惑的目光,薛嵩解释了两句:敌人也不知我有多大,吓吓他们──他把这个东西拴在腰间,拴上护肩甲、护腿甲、护胫甲,薛嵩威风凛凛,有如一位金甲天神。 

但是,所有这些甲胃都只有前面,没有后面;后面用几根皮带系住。所以,薛嵩也只是从前面看时像位金甲天神,从后面一看,裸露着脊梁,光着屁股,甚是不雅观。薛嵩用巨雷般的低沉嗓音说道:敌人只能看到我的前面,休想看到我的后面;这话说得颇有气概。他还穿上了皮底的凉鞋,鞋底有很多的钉子,既有利于翻山越岭,又可以用来踢人。着装以后,薛嵩行动起来颇为不便,他有一把连鞘的青铜大剑放在地下。他让红线给他拿起来,以便拴在腰上。看到那剑又宽又厚,红线就用了很大的力气去拿。结果是连人带剑一起从地下跳了起来,原因是那剑很轻。薛嵩抹了一下鼻子,不好意思地说道:空心的。把剑佩好,他把铜盔上的面具拉了下来,露出一副威猛的面容。然后,这样一位薛嵩就行动了起来,准备向外来的袭击者展开反攻。 

4 

有关薛嵩的院子,必须补充说,它不但可以在柱子上升降,那些柱子又可以水平移动。只要转动一些绞盘,整个院子连同支撑它的柱子就可以像个大螃蟹一样走动,成为一个极为庞大的步行机械。实际上,薛嵩可以使他的院子向寨中的敌人发起冲击,但要有个前提:必须有一百个人呆在上面,按薛嵩的口令扳动绞盘。假如有一百个人,这座院子就会变成一架可怕的战争机器,连同地基向敌人冲击。不幸的是,此时院子里只有两个人,缺少了人手,它就瘫了不能动。细究起来,这又要怪薛嵩自己。他只让自己和红线登上柚木平台,换言之,除了红线,他谁都不信任…… 

白衣女人说,她最讨厌我在小说里写到各种机械、器具;什么绞盘啦、滑轨啦,她都不知道是些什么东西。她说得有道理,但我满脑子全是这种东西,不写它写什么?写高跟鞋?这种东西她倒是很熟悉,但我对它深恶痛绝,尤其是今天被穿着高跟鞋的脚踢了两下以后,就更痛恨了。她听了挑起眉毛来说:哟!记仇了。好吧,以后不穿高跟鞋。她就是不肯说以后不再踢我。我的背后继续受到威胁…… 

红线以为,薛嵩会冲出自己的柚木城堡,向聚集在寨中心的刺客们冲锋。这样他将面对数十倍于己的敌人,前面虽然武装完备,后面却还露着屁股;这样顾前不顾后肯定不会有好的结果。她对于战争虽然一窍不通,但还懂得怎么打群架。所以她也武装了起来:把头发盘在了头上,把家里砍柴、切菜的刀挑了一个遍,找到一把份量适中,使起来趁手的,拿在右手里。至于左手,她拿了一个锅盖。薛嵩家里的一切东西都是他亲手做的,既结实、又耐用,样子也美观,总之,都很像些东西;这个锅盖也不例外。它是用柚木做的。有一寸来厚,完全可以当盾牌用。红线跟在薛嵩后面,准备护住他的后背,满心以为他就要离开家去打交手战;谁知薛嵩不往门外跑,却往后面跑去。他打开了库房的大门,从里面推出一架救火云梯似的东西──那东西架在一辆四轮车上。红线帮他把这个怪东西推到了门前的空地上,薛嵩用三角木把车轮固定住,把原来折叠的部件展开来;这才发现它原来是一张大的不得了的弩。原来,薛嵩并不准备冲出去,他打算呆在城堡里──也就是说,躲在安全的地方施放冷箭。既然如此,红线就不明白薛嵩为什么要作张作势地穿上那么多的铠甲。我觉得这个问题的答案应该是:造造气氛。 

薛嵩的弩车停在城堡的边缘上。弩上的弓是用整整一棵山梨树做成的,弓弦是四股牛筋拧成的绳子。他和红线借助一个绞盘把弓张开,装上一支箭──那箭杆是整整的一根白蜡杆,我以为叫作一支标枪更对。此时,这张弩的样子就像一辆现代的导弹发射架,处于待发的状态。薛嵩登上瞄准手的位子,摇动方向机和高低机,把弩箭对准了敌人。如前所述,这里离寨中心相当远,只能看见影影绰绰的一群人。就这样一箭射出去,大概也能射着某个人。但薛嵩的伎俩远不止此。他还有个光学瞄准镜,由两个青铜阳燧组成。众所周知,阳燧是西周人发明的凹面镜,原来是用来取火的。薛嵩创造性地把它们组装在一起,变成了一个反光式的望远镜。透过它看去,隔了两里多地,人头还有大号西瓜大。他在里面仔细地瞄准,只是不知在瞄谁。这个目标对我自己来说,是一个悬念。 

5 

我说过,从前面看去,薛嵩是一位金甲天神。从反面一看就不是这么回事,因为他光着屁股。假如全身赤裸,这个部位倒是满好看的:既丰满、又紧凑;但单单把它露在外面,就说不上好看,甚至透着点寒碜。这就如一位正面西装革履的现代人,身后却露出肉来,谁看了也不会说顺眼。我们知道,浑身赤裸时,薛嵩是个心地善良的好人;打扮成这个样子以后是个什么人,连红线都不知道。他就这样伏在弩车上,仔细地瞄准,然后搬动了弩机;只听见砰地一声,那支弩箭飞了出去…… 

正午时分,空气里一声呼啸,薛嵩的弩箭穿进了人群,把三个人穿了起来,像羊肉串一样钉在了一棵大树上。这三个人里就有老妓女,她被两个刺客夹在中间,像一块三明治。那根弩箭从她的胃里穿过去,她当然感到钻心的疼痛。她还知道,这是薛嵩搞的鬼,就朝他家的方向愤怒地挥了一下拳头。但马上她的注意力就被别的事情吸引过去了。在她身后那个刺客痛苦地挣扎着,把腰间的蔑条都挣开了,那个东西硬邦邦抵在她的屁股上,总而言之,他就像北京公共汽车上被叫作“老顶”的那种家伙。她极过身去,愤怒地斥责道:往哪儿捅?这儿要加钱的,知道吗?后面那个刺客被射穿了心口下面的太阳神经丛,疼得很厉害,无心答理她。在她前面的那一位被从左背到右前胸斜着贯穿,伤口很长,已经开始临死的抽搐,不听使唤的手臂不停地碰到她身上。老妓女又给了他一巴掌,说道:挤那么紧干嘛,又不是没有地方!那人倒着气,勉强答道:对不起,我也不想这样……再后来,老妓女自己也没有了力气,不再争辩什么,就这样死去了,临死时,朝柚木城堡伸出右手的中指,这是个仇恨的手势。这个老妓女留下了一个不解之谜:到底薛嵩是有意射她呢,还是无意的。小妓女总觉得他是无意,我总觉得他是有意。当然,薛嵩自己总不承认自己是有意的。 

放完了这一箭,薛嵩摇了摇头,没有说什么。倒是红线大叫起来:射错人了!然后,薛嵩在弯上装上一支新弩箭,转动绞车把弩张开时,红线继续呆呆地站着,也不来帮忙,忽然又大叫了一声:射错人了!但薛嵩还是一声不吭地忙着,张好了弩,他又跑回瞄准手的座位上去,继续瞄准,而红线则又一次呐喊道:射错人了!射着自己人了!薛嵩回头一看,发现红线正用反感的眼神看着他,就说:别这么看我!这是打仗,你明白吗?战场上什么事都会发生……说完,他就回过头去继续瞄准了。红线定了定神,回头朝寨心望去,发现那片空场上只剩了一个人──无须我说你就知道,原来那里有一大群人,现在都不见了。只剩下一个人,就是那个小妓女。说来也不奇怪,那些刺客发现自己在远程火力的威胁之下,自然要躲起来。假如那个小妓女坚信薛嵩不会射她,她也可以不躲起来。但实际上却不是这样──实际上,她也信不过薛嵩,但有一大夥人躲在她的身后,还有一个人从背后揪伎她的头发,让她躲不开。现在,她面朝着薛嵩家的方向站着,满脸都是无奈。 

也许我需要补充说,薛嵩一箭射死了老妓女和两个刺客,使田承嗣和他的手下人大惊失色,觉得他很厉害。他们赶紧躲了起来──当然,可以躲到大树后面、躲到河沟里,但他们觉得躲在小妓女背后比较保险。他们以为,这个女孩和薛嵩的交情非比一般,她和薛嵩太太红线又是手帕交,薛嵩决不会射她,因此,她身后一定是最保险的地方了。但薛嵩离他们很远,所在的方位又是逆光,所以他们一点都看不到薛嵩在干啥;假如看到了,一定会冒出红线一样的疑问:敌人都躲了,只剩一个自己人,你瞄的到底是谁呀?假如他们知道这问题的答案,更会大为震惊。实际上,薛嵩瞄的就是小妓女,虽然他不想射死她。他把瞄准镜的十字线对在那女孩的双乳正中,心里想着:天赐良机!他们排成了一串……这一箭可以穿透十二个人。这说明他想要射死的决不是小妓女,而想要穿过她,射死她身后的十一个人。当然,我们知道,这个女孩被穿透后之后,很难继续活下去。但这一点薛嵩已经忘记了。他只记得射死了十一个人以后,就可以夺回凤凰寨了。我发现,只要我开个恶毒的玩笑,就可以得到崇高。薛嵩把弩箭瞄准小妓女,就是个恶毒的玩笑;但崇高不崇高,还要读者来评判。他瞄得准而又准,正待扳动弩机,忽然听见砰地一声响,整个弩车猛地歪到一边──原来是红线一刀砍断了弓弦。薛嵩从歪倒的弩车里爬了出来,扶正头上的头盔,朝红线嚷道:怎么搞的?你搞破坏呀你!但红线一言不发,只是瞪大了眼睛看着他。她的眼睛不瞪就很大,瞪了以后连眼眶都看不到了。 

6 

那个白衣女人看过我的故事,摇摇头,说道:你真糟糕。在这个故事里,薛嵩一箭射死了老妓女,又把箭头对准了小妓女;她就是指这点而言。我问:哪里糟糕?她说:想出这样的故事,你的心已经不好了。我连忙伸手去摸左胸时,她又喝道:往哪儿摸?没那儿的事!我说你品行不好!如你所知,我现在最关心这类问题,就很虚心地问道:什么品行叫作好,什么品行叫作不好?她说出一个标准,很简单,但也很使我吃惊:品行好的男人,好女孩就想和他做爱。品行不好的男人,好女孩宁死也不肯和他做爱。我现在的品行已经不好了,这使我陷于绝望之中。 

实际上,是薛嵩的品行有了问题。我发现他很像我的表弟:如前所述,我表弟的手脚都很小,他的皮肤是棕色的,留着一头板寸。傍晚我们到王府饭店去看他,坐在lobby里,看着大厅中央的假山和人造瀑布。我表弟讲着他的柚木生意,有很多技术性的细节,像天书一样难懂。许多年前,薛嵩就是这样对红线讲起他行将建造的凤凰城。他在砂地上用树枝画了不少波浪状的花纹,说道,长安城虽然美丽,但缺少一个中心,所以是有缺点的。至于他的城市,则以另一种图样来表示,一个圆圈,周围有很多放射出的线条。红线没看出后一个形状有任何优点,相反,她觉得这个图样很不雅,像个屁眼。不过她很明智,没把这种观感说出来。实际上,薛嵩说了些什么,她也没听懂。薛嵩是说,这座城市将以他自己为核心来建造。它会像长安一样美丽,但和长安大不相同。它将由架在众多柱子上的柚木平台组成,其中最大最高的一个平台,就是薛嵩自己的家。这个建筑计划我表弟听了一定会高兴,因为这个工程柚木的用量很大,他的柚木就不愁卖不出去了。 

身在凤凰寨内,薛嵩总要谈起长安城。起初,红线专注地听着,眼睛直视着薛嵩的脸:后来她就表现出不耐,开始搔首弄姿,眼睛时时被偶而飞过的蝴蝶吸引过去。在王府的lobby里当然没有蝴蝶,她的视线时时被偶尔走过的盛装女郎吸引过去,看她们猩红的嘴唇和面颊上的腮红,我猜她是在挑别人化妆的毛病──顺便说一句,我觉得她是枉费心机,在我看来,大家的妆都化得满好──对于我们正在说着的这种语言,她还不至全然不懂,但十句里也就能听懂一到两句。等到薛嵩说完,红线说:能不能问一句?薛嵩早就对她的不专心感到愤怒,此时勉强答道:问吧!这问题却是:雪是什么呀?身为南国少女,红线既没见过雪,也没听说过雪,有此一问是正常的。但薛嵩还是觉得愤怒莫名,因为他这一番唇舌又白费了。我的表弟一面说柚木,一面时时看着我的表弟媳,脸上也露出了不满的神色,看得她说了一声:“Excuse me”,就朝卫生间走去了。那位白衣女人说了一句:“Excuse me”,也朝卫生间走去。后来她们俩再次出现时,走到离我们不远的沙发上坐下了──女人之间总是有不少话可说的。现在只剩下了我,听我表弟讲他乏味的柚木生意。 

我已经知道柚木过去主要用于造船,日本人甚至用它来造兵舰,用这些兵舰打赢了甲午海战──由此可以得到一个结论:这种木头是我们民族的灾星──而现在则主要用来制造高档家具,其中包括马桶盖板。他很自豪地指出,这家饭店的马桶盖就是他们公司的产品,这使我动了好奇心,也想去厕所看看。但我表弟谈兴正浓,如果我去厕所,他必然也要跟去。所以我坐着没有动:两个男人并肩走进厕所,会被人疑为同性恋,我不想和他有这种关系……我还知道了最近五年每个月的柚木期贷和现货行情,我表弟真是一个擅长背诵的人哪。我虽然缺少记忆,但也觉得记着这些是浪费脑子一──这种木头让我烦透了。后来,我们在一起吃了饭。再后来,就到了回家的时刻。我表弟希望我们再来看他,不知道为什么,我有点不想再来了…… 

7 

晚上我回家,追随着那件自色的连衣裙,走上楼梯。走廊里很黑,所有的灯都坏了。我不明白为什么没人来修理。楼梯上满是自行车。我被车把勾住了袖子,发起了脾气,用脚去踢那些自行车。说实在的。穿凉鞋的脚不是对付自行车的良好武器──也许我该带把榔头出门。那个自衣女人从楼梯上跑了下来,把我拉走了。她来得正好,我们刚上了楼,楼下的门就打开了,有人出来看自己的车子,并且破口大骂。假如我把那些骂人话写了出来,离崇高的距离就更远了。此时我们已经溜进了自己的家,关上了门,她背倚着门笑得透不过气来。但我却笑不出来:我的脚受了伤,现在已经肿了起来。后来到了床上,她说:想玩吗?我答道:想,可是我品行不好呀;她又笑了起来,最后一把抱住我说:还记着哪,这似乎是说,白天她说的那些关于品行的话可以不当真。有些话要当真,有些话不能当真。这对我来说是太深奥了…… 

有件事必须现在承认:我和以前的我,的确是两个人。这不仅是因为我一点都记不得他了,还因为怀里这个女人的关系。我一定要证明,我比她以前的丈夫要强。现在我们在做爱。我不知别的夫妇是怎样一种作法,我们抱在一起,像跳贴面舞那样,慢条斯理──我总以为别的姿势更能表达我的感情。于是,我爬了起来,像青蛙一样岔开了腿。没想到她从鼻子里哼了一声说:别乱来啊,就在我头上敲了一下。正好打中了那块伤疤,几乎要疼死了。不管怎么说罢,我还是坚持到底了…… 

我现在相信薛嵩的品行的确是不好的。以前红线不知道他有这个缺点,所以爱过他,很想和他做爱。现在看到他射死了老妓女,又想射死小妓女,觉察出这个问题,就此下定决心,再也不和他做爱。她甚至用仇恨的目光看看薛嵩的头盔,心里想着:这里没盛什么真正的智慧;里面盛着的,无非是一包软塌塌的、历史的脐带…… 

第三节 

薛嵩的所作所为使红线大为不齿,我也被他惊出了一身冷汗。如你所知,我因为写他,品行都不好了。但我总不相信他真有这么坏。他不过是被自己的事业迷了心窍而已。身为一个男人,必须要建功立业…… 

我说过,薛嵩在长安城里长大。后来,他常对红线说起那座城市的美丽之处。他还说,要在湘西的草地上建起一座同样美丽的城市,有同样精致的城墙、同样纵横的水道、同样美丽的水榭;这种志向使红线深为感动。从智力方面来看,薛嵩无疑有这样的能力。遗憾的是,他没有建成这座新长安所需的美德──像这样一座大城,可不是两个人就能建成的啊。 

身在凤凰寨内,薛嵩总要谈起长安城里的雪。他说,雪里带有一点令人赏心悦目的黄色,和早春时节的玉兰花瓣相仿。这些雪片是甜的,但大家都不去吃它,因为雪是观赏用的。等到大地一片茫茫,黑的河流上方就升起了白色的雾;好像这些河是温泉一样……假如能把长安的雪搬到这里就好了──起初,红线专注地听着,眼睛直视着薛嵩的脸;后来她就表现出不耐,开始搔首弄姿,眼睛时时被偶而飞过的蝴蝶吸引过去。 

薛嵩描述的长安城是一片白茫茫的雪地,在雪地上纵横着黑色的河岸。在河岸之间,流着黑色透明的河水,好像一些流动的黑水晶。但这也没什么用处。住在这里的人没有真正的智慧,满脑子塞满了历史的脐带。河水蒸腾着热气,五彩的画肪静止在河中,船上佳丽如云。这也没什么用处,这些女人一生的使命无非是亲近历史的脐带,使之更加疲软而已。她们和那位建造了万寿寺的老佛爷毫无区别…… 

忽然间薛嵩惊呼一声:我的妈呀!我都干了什么事呀……然后他就坐在地上,为射死了老妓女痛心疾首,追悔不已。首先,他在弩车的轮子上撞破了脑袋,然后又用白布把头包了起来。这一方面是给死者带孝,另一方面也是包扎脑袋。然后,他又在肩上挎了一束黄麻,这也是给死者戴孝之意。这都是汉人的风俗,红线是不懂的,但她也看出这是表示哀痛之意。然后,薛嵩就坐在地下嚎啕痛哭,又用十根指头去抓自己的脸,抓得鲜血淋漓。这些哀痛之举虽然真挚,红线却冷冷地说:一箭把人家射死了,怎么哭都有点虚伪。后来薛嵩拿起地上那把青铜剑,在自己身上割了一些伤口,用这种方法来惩罚自己。但红线还是不感动。最后他把自己那根历史的脐带放在侧倒的车轮上,想把它一剑剁下来,给老妓女抵命,红线才来劝止道:她人已经死了,你也用不着这样嘛。薛嵩很听劝,马上就把剑扔掉了。这说明,他本来就不想失掉身体的这一部份。不管你对上述描写有何种观感,我还是要说,薛嵩误杀了老妓女之后,是真心的懊悔。其实,我也不愿给薛嵩辩护。我对他的故事也感到厌恶。假如我记忆无误,这已经不是第一次了。 

薛嵩在凤凰寨里,修理翻掉的弩车�。如前所述,红线一刀砍断了弓弦。假如它只是断了弦,那倒简单了;实际上,这件机器复杂得很,很容易坏,而且是木制的。不像铁做的那么结实;翻车以后就摔坏了。薛嵩把它拆开,看到里面密密麻麻装满了木制的牙轮、涂了腊的木杆、各种各样的木头零件。随便扳动哪一根木杆,都会触发一系列复杂的运动。这就是说,在这个庞大的木箱子里,木头也在思索着。这东西是薛嵩的作品,但它的来龙去脉,他自己已经忘掉了。所以,薛嵩马上就被它吸引住了。他俯身到它上面,全神贯注地探索着,呼之不应.触之不灵。红线在地下找了一根竹签,拿它扎薛嵩的屁股。头几下薛嵩有反应,头也不回地用手撵那不存在的马蝇子;后来就没了反应。这件事使红线大为开心。她也俯身到薛嵩紧凑的臀部上,拿竹签扎来扎去;后来又用颜色涂来涂去,最后纹出一只栩栩如生的大苍蝇。此后,薛嵩在挪动身体时,那苍蝇就会上下爬动,甚至展翅欲飞。这个作品对薛嵩很是不利──以后常有人伸手打他的屁股,打完之后却说:哎呀,原来不是真苍蝇!对不起啊,瞎打了你一下。由此看来,假如红线在他身上纹一只斑鸠,他就会被一箭射死。那射箭的人自会道歉道:哎呀,原来不是真斑鸠!对不起啊,把你射死了…… 

2 

在凤凰寨里,此时到了临近中午的时分。天气已经很热了,所以万籁无声。所有的动物都躲进了林荫──包括那些刺客和小妓女。但薛嵩还在修理他的弩车,全不顾烈日的暴晒,也不顾自己汗下如雨。起初,红线觉得薛嵩这种专注的态度很有趣,就在他屁股上纹了只苍蝇,后来又在他脊梁画了一副棋盘和自己下棋。很不幸的是,这盘棋她输了。再后来,她觉得薛嵩伏在地上像一匹马,就把他照马那样打扮起来一一在他耳朵上挂上两片叶子,假装是马耳朵;此后薛嵩的耳朵就能够朝四面八方转动。搞来一些乾枯的羊胡子草放在他脖子上,冒充鬃毛;此后薛嵩就像马一样的喷起鼻子来了。后来,她拿来一根孔雀翎,插在他肛门里当作马尾巴。这样一来,薛嵩的样子就更古怪了。 

后来,那根孔雀翎转来转去,赶起苍蝇来了──顺便说一句,自从红线在臀部纹上了一只苍蝇,这个部位很能招苍蝇,而且专招公苍蝇。这不仅说明红线纹了只母苍蝇,而且说明这只苍蝇很是性感,是苍蝇界的电影明星──这根羽毛就像有鬼魂附了体一样,简直是追星族。一只金头苍蝇在远处嬉戏,这本是最不引人注意的现象,这根翎毛却已警惕起来,自动指向它的方向。等它稍稍飞近,羽毛的尖端就开始摇动,像响尾蛇摇尾巴一样,发出一种威胁信号;摇动的频率和幅度随着苍蝇逼近的程度越来越大。等到苍蝇逼近翎毛所能及的距离时,它却一动也不动了;静待苍蝇进一步靠近。直到它飞进死亡陷阱,才猛烈地一抽,把它从空中击落。你很难相信这是薛嵩的肛门括约肌创造了这种奇迹,倘如此,人的屁眼儿还有什么做不到的事情呢?我倒同意红线的意见,薛嵩有一部份已经变成马了…… 

这种情形使红线大为振奋,她终于骑到他身上,用脚跟敲他的肋骨,催他走动。而薛嵩则不禁摇首振奋,摇动那根孔雀翎,几乎要放足跑动。照这个方向发展下去,结果是显而易见的:薛嵩变成了一匹马。在红线看来,一个丈夫和一匹马,哪种动物更加可爱是显而易见的。特别是她觉得这匹马没有毛,皮肤细腻,骑起来比别的马舒服多了…… 

但是,故事没有照这个方向发展。薛嵩对红线的骚扰始终无动于衷,只说了一句“别讨厌”,就专注于他的修理工作。这态度终于使红线肃然起敬。她从他身上清除掉一切恶作剧的痕迹,找来了一片芭蕉时,给他打起扇来了……虽然这个故事还没有写完,但我已经大大地进了一步。 

现在,万寿寺里也到了正午时节,所有的蝉鸣声嘎然而止。新粉刷的红墙庄严肃穆,板着脸述说着酷暑是怎样一回事。而在凤凰寨里,薛嵩蹲在地上,膝盖紧贴着腋窝,肩膀紧夹着脑袋,手捧着木制零件,研究着自己制造的弩车──他的姿式纯属怪涎,丝毫也说不上性感。但红线却以为这种专注的精神十足性感。因为她从来也不能专注地做任何事,所以,她最喜欢看别人专注地做事,并且觉得这种态度很性感……与此同时,薛嵩却一点点进入了这架弩车的木头内心,逐渐变成了这辆弩车。就在这时,红线看到垂在他两腿之间的那个东西逐渐变长了,好像是脱垂出来的内脏──众所周知,那个东西有时会变得直撅撅,但现在可不是这个模样。仅从下半部来看,薛嵩像匹刚生了马驹的老母马。那东西色泽深红,一端已经垂到了地上。这景象把庄严肃穆的气氛完全破坏了。开头,红线用手捂着嘴笑,后来就不禁笑出声来了。薛嵩傻呵呵地问了一句:你笑什么?红线顾不上回答。这种嘻皮笑脸的态度当然使薛嵩恼怒,但他太忙,顾不上问了。那个白衣女人对这个故事大为满意,她说:写得好──你们男人就是这样的!这句话使我如受当头棒喝。原来我们男人就是这样的没出息! 

3 

我终于明白了我为什么对自己不满:我是一个男人,有着男性的恶劣品行:粗俗、野蛮、重物轻人。其中最可恨的一点就是:无缘无故地就想统治别人。在这些别人之中,我们最想要统治的就是女人。这就是男人的恶行,我既是男人,就有这种恶行…… 

看过了《甘泽谣》的人都知道红线盗盒的故事是怎么结束的:薛嵩用尽了浑身的解数,也收拾不了田承嗣。最后是红线亲自出马,偷走了田承嗣起卧不离身的一个盒子,才把他吓跑了。现代的女权主义文论家认为,这个故事带有妇女解放的进步意义,美中不足之处在于:不该只偷一个盒子,应该把田承嗣的脑袋也割下来。这真是高明之见,我对此没有不同意见。我要说的是:的确存在着一种可能,就是薛嵩最终领悟到大男子主义并不可取,最终改正了自己的错误。但是冰冻三尺非一日之寒,一个人在改变中,也会有反复�。因为这个缘故,每次看到薛嵩的把把变粗变直,红线就会奋起批判:好啊薛嵩!你�又来父权制那一套了!让大家都看看你,这叫什么样子?而这时薛嵩已被改造好了,听了这样的指责,他感到羞愧难当,面红耳赤地说:是呀是呀。我错了……下次一定不这样。 

可借仅仅认错还不能使那个东西变细变软,它还在那里强项不伏。于是,红线就吹起铜号,把整个寨子里的人都招来,大家开会批判大男子主义者薛嵩,那个直挺挺的器官就是他思想问题的铁证。说实在的,很少有哪种思想问题会留下这样的铁证──而且那东西越挨批就越硬。久而久之,薛嵩也有了达观的态度,一犯了这种错误就坦白道:它又硬了,开会批判罢──这哪叫一种人过的生活呢。好在有时红线也会说:好吧,让你小孩吃巴巴,就躺下来,和薛嵩做爱──像这样的生活能不能叫作快乐,实在大有疑问…… 

这样写过了以后,我忽然发现自己并没有统治女人的恶劣品行。我能把薛嵩的下场写成这个样子,这本身就是证明……我和他们没有任何关系。顺便说一句,我想到了自己对领导的许诺──我在工作报告里写着,今年要写出三篇《精神文明建设考》──既然说了,就要办到。这个故事我准备叫它《唐代凤凰寨之精神文明建设考》。白衣女人对此极感兴奋,甚至倒在双人床上打了一阵滚;这使我感到一定程度的满足。滚完了以后,她爬起来说:可别当真啊。这又使我如坠五里雾中:我最不懂的就是:哪些事情可以当真,哪些事情不能当真。 

4 

不久之前,万寿寺厕所的化粪池堵住了,喷涌出一股碗口粗细的黄水。这件事发生在我撞车之前,这段时间里的事我多半都记不起来,只记起了这一件。它给我带来了极大的痛苦,因为我只要看到那片黄水,就有一种按捺不住的欲望,要用竹片去把下水道捅开──连竹片我都找好了。而那位自衣女人见到我的神情,马上就知道我在想什么。她很坚决地说:你敢去捅化粪池───马上离婚。因为这个威胁,那片黄水在万寿寺里蔓延开来。这种液体带着黄色泡沫,四处流动。领导打了很多电话,请各方面的人来修,但人家都忙不过来。后来,那片黄水漫进了他的房间。他只好在地上摆些砖头以便出入,自己也坐在桌子上面办公。有些黄色的固体也随着那股水四下漂流。黄水也漫进了资料室,里面的几个老太太也照此办理,并且戴上了口罩。与此同时,整个万寿寺弥漫着火山喷发似的恶臭。全城的苍蝇急忙从四面赶来,在寺院上空发出轰鸣……这种情形使我怒发冲冠。没有一种道理说,所有的历史学家都必须是学院派,而且喜欢在大粪里生活。豁出去不做历史学家,我也一定要把壅塞的大粪桶开。 

在此情形之下,那个白衣女人断然命令道:走,和我到北京图书馆查资料去。我坐在图书馆里,想到臭轰轰的万寿寺,心痒难熬。而那位白衣女士却说:连个助研都不给你评(顺便说一句,我还没想起助研是一种什么东西),你却要给人家捅大粪!我的上帝啊,怎么嫁了这么个傻男人!后来,我逃脱了她的监视,飞车前往万寿夺,在路上被面包车撞着了。因为这个缘故,她在医院里看到我时,第一句话就是:你活该!然后却哭了起来。当时我看到一位可爱的女士对我哭,感到庄严肃穆,但也觉得有点奇怪:既然我活该,她哭什么呢?我丝毫也没有想到这种悲伤的起因竟是四处漫延的大粪。当然,大粪并不是肇事的真正原因。真正的原因是:我是现代派,而非学院派。现代派可以不评助研,但不能坐视大粪四处漫延……那白衣女人现在提起此事,还要调侃我几句:认识这么多年,没见过你那个样子。见了屎这么疯狂,也许你就是个屎克螂?我很沉着地答道:我要是屎克螂,你就是母屎克螂。既然连被撞的原因都想了起来,大概没有什么遗漏了。薛嵩走上塔顶去修理锅炉的故事跨过丧失的记忆,从过去延伸到了现在……

\section{第七章}

第一节 

早上我在万寿寺里,在金色的琉璃瓦下。从窗子里看去,这里好像是硫磺的世界,到处闪着硫磺的光芒,还有一股硫磺的气味。我多次出去寻找与硫磺有关的工厂,假如找到的话,我要给市政府写信,揭发这件事,因为硫磺不但污染环境,还是种危险品,不能放在万寿寺边上。结果是既没有找到工厂,也没有找到硫磺,而且一出了寺门气味就小了。事实是:我们正在污染环境,我们才是危险品。面馆里的人还抱怨说,我们发出的气味影响了他们的生意。这样我就不能写这封信了──因为人是不该自己揭发自己的呀。 

从医院里出来已经有一个礼拜了。我有一个好消息:我的记忆正在恢复中,每时每刻都有新的信息闯进我的脑海。但也有很多坏消息,这是因为这些记忆都不那么受我的欢迎。比方说这一则:我不是历史学家。我已经四十八岁了,还是研究实习员,没有中级职称。学术委员会前后十次讨论我的晋升问题。头三次没有通过,我似乎还有点着急。到了第四次我就不再着急。第五次评上了,我又让了出去,让给了一个比我岁数大的人。领导说:这是你自己要让啊,可不要怪我们;我只微笑着点了一下头。第五次以后总能评上,我自己高低不同意晋职,说自己的水平不够。第十次发生在我撞车之前,我还是不同意晋升,并且再三声明,我准备在一百岁时晋升助理研究员,并在翌年死去。谁敢催我早日晋升就是催我早死。但不知为什么,他们收走了我的工作证,发回来时就填上了新职称。不管别人怎么说,我都不承认自己已经晋升了中级职称──就是这样,我还被车撞了,这完全是领导给我强行晋职所致──既然我没有职称,也就不是历史学家。但我还不至于什么人都不是:我大体上是个小说家。 

在香案底下,我找到了一叠积满了尘土的文学刊物,上面都有署我名字的作品。我还出过几本小说集。今天,我还收到了一张汇款单,附言里写明了是稿费。还有一封约稿信,邀请我写篇短篇小说,参加征文比赛,但很婉转地劝我少一点“直露”的描写──我想这是指性描写。这些事我一点都记不得了。但既然是小说家,那就好好写吧。 

我把薛嵩的故事重写了一遍,就是现在这个样子。中午,那个自称我老婆的白衣女人把它从头到尾看了一遍,不置可否地放下了。这使我感到失望。我总觉得,失掉记忆以后,我的才能在突飞猛进,可以从前后写出的手稿中比较出来。现在我正期待着别人来验证。我问她道:怎么样?她反问道:什么怎么样?这使我感到沮丧──她连我的话都听不明白了;或者说,我自己连话都说不明白了。这两种说法中,后一种更为通顺,但我更喜欢前一种。我说:这回的稿子怎么样?她淡淡地答道:你总是这样,反反复复的。说完就从房间里走了出去。按说我该感到更加沮丧才对。但是我没有。她走路的样子姿仪万方,我总是看不够。 

2 

在我失掉记忆之前,写到:盛夏时节,薛嵩走过金色的池塘,去给学院修理一具热水锅炉。现在我必须接着写下去。在写这件事之前,我必须说说这件事使我想到了些什么:我自己念研究生时,就常常背着工具袋,去给系里修理东西,我自己还念过研究生,有硕士学位,这使我不胜诧异。系里领导直言不讳他说:他们录取我,不是看中了我的人品和学业,而是看中了我修理东西的手艺──这就提示我,我的人品和学业都不值得回忆,只有手艺是值得回忆的。历史系和别的文科系不同,有考古实验室,文物修复室,加上资料室、计算机教室,好大的一份家业,要修的东西也很多。顺便说一句,领导对我说这样的话,不是表扬我有手艺,而是提醒我,修理东西是我应尽的义务,不要指望报酬了……对薛嵩来说,学院是什么地方、要修的是一台什么锅炉等等,只要你把薛嵩当成了我这佯的人,就无须解释。只要让他知道有座锅炉坏了,这就够了,他立即就会去修理。 

薛嵩要修的锅炉在一座八角形的楠木大塔上,这座大塔又在一个新月形的半岛的顶端,这个半岛伸在一个荒芜的湖里。在湖水的四周,没有一棵树。湖里也没有一棵芦苇,只有金色透明的湖水。正午时分,塔上金色的琉璃瓦闪着光。我以为,这是很美丽的景色。但薛嵩没有看风景,他走进了塔里。在塔的内部,是一个八角形的天井,有一道楼梯盘旋而上,直抵塔顶。这是很美丽的建筑。但薛嵩也无心去看,只顾拾级而上。在塔的每一层,学院里的姑娘们在打棋谱,研究画法,弹着古琴研究音律,看到有个男人经过,都停下来看他。这都是些很美丽的女人。但他也无心去看,一直登到塔顶去看那个坏了的锅炉。这是因为,这台坏掉的锅炉──说实在的,这算不上是一台锅炉,只是一个大肚子茶炊,是精铜铸成的,擦得光可鉴人──是他的一块心病,是来自内心的奇痒。在茶炊顶上,有一具黑铁制成的送炭器,是个马鞍蹬子一样的东西,用来把炭送进炉膛。这个东西前不久刚修理过,现在又坏了。在折断的铁把手上,留下挫过的痕迹。这是破坏……问题在于,谁会来破坏一具茶炊?薛嵩直起身来,看着塔里来来去去的女人们。在这些女人中,有一个爱上他了。所以她总要破坏茶炊,让他来此修理。现在的问题是:她是谁?在塔里那些像月亮一样美丽的姑娘中,她是哪一个?在我已经写到过的女人里,她又是谁? 

我依稀觉得,这就是我自己的故事,系里的每件仪器我都修过,这不说明别的,只说明历史系拥有一批随时会坏掉的破烂。考古试验室的主任是个有胡子的老太太,我看过一台仪器后,说道:旧零件不行了,得买新的。她说:你把型号写下来,我去买。我二话不说,背起工具包就走;因为我觉得她不让我去买零件,是怀疑我要贪污,这是对我人格的羞辱──这样走了以后,她更加怀疑我要贪污。对于羞辱这件事,我有这样的结论:当一件羞辱的事降临到你头上时,假如你害怕羞辱,就要毫无怨言地接受下来,否则就会有更大的羞辱。但这是真实发生了的事,不是故事。 

有一次,在我的故事里,我走上了一座高塔去修理一具茶炊。在这座塔的内部,到处是一片金黄:金丝捕木做的护壁、楼梯扶手,还有到处张挂的黄缎子;表面上富丽堂皇,实际上俗不可耐。相比之下,我倒喜欢在塔顶上那片铁。它平铺在惺亮的茶炊下面,身上堆满了黑炭。这种金属灰溜溜的,没有光泽,但很坚硬。不漂亮,但也不俗气。 

我走上陡峭的楼梯,从喧嚣的声音中走过。这些琴、瑟、笙、管,假如单独奏起来,没有人会说难听,但在一座塔里混成一团,就能把人吵晕。我又从令人恶心的香烟中走过,这些檀香、麝香、龙涎、冰片,单独闻起来都不难闻,混在一起就叫人恶心。这地方还有很多姑娘,单看起来个个漂亮,但都穿着硬邦邦的黄缎子,描眉画目,乱糟糟地挤在了一起,就不再好看。在这座大塔的天井里,正绞着一道黄色、炽热的旋风。我虽是从风边走过,但已感到头晕。 

在那片黑铁上,紧靠着茶炊有一道板障,板障下面放了一个大板凳,有个姑娘坐在上面。她可没穿黄缎子,几乎是全裸着的,双脚被铁索锁住。仔细一看,她不是自愿坐在这里的。在她身后的板壁上有个铁环,又有一道铁索套住了她的脖子,把她锁在了铁环上,还有一根大拇指粗细的木棍,卡在她的嘴里,后面有铁箍勒住。至于双手,则被反锁在身后。这个姑娘闭着眼睛缩成一团,在热风里出着汗,浑身红彤彤的,好像在洗桑拿浴──这是全楼最热的角落,因为热气是上升的,又有填满了红炭的茶炊在烤着。她脸上没有化妆,头发因酷热而乾枯,看不出是不是漂亮。但我以为她一定是漂亮的,因为她是这样的不同凡响。陪我来的老虔婆介绍说,学院里规矩森严。这个姑娘犯了门规,正在受罚。我顺嘴问道:她吃豆予了吗?随着我的声音在板壁间响起,那个姑娘朝我睁开了眼睛,张开嘴巴,露出咬住木棍的两排整齐的牙齿,朝我做了个鬼脸。与此同时,老虔婆也宣布了她的罪状:“破坏茶炊”。这种罪名完全在我的意料之内。 

在那个老虔婆的监视下,我解开了脚上套着的白布口袋,踏上那片黑铁,套这两个口袋,是要防止我这俗人污染了学院神圣的殿堂──顺便说说,我给考古室修东西时,脚上倒不用套袋子,只是要穿白大褂──把沉重的帆布工具袋放在黑铁上。就在这时,那双被铁链锁在一起的脚对我打出一个手势:左脚把右脚抱住,在趾缝之间透出一根足趾,上下摆动着。这是一条马尾巴。我知道这是讥笑我的袋子,说它像个挂在马尾巴下面的马粪袋子。这个帆布袋子上满是污渍,不用她说我也知道它像什么。对于这种恶毒攻击,我也有反击的手段。我用左手比成一个马头,把右手的食指放到马嘴里去,这是比喻她像马一样戴着衔口。然后,我拿着一把扳手站了、起来,假装无意地看了她一眼,只见她正作出个苦脸,假装在哭。这就是说,我的比方太过恶毒,她不喜欢了。但转眼之间她脸上又带上了娇笑,含情脉脉地看着我。我不动声色地转过身去,开始修理茶炊。如前所述,我早就知道锅炉会坏,坏在哪里,所以我把备件带了来。但我不急于把它修好,慢吞吞地工作着。那个老虔婆耐不住高温,说道:师傅您多辛苦,我去给你倒杯茶来,就离去了。假如我真的相信她会给我倒茶,那我就是个傻爪。此时,茶炉间里只剩下了我们两个人。 

3 

正午时节,那位白衣女人在我房间里,看我的稿子,和我聊天,这使我感到很幸福。一点半以后,我们那位戴白边眼镜的领导就出现在院子里,不顾烈日当头和院子里的恶臭在徘徊着。随着时间的推移,他踱步的路线朝我门日靠近。等到两点整,他乾脆就是在我门前跺着脚绕圈子。有点脑子就能猜出来,他是告诉我们,上班时间已到,应该开始工作。不用有脑子你也能猜到,他就是我故事里的那个老虔婆。因为他的催促,白衣女人只好从我这里走出去,回到自己屋里。 

在我的故事里,离去的却是那个老虔婆。我马上扑到她面前,迅速地松开铁箍,她就把那根木头棍子吐了出来,还连吐了两口唾沫,说道:苦死了。你猜那是根什么木棍?黄连树根。学院派整起人来可真有些本领……然后,我把这个浑身发烫、头发蓬松的姑娘抱在了怀里,一面亲吻她的脖子,一面松掉她脖子上的铁锁,让她可以站起来。然后,轻轻咬着她的耳朵,抚摸着她的乳房。这地方比平常柔软。她说:天热,缺水,蔫掉了。我马上拿出木头水壶,给她喝了几口,又往蔫掉的地方浇了一些。现在我看出这姑娘已经不很年轻,嘴角有了皱纹,脖子上的皮也松弛了。但只有这种不很年轻的姑娘才会真正美丽…… 

我像一个夜间闯进银行的贼,捅开她身上的一重重的锁。看来学院真不缺买锁的钱。这世界上没有捅不开的锁,只是多了就很讨厌──转到她后面才能看到,那一串锁就像那种龙式的风筝。把所有的锁都捅开之后,我就可以和她做爱,在这个闷热、肮脏的茶炉间里大干一场。为此我摊开了工具袋,她也转过身去,蹲了下来,让我在她背上操作。不幸的是,这串锁只开到了一半,楼梯上响起了沉重的脚步声。她小声嚷道:别开了!‘决把我再锁上!于是又开始了相反的过程,而且是手忙脚乱的。但是上锁总比开锁容易,把那个木头衔口放回她嘴里前,我和她热烈地亲吻──她的嘴很苦,黄连树根的味道不问可知。等到那老虔婆走进茶炉间时,她已经在板凳上坐下,我也转过身去,面向着茶炊,作修理之状,如前所述,我早就知道这茶炊要坏,而且知道它会坏在哪里,所以带来了备件。但现在找不到了。怎么会呢?这么大的东西,这么点地方!我满地乱爬着找它,忽然看到那双被铁链重重缠绕的脚在比划着一个手势:右脚的大脚趾指向自己。这下可糟了。那东西锁在她身上了!现在没有机会把它再拿下来…… 

白衣女人离开之后,领导继续在我门口徘徊,谁都不喜欢有人在门口转来转去,所以我起身把窗子全部打开,让他看看我屋里没有藏着人。但他不肯走,还在转着,与此同时,臭味从外面蜂拥而入。所以我只好关上窗子,请领导进来坐。他假作从容地咳嗽一声,进了这间屋子,在白衣女人坐过的方凳上坐下;我也去写自己的小说,直到他咳嗽了最后一声──他咳嗽每一声,我就从鼻子里哼一声,这样重复了很多回,在此期间,我一直埋头写自己的小说──清清嗓子道:看来我们需要谈谈了。我头也不回地答道:我看不需要;嗓音尖刻,像个无赖。他又说:请你把手上的事放一放,我在和你说话。我把句子写完,把笔插回墨水瓶,转过身来。他问我在写什么,我说是学术论文。他说:能不能看看,我说不能。就是领导也不能看我的手稿,等到发表之后我自会送他一份。随着这些弥天大谎的出笼,一股好邪的微笑在我脸上迅速地弥散开来。看来,我不是个良善之辈,我又把自己给低估了…… 

领导和我谈话时并没有注意到,我不是一个人,是一个小宇宙;在其中不仅有红线、有薛嵩、有小妓女和老妓女,还有许多别人。举个例子,连他自己也在内,但不是穿蓝制服、戴白边眼镜,而是个太阳穴上贴着小膏药的老虔婆。假如他发现自己在和如此庞大的一群人说话,一定会大吃一惊,除此之外,我还是相当广阔的一段时空。他要是发现自己对着时空作思想工作,一定以为是对牛弹琴,除了时空,还有诗意──妈的,他怎么会懂得什么叫作诗意。除了诗意,还有恶意。这个他一定能懂。这是他唯一懂得的东西。 

在我这个宇宙里,有两个地方格外引人注目:一处是长安城外金色的宝塔,另一处是湘西草木葱宠的风凰寨。金色的宝塔是阳具的象征,又是学院所在地。看起来堂皇,实际上早就疲软了,是一条历史的脐带……领导对我说,我现在有了中级职称,每年都要有一定的字数(他特别指出,这些字数必须是史学论文,不能拿小说来凑数),如果完不成,就要请我调离此地。不是和我为难──这是上级的规定。说完了这些屁话,他就起身从我屋里踱了出去。他走之后,我感到愤怒不已,决定摔个墨水瓶子来泄愤。然后我就惊诧不已:墨水瓶子根本就摔不碎…… 

我把故事和真实发生的事杂在一起来写,所以难以取信于人。如果我说,我们领导教训了我一顿,一转身就变成了一条老水牛,甩着沾了牛屎的尾巴,得意洋洋地从我房里走了出去,两个睾丸互相撞击,发出檐下风铃的金属声响,你也不会诧异──但墨水瓶子摔不碎不是这类事件。我有很多空墨水瓶,贴着红色的标签,印着中华牌炭素墨水,57ml,还有出厂日期等等。你把它往砖地上一摔,它就不见了,只留下一道白印。与此同时,头上的纸顶棚上出现了一个黑窟窿,再摔一个还是这样,只是地下有了两道白印,头上有两个黑窟窿。这些空瓶子就这样很快地消失了,地上没有一片碎玻璃,顶棚上有很多窟窿──隔壁的人大声说道:顶棚上闹耗子!最后剩下了一个墨水瓶,我把它拿在手里端详了一阵:这种扁扁的瓶子实在是种工程上的奇迹,设计这种瓶子的肯定是个大天才。我把它拿到外面去,灌满了水,在石头台阶上一摔,这回它成了碎片。随着水渍在台阶上摊开,我感到满意,走回自己屋里。 

4 

我站起来,转向老虔婆,一本正经地告诉她,茶炊坏得很厉害,无法马上修好。那个老太太擦着额头上的汗说:那怎么办?楼下这么多姑娘要喝水……越过老虔婆,身后的姑娘在板凳上往后仰,做哈哈大笑之状。我说:我回去做备件,做好了明天再来。现在没有理由再呆在这里。我只好提起工具袋……那个姑娘朝我送了一吻,这一吻好似猩猩的吻──这当然是因为嘴里衔着木棍。这一吻可以把我的左颊和右颊同时包括在内。趁那老虔婆不注意,我朝她做了个鬼脸,走出了这座塔,走到外面金色的风景里去,但也把一缕情丝留在了身后。无论是我,还是薛嵩,对已经发生的事情还算是满意。唯一不满的是那黄连树根,谁也不愿把那么苦的东西放到爱人嘴里。假如有一种木头是甜的就好了。我可用它作根衔口,把塔里的黄连树根换掉……说实在的,塔里的茶炊设得不好,尤其是送炭器。那地方不该做成马蹬状,而是应该做成滚筒状。当然,做成滚筒状,破坏起来就更难了。 

我在金色的风景里徘徊……实际上,我是在万寿寺里,面对着一张白色的稿纸。如前所述,我总是用发黄的旧稿纸写小说,现在换上了这种纸,说明我想写点正经东西。在昏迷之中,我已经写出了题目:《唐代精神文明建设考》。这个题目实在让我倒胃……回头看看那座金色的塔,它已经是金色余晖中的一道阴影。很多窗口都点起了金色的灯火。在这个故事开始时,我走上这座塔,假作修理茶炉,实际上是来会我爱的姑娘;在这个故事结束时,我用重重枷锁把她锁住,把黄连木的衔口塞在了她嘴里。现在我发现,我把这个故事讲错了。实际上,是别人用重重锁链把我锁住,又把黄连木的衔口塞到了我的嘴里,我愤然抓起那张只写了题目的稿纸,把它撕得粉碎,然后在晚风中,追随那件白色的衣裙回到家里;在不知不觉之中就到了午夜──在床上,她拿住了我的把把,问道:怎么,没有情绪?我答道:天热,缺水,蔫掉了……与此同时,我在蔫蔫地想着:能不能用已知的史料凑出个《唐代精神文明建设考》。假如不能,就要编造史料。这件事让人恶心:我是小说家,会编小说,但不编史料…… 

在长安城外的大塔上,在乌黑闷热的茶炉间里,带着重重枷锁缩成一团,我也准备睡了。这个故事对我很是不利:灼热的空气杀得皮肤热辣辣的,嘴里又苦得睡不着。板凳太窄,容不下整个屁股,脖子上的锁链又太紧,让我躺不下来。唯一的希望就是:薛嵩还会再来。他会松开我身上的锁链──起码会把脚腕上的锁链松开。此后,就可以分开双腿,用全身心的欢悦和他做爱。生活里还有这件有趣的事,所以活着还是值得的──这样想着,我忽然感到一种剧烈的疼痛,仿佛很多年后薛嵩射出的标枪现在就射穿了我的胸膛……不管我喜不喜欢,我现在是那个塔里的姑娘,也就是那个后来在凤凰寨里被薛嵩射死的老妓女。对她的命运我真是深恶痛绝──这哪能算是一种人的生活呢?不幸的是,每个人都有自己的命运,你别无选择。假如我能选择,我也不愿生活在此时此地。 

第二天早上,带着红肿的眼睛和无处不在的锁链的压痕,我从板壁上被放了下来,回到自己的房间里。这间房子在塔角上,两面有窗子,还有通向围廊的门。在门窗上钉有丝质的纱网。就是在正午,这里也充满了清凉的风,何况是在灰色的清晨。地板上铺着藤席,假如我倒下去,立刻就会睡着,但现在塔里已是起身的时节。现在已经别无选择,只能用冷水洗脸,以后在镜前描眉画目,遮掩一夜没睡的痕迹,以免被人笑话。再以后,穿上黄缎子的衣服,在席子上端坐。在我面前的案上,放着文房四宝,一大叠宣纸的最上面一张,在雪白的一片上,别人的笔迹赫然写着题目:《先秦精神文明建设考》。很显然,这个题目不能医治,而是只能加重我的瞌睡。现在我有几种选择:一种是勉强瞎制上几句。这么大的人了,连官样文章都写不出,也实在惹人笑话。另一种选择是用左手撑着头,作搜索枯肠状,右手执笔在纸上乱描。实际上我既不是在搜索枯肠,也不是在乱描,而是在打瞌睡。还有一种选择是不管三七二十一,躺倒了就睡。等他们逮到我,想怎么罚就罚好了。但这都不是我的选择。我端坐着,好像在打腹稿,眼睛警惕着在门外巡逡的老虔婆,一只脚却伸到了席子下面,足趾在板缝里搜索着,终于找到了几条硬硬的东西。我把其中一条夹了出来,藏在袖子里──这是一把三角锉。这样,我又能够破坏茶炊。然后被锁在茶炉间里。然后薛嵩就会来修理。然后就有机会和他做爱。性在任何地方都重要,但都不如在这座塔里重要。在这里,除此之外再没有值得一做的事了。 

后来,这个塔里的姑娘离开了长安城,随着薛嵩来到了凤凰寨。在这个绿叶和红土相间的地方,岁月像流水一样过去,转眼之间就到了生命的黄昏。她始终爱着薛嵩,但薛嵩却像黄连木一样的苦──他用情不专,到处留情……而且,不管是有意无意,反正最后还是薛嵩把她射死了。对此,我完全同意红线的意见:薛嵩是不可原谅的。看着他假模假式的哀痛之状,红线几番起了杀心──假如她要杀他,就可以把薛嵩当作一个死人了,因为那就如白衣女人要杀我,是防不胜防的。但是最后红线决定不杀薛嵩,这是因为薛嵩是个能工巧匠──一个勤奋工作的人。一个人只要有了这种好处,就不应该被杀掉。 

5 

上述故事可以发生在薛嵩到凤凰寨之前,也可发生在薛嵩离开凤凰寨之后;所以,它可以是故事的开始,也可以是故事的终结。故事里的女人可以是老妓女,也可以是小妓女、红线,或者是另外一个女人。只有薛嵩总是不变。这是因为我喜欢薛嵩。 

这座金色宝塔里佳丽如云,长安最漂亮女人住在里面。进这座塔是女人最大的光荣,但是在这座塔里面,漂亮绝无用武之地。学院也是这样的地方,能进学院说明你很聪明,但在学院里面又最不需要聪明。在这里呆久了,人会变得癫狂起来──我就是这么解释自己。我学了七年历史,本科四年、研究生三年,又在万寿寺里呆了十年半。再呆下去我也不会更聪明。假如那个塔里的姑娘也呆了这么久,她应该是三十五六岁,在女人最美丽的年龄。再呆下去,她也不会更加美丽。 

转眼之间已经入秋,塔里的人脱下身上的黄缎子,换上开司米的长袍。我大概是最后换季的人,因为我喜欢秋天的凉意──现在已是深秋时节。深秋时的早晨有种深灰色的雾笼罩着一切,穿过窗纱,钻进网里来──既是雾,又是露水。黄缎子不再娑娑做声,开司米表面也笼罩着一层水珠。此时我正对着镜子更衣。这面镜子有粗笨的镜座,厚重的镜片,都用黑色的古铜制作,镜背上错有银丝的图案,镜面上镀了一层锡──但薛嵩骗管总务的老虔婆说,镀的是银。这座塔里的器具多半是薛嵩所制,因为薛嵩做的东西总是最好的。正因为如此,塔门口就立了一块牌子:不通琴棋书画者,以及薛嵩,禁止人内。如你所知,这块牌子拾了古希腊毕达哥拉斯学派的牙慧。在这座宝塔里,人们认为琴棋书画的层次很高,能工巧匠的层次很低。薛嵩是所有的能工巧匠中最出色者,所以他层次最低;即便他琴棋书画无所不通,也不能让他入内。坦白地说,我认为这种算法是有问题的:就算能工巧匠层次低,能工巧匠中最出色者层次应该是较高才对;不应该把他算成层次最低。但是,我也不想去和老虔婆说理。因为女人给自己的爱人说理,层次已经很低,假如说赢了,层次就会更低。既然如此,就不如不说理。 

在那座金色的宝塔下面,所有的苹果树都树起了绿叶,和南方的橡皮树相似;并且挂满了殷红的果实,这些果子会在枝头由红变紫,最后变成棕黑色,同时逐渐萎缩,看上去像枯叶或者状似枯叶的蛾子。所幸这是一些红玉苹果,只好看,不好吃;所以让它们干掉也不特别可惜。全中国只有这个地方有苹果树,别的地方只有“揪子”,它也属苹果一类,树形雄伟,有如数百年的老橡树,但每棵上只结寥寥可数的几个果子,吃起来像棉花套子──虽然是甜的。水边的枫树和山毛榉一片鲜红,湖水却变成了深不可测的墨绿色。在这片景色的上空,弥散着轻罗似的烟雾,一半是雾,一半是露水。 

在镜子里看到的身体形状依旧,依然白皙,但因为它正在变软,就带着一点金黄色。因此它需要薛嵩,薛嵩也因为这身体正在变软,所以格外的需要它。假如一个身体年轻,清新、质地坚实,那就只需要触摸,只有当它变软时,才需要深入它的内部。看清楚以后,她穿上细毛线的长袍,这件衣服朦朦胧胧地遮住了她的全身,有如朦胧的爱意。但是朦陇的爱意是不够的,她需要直接的爱。 

对这个金色宝塔的故事,必须有种通盘的考虑。首先,这塔里有个姑娘,对着一面镀锡的青铜镜子端详自己。她的身体依旧白皙,只是因为秋天来临,所以染上了一丝黄色。秋天的阳光总是带着这种色调,哪怕是在正午也不例外。在窗外,万物都在凋零:这是最美的季节,也是最短暂的季节。所以,要有薛嵩──薛嵩就是爱情。 

其次,薛嵩在塔外,穿着一件黑斗篷在石岸上徘徊,从各个方向打量这座塔,苦思着混进去的方法。他在想着各种门路:夜里爬上宝塔;从下水道钻进地下室,然后摸上楼梯;乘着风筝飞上去。所以,塔里要有一个姑娘,这个姑娘就是爱情。 

除此之外,还有第三种考虑,早上,这个石头半岛上弥漫着灰色的青烟──既是雾,又是露水,青烟所到之处,一切都是湿漉漉的,冰人指尖;令人阴囊紧缩,阴茎突出;或者打湿了毛发,绷紧了皮肤。这种露水就是爱情。所以,要有薛嵩,也要有塔里的女人。我自己觉得这最后一种考虑虽不真实,但颇有新奇之处,是我最喜欢的一种,作为一个现代派,我觉得真实不真实没什么要紧。但白衣女人却要打我的嘴巴:我们不是爱情,露水才是爱情?滚你的蛋吧!这就提出了一种新的思路:对方不是爱情,环境也不是爱情。“我们”才是爱情。现在的问题是:谁是那些“我们”? 

第二节 

我给系里修理仪器时,经常看到那位白衣女人。她穿着一件白大褂,在蓝黝黝的灯光下走来走去;看到我进来就说:哟,贪污分子来了。我一声不吭地放下工具,拖过椅子坐下,开始修理仪器。这种态度使她不安,开始了漫长的解释:怎么,生气了?──开个玩笑就不行吗?──嘿!我知道你没贪污!说话呀!──是我贪污行不行?我贪污了国家一百万,你满意了吧?……我是爱国的,有人贪污了国家一百万,我为什么要满意?但我继续一声不吭,把仪器的后盖揭开,钻研它的内脏。直到一只塑料拖鞋朝我头上飞来,我才把它接住,镇定如常地告诉她:我没有生气,何必用拖鞋来扔我呢。我从来没有贪污过一分钱,却被她叫作贪污分子,又被拖鞋扔了一下,我和那个塔里的姑娘是一样的倒霉。 

秋天的下午,我在塔里等待薛嵩。他的一头乱发乱蓬蓬地支愣着,好像一把黑色的鸡毛掸子;披着一件黑色的斗篷,在塔下转来转去,好像一个盗马贼。在他身后,好像摊开了一个跳蚤市场,散放着各种木制的构架,铁制的摇臂,还有够驾驶十条帆船之用的绳索。除此之外他还在地上支起了一道帷幕,在帷幕后面有不少人影在晃动。这样一来,他又像一个海盗。天一黑他就要支起一座有升降臂的云梯,坐在臂端一头撞进来,现在正在看地势。因为没有办法混进这座塔,他就想要攻进来。通常他只是一个人,但因为他是有备而来,所以今天好像来的人很多。 

对于薛嵩,塔里已经有了防范措施,在塔的四周拉起了绳网。但如此防范薛嵩是枉然的,也许那架绳梯会以一把大剪子为前驱,把绳网剪得粉碎,也许它会以无数高速旋转的挠钩为前驱,把绳网扯得粉碎。塔里的人也知道光有绳网不够,所以还做着别的准备。如前所述,我在等待薛嵩,所以我很积极地帮助拉绳网,用这种方式给自己找点别扭。 

在绳网背后,有一些老虔婆提来了炭炉子,准备把炭火倒在薛嵩头上,把他的云梯烧掉。我也帮着做这件事:用扇子煽旺炭炉子。但做这些又是枉然的。薛嵩的云梯上会带有一个大喷头,喷着水冲过来,连老虔婆带她们的炭炉子都会被浇成落汤鸡。又有一些老虔婆准备了油纸伞,准备遮在炭炉上面。这也是枉然的,薛嵩的云梯上又会架有风车,把她们的油纸伞吹得东歪西倒。塔里传着一道口令:把所有的马桶送到塔顶上来,这就是说,她们准备用秽物来泼他。听到这道命令,我也坐在马桶上,用实际行动给防御工作做点贡献。但这也没有用处,薛嵩的云梯上自会有一个可以灵活转动的喇叭筒,把所有的秽物接住,再用唧筒激射回来。只有一位老虔婆在做着最英明的事情,她把塔外那块牌子上“薛嵩不得入内”的字样涂掉了。这样他就可以好好地进来,不必毁掉塔上的窗子。但这也是枉然的,薛嵩既已做好了准备,要进攻这座塔,什么都不能让他停下来。塔里所有的姑娘都拥到了薛嵩那一侧的围廊上,在那里看他作进攻的准备,这就使人担心塔会朝那一面倒下来…… 

有关这座宝塔,我已经说过,塔里佳丽如云。全长安最漂亮的女人都在里面,所以,能进这座塔就是一种光荣。但是光有这种光荣是不够的。还要有个男人在外面,为你制造爱情的云梯,来进攻这座反爱情的高塔。因为这个原故,那些姑娘在围廊上对薛嵩热情地打招呼、飞吻,而薛嵩正在捆绑木架,嘴里咬着绳索不能回答,只能招招手。因为他是个暂时的哑巴,所以谁是他此次的目标暂时也是个谜。说实在的,我也不想过早揭开谜底。 

天刚黑下来,薛嵩已经把云梯做好,坐在自己的云梯上,就如一个吊车司机。但整个升降臂罩在一片黑布帷幕下面,就如一座待揭幕的铜像。他打算怎样攻击这座塔也是一个谜──所有的姑娘都屏住了呼吸,把双手放在胸前,准备鼓掌。我也想看看他这回又有什么新花样,但我不会傻到站在围栏边,因为所有的老虔婆都在围栏边上找我。我混在防御的队伍里,忙前忙后,这一方面是反抗自己的情人,也就是和自己作对,另一方面也是在躲风头。每当有老虔婆从身边走过,我就把头低下去,因为我很怕被人认出来。但这是现代派的劣根性,有个人老是低着头显得很扎眼,招来了一个老虔婆站在我身边。我把头低下去,她就把头低得更低,几乎躺在了地下。最后,她对我说道:孩子,低着头就能躲过去吗?这时我勇敢地抬起头来,含笑说道:要是抬着头,你早就认出来了。 

那个塔里的姑娘被认出之后,就在一群老虔婆的簇拥之下来到了总监的面前。她勇敢地提出一个建议说:薛嵩大举来犯,意在得到她。虽然她最憎恶薛嵩,但准备挺身而出,把自己交给薛嵩,任凭他凌辱,牺牲自己保全全塔,这是最值得的。一面说着,她一面憋不住笑,看得出说的是反话。因为自己的情人来大举进攻本塔,对她来说是个节日,所以她很是高兴。总监婆婆表扬了她的自我牺牲精神,但又说,我们决不和敌人作交易,宁可牺牲全塔来保全你一人。现在的当务之急是把你藏起来,不让薛嵩找到。这话本该让人感动,但那姑娘却发起抖来,因为总监婆婆说的也是反话。她赶紧提出个反建议,说应该大开塔门,冲出去和薛嵩一拼。很显然,这个建议薛嵩一定大为欢迎;他不可能没有准备──再说,她也可以趁机跑掉。总监婆婆又指出,我们不能冲到外面和男人打架,有失淑女的风范。然后,不管乐意不乐意,她被拥到了塔的底层。这里有一块巨大的青石板,揭开之后,露出了一个地穴,一道下去的石阶和一条通往黄泉的不归路。假如有姑娘犯了不能饶恕的错误,总监婆婆就送她下去,然后自己一个人上来,此后,这姑娘就不再有人提起。总监指着洞边的一个竹筐说道:把衣服脱掉吧,下面脏啊;好像这姑娘还会回来,再次穿上这件衣服,这就显得很虚伪。 

我们知道,总监是舍不得这件开司米的长袍,它值不少钱,不该和这姑娘一样在地下室里烂掉。而她现在很需要这件长袍,因为她冷得发抖:但她没有提出反驳,只是眼圈有点红,嘴唇咬得有点白,但是益增妩媚。她憋了一会气,终于粗声大气他说道:这也没什么;就把衣服脱掉,赤身裸体地站着。然后,总监笑眯眯地看着她说:不是不信任你,但要把你的手绑起来。此时那姑娘的嘴唇动了动,现出要破口大骂的样子。但她还是猛地转过身去,把双手背着伸了出来,说道:讨厌!捆吧!总监婆婆接过别人递过来的皮绳,亲自来捆她的双手,那姑娘恶狠狠他说道:捆紧些啊!挣脱了我会把你掐死。总监婆婆说:这倒说的是。我要多捆几道。于是就把她捆得很结实。然后总监取出一条精致的铁链子,扣在姑娘的脖子上,很熟练地收了几下,就勒得她不能呼吸,很驯眼地倚在自己肩上。顺便说一句、总监婆婆的手指粗大,手掌肥厚,小臂上肌肉坚实,一看就知道她很有力气。她用右手控制住女孩,左手拿起了灯笼,有人提出要跟她去,她说:不用,下面的路知道的人越少越好;就把女孩拖下了右头楼梯──下楼时手上松了一下,让她可以低头看路,一到了底下就勒紧了链子,让那姑娘只能踏着脚尖走路,看着黑洞洞的石头天花板。就这样呼吸了不少霉臭味,转了不少弯,终于走到一面石墙面前。在昏黄的灯光下,可以看到墙上不平之处满是尘土,墙角挂满了蛛网。那女孩想:这个地方怎么会有飞虫?蜘蛛到此来结网,难免要落空。她为蜘蛛的命运操起心来,忘掉了铁索勒住脖子的痛苦…… 

总监婆婆把灯笼插在墙上的洞里,用墙上铁环里的锁链把女孩拦腰锁住,然后松掉了她脖子上的铁链。此后那姑娘就迫不及待地呼吸着地下室里的霉臭气。总监婆婆说道:好啦,孩子,你在这里安全了。没人能到这里来玷污你的清白……那女孩忍着喉头的疼痛,扁着嗓子说:快滚,免得我啐你!总监说,你说话太粗,没有教养。看来早就该来这里反省一下──反省这个词我很熟,人们常对我说,但我对它很是反感──女孩说:反省个狗屁。总监婆婆不想再听这种语言,就拿起灯笼准备离去。此时女孩说了一句:薛嵩一定会来救我的。虽然薛嵩本领很大,却不一定能找到地下来,更不一定能在迷官似的地下室里找到她。她把不一定说成了一定,是在给自己打气。但是总监婆婆却转了回来,插好了灯笼说:你提醒得好。万一薛嵩进到这里来,你开口一叫,他就找到你了。所以,要把你的嘴箍起来。然后,她老人家从长袍的口袋里掏出一根黄连木的衔口来。 

此后,那女孩就把头拼命地扭到一边,紧闭着牙关;直到总监婆婆狠命地揪住了她的头发,使劲扭她的鼻子,她才说道:我真多嘴!算我自己活该吧……于是,她转过头来,使劲张开了嘴巴。总监婆婆以为她要咬她,往后退了退。但她又说:箍上吧。然后像请大夫看喉咙一样张大了嘴,仔细地咬住了黄连木;然后低下了头,让婆婆把衔口的皮绳拴在脑后。再以后,她扬起了头,像个吹口琴的人一样环顾四周。这回总监婆婆真的要走了,但她又觉得必须交待几句,就说:其实,你是个很好看的姑娘。我不想这样待你。那女孩在鼻子里哼出一句话,好像是“操你妈”。总监婆婆又说:等薛嵩走了之后,也许我会来放你。因为这是弥天大谎,所以她自己也有点不好意思,女孩又哼了一句,好像是“操你姥姥”。然后,总监就离去了,把这女孩留在坟墓一样的黑暗里。 

2 

我孤身在黑暗里,品尝着黄连木的苦味,呼吸着地下的霉臭气。生活中重要的是光亮,但这里没有光亮。生活中重要的是风,但这里没有风;生活中重要的是声音,但这里没有声音。地下的寒意从身体的表面侵入到腋下、两腿之间。这种处境和死亡不同的是,我还可以想事情。思维这种乐趣,与生俱来,随死亡而去。当人活着的时候,这种乐趣是不可剥夺的……那位白衣女人看到此处说:你瞎扯什么呀!我从来不这样想问题。这评论使我如受电击:我觉得在写自己,但听她的意思,此处是在写她。实际上,她说得更对。我恍恍惚惚地说:这样一来,你就不是学院派了──这句话招致我额头上的一次敲击和一顿斥骂:混帐!我要是学院派,能嫁给你吗?看来,她的确是嫁给我了。虽然我不愿相信,但对此不应再有疑问。 

我总觉得,说一个人是学院派是一种赞誉。对于男人来说,这是称赞他聪明,对于一个女人来说,这是称赞她漂亮。只有极少数的人不需要这种赞誉,比方说,我和薛嵩。那个地下室里的女孩在黑暗中站着,渐渐感到腿上很累,又不能躺下来休息……地下室里没有一点声音,寂静使耳膜发起疼来。最后她觉得,反正没人看见,可以哭一会。于是,对面响起了抽泣声。这使她知道对面不很远的地方有堵墙壁。忽然她仿佛听到一声嗤笑,赶紧停止了哭泣,凝神去听,什么都没听到。但是她又觉得在霉臭味里杂有薛嵩特有的体臭──这个家伙经常弄到一身大汗,嗅起来有点馊。于是她使劲去嗅,结果马上就被霉味把鼻子呛住了。然后她就叫起来,但那块黄连木压住了她的舌头,只发出了一阵呜呜的声音。然后她又凝神去听,还是什么都听不到……猛然间,没有任何徵兆,她的乳房落进了男人温热的手掌。薛嵩的声音在她耳畔轰鸣着:怎么,不哭了?此后,她就什么都不想,什么都不听,冒了被铁链勒断腰的危险,踢开了薛嵩身上的斗篷,两只脚顺着薛嵩的腿爬了上去,紧紧地盘住了他的腰,和他做爱。 

与此同时,薛嵩像驴鸣一样解释着今天发生的事情:外面扮作薛嵩的那个人是他的表弟。他自己早就钻了进来,一直躲在这里,看到了总监老太太怎么把她揪了进来,锁在墙上,又看到了她们俩怎么吵嘴。他还说,今天的计划非常之好,百分之百地成功了。但那女孩早就不想听他解释,她还觉薛嵩的声音像是驴鸣──但这不是薛嵩之过,他并没有把嗓音放大,是这里过于安静之故──假如不是嘴被勒住,她早就喊他闭嘴了。最后,当薛嵩把她嘴上的嚼子解开时,她才说了一句早就想说的肺腑之言:你可真坏呀你! 

在薛嵩的故事里出现了一个表弟,使我深为不快。如你所知,我也有一个表弟,而且我不喜欢和薛嵩搞得太相像。午夜时分,这位表弟在塔外面辛苦地工作着。他一会爬上云梯,一会儿爬下来跑到幕后,转动一个满是假人的圆盘,借助一个铜皮喇叭发出众多人的呐喊,敲锣打鼓,并且给到处点着的灯笼添油。直到他听到塔上的姑娘们欢声雷动,才松了一口气,从帷幕后面跑了出来。如你所猜到的那样,那些姑娘看到两个人影从塔下的乱石缝里钻了出来。其中一个披着男人的黑斗篷,长发披肩,身材娇小;另一个则身材高大,一丝不挂,长着紧凑的臀部和两条长腿,小腿的下半部还有一些毛。后一个把手搭在前一个肩上,两人从容不迫地走开。只有看到过薛嵩屁股上的肌肉是怎样的一起一伏,你才会知道什么叫作从容不迫。只有看到过薛嵩站定时的样子,你才知道什么叫作男人的屁股──那两块坚实的肌肉此时紧紧地收在他的腰后,托住他的上半身──我只是转述那些姑娘的看法,其实我也不能算见过男人的屁股。总而言之,薛嵩和他的臀部彻底动摇了学院派对爱情的说法:这种说法强调爱情必须以琴会友,在红叶上写情书,爱人之间用诗来对话,从来没有提到过屁股。当然,姑娘们不会把这个不雅的部位挂在嘴上,她们说的是:我就想有这么个人,把我从死亡中救出来,脱下斗篷裹住我的裸体,然后赤身裸体地走在我身边。因为她们都这样想,就给塔里带来了无数的麻烦;不久之后,这座塔就倒掉了。 

从那位表弟的眼里看来,那天晚上的景象就大不相同。薛嵩和那女孩朝黑布帷幕走来,在黑毛毡的笼罩之下,那女孩的脸和从斗篷缝里伸出的手显得特别白。她脸上带着快活的微笑,但笑容里又有几分苦涩。而薛嵩前面的样子,塔里的姑娘们看了更会满意──他上身肌肉匀称,腹部凹陷下去,因为寒冷,阴囊紧缩着,已经松弛下来的阴茎依然很长很大,像大象鼻子一样低垂着。他自己也觉得这样子不雅──虽然赤身裸体地维护爱人可以得到塔上姑娘们的高度评价,但也会着凉的──就对表弟说,脱件衣服给我!那位表弟动手脱外衣,同时盯着表嫂猛看,她只好假作无意地侧过脸去。总而言之,经过短暂的准备,这三个人从幕后走了出来,和塔里的人告别。女孩大声叫着总监婆婆,这位婆婆本不想露面,但又想,不露面更不光彩,就走到围廊上,假作慈爱他说:本想等薛嵩走后再到地下室去放她,不想她已经脱困,真是可喜可贺;她还想说,今后这位姑娘就交付给薛嵩,希望他好好待她──把虚伪扣除在外,这会是很好的演说词,只可惜那女孩不想听下去,猛地转过身去,把斗篷一撩,露出了整个屁股,总监的演说词就被老虔婆们的一片嘘声淹没了。本来大家是要嘘女孩的屁股,结果把总监嘘倒了,她也只好闭嘴,同时恶狠狠地想道:这个小婊子可真狡猾──这种坏女人走掉了也不可惜。然后就轮到了薛嵩,他把双手放到唇上,给塔上送去一个大大的吻,博得了姑娘们的喝彩声。至于那个表弟,他什么都没有说,因为这本不是他的故事。此后,这三个人就转身行去,把这座彻底败坏了的塔留在身后,走进了长安城……这个故事得到了白衣女人的好评,但我对它很不满意。因为故事里的薛嵩敢作敢为,像一个斗士,这不是我的风格。那个白衣女人拍拍我的头说:没关系、用不着你敢作敢为。有我就够了。 

3 

秋天的长安城满街都是落叶,落叶在街道两侧堆积起来,又延伸到街道的中间。在街道中间,露出稀疏的铺街石板。人在街上走着,踩碎了落叶,发出金属碎裂的声响,很不好听。但是深秋时节长安城里人不多。清晨时分,在街上走着的就只有三个人。风吹过时,这些落叶发出叮叮咚咚的声音,这就很好听了。秋天长安城里的风零零落落,总是在街角徘徊。秋天长安城里有雾,而且总是抢在太阳之前升起来,像一堵城墙;所以早上的阳光总是灰蒙蒙的。我们从翻滚的落叶中走过无人的街道,爬上楼梯,走过窄窄的天桥,低下头走进房门,进了一间背阴的房子。这里灰蒙蒙的一片,光线不好,好在顶上有天窗,这房子又窄又高,就是为了超过前面的屋脊,得到一扇天窗──就如一个矮人看戏时要踞脚尖。前面的地板上铺着发暗的草席,靠墙的地方放着几个软垫子,垫子里漏出的白羽毛在我们带进来的风里滚动着,薛嵩说:房子比较差啊。他的嗓子像黄金一样,虽然高亢,但却雍容华贵。这也不足为奇,他毕竟是做过节度使的人哪。那女孩说:没关系,我喜欢。她的声音很纯净,也很清脆。薛嵩抬头看看天窗──天窗不够亮,就说,我该帮你擦擦窗户。女孩说:等等我来擦吧,这是我的家啊。每次说到“我”,她都加重了语气。但她脸上稍有点浮肿,禁不住要打呵欠。按照学院派的规矩,打呵欠该用手遮嘴,但她手在斗篷下很不方便。于是她垂下睫毛、侧着脸,俏俏打着小呵欠,样子非常可爱──但最终她明白这种做作是不必要的,她自由了,就伸了一个大懒腰,使整个斗篷变成了一件蝙蝠衫,同时快乐地大叫一声:现在,我该睡觉了! 

既然人家要睡觉,我们也该走了。薛嵩压低了声音说:要不要我给你买衣服?那姑娘微微愣了一下,看来她想自己去买,但又想到自己没有钱,就说:知道买什么样子的吗?薛嵩当然知道。于是,女孩说:好吧,你去买。我欠你。从这些对话里我明白这个女孩从此自由了,既不倚赖学院,也不倚赖薛嵩──虽然是他把她从学院里救了出来。我非常喜欢这一点。 

后来,那姑娘像主人一样,把我们送到了街上。此时街上依旧无人,只有风在这里打旋。在这里,她把手从斗篷下面伸出来,搂住薛嵩的脖子,纵情地吻他,两件黑斗篷融成了一件。薛嵩大体保持了镇定,那姑娘却在急不可耐地颤抖着──可以看出,她非常的爱他。除此之外,她刚从死亡的威胁中逃出来。这种威胁在我们看来只是计划的一部份,但对她就不一样,她可不知道这个计划啊…… 

后来,那姑娘放开了薛嵩。他们带着尴尬的神情朝我转过身来。我穿着白色的内衣,在冷风里发着抖,流着清水鼻涕,假装轻松地说:没关系,没关系,我可以假装没看见。如你所知,我是那个来帮忙的表弟,在高塔下面狂喊了半夜,嗓子都喊哑了,又敲了半夜的鼓,膀子疼痛不已。最糟的是,在高塔外面陈列着的那些器材──云梯、帷幕、灯笼、火把都是我的,值不少钱。此时回去拿就会被人逮住,只好牺牲了。这件事我决定永不提起,救了一个人,还让她出救命的钱,实在太庸俗。这笔钱她也不便还我,还别人救命的钱也太庸俗。当然,见死不救就更庸俗。不知为什么,我竟是如此的倒霉…… 

后来,那姑娘朝我走过来,拉住我的手说:谢谢你啊,表弟,在我面颊上吻了一下,就把我给打发了。我独自走开。长安城里的凤越来越烈,所有的落叶就如在筛子上一样,剧烈地滚动着。那姑娘的体味就如没有香味的鲜花,停留在我面颊上──这是一种清新之气,一种潜在的芳香,因为不浓烈,反而更能持久。我独自下定了决心:在任何故事里,我都再不作表弟了。 

4 

现在来看这个故事,仿佛它只能发生在薛嵩从湘西回来之后。既然如此,我就必须把湘西发生的事全部交待清楚。我开始考虑红线怎样了,小妓女怎样了,田承嗣又怎样了,觉得不堪重负。尤其是田承嗣,他像只巨大的癞蛤蟆压在我身上,叫我透不过气来。癞蛤蟆长了一身软塌塌、疙疙瘩瘩的皮,又有一股腥味,被它压着实在不好受。史书上说,董卓很肥,又不讨人喜欢,但他有很多妾。董卓的小妾一定熟悉这种被压的滋味。除此之外,我一会儿是薛嵩,一会儿是薛嵩的情人,一会儿又成了薛嵩的表弟;这好像也是一种毛病。但我忽然猛省到,我在写小说。小说就不受这种限制。我可以在任何时间,任何地点。我可以是任何人。我又可以拒绝任一时间,任一地点,拒绝任何一人。假如不是这样,叉何必要有小说呢。 

后来,那个从塔里逃出来的姑娘就住在长安城里。我很喜欢这个姑娘,正如我喜欢此时的长安城:满是落叶的街道,鳞次栉比的两层楼房,还有紧闭的门窗。长安城到处是矮胖的法国梧桐,提供最初的宽大落叶;到处是年轻的银杏树,提供后来的杏黄色落叶,这种落叶像蝴蝶,总是在天上飞舞,不落到地下来;到处是钻天杨树,提供清脆的落叶。最后是少见的枫树,叶子像不能遗忘的鲜血,凝结在枝头。在整个自由奔放的秋季,长安是一座空城。你可以像风一样游遍长安,毫无阻碍。直到最后,才会在一条小街里,在遥远的过街天桥上看到这个姑娘,独自站着,白衣如雪。作为薛嵩,你看到的就是这样的景象,相当令人满意。但我更想作那个姑娘,在天桥上凭栏而立;看到在如血残阳之下,在狂涛般的落叶之中,薛嵩舞动着黑色的斗篷大踏步地走来。这家伙岂止像个盗马贼,他简直像个土匪……我作薛嵩作得有点腻,但远远地看看他,还觉得满有兴趣。 

在长安城里看这篇小说,就会发现,它的起点在千年之后的万寿寺,那里有个穿灰色衣服的男人,活得像个窝囊废;他还敢说“作薛嵩作得有点腻”,把他想出了这一切扣除在外,他简直就是狂妄得不知东西南北。 

在薛嵩到来之前,我走进自己的房间。除了不能改变的,这间房子里的一切都改变了。不能改变的是这座房子的几何形状,窄长、通向天顶,但我喜欢这种形状。以前的草席、软垫子通通不见了,四壁和地板都变成了打磨得平滑的橡木板。当然,推开墙上的某块木板,后面会有一个柜子,里面放着衣物,被褥等等,但在外面是看不到的,头顶的天窗也没有了,代之以一溜亮瓦,像一道狭缝从东到西贯通了整个房间。于是,从头顶下来的光线就把这间房子劈成了两半。这间房子像极北地方的夏季一样,有极长的白天和极短的夜。从南到北的云在转瞬之间就通过了房顶,而从东到西的云则在头上徘徊不去。这个季节的天像北冰洋一样的蓝。这正是画家的季节。 

从塔里逃出来之后,我是一个独立的人。也许,如你所猜测的那样,我是一个画家,也许是别样独立谋生的人,像这样的人不分男女,通通被称作“先生”。我喜欢作一个“先生”,只在一点上例外。这一点就是爱情。薛嵩走进这间房子,转身去关门。此时我体内闹起了地震,想要跳到他身上去,用腿盘住他的腿爬上去……女人就像这间房子,很多地方可以改变,但有一点不能改变。不能改变的地方就是最本质的地方。 

后来,薛嵩朝我走来,我则朝后退去,保持着旧有的距离,好像跳着一种奇异的双人舞。就这样,我们在房子中间站住,中间隐了两臂的距离;黑白两色的衣衫从身上飘落下来,起初还保持着人体的形状,后来终于恢复了本色,委顿于地。薛嵩仿佛永远不会老,肤色稍深,像一个铜做的人,骨架很大,但是削瘦,肌肉发达,身上的毛发不多,只有小腹例外。这家伙有点斗鸡眼,笑起来显得很坏,但他是个好人。我认识他的时候他是这个样子,现在还是这个样子。他低下头去,动了动脚趾,然后带着一脸好笑抬起头来。他是不会随便笑的──果然,他勃起了。那东西可真是难看哪……薛嵩留着八字胡,整个胡子连成了一片,呈一字形。而在他身体的下部,阴毛就像浓烈的胡须,那个东西就如翘起的大鼻子,这张脸真是滑稽得要死…… 

而我自己浑圆而娇小,并紧腿笔直地站着。腿之间有一条笔直的线,在白色的朦胧中几不可见。假如它不是这样的直,本来该是不可见的……我像在塔里时那样端庄,不顾他的好笑,毫无表情。但微笑是不可抑制的,水面凝止的风景终究是会乱的──这道缝隙也因此变显著了──如你所知,我在万寿寺里写这个故事,那位白衣女人在我身边看着。她在我脑袋上敲了一下,叫道:变态哪!我也就写不下去了…… 

不管那位白衣女人说什么,我总愿意变得浑圆、娇小,躺在坚硬的橡木地板上,看亮瓦顶上的天空,躺在露天地上,天绝不会如此的遥远,好像就要消失;云也不会如此近,好像要从屋顶飘进来。起初,我躺得非常平板,好像一块雕琢过的石材平放在地板上,表情平板,灰白的嘴唇紧闭,浑身冰冷,好像已经沉睡千年。然后,双唇有了血色,逐渐变得鲜红,鼻间有了气息;肩膀微微抬了起来,乳房凸现,腹部凹陷,臀部翘了起来。再以后,我抬起一只手,抱住薛嵩的肩头。再以后,这间屋子里无尘无嗅的空气里,有了薛嵩的气味。坦白地说,这味道不能恭维,但在此时此地是好的。我的另一手按在他的腰际。就这样,我离开地板,浮向空中,迎接爱情。爱情是一根圆滚滚、热辣辣的棍子,浮在空中,平时丑得厉害,只有在此时此地才是好的。写完了这一句,我愤怒地跳了起来,对白衣女人吼道:你有什么意见可以直说,不要敲脑袋。这又不是一面鼓,可以老敲!这样一吼,她倒有点不好意思。噎了一下,才说:不是我要敲你──像这种事总不好拿来开玩笑。我说:我很严肃,怎么是开玩笑!她马上答道:得了吧,我又不是今天才认识你。你满肚子都是坏水,整个是个坏东西……说完她就走了。剩下我一个人发愣,想起了维克多·雨果的《笑面人》。那个人谁看他都是一副嘻皮笑脸的样子,只有他还挺拿自己当真──但我又想不起维克多·雨果是谁。我也不知这是怎么回事,但我知道假如去问那个白衣女人,肯定是找挨揍。 

第三节 

现在我终于明白,在长安城里我不可能是别人,只能是薛嵩。薛嵩也不可能是别人,只能是我。我的故事从爱情开始,止于变态,所以这个故事该结束了。此时长安城里金秋已过,开始刮起黑色的狂风。风把地下半腐烂的叶子刮了起来,像膏药一样到处乱贴,就如现在北京刮风时满街乱飞的塑料袋。一股垃圾场的气味弥漫开来。我(或者是薛嵩)终于下定了决心,要离开长安,到南方去了。 

在《暗店街》里,主人公花了毕生的精力去寻找记忆,直到小说结束时还没有找到。而我只用了一个星期,就把很多事情想了起来,这件事使我惭愧,莫迪阿诺没有写到的那种记忆必定是十分激动人心,所以拼老命也想不起来。而我的记忆则令人倒胃,所以不用回想,它就自己往脑子里钻,比方说,我已经想起了自己是怎样求学和毕业的。在前一个题目上,我想起了自己是怎样心不在焉地“坐在阶梯教室里,听老师讲课。老师说,史学无它,就是要记史料;最重要的史料要记在脑子里。脑子里记不下的要写成卡片,放手边备查。他自己就是这样的──同学们如有任何有关古人的问题,可以自由地发问。我一面听讲,一面在心里想着三个大逆不道的字:“计算机”,假如史学的功夫就是记忆,没有人可以和这种不登大雅之堂的机器相比。作为一个史学家,我的脑壳应该是个monitor,手是一台打印机,在我的胸腔里,跳动着一个微处理器,就如那广告上说的Pentium,给电脑一颗奔腾的心。说我是台586,是不是给自己脸上贴金?我的肠胃是台硬磁盘机,肚脐眼是软磁盘机。我还有一肚子的下水,可以和电脑部件一一对应。对应完了,还多了两条腿。假如电脑也长腿,我就更修不过来了。更加遗憾的是,我这台计算机还要吃饭和屙屎。正巧此时,老师请我提问(如前所述,我可以问任何有关古人的问题),我就把最后想到的字眼说了出去:“请问古人是如何屙屎的”。然后,同学笑得要死,老师气得要死。但这是个严肃的问题。没有人知道古人是怎样屙屎的:到底是站着屙,坐着屙,还是在舞蹈中完成这件重要工作……假如是最后一种,就会像万寿寺里的燕子一样,屙得到处都是。 

说到毕业,那是一件更恐怖的事。像我这样冒犯教授,能够毕业也是奇迹。除此之外,系里也希望我留级,以便剥削我的劳动力。在此情况下,白衣女人经常降临我狗窝似的宿舍,辅导我的学业,并带来了大量的史料,让我记住。总而言之,我是凭过硬本领毕了业,但记忆里也塞进了不少屎一样的东西。无怪我一发现自己失掉了记忆,就会如此高兴……根据这项记忆,白衣女人是我的同门。无怪我要说:薛嵩和小妓女作爱,是同门之间切磋技艺──原来这是我们的事。很不幸的是,白衣女人比我早毕业。这样就不是学兄、学妹切磋技艺,而是学姐和学弟切磋技艺。这个说法对我很是不利,难怪我不想记住自己的师门。 

2 

我到医院去复查,告诉治我的大夫,我刚出院时有一段想不起事,现在已经好多了。他露出牙齿来,一笑,然后说:我说嘛,你没有事。等到我要走时,他忽然从抽屉里取出一本书来,说道:差点忘了!这书是你的吧。它就是我放在男厕所窗台上的《暗店街》……我羞怯地说道:我放在那里,就是给病友和大夫们看的。他把手大大地一挥,果断他说:我们不看这种书──我们不想这种事,我只好讪仙地把书拿了起来,放进了自己的口袋。这本书大体还是老样子,只是多了一些黄色的水渍,而且膨胀了起来。走到门诊大厅里,我又偷偷把书放在长条椅子上。然后,我走出了医院,心里想着:这地方我再也不想来了。 

我和莫迪阿诺的见解很不一样。他把记忆当作正面的东西,让主人公苦苦追寻它;我把记忆当成可厌的东西,像服苦药一样接受着,我的记忆尚未完全恢复,但我已经觉得很够了,恨不得忘掉一些,但如你所知,我和他在一点上是相同的,那就是认为,丧失记忆是个重大的题目,而记忆本身,则是个带有根本性的领域,是摆脱不了的。因为这个原故,我希望大家都读读《暗店街》,至于我的书,读不读由你。我就这样离开医院,回到万寿寺里。 

我表弟在北京呆够了,要回泰国。我纳闷他怎么呆到今天才觉得够:成天呆在饭店里不知有什么意思。傍晚时分,我们到机场去送他,他忽然变得很激动,拉着我的手说:表哥,不知什么时候再见。我敷衍他说道:是呀,是呀;心里却盼着他早点登机。只要他通过了安全门,我们就可以回家去。此后就会再也见不到这个不知从哪里来的、我怎么也想不起来的表弟。他语不成声他说道:还记得吗,姥姥给我们做的蒸糕……就如有一个晴空霹雳在头顶炸响,我想起了小时的大灾荒年月。 

那时我在空地上寻找苦苦菜,然后,我们俩共同的外祖母,一个慈祥和蔼的老妇人,用这些野菜和着面粉蒸糕给我们吃。除了找野菜,我们俩还偷东西。半夜里出去,偷别人家自留地里的黄瓜、茄子、胡萝卜,假如有可能,还偷鸡、偷兔子。这些东西拿回来以后,姥姥看了就摇头。但她还是动手把这些东西做熟。然后,我和表弟就把这些没油没盐、煮得软塌塌的蔬菜和肉类吃掉。姥姥一点都不肯吃──我和我表弟是两个孤儿,但有一个满头白发,面颊松弛的姥姥。我一点都不后悔忘掉了自己做过贼的事,但我不该忘掉姥姥。我眼里充满了泪……与此同时,表弟还在喋喋不休他说:现在我可过上人的生活了,要钱有钱,要老婆有老婆──姥姥在天之灵会高兴的。他一句也没提到我。我看着这个满脸流油的家伙,心里暗暗想道:我把他忘掉,这就对了…… 

晚上我们回家去,坐在出租车里,我闷闷不乐。她问我怎么了,我说想起了姥姥,她也黯然伤神。这倒使我吃了一惊:莫不成我姥姥也是她姥姥?假设如此,她就是我的表妹。按照现行法律,表兄妹是近亲,禁止结婚。这件事使我怦然心动。回到家里,她拍我的脑袋说:可怜的孤儿……以后我得对你好一点。这当然是好消息。我问她准备怎样对我好,她说,以后再不敲我脑袋了。这个好消息大小一点了……后来,在床上,我亲热地提出了这个问题:你到底是不是我表妹?回答是:错!我是你姑妈啊。我赶紧丢下她坐了起来,浑身起满了鸡皮疙瘩──我想每个男人在无意中拥抱了自己的姑妈,都会有这种反应。然后,就着塑料百叶窗里漏进的灯光,我看到她满脸笑容,鸡皮疙瘩才消散了。看来她不是我的姑妈──岁数也不像。她说:好个坏蛋啊,提起了姥姥,正经了不到五分钟,又开始胡扯了──真是狗改不了吃屎啊。我正想用这句话来说她──当然,我不会把她比作狗。看来她不会是我表妹:这不像是对表哥的态度。今天的好消息是:我未曾犯下奸污姑母的罪行。坏消息则是表妹也没有了。 

3 

早上我来上班时,万寿寺的下水道又堵了。黄水在低洼地带漫着,很快就要漫到院子里来。我终于抑制不住狂怒,走进领导的办公室,恳请他撤销我助理研究员的职务,把我贬作一个管子工;这样我就可以名正言顺地去捅大粪。我还说,我宁愿自己死掉,也不想见到领导和资料室的老太太们坐在屎里──这种屎里虽然有大量的水来稀释,但仍然是屎。我完全是认真的,但领导的脸却因此而变紫了。他跑了出去,很快又和白衣女人一起走回来;大声大气地吼道:身体既然没有恢复,就不要来上班。那白衣女人朝我快步走来──我不由自主地缩紧了脖子,以为她要打我一耳光──但她没有,只是小声说道:走,回家去…… 

然后,我们走在街上。我就像一只狗,跟着大发脾气的主人,做好了一切准备要挨上一脚,但主人就是不踢。过马路时,她紧紧揪住我的袖口,当我看她时,她又放开,说道:我怕你再被汽车撞了。而我,则在傻愣愣地想着:我是谁,为什么要这样愤怒?她是谁?为什么要这样关心我?我值得她这样关心吗?最后,她把我送到了楼梯口,小声说道:人家愿意坐在屎里,这干你什么事啊;就离去了。剩下我一个人去爬三层的楼梯。爬上第一层时,我对今天发生的一切都不能理解,觉得自己完全是对的──就是不能让人坐在屎里。爬到了第二层,我觉得眼前的世界完全无法理解──那白衣女人说,人家乐意坐在屎里,不干我的事──但别人为什么要乐意坐在屎里?但爬到第三层,手里拿着大串的钥匙,逐一往门上试时,我终于想到,是我自己出了毛病。没有记忆的生活虽然美好,但我需要记忆。

\section{第八章}

第一节 

千年之前的长安城是一座美丽的城市。在它的城外,婉蜒着低矮精致的城墙。在它的城内,纵横着低矮精致的城墙;整个城市是一座城墙分割成的迷宫。这些城墙是用磨过的灰砖砌成,用石膏勾缝,与其说是城墙,不如说是装饰品。在城墙的外面,爬着常青的藤萝,在隆冬季节也不凋零。 

冬天,长安城里经常下雪。这是真正的鹅毛大雪,雪片大如松鼠尾巴,散发着茉莉花的香气。雪下得越久,花香也就越浓。那些松散、潮湿的雪片从天上软软地坠落,落到城墙上,落到精致的楼阁上,落到随处可见的亭榭上,也落到纵横的河渠里,成为多孔的浮冰。不管雪落了多久,地上总是只有薄薄的一层。有人走过时留下积满水的脚印──好像一些小巧的池塘。积雪好像漂浮在水上。满天满地弥散着白雾……整座长安城里,除城墙之外,全是小巧精致的建筑和交织的水路。有人说,长安城存在的理由,就是等待冬天的雪…… 

长安城是一座真正的园林:它用碎石铺成的小径,架在水道上的石拱桥,以及桥下清澈的流水──这些水因为清澈,所以是黑色的。水好像正不停地从地下冒出来。水下的鹅卵石因此也变成黄色的了。每一座小桥上都有一座水榭,水榭上装有黄杨木的窗棂。除此之外,还有渠边的果树,在枝头上不分节令地长着黄色的批耙,和着绿叶低垂下来。划一叶独木舟可以游遍全城,但你必须熟悉长安复杂的水道;还要有在湍急的水流中操舟的技巧,才能穿过桥洞下翻滚的涡流。一年四季,城里的大河上都有弄潮儿。尤其是黑白两色的冬季,更是弄潮的最佳季节;此时河上佳丽如云……那些长发披肩的美洲人在画肪上,脱下白色的亵袍,轻巧地跃入水中。此后,黑色的水面下映出她们白色的身体。然后她们就在水下无声无息地滑动着,就如梦里天空中的云……这座城市是属于我的,散发着冷冽的香气。在这座城中,一切人名、地名都不重要。重要的是实质。 

在长安城里,所有的街道都铺着镜面似的石板,石质是黑色的,但带有一些金色的条纹。降过雪以后,四方皆白,只有街道保持了黑色;并和路边的龙爪槐相映成趣。那些槐树俯下身来,在雪片的掩盖下伸展开它们的叶子,叶心还是碧绿色,叶缘却变成红色的了。受到雪中花香的激励,龙爪槐也在树冠下挂出了零零散散的花序,贡献出一些甜里透苦的香气,能走在这样的街道上真是幸运。她就这样走进画面,走上镜面似的街道,在四面八方留下白色的影子。 

我在一切时间,一切地点追随白衣女人。她走在长安城黑色的街道上,留着短短的头发,发际修剪得十分整齐,只在正后方留了一络长发,像个小辫子的样子。肩上有一块白色的、四四方方的披肩,这东西的式样就像南美洲人套在脖子上的毯子。准确地说,它不是白色,而是米色,质地坚挺,四角分别垂在双肩上、身后和身前。在披肩的下面,是米色的衣裙。在黑色的街道上,米色比白色更赏心悦目。在凛冽的花香中,我从身后打量着她,那身米色的衣服好像是丝制的,又好像是细羊毛──她赤足穿着一双木履,有无数细皮带把木鞋底拴在脚腕上。她向前走去,鞋底的铁掌在石板上留下了一串火花……我写到这些,仿佛在和没有记忆的生活告别。 

2 

我来上班,站在万寿寺门口,久久地看着镌在砖上的寺名。这个名称使我震惊。如你所知,我失掉了记忆,从医院里出来以后,所见到的第一个名称,就是“万寿寺”;这好像是千秋不变的命运。我看着它,心情惨然,白衣女人从我身边走过,说道:犯什么傻,快进去吧。于是,我就进去了。 

早上,万寿寺里一片沉寂,阳光飘浮在白皮松的顶端,飘浮在大雄宝殿的琉璃瓦上。阳光本身的黄色和松树的花粉、琉璃瓦的金色混为一体;整座寺院好像泡在溶了铁锈的水里。就在这时,她到我房间里来坐,搬过四方的木头凳子,倚着门坐着,把裙角仔细压在身下;在阳光中,镇定如常地看着我。就是这个姿势使我起了要使她震惊的冲动……在沉思中,我咬起手来。她站了起来,对我说:别咬手,就走出去了,仪态万方……她就这样走在一切年代里。 

我迫随那位白衣女人。更准确他说,我在追随她的小腿。从后面看,小腿修长而匀称,肌肉发达。后来,我走到她面前,告诉她此事;她因此微笑道:是吗,你这样评价我──这种口气不像是在唐代,不在这个世界里;但是她呵出的白气如烟,马上就混入了漫天的雪雾,带来了真实感。我穿着一套黑粗呢的衣服,上面还带一点轻微的牲畜味。雪花飘到这衣服上就散开,变成很多细碎的水点;而且我还穿了一双黑色的皮靴。但她身上很单薄……这使我感到不好意思,想到:要找个暖和的地方。但是她微笑着说:没关系,我不冷。这些微笑浮在满是红晕的脸上,让人感觉到她真的不冷。再后来,我就和她并肩行去,她把一只手伸了过来,一只冰冷的小手。它从我右手的握持中挣脱出来,滑进宽大的衣袖,然后穿入衣襟的后面,贴在我胸前。与此同时,黑色的街道湿滑如镜。是时候了,我把她拉进怀里,用斗篷罩住。她的短发上带有一层香气,既不同于微酸的茉莉,也不同于苦味的夹竹桃,而是近乎于新米的芳香;与此同时,带来了裸体的滑腻。 

在漫天的雪雾之中,我追随着一件米色的衣裙和一股新米的香气。除了黑色的街道和漫天的白色,在视野中还有在密密麻麻雪片后面隐约可见的屋檐;我们正向那里走去,然后,爬上曲折的楼梯,推开厚厚的板门,看到了这间平整的房子,这里除了打磨得平滑的木头地板之外,再没有别的东西了。与平滑的木头相比,我更喜欢两边的板墙,因为它们是用带树皮的板材钉成的,带有乡野的情调。而在房子的正面,是纸糊的拉门,透进惨白的雪光。我想外面是带扶栏的凉台,但她把门拉开之后,我才发现没有凉台,下面原来是浩浩的黑色江水──那种黑得透明的水,和人的瞳孔相似;从高处看下去,黑色的水像一锅滚汤在翻腾着,水下黄色的卵石清晰可见。那位白衣女人迅速地脱去了衣服,露出我已经见过的身体……她一只手抓住拴在檐下的白色绳子,另一只手抓住我的领子,把修长、紧凑的身体贴在我身上──换言之,贴在黑色的毛毡上。顺便说一句,那条白色的绳子是棉线打成的,虽然粗,却柔软;隔上一段就有个结,所以,这是一条绳梯,一直垂到水里。又过了一会儿,她放开了我,在那条绳子上荡来荡去,分开飞旋的雪片,飘飘摇摇地降到江里去。此时既无声息,又无人迹;只有黑白两色的景色。我不知道这意味着什么。但是,它绝不会毫无意义。 

3 

在古代的长安城里,有一条黑色的江,陡峭的江岸上,有一些木头吊楼。我身在其中一座楼里。我所爱的白衣女人穿过飞旋的雪片到江中去游水。这个女人身体白皙、颀长,在黑色的吊楼里,就如一道天顶射下的光线;就如一只水磨石地板上的猫──这是她下到江里以前的事。我不知道她是谁,只知道她是我之所爱──等到她从江里出来时,皮肤上满是水渍。在水渍下面,身体变得像半透明的玉,或者说像是磨砂玻璃。整个房间充满了雪天的潮湿,皮肤摸起来像玻璃上细腻的水雾……在冷冽的水汽中,新米的香味愈演愈烈。 

我在江边的木屋里,这里的地板很平整,平到可以映出人影。我终于可以听到那条江的声音了,流水在河岸边搅动着。从理论上说,有很多东西比水比重大。但我想象不出有什么比流水更重。每有一个浪头冲到岸上,整座吊楼都在颤动。就在这座摇摇晃晃的房子里,我亲近她的身体。她既冷冽又温暖,既热情又平静。在黑白两色的背景之下,她逐渐变得透明,最后完全不见了。与此同时,新米的香气却越来越浓。与此同时她说,这难道不好吗?声音弥散在整个房间里。这很好,起码什么都不妨碍。我深入她的既虚无又致密的身体,那些不存在的发丝在我面前拂动;在我肩头还有两道若有若无的鼻息……等到一切都结束,她又重新出现在我的怀抱里;带着小巧鼻翼冰凉的鼻子,乳房像一对白鸽子──老实说,形像并不像。我只是说它偎依在怀里的样子,这是我和那位白衣女人的故事,但它也可以是薛嵩和他情人的故事。是谁都可以。在这座城里,名字并无意义。 

在玻璃一样的地板上,我也想要消失。失掉我的名字,失掉我的形体,只保留住在四壁间回响的声音和裸体的滑腻;然后,我就可以飘飘摇摇,乘风而行,漫游雪中的长安城。 

江边吊楼敞开的窗户外面,雪片变得密密麻麻,好像有些蘸满了白浆的刷子不停地刷着,黑色斗篷的外面越来越冷,冷气像锥子一样刺着我的面部神经。而在那件斗篷内部,在这黑白两色的空间里,则温暖如春。她不再散发着新米的香气,而是弥漫着米兰的气味。米兰是一种香气甜得发苦的花。在我看来,黑白两色的空间、冷热分明的温差,加上甜得发苦的花,就叫作“性”。我不同意她再次消失,就紧紧地抓住她的手腕……于是,她挺直了身体,把白色的双肩探到斗篷外面,舔了一下嘴唇。不管怎么说吧,第二次像水流一样自然地过去了。以后,她在我身体两侧跪了起来,转了一个身;再以后,她倚着我,我倚着墙,就这样坐着。我不明白为什么,仅仅坐着会使我感到如此大的满足。 

我不由自主地写下了这个故事,觉得它完全出于虚构。那位白衣女人看了以后说:不管怎么说罢,我不同意你把什么都写上。这句话使我大吃一惊:听她的口气,这好像是发生过的事情。难道我和她在长安城里做过爱?我怎么不记得自己有这么大的年龄……我需要记忆。难道这就是记忆? 

4 

但我又曾生活在灰色的北京城里。这里充满了名字。我有一个姥姥,一个表弟,还有我自己,都有名字。我们住在东城的一条街上,这条街道也有名字。我在这条街上一个大院子里,这座院子也有门牌号数。我很不想吐露这些名字。但是,假如一个名字都不说,这个故事就会有点残缺不全──我长大的院子叫作立新街甲一号,过去这院子门口有一对石头狮子,我和我表弟常在石头狮子之间出入──吐露了这个名字,就暴露了自己。 

因为想起了这些事,我又回到了青年时代。那时候我又高又瘦,穿着一件硬领的学生上衣,双手总是揣在裤兜里。这条蓝布裤子的膝头总是油光银亮,好像涂了一层清漆。春天里,我脸上痛痒难当,皮屑飞扬,这是发了桃花藓。冬天,我的鼻子又总是在流水:我对冷风过敏。我好像还有鬼剃头的毛病──很多委托行都卖大穿衣镜,站在它的面前,很容易暴露毛发脱落的问题。我总是和我表弟在京城各家委托行里转来转去;从前门进去,浏览货架寻找猎物,找到之后,就去委托行的后门找人。走到后门的门口,我表弟站住了,带着嫌恶的表情站住,递过一团马粪也似的手绢,说道:表哥,把鼻涕擦擦──讲点体面,别给我丢人!我总觉得和他的手绢相比,我的鼻涕是世上绝顶清洁之物。实际上,那些液体也不能叫作鼻涕。它不过是些清水而已。 

在我自己的故事里,我修理过一台“禄来福来”相机。“禄来福来”又是一个名字。这是一种德国造的双镜头反光相机,非常之贵。到现在我也买不起这样的相机。然而我确实记得这架相机,它摆在西四一家委托行的货架上。这家委托行有黑暗的店堂,货架上摆着各种电器、仪器,上面涂着黑色的烤漆、皱纹漆遮掩着金属的光泽──总的来说,那是在黑暗的年代。就如纳博科夫所说,这是一个纯粹黑白两色的故事。 

我和我表弟常去看那台禄来福来相机,要求售货员把它“拿下来看看”。人家说:别看了,反正你们也买不起;口气里带着轻蔑。这仿佛是我们未曾拥有这架相机的证明。然而下一幕却是:我和我表弟出现在委托行附近的小胡同里。这个胡同叫作砖塔胡同,胡同口有一个庵,庵里有座醒目的砖塔,总有两三层楼高罢,我们俩在胡同里和个老头子说话,时值冬日,天色昏暗,正是晚饭前的时节。这条胡同黑暗而透明,从头透到尾;两边是灰色的房屋。此人就是委托行的售货员,头很大,屁股也很大,满脸白胡子茬,和我们的领导有点相像之处。我做了很大的努力,才使自己不要想起此人的名字──我成功了。但我也知道,这人的名字,起码他的姓我是记得的──此人姓赵。我们叫他赵师傅。当时叫“师傅”是很隆重的称呼,因为工人阶级正在领导一切…… 

我表弟建议这位可敬的老人,假如有人来问这架禄来福来相机,就说它有种种毛病;还建议他在相机里夹张纸条把快门卡住,这样该相机的毛病就更加显著了。总而言之,他要使这台相机总是卖不去;然后降价,卖给我们。我表弟的居心就是这么险恶。说完了这件事,我们一起向马路对面走去。那里有家饭庄,名叫“砂锅居”……这地方的名菜是砂锅三白,还有炸鹿尾……与这些名字相连的是这样一些事实:姥姥去世以后,我和表弟靠微薄的抚恤金过活,又没有管家的人,生活异常困难,就靠这种把戏维持家用:买下旧货行里的坏东西,把它加价卖出去。做这种事要有奸商的头脑和修理东西的巧手。这两样东西分别长在我表弟和我的身上。从本心来说,我不喜欢这种事,所以,“禄来福来”这个名字使我沉吟不语。 

5 

我表弟到北京来看我,我对他不热情。我讨厌他那副暴发户的嘴脸,而且我也没想到立新街甲一号这个地点和禄来福来这个品牌。假如想到了,就会知道我只有一个表弟,我和他共过患难。把这些都想起来之后,也许我会对他好一点。 

下一个名字属于一架德国出产的电子管录音机,装在漆皮箱子里;大概有三十公斤左右,在箱子的表面上贴了一张纸,上面写了一个“残”字。在西四委托行的库房里,我打开箱盖,揭掉面板,看着它满满当当的金属内脏:这些金属构件使我想起它是一台“格朗地”,电子管和机械时代的最高成就。她复杂得惊人,也美得惊人。我表弟在一边焦急他说:表哥,有把握吗?而我继续沉吟着。我没有把握把它修好,却很想试试。但我表弟不肯用我们的钱让我试试。他又对那个臀部宽广的老头说:赵师傅,能不能给我们一台没毛病的?赵师傅说:可以,但不是这个价。我表弟再次劝说他把好机器做坏机器卖给我们,还请赵师傅说要“哪儿请”,但赵师傅说:哪儿请都不行,别人都去反映我了……这些话的意思相当费解。我没有加入谈话,我的全部注意力都被眼前的金属美人吸引住了。 

那台“格朗地”最终到了我们手里。虽然装在一个漂亮箱子里,它还是一台沉重的机器,包含着很多钢铁。提着它走动时,手臂有离开身体之势。晚上我揭开它的盖子,揭开它的面板,窥视它的内部,像个窥春癖患者。无数奇形怪状的铁片互相啮合着,只要按动一个键,就会产生一系列复杂的运动;引发很复杂的因果关系。这就是说,在这个小小的漆皮精子里,钢铁也在思索着…… 

我把薛嵩写作一位能工巧匠,自己也不知是为什么;现在我发现,他和我有很多近似之处,我花了很多时间修理那台“格朗地”,与此同时,我表弟在我耳边恬噪个不停:表哥,你到底行不行?不行早把它处理掉,别砸在我们手里!起初,我觉得这些话真讨厌,恨不得我表弟马上就死掉;但也懒得动手去杀他;后来就不觉得他讨厌,和着他的吩叨声,我轻轻吹起口哨来。再后来,假如他不在我身边啼叨,我就无法工作。哪怕到了半夜十二点,我也要把他吵起来,以便听到他的唠叨……我表弟却说道:表哥,要是我再和你合伙,就让我天打五雷轰!从此之后,我就没和表弟合过伙。我当然很想再合伙,顺便让天雷把表弟轰掉。但我表弟一点都不傻。所以他到现在还活着。 

因为格朗地,我和表弟吵翻了。我把它修好了,但总说没修好,以便把它保留在手里。首先,我喜欢电子设备,尤其是这一台;其次,人也该有几样属于自己的东西,我就想要这一件,但他还是发现了,把它拿走,卖掉了。此后,我就失掉了这台机器,得到了一些钱。我表弟把钱给我时,还忘不了教育我一番:表哥,这可是钱哪。你想想罢。钱不是比什么都好吗──我就不信钱真有这么重要。如今我回想起这些事,怎么也想象不出,我是怎么忍受他那满身的铜臭的……吵架以后不久,他就去泰国投靠一位姨父。只剩下我一个人。我的过去一片朦胧……现在我正期待着新的名字出现…… 

第二节 

晚上,我在自己家里。因为天气异常闷热,我关着灯。透过塑料百页窗,可以看到对面楼上的窗子亮着昏黄的光。这叫我想起了马雅可夫斯基的诗句──“一张张燃烧的纸牌”。本来我以为自己会想不起马雅可夫斯基是谁,但是我想起来了。他是一个苏俄诗人。他的命运非常悲惨。我的记忆异常清晰,仿佛再不会有记不得的事情──我对自己深为恐惧。 

在我窗前有盏路灯,透进火一样的条纹。白衣女人站在条纹里,背对着我,只穿了一条小小的棉织内裤。我站了起来,朝她走去,尽力在明暗之中看清她。她的身体像少女一样修长纤细,像少女一样站得笔直,欣赏墙上的图案。我禁不住把手放在她背上。她转过身来,那些条纹排列在她的脖子上、胸上,有如一件辉煌的衣装。 

我还在长安城里。下雪时,白昼和黑夜不甚分明,不知不觉,这间房子就暗了很多;除此之外,敞开的窗框上已经积了很厚的雪。雪的轮廓臃肿不堪,好像正在膨胀之中。那个白衣女人把黑色的斗篷分做两下,站了起来,说道:走吧,不能总呆在这里。然后就朝屋角自己的衣服走去。从几何学意义上说,她正在离开我。而在实际上却是相反。任何一位处在我地位的男子都会同意我的意见,只要这位走开的裸体女士长着修长的脖子,在乌青的发际正中还有一缕柔顺的长发低垂下来;除此之外,这位女士的身体修长、纤细,臀部优雅──也就是说,紧凑又有适度的丰满──这些会使你更加同意我的意见。在雪光中视物,相当模糊,但这样的模糊恰到好处……当她躬下身来,钻进自己的衣裙时,我更感到心花怒放……后来,她系好了木履上的每一根皮带子,就到了离去的时节。我对这间已经完全暗下来的房子恋恋不舍。但我也不肯错过这样的机会,和她并肩走进漫天的大雪。如前所述,我不认为自己是学院派。但在这些叙述里,包含了学院派的金科玉律,也就是他们视为真、善、美三位一体的东西。 

我在条纹中打量那位白衣女人,脖子、乳房、小腹在光线中流动。她对我说:什么事?我说,没有什么。就转过身去,欣赏我们留在墙上的图案。在墙上,我们是两个黑色的人影。有风吹过时,闪着电光的鳗鱼在我们身边游动。忽然,她跳到我的背上,用光洁的腿卡住我的腰,双手搂住我的脖子,小声说道:什么叫“没有什么”?此时,在我身后出现了一个臃肿的影子。我不禁小声说道:袋鼠妈妈……这个名称好像是全然无意地出现在我脑海里。白衣女人迅速地爬上我的脖子,用腿夹住它,双手抱住我的头,说道:好呀,连袋鼠妈妈你都知道了!这还得了吗?现在我不像袋鼠妈妈,倒像是大树妈妈,只可惜我脚下没有树根。重心一下升到了我头顶上,使我很难适应。我终于栽倒在床上了。然后,她就把我剥得精光,把衣服鞋袜都摔到墙角去,说道:这么热的天穿这么多,你真是有病了……起初,这种狂暴的袭击使我心惊胆战;但忽然想起,她经常这样袭击我。只要我有什么举动或者什么话使她高兴,就会遭到她的袭击。这并不可怕,她不会真的伤害我。 

2 

我努力去追寻袋鼠妈妈的踪迹,但是又想不起来了,倒想到了一个地名:北草厂胡同。这胡同在西直门附近,里面有个小工厂。和表弟分手以后,我就到这里当了学徒工。在它门口附近,也就是说,在别人家后窗子的下面,放了一台打毛刺的机器。我对这架机器的内部结构十分熟悉,因为是我在操作它。它是一个铁板焊成的大滚筒,从冲压机上下来的零件带着锋利的毛刺送到这里,我把它们倒进滚筒,再用大铁锹铲进一些鹅卵石,此后就按动电门,让它滚动,用卵石把飞刺滚平,从这种工艺流程可以看出我为什么招邻居恨──尤其是在夏夜,他们敞着窗子睡,却睡不着,就发出阵阵呐喊,探讨我的祖宗先人。当然,我也不是吃素的,除了反唇相讥,我还会干点别的。抓住了他们家的猫,也和零件一起放进滚筒去滚,滚完后猫就不见了,在筒壁内部也许能找到半截猫尾巴。 

后来,那家人的小孩子也不见了,就哭哭啼啼地找到厂里来,要看我们的滚筒──他们说,小孩比猫好逮得多;何况那孩子在娘胎里常听我们的滚筒声,变得呆头呆脑,没到月份就跑了出来;这就更容易被逮住了。这件事把我惊出了一头冷汗。谢天谢地,我没干这事。那孩子是掉在敞着盖的粪坑里淹死了──对于他的父母真是很不幸的事,好在还可以再生,以便让他再次掉进粪井淹死──假如对小孩子放任不管,任何事都可能发生。我就是这样安慰死孩子的父母,他们听了很不开心,想要揍我。但我厂的工人一致认为我说了些实话,就站出来保护我这老实人。出了这件事以后,厂领导觉得不能让我再在厂门口呆着,就把我调进里面来,做了机修工。 

进到工厂里面以后,我遇上了一个女孩子,脸色苍白,上面有几粒鲜红的粉刺,梳着运动员式的短头发。那个女孩虽没有这位白衣女人好看,但必须承认,她们的眉眼之间很有一些相似之处。她开着一台牛头刨。这台刨床常坏,我也常去修,我把它拆开、再安装起来,可以正常工作半小时左右;但整个修理工作要持续四小时左右,很不合算;最后,她也同意这机器不值得再修了。这种机床的上半部一摇一摆,带着一把刨刀来刨金属,经常摆着摆着停了摆,此时她就抬起腿来,用脚去踹。经这一踹,那刨床就能继续开动,我从那里经过,看到这个景象,顺嘴说道:狗撤尿。然后她就追了出来,用脚来踹我。她像已故的功夫大师李小龙一样,能把腿踢得很高。但我并非刨床,也没有停摆啊…… 

我怀疑这个女孩就是袋鼠妈妈,她逐渐爱上了我。有一次,我从厂里出来,她从后面追上来,把我叫住,在工作服里搜索了一通之后,掏出一个小纸包来,递给我说:送你一件东西,然后走开了。我打开重重包裹的纸片,看到里面有些厚厚的白色碎片,是几片剪下的指甲。我像所罗门一样猜到了这礼物的寓意:指甲也是身体的一部份。她把自己裹在纸里送给我,这当然是说,她爱我。下次见到她时,我说,指甲的事我知道了。本来我该把耳朵割下来做为回礼。但是我怕疼,就算了吧。这话使她处于颠狂的状态,说道:连指甲的秘密你都知道了,这还得了吗?马上就来抢这只耳朵,等到抢到手里时又变了主意,决定不把它割下来,让它继续长着。 

3 

我有一件黑色的呢子大衣,又肥又长,不记得是从哪个委托行里买来的,更不知道原主是谁。我斗胆假设有一位日本的相扑力士在北京穷到了卖大衣的地步,或者有一位马戏班的班主十分热爱他的喜马拉雅黑熊,怕它在冬天冻着;否则就无法解释在北京为什么会有件如此之大的衣服,假如我想要穿着这件衣服走路的话,必须把双臂平伸,双手各托住一个肩头,否则就会被下摆绊倒──假如这样走在街上,就会被人视为一个大衣柜。当然,这种种不利之处只有当白天走在一条大街上才存在。午夜时分穿着它坐在一条长椅上,就没有这些坏处,反而有种种好处。北京东城有一座小公园,围着铁栅栏,里面有死气沉沉的假山和乾涸的池塘,冬天的夜里,树木像一把把秃扫帚,把儿朝下地栽在地上。这座公园叫作东单公园──它还在那里,只是比当年小多了。 

此时公园已经锁了门,但在公园背后,有一条街道从园边穿过,这里也没有围墙。在三根水泥竿子上,路灯彻夜洒落着水银灯光……我身材臃肿,裹着这件呢子大衣坐在路边的长凳上,脸色惨白(在这种灯下,脸色不可能不惨白),表情呆滞,看着下夜班的人从面前骑车通过。这是七五年的冬夜,天上落着细碎、零星、混着尘土、像微型鸟粪似的雪。 

想要理解七五年的冬夜,必须理解那种灰色的雪,那是一种像味精一样的晶体,它不很凉,但非常的脏。还必须理解惨白的路灯,它把天空压低,你必须理解地上的尘土和纷飞的纸屑。你必须理解午夜时的骑车人,他老远就按动车铃,发出咳嗽声,大概是觉得这个僻静地方坐着一个人有点吓人。无论如何,你不能理解我为什么独自坐在这里。我也不希望你能理解。 

午夜十二点的时候,有一辆破旧的卡车开过。在车厢后面的木板上,站了三个穿光板皮袄、头戴着日本兵式战斗帽的人。如果你不曾在夜里出来,就不会知道北京的垃圾工人曾是这样一种装束。离此不远,有一处垃圾堆,或者叫作渣土堆,因为它的成份基本上是烧过的蜂窝煤。在夜里,汽车的声音很大,人说话的声音也很大。汽车停住以后,那些人跳了下来,用板锹撮垃圾,又响起了刺耳的金属摩擦声。说夜里寂静是一句空话──一种声音消失了,另一种声音就出来替代,寂静根本就不存在。垃圾工人们说:那人又在那里──他大概是有毛病罢。那人就是我。我继续一声不响地坐着,好像在等待戈多……因为垃圾正在被翻动,所以传来了冷冰冰的臭气。 

垃圾车开走以后,有一个人从对面胡同里走出来。他穿了一件蓝色棉大衣,戴着一个红袖标,来回走了几趟,拿手电到处晃──仿佛是无意的,有几下晃到了我脸上。我保持着木讷,对他不理不采。这位老先生只有一只眼睛能睁开,所以转过头来看我,好像照像馆用的大型座机……他只好走回去,同时自言自语道:什么毛病。再后来,就没有什么人了。四周响起了默默的沙沙声……她从领口处钻了出来,深吸了一口气说:憋死我了──都走了吗?是的,都走了。要等到两点钟,才会有下一个下夜班的人经过。从表面上,我一个人坐在黑夜里;实际上却是两个人在大衣下肌肤相亲。除了大衣和一双大头皮鞋,我们的衣服都藏在公园内的树丛里,身上一丝不挂,假如我记忆无误,她喜欢缩成一团,伏在我肚子上。所以,有很多漫漫长夜,我是像孕妇一样度过的……但此时我们正像袋鼠一样对话,她把我称作袋鼠妈妈。原来,袋鼠妈妈就是我啊。 

4 

虽然是太平盛世,长安城里也有巡夜的士兵,捉拿夜不归宿的人。那些人在肩上扛着短载,手里拿着火把,照亮了天上飘落的雪片──每个巡夜的士兵都是一条通天的光柱,很难想象谁会撞到这些柱子上。在我看来,他们就像北京城里的水银灯。假如你知道巡逻的路线,他们倒是很好的引路人。因为这个缘故,我们走在一队巡逻兵的后面,跟得很紧,甚至能听见他们的交谈,即便被他们逮住,也不过是夜不归宿──很轻的罪名。在北京城里也有守夜的人,他们从我面前走过,对我视而不见。因为他们要逮的是两个人,而非一个人……但我多少有点担心,被逮住了怎么办。为此曾请教过她的意见。她马上答道:“那就嫁给你呗。”在公园里被逮住之后,嫁过来也是遮丑之法。然后她又说:讨厌,不准再说这个了。看来她 很不想嫁给我。 

我最终明白,对我来说,雪就是性的象征。我和她走在长安城的漫天大雪之中;这些雪就像整团的蒲公英浮在空中。因为夜幕已经降临,所以每一团松散的雪都有蓝色的荧火裹住,就这样走到了分手的时节。雪蒙蒙的夜空传来了低哑的雷声,模糊不清的闪电好像是遥远的焰火。而在遥远的北京城里,分手的时节还没有到来。它是在黎明,而不是在午夜……后来,在北京城的冬夜里,我想到了这些事,就说:性是人间绝顶美丽之事。她马上就从大衣里钻了出来,惊叫道:袋鼠妈妈!你是一个诗人!再后来,在北京城的夏夜里,我喃哺说道:袋鼠妈妈是个诗人……她马上在飘浮着的灯光里跪了起来,拿住我的把把说:连他是诗人你都知道了──咱们来庆祝一下吧!这使我想了起来,我经常假装失掉了记忆,过一段时间再把它找回来,以便举行庆祝活动。现在庆祝活动在举行中,看来,我没有什么失落的东西了。 

从她的角度来看,我和我的黑大衣想必像是一片黑黝黝的海水,而她自己像一只海狗(假如这世界上有白色的海狗)一样在其中潜水,当然这海里也不是空无一物……她浮出水面向我报告说:一个硬邦邦的大蘑菇哎。我无言以对。她又说:咬一口。我正色告诉她:不能咬,我会疼的。后来她又潜下去,用齿尖和舌头去碰那个大蘑菇。而我继续坐在那里,忍受着从内部来的奇痒。外面黑色的夜空下,才真正的空无一物。再过一会儿,她又来报告说:大蘑菇很好玩。我由衷地问道:大蘑菇是什么呀? 

夜里,我们的床上是一片珊瑚海,明亮的波纹在海底游曳,她就躺在波纹之中,好像一块雨花石;伸出手来,对我说道:快来。在闷热的夜里,能够潜入水底真是惬意。有一只鳐鱼拖着乌云般的黑影侵入了这片海底,这就是我,我们以前举行的庆祝活动却不是这一种。这是因为,当时我们还没有被人逮住。午夜巡逻的工人民兵在走过,但只是惊诧地看着我的大肚子──那年月的伙食很难把肚子吃到这么大。当然,人家也不全是傻瓜。有一夜,一个小伙子特意掉了队,走到我面前借火。我摇摇头说,我不吸烟。他却进一步凑了过来,朝我的大肚子努努嘴,低声说道:这里面还有一个吧?我朝他笑了一笑。所以,在这个世界上,可能还有人记得,在七五年的寒夜里,水银灯光下马路边上那一缕会心的微笑。 

5 

在北京城的冬夜里,分手时节是在公园里的假山边上。那件黑大衣就如蛇蜕一般委顿于地,地面上有薄薄的一层白粉,与其说是雪,不如说是霜。曙光给她的身体镀上一层灰色,因为寒冷,乳房紧缩于胸前,对于女人来说,美丽就是裸体直立时的风度──带着这种风度,她给自己穿上一条面口袋似的棉布内裤一然后是红毛裤,红毛衣,蓝布工作服。最后,她用一条长长的绒围巾把头裹了起来,只把脸露在外面──想必你还记得七十年代的女孩流行过一种裹法,裹出来像海带卷,现在则很少见──戴上毛线手套,从树丛里推出一辆自行车,说道:厂里见,就骑走了。我影影绰绰地记得,在厂里时,她并不认识我,她看我的神情像条死带鱼。在街上见面时她也不认识我,至多侧过头来,带着嫌恶的神情看上一眼。晚上,在公园里见面时,她也不认识我,顶多公事公办他说一句:在老地方等我。只有在那件大衣的里面她才认识我,给我无限的热情和温存。 

在那件旧大衣底下,我是一个谦谦君子。我总把手背在身后,好像一年级的小学生在课堂上听讲。很快我就忘掉自己长着手了。我很能体会一条公蛇能从性中体验到什么,而且我总觉得,只有蛇这种动物才懂得什么叫作性感。我不是一条蛇,这正是我的不幸之处。有时候她对我发出邀请,说道:摸摸我!我想把手伸出来,但同时想到,我是一个蛇一样的君子,就把手又背过去,简短地回答道:不摸。这种争论可以持续很久,到了后来,她只说一个字:摸!我只说两个字:不摸。听起来就是:摸!──不摸。在对答之间,隔了一分钟。按照这种情节,她能够保持处女之身,都是因为我坐怀不乱──我就是这么回想起来的,但又影影绰绰地觉得有点不对。也有可能是我要摸她,但她不让。需要说明,不论是公园还是校园,都常常不止我们两个人。别人把这种问答听了几十遍,自然会对我们产生兴趣。在黎明前的曙光里,常有一个男孩子(有时也有女孩子怯生生地跟着)走过来。听到脚步声,她赶紧把头从衣领处探出来,和我并肩坐着,像一个双头怪胎。这位男孩子笑笑说:我来看看你们在干什么呢。她就答道:没干什么。没干什么。然后,那个男孩就又笑了一笑,说:认识你们很高兴。她又抢答道:我们也很高兴。然后从袖筒里伸出手来,和他握手告别。我也很想和这个小伙子握手告别,但伸不出手来──在这种地方,遇上的都是夜不归宿的人。而夜不归宿的都是些文明人。但我影影绰绰地觉得,这故事我讲得有点不对头了。 

和分手时节紧密相接的是相见时节──中间隔了一个无聊的白天,这是很容易忘掉的──也是在这座假山边上,夜幕刚刚降临,游人刚刚散尽。她就是不肯钻进这件黑大衣。夜晚最初的灯光并不明亮,所以,白色的身体份外醒目。我说道:快进来,别让别人看到了。她说:我不。坏东西,你让我怎能相信你。我说:我不是坏东西。我是袋鼠妈妈。她却说:袋鼠妈妈是谁呀?最后,我只能像事先商量好的那样,背过身去,让她用一根棉线绳子把手绑在了背后。然后她才肯钻进大衣,捏捏那个硬邦邦的家伙,说道:好恶毒啊……幸亏我防了一手。还想帮它骗我吗?坐在长椅上时,我想,假如这样被人逮到,多少有点糟糕,然后,我就把这件事忘掉了。 

第三节 

我的过去不再是一片朦胧。过去有一天我结婚,乘着一辆借来的汽车前去迎亲,我的大姨子对我说:我妹妹是个疯子。晚上她要是讨厌,你别理她,径直干好事──很难想象哪个大姨子会建议未来的妹夫强奸自己的妹妹,除非他们以前就认识。但我分明不认识这个大姨子。这个女人的头很大,梳了两条大辫子,前面留了很重的刘海,背上背了一个小孩子。她弯着腰,让小孩骑在背上,头顶就在我眼前;三道很宽的发缝和满头的头皮屑就在我眼前。这个景象和晚上十点钟的农贸市场相似:那里满地是菜叶和烂纸。我可以发誓,这个背孩子的女人我见过不到三次,其中一次就是这一次,在这间低矮的房子里。头顶有一片低垂的顶棚,上面满是黄色的水渍。屋子里弥漫着浓郁的尿骚味…… 

从窗户看出去,是个陌生的院子,带着灰色的色调,像一张用一号相纸洗印的照片。院里有棵枣树,从树干到枝头到处长满了瘤子。这个院子我也很是陌生。院子里有个老太太的声音在吵吵闹闹,院子外面汽车喇叭不停地叫,好像电路短路了。我按捺不住手艺人的冲动,想冲出去把它修好。但我还是按捺住了──作为新郎,显然不宜有一双黑油手,这位新娘子是别人介绍我认识的──但愿她和白衣女人不是一个。我一面这样想,一面又隐隐地觉得这种想法不切实际。然后,她哇地一声从里屋冲了出来,穿着白色的睡袍,赤着脚,手里拿了一把小镜子,苍白的脸上每粒粉刺都鲜艳地红着,看来都是挤过的,嘴边还有一处流了血:“哇,真可怕,要结婚就长疙瘩啦”,到脸盆架边撕了一块棉花,又跑回去了。她和我以前认识的女孩显然是一个,和现在的白衣女人又很像。我马上就会想到她是谁。 

我终于纠正了自己的错误,早上起来,我向那位白衣女人坦白说,我失去了记忆,过去的事有很多记不得了。一个人失去记忆,就是变成了另一个人。我变成了另一个人,又不自觉声明,就这样过了半个多礼拜,在这期间,我一再犯下非法占有对方身体之罪。这个错是如此的罪大恶极,简直没有什么希望得到原谅。但是她听了以后,只略呈激动之态,还微笑着说:是吗,还有什么?快说呀。此时我也想给自己说几句话,就说:想必你也看出来了,我心地善良、作风朴实,有各种各样的优点,而且热爱性生活──我的本意是说,我虽已不是以前的王二,但也不无可取之处,希望她继续接受我。谁知她听了这末一句(热爱性生活)就大笑起来,并且挣扎着说道:Me too!Me too!那声音好像是在打嗝。一位可爱的女士这样说话,多少有点失态,我不禁皱起眉毛来。后来她终于不笑了,走过来拍拍我的脸说:你已经够逗的了,别再逗啦。直到此时我才明白,原来我是很逗的。 

2 

如你所知,毕业以后,我到万寿寺里工作。起初,我严守着这两条戒律:不要修理任何东西,不要暴露自己是袋鼠妈妈。所以我无事可做,只能端坐在配殿里写小说。因为一连好几年交不出一篇像样的论文,领导对我的憎恶与日俱增。夜里,在万寿寺前的小花坛里,一谈到这些憎恶,她就赞叹不止:袋鼠妈妈,好硬呀。然后我就谈到让我软一些的事:别人给我介绍对象。他们说,女孩很漂亮,和我很般配。就在我们所里工作,和我又是同学。假如我乐意,他们就和女方去说。她马上大叫一声,从大衣底下钻了出来,赤条条地跑到花坛里去穿衣服,嘴里叫着:讨厌,真讨厌!这样大呼小叫,招来了一些人,手扶着自行车站在灯光明亮的马路上,看她白色的脊背,但她对来自背后的目光无动于衷。我木然坐在花坛的水泥沿上,她又跑了回来,在我背上踢了一脚说,还坐在这里干什么?还不快点滚?而我则低沉地说道:可你也得把我放开呀……后来,我和她一起走进黑暗的小胡同,还穿着那件黑大衣,推着一辆自行车,车座上夹着我的衣服。我微微感到伤感,但不像她那样痛心疾首。但她后来又恢复了平静,说道:既然如此,那就结婚罢。这就是说,如果不是有人发现我和她般配,我到现在还是袋鼠妈妈。 

……那一天她不停地磕瓜子,从早上磕到了午夜,所到之处,到处留下了瓜子皮。那一天她穿了一件红缎子旗袍和一双高跟鞋,这在她是很少有的装束。除此之外,她还在读安加沙·克里斯蒂的侦探小说,对任何人都不理不睬。我的丈母娘对此感到愤怒,就去抢她的书,抢掉一本她又拿出一本,好像在变古彩戏法。但是变古彩戏法的人身上总是很臃肿的,而这位新娘子则十分苗条,简直苗条得古怪;衣服也十分单薄,连乳头的印子都从胸前的衣服上凸了出来──我的丈母娘老想把印子抚平,并且用身体挡住我的视线,她说:妈,别挑逗我好不好──把老太太气得两眼翻白。时至今日,我也不知这戏法是怎么变的,唯一可行的解释是:我丈母娘和她通同作弊,明里抢走一本,暗里又送回来,用这种把戏来恫吓新女婿,让他以为自己未来的妻子有某种魔力。但我又觉得不像:我丈母娘是个很严肃的人,鼓着肥胖的双腮,不停地唠叨。我很讨厌别人唠叨,如果不是要娶她女儿,我绝不会和她打任何交道…… 

我记得这是我们结婚的日子,这一天俗不可耐。所有的婚礼大概都是这个样子。因此她把自己对准了一本侦探小说,鼻梁上架了一副白边眼镜──她有四百度的近视。等到眼镜被抢走之后,她就眯起眼睛来,好像一只守宫(一种变色龙)在端详蚊子。到酒宴临近结束时,大家要求新娘子给男宾点烟。她把书收好站了起来。此时大家才看到,这位新娘子长了两只硕大的白眼珠,上面各有一个针尖大小的黑色瞳孔──都是没眼镜看书看的。她从桌子底下拿出一支大号手枪,把所有的男宾一一枪毙掉。你当然知道我的意思,她用手枪式的打火机给大家点烟。每点一位,就扭过头去闻闻自己的腋窝说:天热,有味了。这当然是说所有的宾客都早已死掉,已经有味了。 

喜宴过后,到了新房里,这位新娘子又歪在了床上看克里斯蒂。我无事可干,只好抽烟。把身上带的四盒烟都抽完以后,很想再去买一盒。当时午夜时分,要买烟就得去北京站,那地方实在远了一点,所以我没有去。这些事说明她很能沉得住气。这好像也是我的长处。但我很不想往这方面来想。假如我们俩也可以贯通,那就要变成一个人。这样人数就更少了。那天晚上我把烟抽完后,就开始磕瓜籽。假如是葵花籽,我磕起来就没有问题。不幸是些西瓜籽,籽皮又滑又硬,我不会磕,磕来磕去,磕不到籽仁,只是吐出些黑白相间、鸡屎也似的残渣…… 

3 

在长安城里,我和白衣女人分手,走过黑白两色的街道。现在飘落的雪片像松鼠的尾巴,雪幕因此而稀疏。这样的雪片像落叶一样在街道两侧堆积着。在我身后,留着残缺不全的脚印。也许我的下一篇论文该考一考长安城里的雪?它又要把领导气得要死。在他狭隘的内心里,容不下一点诗意。 

在我自己的故事里,早已经过了午夜,但我还没按大姨子的告诫行事。她终于看完了那本克里斯蒂,并给它两个字的评价:瞎编;把它丢开。然后,她朝我皱起了眉头,说道:咱们要干什么来的?我摇摇头说:我也不记得。看来,我失去记忆不是头一次了……后来,还是她先想了起来:嗅!今天咱们结婚!当然,这不是认真忘了又想起来,是卖弄她的镇定从容。我那次也不是认真失去了记忆,而是要和她比赛健忘。无怪乎本章开始的时候,我告诉她自己失去了记忆时,她笑得那么厉害──她以为我在拾新婚之夜的牙慧──但我觉得自己还不致于那么没出息…… 

后来,她朝我张开双臂,说道:来吧,袋鼠妈妈……必须承认,这个称呼使我怦然心动。那根大蘑菇硬得像搏面杖一样。我说的不仅是过去,还有现在──用当时的口吻来说,那就是:不仅是现在,还有将来。但我还是沉得住气,冷静地答道:别着急嘛。我一点都不急──我看你也不急。她说道:谁说我不急?就把旗袍脱掉,并且说:把你的大蘑菇拿出来!好像在野餐会上的口气。在旗袍下面,她什么都没有穿,只有光洁、白亮的肉体──难怪她白天苗条得那么厉害──于是我就把大蘑菇拿了出来。那东西滚烫滚烫,发着三十九度的高烧。请相信,底下的事我一点都记不得了。只记得她说了一句:你真讨厌哪,你……因为想不起来,所以那个关节还在,我的过去还是一个故事,可以和现在分开。 

现在,我除了长安城已经无处可去。所以我独自穿过雪幕,走过曲折的小桥,回到自己家里。在池塘的中央,有一道孤零零的水榭;它是雪光中一道黑影,是一艘方舟,漂浮在无穷无尽的雪花之上……那道雪白的小桥变得甚胖。这片池塘必定有水道与大江大河连接,因为涌浪正从远处涌来,掀起那厚厚的雪层。在我看来,不是池水、层积在上面的雪在波动,而是整个大地在变形,水榭、小桥、黑暗中的树影,还有灰色、朦胧、几不可辨的天空都在错动。实际上,真正错动变形的不是别的,而是我。这是我的内心世界。所以就不能说,我在写的是不存在的风景。我在错动之中咬紧牙关,让“格支格支”的声音在我头后响起。好像被夹在挪动的冰缝里,我感觉到压迫、疼痛。这片错动中的、黑白两色的世界不是别的,就是“性”。 

我在痛苦中支持了很久,而她不仅说我讨厌,还用拳头打我。等到一切都结束,我已经松弛下来,她还不肯甘休,追过来在我胸前咬了一口,把一块皮四面全咬破了,但没有咬下来。据说有一种香猪皮薄肉嫩,烤熟之后十分可口。尤其是外皮,是绝顶美味。这件事开始之前我是袋鼠妈妈,在结束时变成了烤乳猪。那天晚上,我被咬了不止一口──她很凶暴地扑上来,在我肩头、胸部、腹部到处乱咬,给我一种被端上了餐桌的感觉……但是,她的食欲迅速地减退,我们又和好如初了。 

4 

当一切都无可挽回地沦为真实,我的故事就要结束了。在玫瑰色的晨光里,我终于找到了我们的户口本,第一页上写着她的名字,在另一栏上写着:户主。我的名字在第二页上,另一栏上写着:户主之夫。我终于知道了她的名字,但现在不敢说;恐怕她会跳到我身上来,叫道:连我的名字你都知道了!这怎么得了啊!现在不是举行庆祝活动的适当时节,不过,我迟早会说的。 

你已经看到这个故事是怎么结束的:我和过去的我融汇贯通,变成了一个人。白衣女人和过去的女孩融汇贯通,变成了一个人,我又和她融汇贯通,这样就越变越少了。所谓真实,就是这样令人无可奈何的庸俗。 虽然记忆已经恢复,我有了一个属于自己的故事,但我还想回到长安城里──这已经成为一种积习。一个人只拥有此生此世是不够的,他还应该拥有诗意的世界。对我来说,这个世界在长安城里。我最终走进了自己的屋子──那座湖心的水榭,在四面微白的纸壁中间,黑沉沉的一片睁大红色的眼睛──火盆在屋子里散发着酸溜溜的炭味儿。而房外,则是一片沉重的涛声,这种声音带着湿透了的雪花的重量──水在搅着雪,雪又在搅着水,最后搅成了一锅粥。我在黑暗里坐下,揭开火盆的盖子,乌黑的炭块之间伸长了红蓝两色的火焰。在腿下的毡子上,满是打了捆的纸张,有坚韧的羊皮纸,也有柔软的高丽纸。纸张中间是我的铺盖卷。我没有点灯,也没有打开铺盖,就在杂乱之中躺下,眼睛绝望地看着黑暗。这是因为,明天早上,我就要走上前往湘西风凰寨的不归路。薛嵩要到那里和红线汇合,我要回到万寿寺和白衣女人汇合。长安城里的一切已经结束。一切都在无可挽回地走向庸俗。
