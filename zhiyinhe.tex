\chapter{致银河}

你好哇,李银河。 

你走了以后我每天都感到很闷,就像堂吉诃德一样,每天想念托波索的达辛尼亚。请你千万不要以为我拿达辛尼亚来打什么比方。我要是开你的玩笑天理不容。我只是说我自己现在好像那一位害了相思病的愁容骑士。你记得塞万提斯是怎么描写那位老先生在黑山里吃苦吧?那你就知道我现在有多么可笑了。  

我现在已经养成了一种习惯,就是每三二天就要找你说几句不想对别人说的话。当然还有更多的话没有说出口来,但是只要我把它带到了你面前,我走开时自己就满意了,这些念头就不再折磨我了。这是很难理解的是吧?把自己都把握不定的想法说给别人是折磨人,可是不说我又非常闷。 

我想,我现在应该前进了。将来某一个时候我要来试试创造一点美好的东西。我要把所有的道路全试遍,直到你说“算了吧王先生,你不成”为止。我自觉很有希望,因为认识了你,我太应该有一点长进了。 

我发觉我是一个坏小子,你爸爸说的一点也不错。可是我现在不坏了,我有了良心,我的良心就是你。真的。 

你劝我的话我记住了。我将来一定把我的本心拿给你看。为什么是将来呢?啊,将来的我比现在好,这一点我已经有了把握。你不要逼我把我的坏处告诉你。请你原谅了这一点男子汉的虚荣心吧。我会在暗地里把坏处去掉,我要自我完善起来。为了你我要成为完人。 王小波 5月20日

——致银河(书简2) 

你好哇,李银河。今天我走出诌了一首歪诗。我把它献给你。这样的歪诗实在拿不出手送人,我都有点不好意思了。 

今天我感到非常烦闷 

我想念你 

我想起夜幕降临的时候 

和你踏着星光走去 

想起了灯光照着树叶的时候 

踏着婆娑的灯影走去 

想起了欲语又塞的时候 

和你在一起 

你是我的战友 

因此我想念你 

当我跨过沉沦的一切 

向着永恒开战的时候 

你是我的军旗 

过去和你在一块儿的时候我很麻木。我有点两重人格,冷漠都是表面上,嬉皮也是表面上的。承认了这个非常不好意思。内里呢,很幼稚和傻气。啊哈,想起来你从来也不把你写的诗拿给我看,你也有双重人格呢。萧伯纳的剧本《匹克梅梁》里有一段精彩的对话把这个问题说得很清楚: 

息金斯:杜特立尔,你是坏蛋还是傻瓜?  

杜特立尔:两样都有点,老爷。但凡人都是两样有一点。 

当然你是两样一点也没有。我承认我两样都有一点:除去坏蛋,就成了有一点善良的傻瓜;除去傻瓜,就成了愤世嫉俗、嘴皮子伤人的坏蛋。对你我当傻瓜好了。祝你这一天过得顺利。 王小波 21日 







——致银河(书简3) 

你好哇李银河。 

今天又写信给你。我一点也不知道你在干什么,所以就不能谈论你的工作。那么怎么办呢?还是来谈论我自己。这太乏味了。我自觉有点厚颜,一点也听不见你的回答,坐在这里唠叨。 

今天我想,我应该爱别人,不然我就毁了。家兄告诉我,说我写的东西里,每一个人都长了一双魔鬼的眼睛。就像《肖像》里形容那一位画家给教堂画的画的评语一样的无情。我想了想,事情恐怕就是这样。我呀,坚信每一个人看到的世界都不该是眼前的世界。眼前的世界无非是些吃喝拉撒睡,难道这就够了吗?还有,我看见有人在制造一些污辱人们智慧的粗糙的东西就愤怒,看见人们在鼓吹动物性的狂欢就要发狂。我总以为,有过雨果的博爱,萧伯纳的智慧,罗曼罗兰又把什么是美说得那么清楚,人无论如何也不该再是愚昧的了。肉麻的东西无论如何也不应该被赞美了。人们没有一点深沉的智慧无论如何也不成了。你相信吗?什么样的灵魂就要什么样的养料。比方说我,只让我看什么《铁道游击队》、《激战无名川》,我势必要沉沦。没有象样的精神生活就没有一代新人。 

出于这种信念,我非常憎恨那些浅薄的人和自甘堕落的人,他们要把世界弄到只适合他们生存。因此我“愤懑”,看不起他们。却不想这样却毒害了自己,因为人不能总为自己活着啊。我应该爱他们。人们不懂应当友爱,爱正义,爱真正美的生活,他们就是畸形的人。也不会有太崇高的智慧,我们的国家也就不会太兴盛,连一个渺小的我也在劫难逃要去作生活的奴隶,如果我不爱他们,不为他们变得美好做一点事情得话。这就是我的忏悔。你宽恕我吗我的牧师? 

你没有双重人格,昨天是我恶毒的瞎猜呢。否则你从哪里来的做事的热情呢。这也算我的罪恶之一,我一并忏悔,你也一并宽恕了吧。祝你今天愉快。你明天的愉快留着我明天再祝。 王小波 22日 







——致银河(书简4) 

你好哇李银河。我今天又想起过去的事情。你知道我过去和你交往时最害怕的是什么?我最害怕你从鼻子里发出一声冷笑(如果这样的形容使你愤怒我立刻就收回)。我甚至怀疑这是一把印第安战斧,不知什么时候就要来砍掉我的脑袋。因为我知道我们的思想颇有差距。我们的信仰是基本一致的,但是不是一个教派?过去天主教徒也杀东正教徒,虽然他们都信基督。这件事情使我一直觉得不妙。比方说我就不以为“留痕迹”是个毕生目标,我曾经相信只要不虚度光阴,把命运赐给我的全部智力发挥到顶点,做成一件无愧于人类智慧的事情就对得起自己,并且也是对未来的贡献。这曾经是我的信仰,和你的大不一致吧?那时候我们只有一点是一致的,就是要把生命贡献给人类的事业,决不作生活的奴隶。 

现在我很高兴地告诉你,我的信仰和你又一致了。我现在相信世界上有正义,需要人为正义斗争。我宣誓成为正义的战士。我重又把我的支点放到全人类上,你高兴吗? 

总而言之,我现在决定,从现在开始,只要有一点益处的事情我都干,决不面壁苦思了。现在就从眼前做起,和你一样。我发现我以前爱唱高调偷懒,现在很惭愧。 

祝你今天愉快。 王小波 23日 







——致银河(书简5) 

你好哇李银河。今天收到你25日的来信。你的祝福真使我感动,因此我想到了很多事情。你回来我讲给你听。 

可是你呀!你真不该说上一大堆什么“崇敬”之类的话。真的,如果当上一个有才气的作家就使你崇敬,我情愿永世不去试一下。我的灵魂里有很多地方玩世不恭,对人傲慢无礼,但是它有一个核心,这个核心害怕黑暗,柔弱得像绵羊一样。只有顶平等的友爱才能使它得到安慰。你对我是属于这个核心的。 

我想了一想:是什么使你想起哭鼻子来呢?一定是雨果所说的“幽冥”。这个“幽冥”存在于天空的极深处,也存在于人的思想的极深处,是人类智力所永远不能达到的。有人能说出幽冥里存在着什么吗?啊,有人能。那就是主观唯心主义者和基督教徒。雨果说他是深深敬畏幽冥的。我呢?我不敬畏。幽冥是幽冥,我是我。我对于人间的事倒更关心。 

不过说实在的,我很佩服天文学家。他们天天沉溺在幽冥之中,却还很正常,多么大的勇气啊!简直是写小说的材料。 

真的一种新学科的萌芽诞生了吗?啊,世界上还真有一些有勇气的人,他们是好孩子。我想到这些年来,人对人太不关心了。人活在世上需要什么呀?食物、空气、水和思想。人需要思想,如同需要空气和水一样。人没有能够沉醉自己最精深智力思想的对象怎么能成?没有了这个,人就要沉沦的和畜生一样了。我真希望人们在评价善恶的时候把这个也算进去呀。我想这个权利(就是思想的权利)就是天赋人权之一。不久以前有人剥夺了很多人的思想的权利。这是多么大的罪孽呀。你也看见了,多少人沉沦得和畜生一样了。 

我觉得我无权论是非,没这个勇气,我觉得你可以。你来救我的灵魂吧。 

我整天在想,今天快过去吧,日子过得越快,李银河就越快回来了。你不要觉得这话肉麻,真话不肉麻。祝你愉快。 王小波 29日 







——致银河(书简6) 

你好哇李银河。今天是6月1日,就是说,今天已经是6月初了。可是不知道你在哪里。也许在归途上吧。但愿如此,阿门! 

真应该回想一下童年。有人说当孩子的时候最幸福,其实远非如此。如果说人在童年可以决定自己生命的前途,那么就是当孩子的时候最幸福,其实有一种我们不能左右的力量参加进来决定我们的命运,也就是说,我们被天真欺骗了。 

我从童年继承下来的东西只有一件,就是对平庸生活的狂怒,一种不甘没落的决心。小时候我简直狂妄,看到庸俗的一切,我把它默默地记下来,化成了沸腾的愤怒。不管是谁把肉麻当有趣,当时我都要气得要命,心说:这是多么渺小的行为!我将来要从你们头上飞腾过去!现在这一切都已经过去。要把童年的每一瞬间都呼唤到脑海里,就是花上一个月时间也难办到了。但是这件事我还记得很清楚。我现在还是这样,只是将来不再属于我了。 

我很想把前面写的乱七八糟扯了,但是那就是对你不老实。留着你看看吧。总之,这一段时间比原来想象的苦。你就要回来了是吧?祝你愉快。 王小波 6月1日 







——致银河(书简7) 

你好哇李银河。我们接着来谈幽冥吧。我记得有一次我站在海边,看着海天浑为一色,到处都是蔚蓝色的广漠的一片。头上是蓝色的虚空,面前是浩荡的大海,到处看不见一个人。这时我感到了幽冥;无边无际。就连我的思想也好象在海天之间散开了,再也凝结不起来。我是非常喜欢碧色的一切的。 

后来呢?后来我拍拍胸膛,心满意足地走开了。虽然我胸膛里跳着一颗血污的心脏,脑壳里是一腔白色泥浆似的脑髓(仅此而已),但是我爱我自己这一团凝结的、坚实的思想。这是我生命的支点。浩荡空虚的幽冥算什么? 

接下来又要谈到把肉麻当有趣。这里有一个大矛盾。我极端地痛恨把肉麻当有趣。我有时听到收音机里放几句河南坠子,油腔滑调的不成个东西,恨不得在地上扒个坑把头埋进去。还有一次规模宏大的把肉麻当有趣,就是68年、69年闹林彪的时候。肉麻的成分是无所不在的,就连名家的作品(如狄更斯、歌德等等)里也有一点。可是有人何等地喜欢肉麻!肉麻是什么呢?肉麻就是人们不得不接受降低人格行为时的感觉。有人喜爱肉麻是因为什么呢?是因为他们太爱卑贱,就把肉麻当成了美。肉麻还和现在文学作品中的简单粗糙不同,它挺能吸引人呢。所谓肉麻的最好注脚就是才子佳人派小说,它就是本身不肉麻,也是迎合肉麻心理的。鲁迅是最痛恨肉麻的,我的这个思想也是从他老人家那里批发来的。 

你有一次诧异我为什么痛恨激情,其实我是痛恨肉麻呀!我们是中国人,生活在北京城里,过了26年的平庸生活。天天有人咂着嘴赞美肉麻,你焉能不被影响?你激情澎湃的时候做出的事情,谁敢打保票不是肉麻的? 

我有点害怕自己,怕我也是30%的肉麻人物,所以只有头脑清醒时才敢提笔。这样是不成的。这样达不到美的高度,人家说我没什么革命意识。说得多对呀。 

你也知道了幽冥和肉麻全都不合我的心意。还有什么呢?我看我不要废话了。别人知道了要笑话的:王先生给李银河写情书,胡扯又八道,又是幽冥,又是肉麻。这不是一件太可笑的事实吗?就此打住,祝你愉快。 

王小波 6月2日 你快该回来了吧!我要疯了。————又及 







——致银河(书简8) 

你好哇李银河: 

今天还不见你出现。我脑子里出现了许多宿命论的狂想。比方说,我很想抛一个硬币来占一占你是否今天回来。这说明我开始有点失常了。 

人呀,无可奈何的时候就要丑态百出。我来揣测你遇到什么了。 

也许是会议整风,鸣放的太过了吧?北京来的记者也有一份。留在那里走不了。呜呜!但愿不是这样! 

也许是你去游山玩水。太好了!好好地玩玩吧,我真希望你玩得好。天热吗?千万不要太热。下雨吗?千万不要下雨。下雨什么也看不清楚。刮风吗?不要乱刮大风。最好是迎面而来的洁净的风。你迎风而去,风来涤荡你的胸怀,仰望着头上的蓝天,好像走在天空的道路上。真的吗?真的是这样吗?真是这样就太好了。我要给你写诗,心里太乱写不了。俾德丽采!俾德丽采! 

在回家吗?在火车上吗?想到我了吗?别想,好好睡一觉吧。祝你心里平静而愉快。为什么没有高速火车呢?飞机!协和式飞机!我想一头穿过墙壁奔出去找你。去不了,我太无能。 

我发誓,你不回来我也不给你写信了。再写我就要胡说八道了。绝对不写。不写。祝你愉快!  

王小波 6月5日 

我没有怨恨吧?一点也没有吧。——又及 还有,我瞎扯。不是俾德丽采。那不是咒你吗?不怪我,怪但丁。打倒但丁!打倒意大利!打倒佛罗伦萨! 







——致银河(书简9) 

李银河,你好! 

我自食其言,又来给你写信。按说世界上有很多的人。可是我今天病歪歪地躺了一天,晚上又睡不着觉,发作了一阵喋喋不休的毛病,又没有人来听我说。 

我又在想,什么是文学的基本问题。今天下午3点45分我的答案是:人可以拥有什么样的生活。谁能对这个问题给出美妙的答案呢?当人们被污泥淹着脖子的时候? 

有很多的人在从少年踏入成人的时候差了一步,于是生活中美好的一面就和他们永别了,真是可惜。在所有的好书中写得明明白白的东西,在人步入卑贱的时候就永远看不懂,永远误解了,真是可惜。在人世间有一种庸俗势力的大合唱,谁一旦对它屈服,就永远沉沦了,真是可惜。有无数为人师表的先生们在按照他们自己的模样塑造别人,真是可惜。 

中国人真是可怕!有很多很多中国人活在世上什么也不干,只是在周围逡巡,发现了什么就一拥而上。比方说,刘心武写了《班主任》,写得不坏,说了一声“生活不仅如此!”就有无数的人拥了上去,连声说:“太对太对!您真了不起!您是班主任吧?啧啧,这年头孩子是太坏。”肉麻得叫人毛骨怵然。我觉得这一切真是糟透了。 

人可以拥有什么样的生活呢?这问题真是深奥。我回答不上来。我知道已往的一切都已经过去。雨果博爱的暴风雨已经过去。罗曼·罗兰“爱美”的风暴已经过去。从海明威到别的人,消极的一切已经过去,海面已经平静,人们又可以安逸地生活了。小汽车,洗衣机,中国人买电视,造大衣柜,这一切和我的人格格格不入。有人学跳舞,有人在月光下散步,有人给孩子洗尿布,这一切和我格格不入。有人解释革命理想,使它更合理。这是件很好的工作。 

可是我对人间的事情比较关心。人真应该是巨人。世界上人可以享有的一切和道貌岸然的先生们说的全不一样。他们全是白痴。人不可以是寄生虫,不可以是无赖。谁也不应该死乞白咧地不愿意从泥坑里站起来。 

我又想起雅典人雕在石头上的胜利女神了。她扬翅高飞。胜利真是个美妙的字眼,人应该爱胜利。胜利就是幸福。我相信真是这样。祝你愉快。 王小波 6月6日 







——致银河(书简10) 

银河,你好:  

想你了,跟你胡扯一通。我这样的喋喋不休可能会招你讨厌 

告诉你,我有一种喜欢胡扯的天性。其实呢,我对什么事都最认真了。什么事情我都不容许它带有半点儿戏的性质,可惜我们这里很多事情全带有儿戏的性质。我坚信人是从爬行动物进化来的,因为有好多好多的人身上带有爬性动物那种低等、迟钝的特性。他们办起事情来简直要把人气疯。真的,我不骗你!早几年我已经气疯一百多次了,那时候从学校到舞台到处不都是儿戏?那时候的宣传、运动不是把大家当大头傻子吗?后来我对这些事情都不加评论、不置一辞,只报之以哈哈大笑。后来我养成一种习惯,遇到任何事情先用鼻子闻一下,闻出一丁点儿戏的气味就狂笑起来。真的,我说实话,你别生气。我以为凡把文学当成沽名钓利手段的全是儿戏……我原准备到处哈哈大笑,连自己在内,笑到寿终正寝之时。可是我现在想认真了,因为你是个认真的人。 王小波 8月30日 







——致银河(书简11) 

银河,你好。 

看了你的信,你为什么把你自己说得那么坏,把我说得那么好呢?你真傻,那不是事实啊。 

我知道你因为什么事情在难过。我猜得出来,怎么办呢,这么办吧。假如你愿意,你就恋爱吧,爱我。恋爱可以把什么问题都解决了的。恋爱要结婚就结婚,不要结婚就再恋爱,一直恋到十七八年都好啊,而且更好呢。如果你不要恋爱,那我还是愁容骑士。如果你想喜欢别人,我还是你的朋友。你不能和我心灵相通,却和爱的人心灵不通吧,我们不能捉弄别人的,是吧?所以我就要退后一步了。不过我总觉得你应该是爱我的。这是我瞎猜,不过我总觉得我猜得有道理,因为什么都不是爱的对手,除了爱。 

…… 小波 9月4日  







——致银河(书简12) 

银河,你好! 

我在家里给你写信。你问我人为什么活着,我哪能知道啊,我又不是牧师。释迦摩尼为了解决这个问题出了家,结果得到的结论是人活着为了涅磐,就是死。这简直近乎开玩笑了。 

不过活着总得死,这一点是不错的,我有时对这一点也很不满意呢。还有,人活着有时候有点闷,这也是很不愉快的。过去我想,人活着都得为别人,为别人才能使自己得到超生。那时大家都这么想吧?结果大家都不近人情的残酷,都走上宗教的道路了呢。我们经过了那个时代,把生活都变成一个连绵不断的宗教仪式了呢。后来我见过活着全然为自己的人,他们是真正的唯物主义者,把自己当成物质,需要的东西也是物质,所以就分不出有什么区别。比方说,物质生活就是生活本身吗?有人分不出来。 

总之,我认为人不应当忽视自己,生活就是自己啊。总要无愧于自己才好。比方说我要无愧于自己就要好好地爱你才对。也不能让人家来造自己,谁要来造我我都不干。有人要我们这样要我们那样,我们就不知道什么是生活本身了。过去我们在顶礼膜拜中度过光阴的时候,我们知道什么是生活吗?不过我现在要爱你,我觉得我很对,你也觉得我很对,别人与此有何相干。 

我这么说你恐怕要怕我了。我一点也不可怕。不管你是谁,是神仙也好,是伟人也好,请你来共享我们的爱情。这不屈辱谁,不屈辱你。 

我们的生活就是我们本身。我们本身不傻,也不斤斤计较大衣柜一头沉。干吗要求我们有什么外在的样子,比方说,规规矩矩,和某些人一样等等。真的,我们的生活是一些给人看的仪式吗? 

我有时对自己挺没信心的,尤其是你来问我。我生怕你发现我是个白痴呢。不过你也该知道,我也肯为别人牺牲,也接受一切人们的共同行动,也尽义务,只要是为大家好;却不肯为了仪式去牺牲、共同行动、尽义务,顶多敷衍一下。别人也许就为这个说我坏吧?我很爱开发智力,我怪吗?不怪吧。我还爱一个美的世界,美是为人的幸福才存在的。我也不肯因为什么仪式性的东西去写什么、唱什么、画什么,顶多敷衍它一下。 

总之,我是这样。为了大家好,还为了我自己好,才能正经做事。为了什么仪式,为了看起来挺对路,我就混它。我决不为了仪式爱你,我是正经爱你呢。我一正经起来,就觉得自己不坏,生活也真不坏。真的,也许不坏?我觉得信心就在这里。 

我对自己有点信心。我爱你呢,爱你! 小波 10月29日夜 







——致银河(书简13) 

银河,你好! 

看了你2日的信,我很喜欢你的看法。不过还有一点我不能同意你,你不生气吧?我要说的是:只要我们真正相爱,哪怕只有一天,一个小时,我们就不应该再有一刀两断的日子。也许你会在将来不爱我,也许你要离开我,但是我永远对你负有责任(我也希望你也负起这个责任),就是你的一切苦难就永远是我的。社会的力量是很大的吧?什么排山倒海的力量也止不住两个相爱过的人的互助。我觉得我爱了你了,从此以后,不管什么时候我都不能对你无动于衷。我可不能赞成爱里面一点责任没有。我当然反对它成为一种枷锁,我也不能同意它是一场宴会。我以为它该是终身不能忘却的。比如说,将来你不爱我了,那你就离开我,可是别忘了它。这是不该忘记的东西。 

有时我有点担心你和我是很不同的人。我正是为这一点爱你,可是我怕你会为这一点不爱我。你呀,你是一个热情的人,你很热。我恐怕我有点儿温。我不经常大喜,几乎不会狂喜。你欣喜若狂的时候,也许我只会点头微笑。不,我说这个你也许不会懂呢。我带有一点宿命的情调。我一丁点也不迷信,只不过有一点该死的这种情调罢了。所以我对你的爱不太像火,倒像烧红的石头呢。不过我太喜欢你了,太想爱护你了。你不知道我呢。我爱谁就觉得谁就是我本人,你能自由也就是我自由。不过我可不高兴你把我全忘了。这件事你可不能干。 

下星期日我们到郊外去吧,去看看我的精神巢穴。在那儿你就知道我是一个什么样的穴居野人了。 

说真的,我喜欢你的热情,你可以温暖我。我很讨厌我自己不温不凉的思虑过度,也许我是个坏人,不过我只要你吻我一下就会变好呢。 小波 11月5日 







——致银河(书简14) 

银河,你好! 

你给我带来一个多么美好的东西,就是说,一个多么好的夜晚!想你,想着呢。 

你呀,又勾起我想起好多事情。我们生活的支点是什么,就是我们自己。自己要一个绝对美好的不同凡响的生活,一个绝对美好的不同凡响的意义。你让我想起光辉、希望、醉人的美好。今生今世永远爱美,爱迷人的美。任何不能令人满意的东西,不值得我们屈尊。 

我不要孤独,孤独是丑的,令人作呕的,灰色的。我要和你相通,共存,还有你的温暖,都是最迷人的啊!可惜我不漂亮。可是我诚心诚意呢,好吗我?我会爱,入迷,微笑,陶醉。好吗我? 

你真可爱,让人爱得要命。你一来,我就决心正经地、不是马虎地生活下去,哪怕要费心费力呢,哪怕我去牺牲呢。说傻话不解决问题。我知道为什么要爱,你也知道为什么要了吧?我爱,好好爱,你也一样吧。 小波 12月1日晚 



书信14 

——致银河(书简14) 

银河,你好! 

你给我带来一个多么美好的东西,就是说,一个多么好的夜晚!想你,想着呢。 

你呀,又勾起我想起好多事情。我们生活的支点是什么,就是我们自己。自己要一个绝对美好的不同凡响的生活,一个绝对美好的不同凡响的意义。你让我想起光辉、希望、醉人的美好。今生今世永远爱美,爱迷人的美。任何不能令人满意的东西,不值得我们屈尊。 

我不要孤独,孤独是丑的,令人作呕的,灰色的。我要和你相通,共存,还有你的温暖,都是最迷人的啊!可惜我不漂亮。可是我诚心诚意呢,好吗我?我会爱,入迷,微笑,陶醉。好吗我? 

你真可爱,让人爱得要命。你一来,我就决心正经地、不是马虎地生活下去,哪怕要费心费力呢,哪怕我去牺牲呢。说傻话不解决问题。我知道为什么要爱,你也知道为什么要了吧?我爱,好好爱,你也一样吧。 小波 12月1日晚 







——致银河(书简16) 

银河,你好! 

我又来对你瞎扯一通了。我这么胡说八道你生气了吗?可是我真爱你,只要你乐意听,我就老说个不停,像不像个傻子? 

真的,我那么爱你,你是个可爱的女孩子。男孩子们都喜欢女孩子,可是谁也没有我喜欢你这么厉害。我现在就很高兴,因为你又好又喜欢我,希望我高兴,有什么事情也喜欢说给我听。我和你就好像两个小孩子,围着一个神秘的果酱罐,一点一点地尝它,看看里面有多少甜。你干过偷果酱这样的事儿吗?我就干过,我猜你一定从来没干过,因为你乖。 

只有一件事情不好。你见过我小时候的相片了吧?过去我就是他,现在我不是他了,将来势必变成老头。这就不好了。要是你爱我,老和我好,变成老头我也不怕。咱们先来吃果酱吧,吃完了两个人就更好了,好到难舍难分,一起去见鬼去。你怕吗?我就不怕,见鬼就见鬼。我和你好。 

今天我累死啦!烦死啦!我整天在洗试管,洗烧杯,洗漏斗,洗该死的坛坛罐罐。我多倒霉,上这个劳什子大学。更倒霉的是一星期只能见你一次,其他时间只能和我不爱见的人在一起。 

昨天我看见了好多情侣,我觉得很喜欢那些人。过去我在马路边看见别人依依不舍就觉得肉麻,现在我忏悔。居然我能到了敢在大街上接吻的地步,我很自豪。 

爱情真美,倒霉的是咱们老不能爱个够。真不知我过去做过什么孽遭此重罚,因而连累了你。 

真希望下个星期日早来,并且那一天春光明媚。 小波 3月5日 





——致银河(书简17) 

「以下书信写于1978年冬李银河在外地调查期间」 

银河,你好! 

我收到你的信了。可是我仍然闷闷不乐,只有等你回来我才高兴呢。 

你可要我告诉你我过的是什么生活?可以告诉你,过的是没有你的生活。这种生活可真难挨。北京天气很冷,有时候天阴沉沉的,好像要开始一场政治说教,可真叫人腻歪。有时我沮丧得直想睡觉去。说实在的,我没有像堂吉诃德一样用甜甜的相思来度过时间,我没有,我的时间全在沮丧中度过。我很想你。 

我好像在挨牙痛,有一种抑郁的心情我总不能驱散它。我很想用一长串排比句来说明我多么想要你。可是排比句是头脑浅薄的人所好,我不用这东西,这种形式的东西我讨厌。我不用任何形式,我也不喜欢形容词。可以肯定说,我喜欢你,想你,要你。 

总之,爱人和被人爱都是无限的。 

你走了以后我写了几页最糟糕、顶顶要不得的东西,我真想烧了它。快考试了,没有时间再写啦。我写一个女孩子爱上一个男孩子之后想到:“我要和他一起深入这个天地,一去再也不回来。”我总也写不好爱情,什么热烈和温情也到不了我的笔端,我实在是低能透啦。我觉得爱情里有无限多的喜悦,它使人在生命的道路上步伐坚定。 

…… 

告诉你,我现在都嫉妒起别人的爱情来啦。我看到别人急急忙忙忙回家去找谁,或者看到别人在一起,心里就有一种不快,好像我被人遗弃了一样。吁,我好孤单! 





——致银河(书简18) 

银河,你好! 

我现在忙着应付期中考试和等你回来。你在外面过得好吗?我梦见过你几次了。 

北京好冷啊,还是南方暖和吧?我有点羡慕候鸟的生活:到了冬天就和你一起飞到南方去,飞到南太平洋的小岛上去。 

我要是个作曲家,我现在的心境做起“葬礼进行曲”来才叫才思不绝呢。我整天哭丧着脸。 

你要是回来我就高兴了,马上我就要放个震动北京城的大炮仗。银河,我爱你。我们来过快乐的生活吧!银河,快回来。 

银河,你好! 

你星期六就要回来了吧?那么说,只差两天了。啊,我盼望了好久了! 

你的信真好玩,你把所有的英文词都写错了,只有“党员”写对了,这件事儿真有趣。 

银河,我离党的要求越来越远啦。真的,我简直成了个社会生活中的叛逆。怎么说呢?我越来越认为,平庸的生活、为社会扮演角色,把人都榨干了。我们做的每一件事都是尽义务,我们自己的价值标准也是被规定了的。作人的乐趣不是太可怜了吗?难怪有人情愿作一只疯狗呢。 

最可憎的是人就此沉入一种麻木状态。既然你要做的一切都是别人做过一千万次的,那么这事还不令人作呕吗?比方说你我是26岁的男女,按照社会的需要26岁的男女应当如何如何,于是我们照此做去,一丝不苟,那么我们做人又有什么趣味?好像舔一只几千万人舔过的盘子,想想都令人作呕。 

我现在一拿笔就想写人们的相爱棗目空一切的那种相爱。可以说这样爱是反社会的。奥威尔说的不错,可是他的直觉有误,错到性欲上去了。总的来说,相爱是人的“本身”的行为,我们只能从相爱上看出人们的本色,其他的都沉入一片灰蒙蒙。也许是因为我太低能,所以看不到。也许有一天我会明白人需要什么,也就是撇开灰色的社会生活(倒霉的机械重复,乏味透顶的干巴巴的人的干涉),也撇开对于神圣的虔诚,人能给自己建立什么生活。如果人到了不受限制的情境,一点也不考虑人们怎么看自己,你看看他能有多疯吧。我猜人能做到欢乐之极,这也看人的才能大小。出于爱,人能干出透顶美好的事情,比木木痴痴的人胜过一万倍。 

我一想到你要回来就可高兴啦,我想你想得要命。现在可该结束了,就要和你在一起了。 

爱你。小波 





——致银河(书简19) 

王小波 

[以下书信写于1979年李银河在北京怀柔学习日语期间」 

银河,你好! 

我在这里想你想得要命,你想我了吗?我觉得我们在一起过的这几天好得要命,就是可惜你老有事,星期天我又像个中了风的大胖子一样躺下了,这真不好,扫了你的兴。 

我喜欢夏天,夏天晚上睡得晚,可以和你在一起,只要你不腻的话。我真希望你快点回来。等我考完了试,你又调成了工作,咱们就可以高兴地多在一起呆一会儿,不必像过去一样啦!过去像什么呢?我就像一个小鬼,等着机会溜进深宅大院去幽会,你就像个大家闺秀被管得死死的棗我是说你老在坐机关。你可别说我拉你后腿呀! 

咱们一定要学会在一起用功,像两个毛主席的好孩子。我们院过去有一个刷厕所的老头,有一天他问我厕所刷的白不白,我说白,他就说我是毛主席的好孩子,现在我还是呢。 

说真的希望你把日语学得棒棒的,你好好用功吧,我不打搅你。真的,你觉得我们在一起过的还好吗?夏天好吗? 

麦子熟了, 

天天都很热。 

等到明天一早, 

我就去收割。 

我的爱情也成熟了, 

很热的是我的心, 

但愿你,亲爱的,  

就是收割的人! 

这诗怎么样?喜欢吗?猜得出是谁的诗吗?是个匈牙利人写的呢。还有一首译得很糟: 

爱神,你干吗在这里,一手拿一只沙漏时计? 

怎么,轻浮的神,你用两种方法计时? 

这只慢的给分处两地的爱人们计时, 

另一只漏得快的给相聚一地的爱人们计时。 

这诗油腔滑调的不成个样子对不对?俗的好像姚文元写的呢。这可是诗哲歌德所做,亵渎不得。唉,说什么也是白搭,我还是耐心等你回来吧! 

小波 5月27日 





——致银河(书简20) 

银河,你好! 

收到你的信了。知道你过得还好,我挺高兴。 

我可是六神不安的,盼着你能早回来。你到底几号能回来呢?到底是16号呢还是20号?我以为这挺重要。过去我特别喜欢星期天,现在可是不喜欢了 

我在《德国诗选》里又发现一首好诗: 

他爱在黑暗中漫游,黝黑的树荫 

重重的树荫会冷却他的梦影。 

可是他的心里却燃烧着一种愿望,渴慕光明!渴慕光明! 

使他痛苦异常。 

他不知道,在他头上,碧空晴朗, 

充满了纯洁的银色的星光。 

我特别喜欢这一首。也许我们能够发现星光灿烂,就在我们中间。我尤其喜欢“银色的月光”。多么好,而且容易联想到你的名字。你的名字美极了。真的,单单你的名字就够我爱一世的了。 

我觉得我笨嘴笨舌不会讨你喜欢。就像马雅可夫斯基说的:“假如我像但丁或彼得拉那样口齿不灵!”真得,如果我像但丁或者彼得拉,我和你单独在一起、悄悄在一起时,我就在你耳边,悄悄地念一首充满韵律的诗,好象你的名字一样充满星光的诗。要不就说一个梦,一个星光下的梦,一个美好的故事。可惜我说不好。我太笨啦!真的,我太不会讨你喜欢啦!我一定还要学会这个。我能行吗?也就是说,你对我有信心吗?说真的,你说我前边说的重要吗? 

小波 6月6日 

银河,你好! 

你为什么不肯给我写信哪?难道非等接到我的信才肯写信吗?那样就要等一个星期才能有一封信,你不觉得太长了吗? 

我猜这封信到你手里恐怕要等不到你回信你就回来了。所以我也不能写些别的了。只能写爱你爱你爱你。你不在我多难过,好像旗杆上吊死的一只猫。猫在爱的时候怪叫,讨厌死啦!可是猫不管情人在哪儿都能找到她。但是如果被吊死在旗杆上它就不能了。我就像它。 

我现在感到一种凄惨的情绪,非马上找到你不可,否则就要哭一场才痛快。你为什么不来呢?我现在爱你爱的要发狂。我简直说不出什么有意思的话,只是直着嗓子哀鸣。人干吗要说咱们整天呆在一起不可思议?如果一天有48个小时,我恨不得49小时和你呆在一块呢!告诉你,我现在的感觉就像得不到你的爱,就像一个刚刚懂事的孩子那种说不出口的哑巴爱一样,成天傻想。喂,你干什么呢?你回来时我准比上次还爱呢。 

我认为你爱我和我爱你一边深,不然我的深从哪儿来呢?只不过我没出息,见不到你就难受极啦。所以,希望你快回来,回来快来找我,早一分钟都好得不得了。 

我爱你。 小波 6月9日 





——致银河(书简21) 

王小波 

[写在五线谱上的信]  

银河, 

你好!做梦也想不到我把信写到五线谱上吧?五线谱是偶然来的,你也是偶然来的。不过我给你的信值得写在五线谱里呢。但愿我和你,是一只唱不完的歌。 

谁也管不住我爱你,真的,谁管谁就真傻,我和你谁都管不住呢。你别怕,真的你谁也不要怕,最亲爱的好银河,要爱就爱个够吧,世界上没有比爱情更好的东西了。爱一回就够了,可以死了。什么也不需要了。这话傻不傻,我觉得我的话不能孤孤单单地写在这里,你要把你的信写在空白的地方。这可不是海誓山盟。海誓山盟是把现在的东西固定住。两个人都成了活化石。我们用不着它。我们要爱情长久。真的,它要长久我们就老在一块,不分开。你明白吗?你,你,真的,和你在一起就只知道有你,没有我,有你,多快活! 

我现在一想起有人写的爱情小说就觉得可怕极了。我决心不写爱情了。你看过缪塞的《提香的儿子》吗?提香的儿子给爱人画了一幅肖像,以后终身不作画了,它把画笔给了爱了。他做得对。奥,真的,我们为什么不早认识呢?那样我们到现在就已经爱了好多年。多么可惜啊!爱才没够呢。 

傻子才以为过家家是爱情呢,世俗的心理真可怕。不听他们的,不听。不管天翻地覆也好,昏天黑地也好,我们到一起来寻找安谧。我觉得我提起笔来冥想的时候,还有坐在你面前的时候,都到了人所不知的世界。世界没有这个哪成呢?过去是没有它就活的没意思,现在没有你也没意思。 

小波,让找们爱个够,爱个够!但愿我和你是一支唱不完的歌! 我看过100本小说,也许还要多,但是这句话是我生平所见过的最美的一句。你的心是多么美呵,太美了。你给我带来多么巨大的快乐和幸福。你为什么不写爱情呢?人生的全部的美都在这里呢,不写它写什么呢?爱把我们平淡的日子变成节日,把我们暗淡的生活照亮了,使它的颜色变得鲜明,使它的味道从一杯清淡的果汁变成浓烈的美酒。我们不该感谢它吗?不该为它歌唱吗?你这把钥匙就是开我这把锁的(或者反过来说)。我怕世俗那一套怕得要死,你比我一点不差。那就让我们一起远远地躲开它们,逃遁到我们那美好的、人所不知的世界里去吧。找我们的幸福,找我们的快乐,找我们灵魂的安谧,找我们生命的归宿。我们一起去找,找它一辈子,对吗? 





——致银河(书简22) 

银河,你好! 

今天你就要来了吧?我等得太久了。 

我很想天天看见你。真的,我们为什么不敢到一起来呢?我会妨碍你吗?你会妨碍我吗?爱情会妨碍我们两个吗?我们都不是神,不过这个问题我们一定能解决。只管爱吧,好银河,什么事也不会有。 

只要我们能在一起,我们什么都能找到。也许缺乏勇气是到达美好境界的障碍。你看我是多么适合你的人。我的勇气和你的勇气加起来,对付这个世界总够了吧?要无忧无虑地去抒情,去歌舞狂欢,去向世界发出我们的声音,我一个人是不敢的,我怕人家说我疯。有了你我就敢。只要有你一个,就不孤独! 

你真好,我真爱你,可惜我不是诗人,说不出再动人一点的话了。 

小波,你好! 

我今天晚上难过极了,想哭,也不知是为什么,我常有这种不正常的心情,觉得异常的孤独。生活也许在沸腾着,翻着泡沫,但我却忽然觉得我完全在它之外,我真羡慕那些无忧无虑的从不停歇的干下去的人。这个时候,谁也不能安慰,也许连你也不能。就像那首诗说的,像在雾中一样。我可能有一个致命的缺点,生命力还不够强。我的灵魂缺燃料,它有时虽然能迸出火花,但是不能总是熊熊燃烧。你的生命力比我强,我觉得你总是那么兴致勃勃的,就像居里说的,像一个飞转的陀螺。你该用你的速度来带动我,用你的火来燃烧我,用你的欢快的浪花把我从死水潭里带走。你会这样做吗?会吗,你一定会的。你应该这样做呀!为什么不给我打电话?难道你的热情已经过去了? 

银河,你好! 

你那天是多么悲伤啊,为什么我不在你身边呢?你孤独了,孤独就是黑暗,黑暗中的寂寞,多么让人害怕啊。 

你害怕雾吗?有一首诗,叫雾中散步。雾中散步,真正奇妙。谁都会有片刻的恍惚,觉得一切都走到了终结,也许再不能走下去了。其实我们的大限还远远没到呢。在大限到来之前,我们要把一切都做好,包括爱。这也是很重要的呀:爱你,真爱! 

我老把和你在一起的时间当节日来度过,我看你也是。其实这也不对。我们应当把我们的生活交织起来。不光有节日,还有艰苦的工作日。你说对吗?也许我是胡说。 你真坏,又说我热情过去了。 





——致银河(书简23) 

小波,你好: 

你是我的天堂,可我是你的地狱。我给你带来了太多的痛苦和烦恼。我们的爱情虽然很甜,但也有太多的苦味。这都怪我,都怪我。我有时十分痛恨自己,觉得我是一个坏人。昨天你说,我们两个都是好人,是特别好的人,真是这样吗?有时我觉得我自己真不怎么样,真坏。你来救我吧,你是我的天使,你总是把最美好的感情给成,你真好。我愿意要,我永远要不够,因为我常常觉得自己是很贫乏的,有时甚至很空虚。记得你也说过:我要。那么我也给,我也愿意给呵!!我们的幸福呵,让它再浓烈些,再浓烈些吧! 

我们常常把事情弄得太沉重了,咱们该轻松些,咱们应该像一对疯子那样歌舞狂欢,对吗?生活本来是很美好、很美好的呵! 

…… 

小波,你以为你找到了一个好朋友,可是你想到了吗?也许你为之要付出太多的代价,其中最主要的是:你将永远失去你的安静。我不会让你安静的,因为我是一个十分不安静的、过于敏感、甚至有点神经质的灵魂。我最害怕冷漠,哪怕有一点点,你就会失去我。我一点也受不了冷漠,真的。你能永远像现在这样热烈甚至还要超过它吗?你能永远满足我的“要”吗?你说过:要,对我来说,就是给,你能永远这样想吗?而且我还很爱妒忌,我甚至妒忌你小说里的女主角和那个被迷恋过的小女孩。我是不是很可笑?简直有点变态心理。你受得了吗?听人家说,女人的忌妒是美德,是吗?那证明我很爱你,不愿意你的感情被别的什么分去,不过你别听我的,好好写下去吧,好好写吧。 

…… 

我们能够幸福吗?能吗?这问题常常烦扰着我。你昨天的话使我似乎放心了。你是又聪明、又真挚的。你总是能为我们找到出路。但愿你永远能成功。 

我抄给你1月8日的日记,那是我满怀着热望和一颗跳动的心,但是发现你竟没给我写,我看着自己那些热情的话像一张树叶扔在水面上并没有激起什么波纹,觉得羞耻、觉得自尊心受到损害时写的。 

“我感到一阵失望,他这是怎么回事?难道他对我所有的也仅仅是那种动物式的感情?我真的爱他吗?我为什么那么容易动摇?我的心像一头不安的小鹿,总要跑掉,任何一点刺激,任何一点过失、松懈,都会使它脱缰而去,这怎么行呢?这样我们能够幸福吗?我应该告诉他。”  

如果我伤了你的心,请你原谅我,因为我们过去说过,要把心中发生的一切告诉对方。否则,它就会变成一种潜伏的危机。 

自从初恋之后,我好像违反一般规律地反而不懂得什么是爱了。你昨天说,要,就是爱。我相信你的话。我是一个内心时常会感到孤独的人,虽然我和朋友、家人亲密无间,但我仍常常感到可怕的孤独。我并不自命不凡,就像你也并不自命不凡一样。我也并不是很难了解的人。但是我觉得真正懂得我的只有你。我愿意爸爸妈妈都高兴,都满意,但是他们不高兴不满意我也会不顾一切的。我是一个自由人,谁也管不着。只要我们能够幸福。而这一点恰恰是我最担心的,我们能吗?能吗?我常常这样问自己。你那么热烈地爱我,想我,我也特别愿意投合你,满足你。我觉得能给你带来快乐,因为我你能快乐,这是我最高兴的事,也是引以为自豪自慰的。一个给别人带来快乐的人是幸福的,你知道吗?我还常常想,为了你我想变得美一些,我希望你爱我的全部肉体,我愿意它因为你而变得美。我甚至问你喜欢不喜欢香味。我愿意变成你所希望的样子,希望给你一切。你懂得我说的话吗?我好像是在胡说八道,说胡话。我也希望你变得美,你知道吗?我做梦还梦见你变得很美呢。 

我们可以拥有什么样的生活?对了,你说你和XX他们都不是一路人,这我也有感觉,我喜欢的也许就是这个,我从那么多人里一下子就把你和他们区别开来(用我妈妈的话说:一头就扎在……)也许就因为这个呢。但是我不是觉得什么一路不一路,我觉得质量不同。如果说他们的心是黄铜(或银子),那么你是金子。你不应该把自己和他们相提并论,有时,对自己的才能不自觉、羞怯,会毁了自己、糟踏了自己的。但是我觉得你不是很勤奋,韧性不太够,不知说得对不对。 

你也希望变成我所希望的样子吗?你愿意吗?你是不爱改造的,我也不愿改造你,但是我希望你怎样,有时会告诉你的,你愿意听吗? 

银河,你好! 

你责备我了。我觉得我近来是有点不像话,不过我总觉得是因为我忙。现在我知道我有点不好了。不,是有点坏。 

不过你的责备也过重。真的,过重!你以后会知道的。为什么怀疑我?你不应该。从来我都是这样,有时候大大咧咧,有时候马马虎虎,不过你要因为这个否定我,我可就太冤了!不要难道!你说的事情根本没有。也许你在日记里都把我说成是个山羊了。 

别怀疑我们会不会幸福。我来告诉你吧:我爱你爱的要命。我有时想起你就不能自已的狂喜,因为你是那样一个人。你也许不知人和人是多么不同:我哥哥说他是对一切充满了智慧的体系,不管哲学体系还是数学,哪怕它已经过时,只要它深刻、周密,他对它会有一种审美式的爱好。我也有一点。我也爱一切人类想出来的美好的东西,它们就像天外来客一样突然来到人间:有时候来龙去脉丝毫也没有呢。没有它们我们就太苦了。 

可是你最可爱。我想过的东西你想都不想,可是你从本性里爱美,不想就知道。你心里还有很多感情的波澜,你呀,就像波涛上的一只白帆船,波涛下面是个谜,这个谜就是女性。我很爱这些!不管你是哭是笑我全喜欢你。 有时候你难过了,这时候我更爱你。只要你不拒绝我就拥抱你,我会告诉你这是因为什么。就是我不知是为了什么。我会告诉你爱,爱可以把一切都容下。如果我的爱不能容下整个的你,算个什么爱!也许你的爱也能容下整个的我吧?不管怎么说,你要我的爱就够了。 





——致银河(书简24) 

小波,你好呵。 

…… 

噢,刚才我说爱情,有时我心里错综复杂,一会儿觉得美国人那种自由的随便的随心所欲的关系非常好,一会儿又觉得钟情的热恋始终如一好。我真不知哪种更好。看来你是后一种,你说过不赞成没有责任感。不愿我忘掉你。我不会忘掉你,永远不会,怎么可能呢?故意忘也忘不掉的。你不要怕失去我,我很愿意和你在一起,但是自由地和你在一起,你也保留你的自由权利吧。我看报看参考,越来越感到海誓山盟的时代过去了。如果没有感情我们就分离,我坚持这一点,不过我们可以约好互相安慰的义务,即一个人虽然已经不喜欢对方,但如果对方要求安慰,那个人有义务安慰对方,使这个人的心里好受一些,你同意吗?另外,我们不要大人,不论现在和将来,让我们把他们抛开,我们只是两个人,不是两家人,我们是两个在宇宙中游荡的灵魂,我们不愿孤独,走到一起来,别人与我们无关,好吗? 

银河,你好! 

我是有点懒,为什么不早给你写信呢? 

你说的话是对的,但是有一点不对。为什么要看报看参考看时代呢?我觉得这些完全与我们无关。不光美国人怎么做与我们关系不大,就是中国人怎么做也不用去考虑它。我就讨厌在这个问题上参考别人。 

海誓山盟,海誓山盟,这些别人的事情与我们无关。主要的是我对你的爱情。你想知道吗?这棵歪脖子树是怎么长着的。真的,我可不喜欢把它说成是花儿,这么说太大言不惭了。也许它会把我挂在上面呢。 

我老觉得爱情奇怪,它是一种宿命的东西。对我来说,它的内容就是“碰上了,然后就爱了,然后一点办法也没有了”。它就是这样!爱上,还非要人家也来爱不可。否则不叫爱,要它也没有意思。海誓山盟有什么用?我要的不就是我爱了人家人家也爱我吗?我爱海誓山盟拉来的一个人吗?不呢,爱一个爱我的人,就这样。 

我总觉得爱情神秘。不,我对你什么要求也没有,什么要求也没有,只要你来看我。我也不知道为什么。你愿意要什么,就给什么。你知道吗?要,对我来说,就是给啊。你要什么就是给我什么。随你吧。 

我是一个很有点反常的人呢。你不知道吧,我很愿意很愿意随和你呢。你不懂吧。我早就对你说过,我很爱嘲弄人,和别人老不能真心相处。我和朋友们之间都有一点心照不宣的东西,就是别人不告诉的东西也不打听,各人保守个人秘密。只有你,我不知为什么特别愿意随你的意。对于我和你,你要什么都是好的。 

上次行了一次骗,骗你上我这儿来,恐怕再不能取信于你了。那一天特别想看见你,你要不来我就像害牙疼一样难熬呢。我一下午都在编谎,后来编了一个关于法治的所谓想法,要你来讨论。不过你来了之后我可慌了,因为我说不出个道道来。你知道吗?我这人政治水平低,上政治课我睡得脖子都痛了。我能和你讨论什么政治吗?可是我居然能编出一些话来说,你说,这是不是我的胜利?也许是爱情的胜利?我现在沾沾自喜,告诉你也不怕,你来罚我也不怕,我太得意了,告诉你,那5页备忘录全是我星期三下午编出来的,还装着上星期就在酝酿的想法呢,还装着有所发现呢。 

你要知道,有时想你想得发疯呢。我不愿意等星期天,写信也是望梅止渴,我只好骗你来了。我也不愿意上门房找你,在门房里见面,那不是探监吗? 我这屋真冷,我手虽不抖,身上抖了。不行,我得睡了,再写下去你就不认得了。 





——致银河(书简25) 

  

小波,你好!  

你现在干什么呢?作业做完了,该看看小说了,又抽烟了吗? 

你谈到爱的神秘性,有时我心里很恐怖地想:爱也许是人对自己的一种欺骗,是一种奇异的想象力造出来的幻影。你的想象力强,所以总在我的周围看到一层光环,其实呢?那光芒并不存在。我怕你早晚会看到这一点,变得冷漠。爱也许就是这样一种神秘的想象力的发作,它会过去。人在最初的神秘感过去之后,会发现一个完全不同的世界,你以为神秘感会永远跟着你吗?它一旦过去,爱就会终结,是吗?多可怕。 

…… 

银河,你好! 

我想我不能同意你关于爱的神秘性的解释。不对,你说的不对。 

我想,人的生活其实是平淡无奇的。也许,我们都能做一次浪漫的梦是一种天赋人权吧?总之,你说是梦也好,它总是好的,比平淡无奇好得多。谁说是欺骗呢? 

我天生不喜欢枯燥的一切,简直不能理解人们总爱把有趣的事情弄的干巴起来。我要活化生活,真的,活化它。要活就活一个够。干什么要把什么事情都弄到一个死气沉沉的轨道里呢,好朋友?干什么你要总结什么是爱呢?你说那些可怕的话是吓唬我吧? 

你呀,你太该过一种真正幸福的生活了:一切都让它变幻无穷,不让它死气沉沉。我也许算不上一个好人,但是就是我死也要把你举高一点呢。说实话我对你将来如何看我一点也不在乎,总之,现在我们要好,对吗? 

…… 我特别相信你。世界上好人不少,不过你是最重要的一个。你要是愿意,我就永远爱你,你要不愿意,我就永远相思。对了,永远“相思”你。 





——致银河(书简26) 

小波,你好! 

你是我的光明,我的快乐,我的幸福。我们谁也不会妨碍对方。只会互相带来人生最宝贵的礼物。生活是有趣的,它绝不能变得死气沉沉。你说,我们将来也会把它弄成死气沉沉的吗?我在人群中看来看去,只有你有最大的可能性使我得到永远不枯燥的生活。你天生不喜欢枯燥,我也是呀。我真是怕它怕得要命呢。如果我们的精神枯竭了,我们的生活变得枯燥,那不如立刻去死了的好。 

你否认爱是人的自我欺偏,你说即使是梦也是好的,那我们就一起来作梦吧。我们生活在梦中,让生活变得像梦那么美,那么变幻无穷。但是我仍要让你想一下,并且回答我:这梦真能做一辈子吗?它会不会醒?醒来又怎么办?我们凭什么比其他和我们一样的人幸福,能一辈子生活在这美好的诗一般的梦里呢?我不是跟你说着玩,我是真不知道我们凭什么,而且对于将来的变化不敢想象。 

银河,你好! 

看了你的信。我来回答你的问题吧! 

真的,也许梦是做不了一辈子,那就让它成为真的好了!我和你就要努力进取,永不休止。对事业是这样,对美也是这样。有限的一切都不能让人满足,向无限进军中才能让人满足。无限不可能枯燥啊,好银河。永远会有新东西在我们面前出现的。哥伦布发现了新大陆,哥白尼发现了新宇宙,这是一条光荣的荆棘路。 

美也是无穷的,可怜的就是人的生命、人的活力是有穷的。可惜我看不到无穷的一切。但是我知道它存在,我向往它。我会老也会死,势必有一天我也会衰老得无力进取的。可是我不怕。在什么事物消失之前,我们先要让它存在啊。我记得有这么一只歌:“在门前清泉旁边,有一棵菩提树,在它的树荫下面,我做过甜蜜的梦……在它的树荫下面,我做过甜蜜的梦,无论是欢乐和悲伤,我总到那里去。”我愿作你的菩提树,你也来作我的吧。 

别怕美好的一切消失,咱们先来让它存在。还有一个美好的东西不会消失,就是菩提树。真希望你是我的菩提树,我愿作你的菩提树。你知道歌里是怎么唱吗?“如今我远离故乡,已经有许多年,我仍然听到呼唤,到这里寻找安谧。”灵魂是活生生的,它的安慰才能使人满足。 还有凭什么:凭着满心的热望,凭着活力。我不是说着玩的。 




——致银河(书简27) 

小波,好朋友,你好。 

…… 

自从我认识了你,我觉得所有的人都黯然失色,再也没有谁比你更好了,我的菩提树!……“无论是欢乐和悲伤,我总到那里去。”是呵,我的心总向往你,特别是在悲伤的时候。你的信太让我感动了。真的永远有新东西在前面吗?我说过了,我的活力不够,这一点从第一天见到你时我就看出来了:你的生命的活力在吸引我,我不由自主地要到你那里去,因为你那里有生活,有创造,有不竭的火,有不尽的源泉。我们一起请求上帝,愿它永远不要枯竭吧! 

我非常非常地想你,特别是在紧张工作的间歇。我觉得这世界上好像除了你和工作,什么都不存在了。你也这样想我吗? 

银河,你好! 

你真好,给我写了那么多信,这多好哇! 

冬天真可恨,把咱们弄得流离失所。让它快点过去吧!该死的天,还下起雪来了。冬天太可恨了。 

春天来了就好了。春天来了咱们一起去玩去。记得老歌德的五月之歌吗?爱情,爱情,灿烂如云……咱们约好了吧,春天一起去玩。我不太喜欢山,我喜欢广阔的田野、树林和河。咱们一定去吧。 

你说我太爱说,真的,我很有一点惭愧,我真是废话太多。不过我太爱你,我能不说吗?真的,我除了乱扯一通什么也不会,只好傻说了。我应当会写诗,写好多美丽的诗给你,可是我这笨蛋,我就不会把话说得响亮。我要是会了这个,再加上会把话说得精练,我就会写诗了。不管我本人多么平庸,我总觉得对你的爱很美。 

我真喜欢你的一举一动,多愁善感也喜欢。我总觉得你的心灵里有一种稚气得让人疼爱的模样,我这么说你不会生气吧?不过我不怕你生气,我也不和你见外。不管你怎么想我都这么说。我也不老成,疯起来我也和傻小子一样。只要你别趁我疯起来欺负我就成了。 

你说我上学苦,真的,真苦。什么时候我们可以自由自在地爱就好了。我不爱让人知道我是怎么想的,不过我永远不怕对任何人承认我爱你。爱呀,写呀,自由自在,可以自由自在地在一起。然后就是让我再和你分开,你到红墙后面,我去上学,咱们各做各的苦工,互相思念。一年有这么一个月就好。 

银河,你好! 

我越来越觉得冬天简直是我们的活灾星。你都不知道我多么希望你明天来看我。可是天多冷啊!路多难走哇!你怎么能来呢,千万不要来。 

静下来想你,觉得一切都美好得不可思议。以前我不知道爱情这么美好。爱到深处这么美好。真不想让任何人来管我们。谁也管不着,和谁都无关。告诉你,一想到你,我这张丑脸上就泛起微笑。还有在我安静的时候,你就从我内心深处浮现,就好像阿芙罗蒂从浪花里浮现一样。你别笑,这个比喻太陈腐了,可是你也知道了吧?亲爱的,你在这里呢。 

你瞧,你从我内心深处经常出现,给我带来幸福,还有什么离间得了我们?咱们可不会变成火炉边的两个傻瓜。别人也许会诧异咱们的幸福和他们的不一样,可那与我们有何相干?他们的我们不要,我们的他们也不知道。 

你要我多给你写,可是我写的总不如你好。上气不接下气的。不过上气不接下气的也不要紧,是给你的,是要你知道这颗心怎么跳。难道我还不能信赖你吗?难道对你还要像对社会一样藏起缺点抖擞精神吗?人对自己有时恍惚一点,大大咧咧,自己喜欢自己随便一点。你也对我随便好了。主要是信赖啊!将来啊,我们要是兴致都高涨就一起出去疯跑,你兴致不高就来吧:哭也好,说也好,懒也好,我都喜欢你。有时候我也会没精打彩,那时候不许你欺负我!不过我反正不怕你笑话。 小波
